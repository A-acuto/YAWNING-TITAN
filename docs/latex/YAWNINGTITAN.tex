%% Generated by Sphinx.
\def\sphinxdocclass{report}
\documentclass[letterpaper,10pt,english]{sphinxmanual}
\ifdefined\pdfpxdimen
   \let\sphinxpxdimen\pdfpxdimen\else\newdimen\sphinxpxdimen
\fi \sphinxpxdimen=.75bp\relax
\ifdefined\pdfimageresolution
    \pdfimageresolution= \numexpr \dimexpr1in\relax/\sphinxpxdimen\relax
\fi
%% let collapsible pdf bookmarks panel have high depth per default
\PassOptionsToPackage{bookmarksdepth=5}{hyperref}

\PassOptionsToPackage{warn}{textcomp}
\usepackage[utf8]{inputenc}
\ifdefined\DeclareUnicodeCharacter
% support both utf8 and utf8x syntaxes
  \ifdefined\DeclareUnicodeCharacterAsOptional
    \def\sphinxDUC#1{\DeclareUnicodeCharacter{"#1}}
  \else
    \let\sphinxDUC\DeclareUnicodeCharacter
  \fi
  \sphinxDUC{00A0}{\nobreakspace}
  \sphinxDUC{2500}{\sphinxunichar{2500}}
  \sphinxDUC{2502}{\sphinxunichar{2502}}
  \sphinxDUC{2514}{\sphinxunichar{2514}}
  \sphinxDUC{251C}{\sphinxunichar{251C}}
  \sphinxDUC{2572}{\textbackslash}
\fi
\usepackage{cmap}
\usepackage[T1]{fontenc}
\usepackage{amsmath,amssymb,amstext}
\usepackage{babel}



\usepackage{tgtermes}
\usepackage{tgheros}
\renewcommand{\ttdefault}{txtt}



\usepackage[Bjarne]{fncychap}
\usepackage{sphinx}

\fvset{fontsize=auto}
\usepackage{geometry}


% Include hyperref last.
\usepackage{hyperref}
% Fix anchor placement for figures with captions.
\usepackage{hypcap}% it must be loaded after hyperref.
% Set up styles of URL: it should be placed after hyperref.
\urlstyle{same}

\addto\captionsenglish{\renewcommand{\contentsname}{Contents:}}

\usepackage{sphinxmessages}
\setcounter{tocdepth}{1}



\title{YAWNING TITAN Documentation}
\date{Apr 21, 2022}
\release{1.0}
\author{Josh Collyer, Alex Andrew, Liam Quantrill}
\newcommand{\sphinxlogo}{\vbox{}}
\renewcommand{\releasename}{Release}
\makeindex
\begin{document}

\pagestyle{empty}
\sphinxmaketitle
\pagestyle{plain}
\sphinxtableofcontents
\pagestyle{normal}
\phantomsection\label{\detokenize{index::doc}}



\chapter{Introduction}
\label{\detokenize{source/intro:introduction}}\label{\detokenize{source/intro::doc}}

\section{About}
\label{\detokenize{source/intro:about}}
\sphinxAtStartPar
YAWNING TITAN is an abstract, graph based cyber\sphinxhyphen{}security simulation environment that supports the training of
intelligent agents for autonomous cyber operations. YAWNING TITAN currently only supports defensive autonomous agents
who face off against probabilistic red agents.
\begin{description}
\item[{YAWNING TITAN has been designed with the following things in mind:}] \leavevmode\begin{itemize}
\item {}
\sphinxAtStartPar
Simplicity over complexity

\item {}
\sphinxAtStartPar
Minimal Hardware Requirements

\item {}
\sphinxAtStartPar
Support for a wide range of algorithms

\item {}
\sphinxAtStartPar
Enhanced agent/policy evaluation support

\item {}
\sphinxAtStartPar
Flexible environment and game rule setup

\item {}
\sphinxAtStartPar
Generation of evaluation episode visualisations (gifs)

\end{itemize}

\end{description}


\section{What is YAWNING TITAN built with}
\label{\detokenize{source/intro:what-is-yawning-titan-built-with}}
\sphinxAtStartPar
YAWNING TITAN is built using the following libraries:
\begin{itemize}
\item {}
\sphinxAtStartPar
{[}OpenAI’s Gym{]}(\sphinxurl{https://gym.openai.com/})

\item {}
\sphinxAtStartPar
{[}Networkx{]}(\sphinxurl{https://github.com/networkx/networkx})

\item {}
\sphinxAtStartPar
{[}Stable Baselines 3{]}(\sphinxurl{https://github.com/DLR-RM/stable-baselines3})

\item {}
\sphinxAtStartPar
{[}Rllib (part of Ray){]}(\sphinxurl{https://github.com/ray-project/ray})

\end{itemize}


\chapter{Getting Started}
\label{\detokenize{source/getting_started:getting-started}}\label{\detokenize{source/getting_started::doc}}

\section{Pre\sphinxhyphen{}Requisites}
\label{\detokenize{source/getting_started:pre-requisites}}\begin{description}
\item[{In order to get YAWNING TITAN installed, you will need to have the following installed:}] \leavevmode\begin{itemize}
\item {}
\sphinxAtStartPar
python3.6+

\item {}
\sphinxAtStartPar
python3\sphinxhyphen{}pip

\item {}
\sphinxAtStartPar
virtualenv

\end{itemize}

\end{description}


\section{Installation}
\label{\detokenize{source/getting_started:installation}}\begin{enumerate}
\sphinxsetlistlabels{\arabic}{enumi}{enumii}{}{.}%
\item {} \begin{description}
\item[{Navigate to the YAWNING TITAN folder and create a new python virtual environment}] \leavevmode
\sphinxAtStartPar
python3 \sphinxhyphen{}m venv \textless{}name\_of\_venv\textgreater{}

\end{description}

\item {} \begin{description}
\item[{Activate virtual environment}] \leavevmode
\sphinxAtStartPar
source \textless{}name\_of\_venv\textgreater{}

\end{description}

\item {} \begin{description}
\item[{Install yawning\sphinxhyphen{}titan into the environment along with all of it’s dependencies}] \leavevmode
\sphinxAtStartPar
python3 \sphinxhyphen{}m pip install \sphinxhyphen{}e .

\end{description}

\end{enumerate}


\section{Creating a generic environment}
\label{\detokenize{source/getting_started:creating-a-generic-environment}}
\sphinxAtStartPar
Yawning Titan also includes the functionality to create a
generic network environment. The generic environment needs to
be given a couple of things to run:
\begin{itemize}
\item {}
\sphinxAtStartPar
A Red agent (Probabilistic Red agent supplied)

\item {}
\sphinxAtStartPar
A Blue agent  (RL Blue agent supplied)

\item {}
\sphinxAtStartPar
A settings file (Examples supplied)

\item {}
\sphinxAtStartPar
A adjacency matrix for the network

\item {}
\sphinxAtStartPar
A dictionary of positions of points if the graph is displayed on a 2d graph

\end{itemize}

\sphinxAtStartPar
A network creator file has been supplied that can generate networks through
some standard methods (mesh, star, ect) to create the adjacency matrix and
the dictionary of positions

\sphinxAtStartPar
You can also supply the environment with the following optional information
(defaults are used if not supplied):
\begin{itemize}
\item {}
\sphinxAtStartPar
Entry Nodes (The gateway nodes that the red uses to attack the network)

\item {}
\sphinxAtStartPar
Vulnerabilities (How vulnerable each node is to attack)

\end{itemize}

\sphinxAtStartPar
To see examples of how the generic environment can be set up and run
see the notebook file:
\begin{quote}

\sphinxAtStartPar
\sphinxcode{\sphinxupquote{notebooks/Creating a new generic environment.ipynb}}
\end{quote}


\subsection{Setting up the configuration file}
\label{\detokenize{source/getting_started:setting-up-the-configuration-file}}
\sphinxAtStartPar
YAWNING TITANS generic network is highly configurable. The main way that you can affect
and change the environment is through the config file. The config file is a yaml file that
contains settings such as what actions the blue agent can take and how the blue agent
wins. The config file is broken into the following sections:
\begin{itemize}
\item {} \begin{description}
\item[{RED}] \leavevmode
\sphinxAtStartPar
Settings relating to the red agent

\end{description}

\item {} \begin{description}
\item[{BLUE}] \leavevmode
\sphinxAtStartPar
Settings relating to the blue agent

\end{description}

\item {} \begin{description}
\item[{GAME RULES}] \leavevmode
\sphinxAtStartPar
Settings relating to the way the game is played and how it is won

\end{description}

\item {} \begin{description}
\item[{REWARDS}] \leavevmode
\sphinxAtStartPar
Settings relating to the rewards that the blue agent gets for its actions

\end{description}

\item {} \begin{description}
\item[{MISCELLANEOUS}] \leavevmode
\sphinxAtStartPar
Other settings that do not fit into any other category

\end{description}

\end{itemize}

\sphinxAtStartPar
The network interface comes with a config file check that ensures that the settings file
has not become corrupted and that the settings values chosen are valid.

\sphinxAtStartPar
To see a description of every setting see: {\hyperref[\detokenize{source/config_file:config-file}]{\sphinxcrossref{\DUrole{std,std-ref}{The Config File Explained}}}}

\sphinxAtStartPar
You can create your own settings file or use one of the pre\sphinxhyphen{}made ones.

\sphinxAtStartPar
If you do not supply the network with a config file then it will use a default one.
To supply one of your own config files then pass the file path of your config file
to the network\_interface.


\subsection{Creating a network}
\label{\detokenize{source/getting_started:creating-a-network}}
\sphinxAtStartPar
You will need to define the network that the agents will compete in. The network interface
requires an adjacency matrix and a dictionary of point locations (used to render the network).

\sphinxAtStartPar
YT contains a couple of builtin methods to create networks based of standard topologies.
Theses include:
\begin{itemize}
\item {} \begin{description}
\item[{create\_18\_node\_network}] \leavevmode
\sphinxAtStartPar
Creates the 18 node network for the research paper: \sphinxurl{https://www.nsa.gov.Portals/70/documents/resources/everyone/digital-media-center/publications/the-next-wave/TNW-22-1.pdf\#page=9}

\end{description}

\item {} \begin{description}
\item[{create\_mesh}] \leavevmode
\sphinxAtStartPar
Creates a mesh network with variable connectivity

\end{description}

\item {} \begin{description}
\item[{create\_star}] \leavevmode
\sphinxAtStartPar
Creates a network based of the star topology

\end{description}

\item {} \begin{description}
\item[{create\_p2p}] \leavevmode
\sphinxAtStartPar
Creates a network based of two “peers” connecting

\end{description}

\item {} \begin{description}
\item[{create ring}] \leavevmode
\sphinxAtStartPar
Creates a network based of the ring topology

\end{description}

\item {} \begin{description}
\item[{custom\_network}] \leavevmode
\sphinxAtStartPar
Creates a network using console input from the user

\end{description}

\item {} \begin{description}
\item[{procedural\_network}] \leavevmode
\sphinxAtStartPar
Creates a network with defined amounts of nodes with certain connectivity

\end{description}

\item {} \begin{description}
\item[{gnp\_random\_connected\_graph}] \leavevmode
\sphinxAtStartPar
Creates a mesh that is guaranteed for each node to have at least one connection

\end{description}

\end{itemize}

\sphinxAtStartPar
To create the data for a network:

\begin{sphinxVerbatim}[commandchars=\\\{\}]
\PYG{n}{matrix}\PYG{p}{,} \PYG{n}{node\PYGZus{}positions} \PYG{o}{=} \PYG{n}{network\PYGZus{}creator}\PYG{o}{.}\PYG{n}{create\PYGZus{}18\PYGZus{}node\PYGZus{}network}\PYG{p}{(}\PYG{p}{)}

\PYG{n}{network\PYGZus{}creator}\PYG{o}{.}\PYG{n}{save\PYGZus{}network}\PYG{p}{(}\PYG{l+s+s2}{\PYGZdq{}}\PYG{l+s+s2}{current\PYGZus{}net.txt}\PYG{l+s+s2}{\PYGZdq{}}\PYG{p}{,} \PYG{n}{matrix}\PYG{p}{,} \PYG{n}{node\PYGZus{}positions}\PYG{p}{)}
\PYG{n}{matrix}\PYG{p}{,} \PYG{n}{node\PYGZus{}positions} \PYG{o}{=} \PYG{n}{network\PYGZus{}creator}\PYG{o}{.}\PYG{n}{load\PYGZus{}network}\PYG{p}{(}\PYG{l+s+s2}{\PYGZdq{}}\PYG{l+s+s2}{current\PYGZus{}net.txt}\PYG{l+s+s2}{\PYGZdq{}}\PYG{p}{)}
\end{sphinxVerbatim}

\sphinxAtStartPar
The above code also shows how to save and load networks


\subsection{Generating the network interface}
\label{\detokenize{source/getting_started:generating-the-network-interface}}
\sphinxAtStartPar
Create the Network Interface object:

\begin{sphinxVerbatim}[commandchars=\\\{\}]
\PYG{n}{network\PYGZus{}interface} \PYG{o}{=} \PYG{n}{NetworkInterface}\PYG{p}{(}\PYG{n}{matrix}\PYG{p}{,} \PYG{n}{node\PYGZus{}positions}\PYG{p}{)}
\end{sphinxVerbatim}

\sphinxAtStartPar
There are also some optional parameters that you can use in the the network interface
\begin{itemize}
\item {} \begin{description}
\item[{entry\_nodes}] \leavevmode
\sphinxAtStartPar
A list of node names that act as doorways to the network for the red agent. If left
blank then the network generates some automatically. There are options in the settings
file to choose how entry nodes are generated if they are left blank.

\end{description}

\item {} \begin{description}
\item[{vulnerabilities}] \leavevmode
\sphinxAtStartPar
A dictionary of node names and vulnerabilities. A vulnerability is a number between 0 and 1
that represents how easy a node is to compromise (1 very easy, 0 very hard). If left
blank then generated randomly.

\end{description}

\item {} \begin{description}
\item[{high value node}] \leavevmode
\sphinxAtStartPar
If the config is set up so that the red agent wins if it compromises a high value
node then you can set the name of the node to be the target. Generated automatically
if left blank.

\end{description}

\item {} \begin{description}
\item[{settings\_path}] \leavevmode
\sphinxAtStartPar
The path to the settings file. If left blank a default settings file is used. To see
more information on the settings file see: {\hyperref[\detokenize{source/config_file:config-file}]{\sphinxcrossref{\DUrole{std,std-ref}{The Config File Explained}}}}

\end{description}

\end{itemize}


\subsection{Settings up the Red and Blue agents}
\label{\detokenize{source/getting_started:settings-up-the-red-and-blue-agents}}
\sphinxAtStartPar
To run an experiment through the generic network environment you will need a red and a
blue agent.

\sphinxAtStartPar
YAWNING TITAN comes supplied with a probabilistic customisable red agent and a
customisable RL blue agent.

\sphinxAtStartPar
Both the red and blue agents can be modified by changing the settings in the configuration
file under the appropriate section.

\sphinxAtStartPar
To create a blue agent:

\begin{sphinxVerbatim}[commandchars=\\\{\}]
\PYG{n}{blue\PYGZus{}agent} \PYG{o}{=} \PYG{n}{BlueInterface}\PYG{p}{(}\PYG{n}{network\PYGZus{}interface}\PYG{p}{)}
\end{sphinxVerbatim}

\sphinxAtStartPar
To create a red agent:

\begin{sphinxVerbatim}[commandchars=\\\{\}]
\PYG{n}{red\PYGZus{}agent} \PYG{o}{=} \PYG{n}{RedInterface}\PYG{p}{(}\PYG{n}{network\PYGZus{}interface}\PYG{p}{)}
\end{sphinxVerbatim}


\subsection{Creating the environment}
\label{\detokenize{source/getting_started:creating-the-environment}}
\sphinxAtStartPar
Create the open AI gym environment

\begin{sphinxVerbatim}[commandchars=\\\{\}]
\PYG{n}{number\PYGZus{}of\PYGZus{}actions} \PYG{o}{=} \PYG{n}{blue\PYGZus{}agent}\PYG{o}{.}\PYG{n}{get\PYGZus{}number\PYGZus{}of\PYGZus{}actions}\PYG{p}{(}\PYG{p}{)}

\PYG{n}{env} \PYG{o}{=} \PYG{n}{GenericNetworkEnv}\PYG{p}{(}\PYG{n}{red\PYGZus{}agent}\PYG{p}{,} \PYG{n}{blue\PYGZus{}agent}\PYG{p}{,} \PYG{n}{network\PYGZus{}interface}\PYG{p}{,} \PYG{n}{number\PYGZus{}of\PYGZus{}actions}\PYG{p}{)}
\end{sphinxVerbatim}


\chapter{Tutorials}
\label{\detokenize{source/tutorials:tutorials}}\label{\detokenize{source/tutorials::doc}}

\section{Example notebooks}
\label{\detokenize{source/tutorials:example-notebooks}}
\sphinxAtStartPar
There are many example notebooks that come with YAWNING TITAN to help show you have to
use certain features.

\sphinxAtStartPar
These notebooks take you through step by step explaining how to do certain tasks such as
creating a custom environment or extracting information from a tensorboard.
\begin{description}
\item[{The notebooks can be found at:}] \leavevmode
\sphinxAtStartPar
\sphinxcode{\sphinxupquote{yawning\sphinxhyphen{}titan/notebooks}}

\end{description}

\sphinxAtStartPar
Some good notebooks to get started are:
\begin{itemize}
\item {} \begin{description}
\item[{Creating a custom environment \& training an agent \& rendering the agents performance.ipynb}] \leavevmode
\sphinxAtStartPar
Shows you how to create a custom environment from the very beginning and takes you through all the way
to training the agent and then rendering its performance at the end.

\end{description}

\item {} \begin{description}
\item[{Creating and playing as a Keyboard Agent.ipynb}] \leavevmode
\sphinxAtStartPar
Shows you how to create a Keyboard Agent that will allow you to be able to play the game yourself.

\end{description}

\item {} \begin{description}
\item[{Using an Evaluation Callback to monitor progress during training.ipynb}] \leavevmode
\sphinxAtStartPar
Shows you how to create a simple environment and an agent that can give regular updates on its
performance throughout training.

\end{description}

\end{itemize}


\chapter{Monitoring Experiments}
\label{\detokenize{source/experiments:monitoring-experiments}}\label{\detokenize{source/experiments::doc}}

\section{Mid Experiment}
\label{\detokenize{source/experiments:mid-experiment}}
\sphinxAtStartPar
When you create an agent you can include the verbose tag to give you updates on the training:

\begin{sphinxVerbatim}[commandchars=\\\{\}]
\PYG{k+kn}{from} \PYG{n+nn}{stable\PYGZus{}baselines3} \PYG{k+kn}{import} \PYG{n}{PPO}
\PYG{k+kn}{from} \PYG{n+nn}{stable\PYGZus{}baselines3}\PYG{n+nn}{.}\PYG{n+nn}{ppo} \PYG{k+kn}{import} \PYG{n}{MlpPolicy} \PYG{k}{as} \PYG{n}{PPOMlp}

\PYG{n}{agent} \PYG{o}{=} \PYG{n}{PPO}\PYG{p}{(}\PYG{n}{PPOMlp}\PYG{p}{,} \PYG{n}{env}\PYG{p}{,} \PYG{n}{verbose}\PYG{o}{=}\PYG{l+m+mi}{1}\PYG{p}{)}
\end{sphinxVerbatim}

\sphinxAtStartPar
If you are using experiment\_runner.py then this will automatically be turned on.

\sphinxAtStartPar
This verbose tag will create these at set intervals during training

\noindent\sphinxincludegraphics[width=400\sphinxpxdimen]{{mid_experiment_example_1}.png}
\begin{description}
\item[{The update contains information such as:}] \leavevmode\begin{itemize}
\item {} \begin{description}
\item[{ep\_len\_mean}] \leavevmode
\sphinxAtStartPar
The average length of each game over this episode

\end{description}

\item {} \begin{description}
\item[{ep\_rew\_mean}] \leavevmode
\sphinxAtStartPar
The average reward the blue agent received over this episode

\end{description}

\item {} \begin{description}
\item[{loss}] \leavevmode
\sphinxAtStartPar
Indicates how bad a prediction is that a model took. A perfect prediction is zero and the bigger the loss, the bigger the negative reward incurred from the prediction

\end{description}

\end{itemize}

\end{description}

\sphinxAtStartPar
The verbose tag will also give you a brief evaluation at the end of training

\noindent\sphinxincludegraphics[width=400\sphinxpxdimen]{{verbose_eval}.png}

\sphinxAtStartPar
If using the generic network environment there are a couple of toggles that you can use to collect more information about the training process and env:

\begin{sphinxVerbatim}[commandchars=\\\{\}]
\PYG{n}{env} \PYG{o}{=} \PYG{n}{GenericNetworkEnv}\PYG{p}{(}
    \PYG{n}{red}\PYG{p}{,}
    \PYG{n}{blue}\PYG{p}{,}
    \PYG{n}{network\PYGZus{}interface}\PYG{p}{,}
    \PYG{n}{number\PYGZus{}of\PYGZus{}actions}\PYG{p}{,}
    \PYG{n}{print\PYGZus{}metrics}\PYG{o}{=}\PYG{k+kc}{True}\PYG{p}{,}
    \PYG{n}{show\PYGZus{}metrics\PYGZus{}every}\PYG{o}{=}\PYG{l+m+mi}{10}\PYG{p}{,}
    \PYG{n}{collect\PYGZus{}data}\PYG{o}{=}\PYG{k+kc}{False}\PYG{p}{,}
\PYG{p}{)}
\end{sphinxVerbatim}

\sphinxAtStartPar
When creating an env you can toggle \sphinxtitleref{print\_metrics} to True. This will print out stats at the end of each game
showing you what actions were taken, how long the game lasted and who won.

\sphinxAtStartPar
You can also toggle \sphinxtitleref{collect\_data} to True. This will collect a lot of data about each action and how it has affected the
environment at each “step” of the step method. All this data will then be returned from the step method as a dictionary.


\section{End of Experiment}
\label{\detokenize{source/experiments:end-of-experiment}}
\sphinxAtStartPar
If using the experiment\_runner.py then after training some summary stats will be printed out as shown below.

\noindent\sphinxincludegraphics[width=400\sphinxpxdimen]{{end_of_eval_example_1}.png}


\section{Rendering}
\label{\detokenize{source/experiments:rendering}}
\sphinxAtStartPar
Most of the environments in YAWNING TITAN support rendering and have a render method. This
method will create a matplotlib graph showing the current state of the environment.
Here are some images showing how the rendering looks:

\noindent\sphinxincludegraphics[width=800\sphinxpxdimen]{{example_render_1}.png}

\sphinxAtStartPar
Creating an ActionLoop with the environment and agent will run the agent through a game rendering each step. A tutorial on how
to do this can be found at:
\begin{quote}

\sphinxAtStartPar
\sphinxcode{\sphinxupquote{yawning\sphinxhyphen{}titan/notebooks/Creating a custom envrionment \& training an agent \& rendering the agents performance.ipynb}}
\end{quote}

\sphinxAtStartPar
Once the rendering is complete the entire episode will be saved as a gif so that you can watch it again at any time.

\sphinxAtStartPar
When rendering an environment it is automatically set up to show you the true state of the environment. You can however set it to only show the blue agent’s
view of the environment. This can be toggled by passing in:
\begin{quote}

\sphinxAtStartPar
show\_only\_blue\_view=True
\end{quote}


\chapter{The Config File Explained}
\label{\detokenize{source/config_file:the-config-file-explained}}\label{\detokenize{source/config_file:config-file}}\label{\detokenize{source/config_file::doc}}

\section{Red agent settings:}
\label{\detokenize{source/config_file:red-agent-settings}}\begin{itemize}
\item {} \begin{description}
\item[{\sphinxstylestrong{red\_skill:}}] \leavevmode
\sphinxAtStartPar
The red agents skill level. Higher means that red is more likely to succeed in attacks

\end{description}

\item {} \begin{description}
\item[{\sphinxstylestrong{red\_uses\_skill:}}] \leavevmode
\sphinxAtStartPar
Red uses its skill modifier when attacking nodes

\end{description}

\item {} \begin{description}
\item[{\sphinxstylestrong{red\_ignores\_defences:}}] \leavevmode
\sphinxAtStartPar
The red agent ignores the defences of nodes

\end{description}

\item {} \begin{description}
\item[{\sphinxstylestrong{red\_always\_succeeds:}}] \leavevmode
\sphinxAtStartPar
Reds attacks always succeed

\end{description}

\item {} \begin{description}
\item[{\sphinxstylestrong{red\_can\_only\_attack\_from\_red\_agent\_node:}}] \leavevmode
\sphinxAtStartPar
The agent has a central “command” node. the red agent can only attack from its command node and so has to move around the environment to be able to be able to conquer the whole thing

\end{description}

\item {} \begin{description}
\item[{\sphinxstylestrong{red\_can\_attack\_from\_any\_red\_node:}}] \leavevmode
\sphinxAtStartPar
The red agent can act from and control any node that it has compromised

\end{description}

\item {} \begin{description}
\item[{\sphinxstylestrong{red\_can\_naturally\_spread:}}] \leavevmode
\sphinxAtStartPar
The red agent naturally spreads its influence every time\sphinxhyphen{}step

\end{description}

\item {} \begin{description}
\item[{\sphinxstylestrong{chance\_to\_spread\_to\_connected\_node:}}] \leavevmode
\sphinxAtStartPar
If a node is connected to a compromised node what chance does it have to become compromised every turn through natural spreading

\end{description}

\item {} \begin{description}
\item[{\sphinxstylestrong{chance\_to\_spread\_to\_unconnected\_node:}}] \leavevmode
\sphinxAtStartPar
If a node is not connected to a compromised node what chance does it have to become randomly infected through natural spreading

\end{description}

\item {} \begin{description}
\item[{\sphinxstylestrong{Red Actions}}] \leavevmode\begin{itemize}
\item {}
\sphinxAtStartPar
\sphinxstylestrong{red\_uses\_spread\_action:}
\begin{quote}

\sphinxAtStartPar
Every compromised node has a chance to infect every neighbouring safe node
\begin{itemize}
\item {} \begin{description}
\item[{\sphinxstylestrong{spread\_action\_likelihood:}}] \leavevmode
\sphinxAtStartPar
The weighting for the action \sphinxhyphen{} how often the red agent performs the action

\end{description}

\item {} \begin{description}
\item[{\sphinxstylestrong{chance\_for\_red\_to\_spread:}}] \leavevmode
\sphinxAtStartPar
The chance for each “spread” to occur

\end{description}

\end{itemize}
\end{quote}

\item {}
\sphinxAtStartPar
\sphinxstylestrong{red\_uses\_random\_infect\_action:}
\begin{quote}

\sphinxAtStartPar
Red has a chance to infect every safe node irrespective of its position in the network
\begin{itemize}
\item {} \begin{description}
\item[{\sphinxstylestrong{random\_infect\_action\_likelihood:}}] \leavevmode
\sphinxAtStartPar
The weighting for the action \sphinxhyphen{} how often the red agent performs the action

\end{description}

\item {} \begin{description}
\item[{\sphinxstylestrong{chance\_for\_red\_to\_random\_compromise:}}] \leavevmode
\sphinxAtStartPar
The chance for each “infect” to succeed

\end{description}

\end{itemize}
\end{quote}

\item {}
\sphinxAtStartPar
\sphinxstylestrong{red\_uses\_basic\_attack\_action:}
\begin{quote}

\sphinxAtStartPar
The red agent picks a single node connected to an infected node and tries to attack and take over that node
\begin{itemize}
\item {} \begin{description}
\item[{\sphinxstylestrong{basic\_attack\_action\_likelihood:}}] \leavevmode
\sphinxAtStartPar
The weighting for the action \sphinxhyphen{} how often the red agent performs the action

\end{description}

\end{itemize}
\end{quote}

\item {}
\sphinxAtStartPar
\sphinxstylestrong{red\_uses\_do\_nothing\_action:}
\begin{quote}

\sphinxAtStartPar
The red agent does nothing on this turn
\begin{itemize}
\item {} \begin{description}
\item[{\sphinxstylestrong{do\_nothing\_action\_likelihood:}}] \leavevmode
\sphinxAtStartPar
The weighting for the action \sphinxhyphen{} how often the red agent performs the action

\end{description}

\end{itemize}
\end{quote}

\item {}
\sphinxAtStartPar
\sphinxstylestrong{red\_uses\_move\_action:}
\begin{quote}

\sphinxAtStartPar
The red agent moves it central “control” node to a connected compromised node
\begin{itemize}
\item {} \begin{description}
\item[{\sphinxstylestrong{move\_action\_likelihood:}}] \leavevmode
\sphinxAtStartPar
The weighting for the action \sphinxhyphen{} how often the red agent performs the action

\end{description}

\end{itemize}
\end{quote}

\item {}
\sphinxAtStartPar
\sphinxstylestrong{red\_uses\_zero\_day\_action:}
\begin{quote}

\sphinxAtStartPar
Red builds up and uses an attack that has a 100\% success chance to compromise a single node
\begin{itemize}
\item {} \begin{description}
\item[{\sphinxstylestrong{zero\_day\_start\_amount:}}] \leavevmode
\sphinxAtStartPar
The number of zero day attacks the red agent starts with

\end{description}

\item {} \begin{description}
\item[{\sphinxstylestrong{days\_required\_for\_zero\_day:}}] \leavevmode
\sphinxAtStartPar
The number of progress “points” that the red agent needs to earn to perform a zero day action

\end{description}

\end{itemize}
\end{quote}

\end{itemize}

\end{description}

\end{itemize}
\begin{itemize}
\item {} \begin{description}
\item[{\sphinxstylestrong{Red Targeting methods}}] \leavevmode\begin{itemize}
\item {} \begin{description}
\item[{\sphinxstylestrong{red\_chooses\_target\_at\_random:}}] \leavevmode
\sphinxAtStartPar
Red picks nodes to attack at random

\end{description}

\item {} \begin{description}
\item[{\sphinxstylestrong{red\_prioritises\_connected\_nodes:}}] \leavevmode
\sphinxAtStartPar
Red sorts the nodes it can attack and chooses the one that has the most connections

\end{description}

\item {} \begin{description}
\item[{\sphinxstylestrong{red\_prioritises\_un\_connected\_nodes:}}] \leavevmode
\sphinxAtStartPar
Red sorts the nodes it can attack and chooses the one that has the least connections

\end{description}

\item {} \begin{description}
\item[{\sphinxstylestrong{red\_prioritises\_vulnerable\_nodes:}}] \leavevmode
\sphinxAtStartPar
Red sorts the nodes is can attack and chooses the one that is the most vulnerable

\end{description}

\item {} \begin{description}
\item[{\sphinxstylestrong{red\_prioritises\_resilient\_nodes:}}] \leavevmode
\sphinxAtStartPar
Red sorts the nodes is can attack and chooses the one that is the least vulnerable

\end{description}

\end{itemize}

\end{description}

\end{itemize}


\section{Observation Space settings}
\label{\detokenize{source/config_file:observation-space-settings}}\begin{itemize}
\item {} \begin{description}
\item[{\sphinxstylestrong{compromised\_status:}}] \leavevmode
\sphinxAtStartPar
The blue agent can see the compromised status of all the nodes

\end{description}

\item {} \begin{description}
\item[{\sphinxstylestrong{vulnerabilities:}}] \leavevmode
\sphinxAtStartPar
The blue agent can see the vulnerability scores of all the nodes

\end{description}

\item {} \begin{description}
\item[{\sphinxstylestrong{node\_connections:}}] \leavevmode
\sphinxAtStartPar
The blue agent can see what nodes are connected to what other nodes

\end{description}

\item {} \begin{description}
\item[{\sphinxstylestrong{average\_vulnerability:}}] \leavevmode
\sphinxAtStartPar
The blue agent can see the average vulnerability of all the nodes

\end{description}

\item {} \begin{description}
\item[{\sphinxstylestrong{graph\_connectivity:}}] \leavevmode
\sphinxAtStartPar
The blue agent can see a graph connectivity score

\end{description}

\item {} \begin{description}
\item[{\sphinxstylestrong{attacking\_nodes:}}] \leavevmode
\sphinxAtStartPar
The blue agent can see all of the nodes that have recently attacked a safe node

\end{description}

\item {} \begin{description}
\item[{\sphinxstylestrong{attacked\_nodes:}}] \leavevmode
\sphinxAtStartPar
The blue agent can see all the nodes that have recently been attacked

\end{description}

\item {} \begin{description}
\item[{\sphinxstylestrong{special\_nodes:}}] \leavevmode
\sphinxAtStartPar
The blue agent can see all of the special nodes (entry nodes, high value nodes)

\end{description}

\item {} \begin{description}
\item[{\sphinxstylestrong{red\_agent\_skill:}}] \leavevmode
\sphinxAtStartPar
The blue agent can see the skill level of the red agent

\end{description}

\end{itemize}


\section{Blue Agent settings}
\label{\detokenize{source/config_file:blue-agent-settings}}
\sphinxAtStartPar
The blue agent does not have to have perfect detection. In these settings you can change how much information blue
can gain from the red agents actions. There are two different pieces of information blue can get: intrusions and
attacks.

\sphinxAtStartPar
\sphinxstylestrong{Intrusions}

\sphinxAtStartPar
An intrusion is when the red agent takes over a node and compromises it. You can change the chance that blue has to
be able to detect this using the “chance\_to\_immediately\_discover\_intrusion”. If blue does not detect an intrusion
then it can use the scan action to try and discover these intrusions with “chance\_to\_discover\_intrusion\_on\_scan”.

\sphinxAtStartPar
There are also deceptive nodes that blue can place down. These nodes are used as detectors to inform blue when they
are compromised. They should have a chance to detect of 1 so that they can detect everything (at the very least
they should have a chance to detect higher than the normal chance to detect) but you can modify it if you so wish
with “chance\_to\_immediately\_discover\_intrusion\_deceptive\_node” and “chance\_to\_discover\_intrusion\_on\_scan\_deceptive\_node”

\sphinxAtStartPar
\sphinxstylestrong{Attacks}

\sphinxAtStartPar
Attacks are the actual attacks that the red agent does to compromise the nodes. For example you may be able to see
that node 14 is compromised but using the attack detection, the blue agent may be able to see that it was node 12
that attacked node 14. You can modify the chance for blue to see attacks that failed, succeeded (and blue was able
to detect that the node was compromised) and attacks that succeeded and the blue agent did not detect the intrusion.

\sphinxAtStartPar
Again there are settings to change the likelihood that a deceptive node can detect an attack. While this should
remain at 1, it is open for you to change.
\begin{itemize}
\item {} \begin{description}
\item[{\sphinxstylestrong{max\_number\_deceptive\_nodes:}}] \leavevmode
\sphinxAtStartPar
The number of deceptive nodes that blue can place down. Deceptive nodes are special nodes that have a higher chance (usually 100\%) of being able to detect when they are being attacked/compromised.

\end{description}

\item {} \begin{description}
\item[{\sphinxstylestrong{can\_discover\_failed\_attacks:}}] \leavevmode
\sphinxAtStartPar
If the blue agent can detect attacks that have failed

\end{description}

\item {} \begin{description}
\item[{\sphinxstylestrong{Intrusions}}] \leavevmode\begin{itemize}
\item {} \begin{description}
\item[{\sphinxstylestrong{chance\_to\_immediately\_discover\_intrusion:}}] \leavevmode
\sphinxAtStartPar
Chance for blue to discover when red compromises a node the instant it is taken

\end{description}

\item {} \begin{description}
\item[{\sphinxstylestrong{chance\_to\_discover\_intrusion\_on\_scan:}}] \leavevmode
\sphinxAtStartPar
Chance for blue to find a node red has compromised during the scan action

\end{description}

\item {} \begin{description}
\item[{\sphinxstylestrong{chance\_to\_immediately\_discover\_intrusion\_deceptive\_node:}}] \leavevmode
\sphinxAtStartPar
Chance for blue to discover when red compromises a deceptive node the instant it is taken

\end{description}

\item {} \begin{description}
\item[{\sphinxstylestrong{chance\_to\_discover\_intrusion\_on\_scan\_deceptive\_node:}}] \leavevmode
\sphinxAtStartPar
Chance for blue to find a deceptive node red has compromised during the scan action

\end{description}

\end{itemize}

\end{description}

\item {} \begin{description}
\item[{\sphinxstylestrong{Attacks}}] \leavevmode\begin{itemize}
\item {} \begin{description}
\item[{\sphinxstylestrong{chance\_to\_discover\_failed\_attack:}}] \leavevmode
\sphinxAtStartPar
Chance for blue to discover a red attack that did not compromise a node

\end{description}

\item {} \begin{description}
\item[{\sphinxstylestrong{can\_discover\_succeeded\_attacks\_if\_compromise\_is\_discovered:}}] \leavevmode
\sphinxAtStartPar
If an attack compromises a node and blue detected the intrusion can blue detect the attack

\end{description}

\item {} \begin{description}
\item[{\sphinxstylestrong{can\_discover\_succeeded\_attacks\_if\_compromise\_is\_not\_discovered:}}] \leavevmode
\sphinxAtStartPar
If an attack compromises a node and blue did not detect the intrusion can blue detect the attack

\end{description}

\item {} \begin{description}
\item[{\sphinxstylestrong{chance\_to\_discover\_succeeded\_attack\_compromise\_known:}}] \leavevmode
\sphinxAtStartPar
Chance for blue to discover a successful attack the blue can see has compromised the node

\end{description}

\item {} \begin{description}
\item[{\sphinxstylestrong{chance\_to\_discover\_succeeded\_attack\_compromise\_not\_known:}}] \leavevmode
\sphinxAtStartPar
Chance for blue to discover a successful attack the blue cannot see has compromised the node

\end{description}

\item {} \begin{description}
\item[{\sphinxstylestrong{chance\_to\_discover\_failed\_attack\_deceptive\_node}}] \leavevmode
\sphinxAtStartPar
Chance for blue to discover an attack that failed to compromise a deceptive node

\end{description}

\item {} \begin{description}
\item[{\sphinxstylestrong{chance\_to\_discover\_succeeded\_attack\_deceptive\_node:}}] \leavevmode
\sphinxAtStartPar
Chance for blue to discover an attack that succeeded to compromise a deceptive node

\end{description}

\end{itemize}

\end{description}

\item {} \begin{description}
\item[{\sphinxstylestrong{making\_node\_safe\_modifies\_vulnerability:}}] \leavevmode
\sphinxAtStartPar
Using the make\_node\_safe action also modifies the vulnerability of a node by a fixed amount

\end{description}

\item {} \begin{description}
\item[{\sphinxstylestrong{vulnerability\_change\_during\_fix:}}] \leavevmode
\sphinxAtStartPar
The vulnerability change the occurs during the make\_node\_safe action

\end{description}

\item {} \begin{description}
\item[{\sphinxstylestrong{making\_node\_safe\_gives\_random\_vulnerability:}}] \leavevmode
\sphinxAtStartPar
Using the make\_node\_safe action modifies the vulnerability to a new random number

\end{description}

\item {} \begin{description}
\item[{\sphinxstylestrong{Blue Actions}}] \leavevmode\begin{itemize}
\item {} \begin{description}
\item[{\sphinxstylestrong{blue\_uses\_reduce\_vulnerability:}}] \leavevmode
\sphinxAtStartPar
Blue can use the reduce vulnerability of a node

\end{description}

\item {} \begin{description}
\item[{\sphinxstylestrong{blue\_uses\_restore\_node:}}] \leavevmode
\sphinxAtStartPar
Blue can restore a node to its default (initial) state

\end{description}

\item {} \begin{description}
\item[{\sphinxstylestrong{blue\_uses\_make\_node\_safe:}}] \leavevmode
\sphinxAtStartPar
Blue patches a node and removes the red agent from it

\end{description}

\item {} \begin{description}
\item[{\sphinxstylestrong{blue\_uses\_scan:}}] \leavevmode
\sphinxAtStartPar
Blue tries to check the status of all of the nodes to detect and red intrusions

\end{description}

\item {} \begin{description}
\item[{\sphinxstylestrong{blue\_uses\_isolate\_node:}}] \leavevmode
\sphinxAtStartPar
Blue isolates a node and removes all connections to and from the node

\end{description}

\item {} \begin{description}
\item[{\sphinxstylestrong{blue\_uses\_reconnect\_node:}}] \leavevmode
\sphinxAtStartPar
Blue reconnects a node adding back any lost connections

\end{description}

\item {} \begin{description}
\item[{\sphinxstylestrong{blue\_uses\_do\_nothing:}}] \leavevmode
\sphinxAtStartPar
Blue does nothing

\end{description}

\item {} \begin{description}
\item[{\sphinxstylestrong{blue\_uses\_deceptive\_nodes:}}] \leavevmode
\sphinxAtStartPar
Blue can place down deceptive nodes on an existing edge. Deceptive nodes can more accurately detect when red tries to compromise them

\end{description}

\end{itemize}

\end{description}

\end{itemize}


\section{Game Rules}
\label{\detokenize{source/config_file:game-rules}}\begin{itemize}
\item {} \begin{description}
\item[{\sphinxstylestrong{node\_vulnerability\_lower\_bound:}}] \leavevmode
\sphinxAtStartPar
The lowest value that could be generated (or reached) for vulnerability (lower means more resilient nodes)

\end{description}

\item {} \begin{description}
\item[{\sphinxstylestrong{node\_vulnerability\_upper\_bound:}}] \leavevmode
\sphinxAtStartPar
The highest value that could be generated (or reached) for vulnerability (higher means easier to compromise nodes)

\end{description}

\item {} \begin{description}
\item[{\sphinxstylestrong{max\_steps:}}] \leavevmode
\sphinxAtStartPar
How many steps blue has to survive for before winning

\end{description}

\item {}
\sphinxAtStartPar
\sphinxstylestrong{lose\_when\_all\_nodes\_lost:}
\begin{quote}

\sphinxAtStartPar
Does the red agent win if it takes all of the nodes
\begin{itemize}
\item {} \begin{description}
\item[{\sphinxstylestrong{lose\_when\_n\_percent\_of\_nodes\_lost:}}] \leavevmode
\sphinxAtStartPar
Does the red agent win if it takes n\% of all the nodes

\end{description}

\end{itemize}
\end{quote}

\item {} \begin{description}
\item[{\sphinxstylestrong{percentage\_of\_nodes\_compromised\_equals\_loss:}}] \leavevmode
\sphinxAtStartPar
If red wins if it takes n\% of nodes what value is n

\end{description}

\item {}
\sphinxAtStartPar
\sphinxstylestrong{lose\_when\_high\_value\_target\_lost:}
\begin{quote}

\sphinxAtStartPar
Does red win if a special “high value” node is taken
\begin{itemize}
\item {} \begin{description}
\item[{\sphinxstylestrong{choose\_high\_value\_target\_placement\_at\_random:}}] \leavevmode
\sphinxAtStartPar
Choose the “high value” node at random

\end{description}

\item {} \begin{description}
\item[{\sphinxstylestrong{choose\_high\_value\_target\_furthest\_away\_from\_entry:}}] \leavevmode
\sphinxAtStartPar
Choose the “high value” to be one of the nodes furthest from all of the entry points

\end{description}

\end{itemize}
\end{quote}

\item {} \begin{description}
\item[{\sphinxstylestrong{choose\_entry\_nodes\_randomly:}}] \leavevmode
\sphinxAtStartPar
If no entry nodes are supplied choose some at random

\end{description}

\item {} \begin{description}
\item[{\sphinxstylestrong{number\_of\_entry\_nodes:}}] \leavevmode
\sphinxAtStartPar
If no entry nodes are supplied and are automatically generated, how many should be generated

\end{description}

\item {} \begin{description}
\item[{\sphinxstylestrong{prefer\_central\_nodes\_for\_entry\_nodes:}}] \leavevmode
\sphinxAtStartPar
If no entry nodes are supplied then when auto\sphinxhyphen{}generating new entry nodes apply a bias towards more central/connected nodes

\end{description}

\item {} \begin{description}
\item[{\sphinxstylestrong{prefer\_edge\_nodes\_for\_entry\_nodes:}}] \leavevmode
\sphinxAtStartPar
If no entry nodes are supplied then when auto\sphinxhyphen{}generating new entry nodes apply a bias towards more edge/outer nodes

\end{description}

\item {} \begin{description}
\item[{\sphinxstylestrong{grace\_period\_length:}}] \leavevmode
\sphinxAtStartPar
The length of a grace period at the start of the game. During this time the red agent cannot act. This gives the blue agent a chance to “prepare” (A length of 0 means that there is no grace period)

\end{description}

\end{itemize}


\section{Reset}
\label{\detokenize{source/config_file:reset}}\begin{itemize}
\item {} \begin{description}
\item[{\sphinxstylestrong{randomise\_vulnerabilities\_on\_reset:}}] \leavevmode
\sphinxAtStartPar
Pick new vulnerabilities for all the nodes on every reset

\end{description}

\item {} \begin{description}
\item[{\sphinxstylestrong{choose\_new\_high\_value\_target\_on\_reset:}}] \leavevmode
\sphinxAtStartPar
Pick a new high value node on every reset

\end{description}

\item {} \begin{description}
\item[{\sphinxstylestrong{choose\_new\_entry\_nodes\_on\_reset:}}] \leavevmode
\sphinxAtStartPar
Pick new entry nodes every reset

\end{description}

\end{itemize}


\section{Rewards}
\label{\detokenize{source/config_file:rewards}}\begin{itemize}
\item {} \begin{description}
\item[{\sphinxstylestrong{rewards\_for\_loss:}}] \leavevmode
\sphinxAtStartPar
The reward blue gets for losing

\end{description}

\item {} \begin{description}
\item[{\sphinxstylestrong{rewards\_for\_reaching\_max\_steps:}}] \leavevmode
\sphinxAtStartPar
The reward blue gets for winning

\end{description}

\item {} \begin{description}
\item[{\sphinxstylestrong{reward\_function:}}] \leavevmode
\sphinxAtStartPar
Choose the reward method. There are several built in example reward methods that you can choose from (shown below) You can also create your own reward method by copying one of the built in methods and calling it here
\begin{description}
\item[{Built in reward methods:}] \leavevmode\begin{itemize}
\item {}
\sphinxAtStartPar
standard\_rewards

\item {}
\sphinxAtStartPar
one\_per\_timestep

\item {}
\sphinxAtStartPar
safe\_nodes\_give\_rewards

\item {}
\sphinxAtStartPar
punish\_bad\_actions

\end{itemize}

\end{description}

\end{description}

\end{itemize}


\section{Miscellaneous}
\label{\detokenize{source/config_file:miscellaneous}}\begin{itemize}
\item {} \begin{description}
\item[{\sphinxstylestrong{output\_timestep\_data\_to\_json:}}] \leavevmode
\sphinxAtStartPar
Toggle to output a json file for each step that contains the connections between nodes, the states of the nodes and the attacks that blue saw in that turn

\end{description}

\end{itemize}


\chapter{Using the Experiment Runner}
\label{\detokenize{source/quick_start_experiment_runner:using-the-experiment-runner}}\label{\detokenize{source/quick_start_experiment_runner::doc}}
\sphinxAtStartPar
YAWNING TITAN comes with a command line interface tool called \sphinxtitleref{experiment\_runner.py} which allows new users to begin experimenting with Reinforcement learning approaches in the environments
included. The experiment runner is a way to run some of specific environments such as the 4 node environment

\begin{sphinxVerbatim}[commandchars=\\\{\}]
\PYG{o}{\PYGZgt{}} \PYG{n}{python3} \PYG{n}{experiment\PYGZus{}runner}\PYG{o}{.}\PYG{n}{py} \PYG{o}{\PYGZhy{}}\PYG{o}{\PYGZhy{}}\PYG{n}{help}

\PYG{n}{usage}\PYG{p}{:} \PYG{n}{experiment\PYGZus{}runner}\PYG{o}{.}\PYG{n}{py}
        \PYG{o}{\PYGZhy{}}\PYG{o}{\PYGZhy{}}\PYG{n}{agent} \PYG{p}{\PYGZob{}}\PYG{n}{random}\PYG{p}{,}\PYG{n+nb}{all}\PYG{p}{,}\PYG{n}{ppo}\PYG{p}{,}\PYG{n}{a2c}\PYG{p}{,}\PYG{n}{dqn}\PYG{p}{\PYGZcb{}}
        \PYG{o}{\PYGZhy{}}\PYG{o}{\PYGZhy{}}\PYG{n}{env} \PYG{p}{\PYGZob{}}\PYG{n}{five}\PYG{o}{\PYGZhy{}}\PYG{n}{node}\PYG{o}{\PYGZhy{}}\PYG{n}{def}\PYG{o}{\PYGZhy{}}\PYG{n}{v0}\PYG{p}{,}\PYG{n}{four}\PYG{o}{\PYGZhy{}}\PYG{n}{node}\PYG{o}{\PYGZhy{}}\PYG{n}{def}\PYG{o}{\PYGZhy{}}\PYG{n}{v0}\PYG{p}{,}\PYG{n}{network}\PYG{o}{\PYGZhy{}}\PYG{n}{graph}\PYG{o}{\PYGZhy{}}\PYG{n}{explore}\PYG{o}{\PYGZhy{}}\PYG{n}{v0}\PYG{p}{,}\PYG{l+m+mi}{18}\PYG{o}{\PYGZhy{}}\PYG{n}{node}\PYG{o}{\PYGZhy{}}\PYG{n}{env}\PYG{o}{\PYGZhy{}}\PYG{n}{v0}\PYG{p}{\PYGZcb{}}
        \PYG{o}{\PYGZhy{}}\PYG{o}{\PYGZhy{}}\PYG{n}{training}\PYG{o}{\PYGZhy{}}\PYG{n}{period} \PYG{n}{TRAINING\PYGZus{}PERIOD}
        \PYG{p}{[}\PYG{o}{\PYGZhy{}}\PYG{o}{\PYGZhy{}}\PYG{n}{help}\PYG{p}{]}
        \PYG{p}{[}\PYG{o}{\PYGZhy{}}\PYG{o}{\PYGZhy{}}\PYG{n}{action}\PYG{o}{\PYGZhy{}}\PYG{n}{loop} \PYG{p}{\PYGZob{}}\PYG{n}{gif}\PYG{p}{,}\PYG{n}{standard}\PYG{p}{\PYGZcb{}}\PYG{p}{]}
        \PYG{p}{[}\PYG{o}{\PYGZhy{}}\PYG{o}{\PYGZhy{}}\PYG{n}{algo}\PYG{o}{\PYGZhy{}}\PYG{n}{backend} \PYG{p}{\PYGZob{}}\PYG{n}{sb3}\PYG{p}{,}\PYG{n}{rllib}\PYG{p}{\PYGZcb{}}\PYG{p}{]}
        \PYG{p}{[}\PYG{o}{\PYGZhy{}}\PYG{o}{\PYGZhy{}}\PYG{n}{dl}\PYG{o}{\PYGZhy{}}\PYG{n}{backend} \PYG{n}{DL\PYGZus{}BACKEND}\PYG{p}{]}
        \PYG{p}{[}\PYG{o}{\PYGZhy{}}\PYG{o}{\PYGZhy{}}\PYG{n+nb}{eval}\PYG{o}{\PYGZhy{}}\PYG{n}{ep}\PYG{o}{\PYGZhy{}}\PYG{n}{count} \PYG{n}{EVAL\PYGZus{}EP\PYGZus{}COUNT}\PYG{p}{]}
        \PYG{p}{[}\PYG{o}{\PYGZhy{}}\PYG{o}{\PYGZhy{}}\PYG{n}{post}\PYG{o}{\PYGZhy{}}\PYG{n}{train}\PYG{p}{]}
        \PYG{p}{[}\PYG{o}{\PYGZhy{}}\PYG{o}{\PYGZhy{}}\PYG{n}{debug}\PYG{p}{]}
        \PYG{p}{[}\PYG{o}{\PYGZhy{}}\PYG{o}{\PYGZhy{}}\PYG{n}{debug}\PYG{o}{\PYGZhy{}}\PYG{n}{to}\PYG{o}{\PYGZhy{}}\PYG{n}{file}\PYG{p}{]}
        \PYG{p}{[}\PYG{o}{\PYGZhy{}}\PYG{o}{\PYGZhy{}}\PYG{n}{save}\PYG{o}{\PYGZhy{}}\PYG{n}{agent}\PYG{p}{]}
        \PYG{p}{[}\PYG{o}{\PYGZhy{}}\PYG{o}{\PYGZhy{}}\PYG{n}{output}\PYG{o}{\PYGZhy{}}\PYG{n}{raw}\PYG{o}{\PYGZhy{}}\PYG{n}{metrics}\PYG{p}{]}
\end{sphinxVerbatim}

\sphinxAtStartPar
Arguments:
\begin{itemize}
\item {} \begin{description}
\item[{\sphinxcode{\sphinxupquote{\sphinxhyphen{}h, \sphinxhyphen{}\sphinxhyphen{}help}}}] \leavevmode
\sphinxAtStartPar
Show this help message and exit

\end{description}

\item {} \begin{description}
\item[{\sphinxcode{\sphinxupquote{\sphinxhyphen{}\sphinxhyphen{}agent \{random,all,ppo,a2c,dqn\}, \sphinxhyphen{}a \{random,all,ppo,a2c,dqn\}}}}] \leavevmode
\sphinxAtStartPar
Which algorithm to use to train an agent

\end{description}

\item {} \begin{description}
\item[{\sphinxcode{\sphinxupquote{\sphinxhyphen{}\sphinxhyphen{}env \{five\sphinxhyphen{}node\sphinxhyphen{}def\sphinxhyphen{}v0,four\sphinxhyphen{}node\sphinxhyphen{}def\sphinxhyphen{}v0,network\sphinxhyphen{}graph\sphinxhyphen{}explore\sphinxhyphen{}v0,18\sphinxhyphen{}node\sphinxhyphen{}env\sphinxhyphen{}v0\}, \sphinxhyphen{}e \{five\sphinxhyphen{}node\sphinxhyphen{}def\sphinxhyphen{}v0,four\sphinxhyphen{}node\sphinxhyphen{}def\sphinxhyphen{}v0,network\sphinxhyphen{}graph\sphinxhyphen{}explore\sphinxhyphen{}v0,18\sphinxhyphen{}node\sphinxhyphen{}env\sphinxhyphen{}v0\}}}}] \leavevmode
\sphinxAtStartPar
Which environment to use

\end{description}

\item {} \begin{description}
\item[{\sphinxcode{\sphinxupquote{\sphinxhyphen{}\sphinxhyphen{}action\sphinxhyphen{}loop \{gif,standard\}, \sphinxhyphen{}l \{gif,standard\}}}}] \leavevmode
\sphinxAtStartPar
Which non\sphinxhyphen{}training loop to use. Render/Gif output or no output

\end{description}

\item {} \begin{description}
\item[{\sphinxcode{\sphinxupquote{\sphinxhyphen{}\sphinxhyphen{}training\sphinxhyphen{}period TRAINING\_PERIOD, \sphinxhyphen{}tt TRAINING\_PERIOD}}}] \leavevmode
\sphinxAtStartPar
Length of agent training period

\end{description}

\item {} \begin{description}
\item[{\sphinxcode{\sphinxupquote{\sphinxhyphen{}\sphinxhyphen{}algo\sphinxhyphen{}backend \{sb3,rllib\}, \sphinxhyphen{}ab \{sb3,rllib\}}}}] \leavevmode
\sphinxAtStartPar
Which Deep Reinforcement Learning library to use

\end{description}

\item {} \begin{description}
\item[{\sphinxcode{\sphinxupquote{\sphinxhyphen{}\sphinxhyphen{}dl\sphinxhyphen{}backend DL\_BACKEND, \sphinxhyphen{}db DL\_BACKEND}}}] \leavevmode
\sphinxAtStartPar
Which deep learning backend to use, only important for Ray based experiments

\end{description}

\item {} \begin{description}
\item[{\sphinxcode{\sphinxupquote{\sphinxhyphen{}\sphinxhyphen{}eval\sphinxhyphen{}ep\sphinxhyphen{}count EVAL\_EP\_COUNT, \sphinxhyphen{}ec EVAL\_EP\_COUNT}}}] \leavevmode
\sphinxAtStartPar
Number of episodes to run post train

\end{description}

\item {} \begin{description}
\item[{\sphinxcode{\sphinxupquote{\sphinxhyphen{}\sphinxhyphen{}post\sphinxhyphen{}train}}}] \leavevmode
\sphinxAtStartPar
Toggle to run the agent once trained and render if available

\end{description}

\item {} \begin{description}
\item[{\sphinxcode{\sphinxupquote{\sphinxhyphen{}\sphinxhyphen{}debug}}}] \leavevmode
\sphinxAtStartPar
Toggle to turn on debugging to the terminal

\end{description}

\item {} \begin{description}
\item[{\sphinxcode{\sphinxupquote{\sphinxhyphen{}\sphinxhyphen{}debug\sphinxhyphen{}to\sphinxhyphen{}file}}}] \leavevmode
\sphinxAtStartPar
Toggle to save debugging info to file

\end{description}

\item {} \begin{description}
\item[{\sphinxcode{\sphinxupquote{\sphinxhyphen{}\sphinxhyphen{}save\sphinxhyphen{}agent}}}] \leavevmode
\sphinxAtStartPar
Toggle to save the trained agent

\end{description}

\item {} \begin{description}
\item[{\sphinxcode{\sphinxupquote{\sphinxhyphen{}\sphinxhyphen{}output\sphinxhyphen{}raw\sphinxhyphen{}metrics}}}] \leavevmode
\sphinxAtStartPar
Toggle to output raw evaluation metrics alongside summary statistics

\end{description}

\end{itemize}

\sphinxAtStartPar
An example command might look like this:

\begin{sphinxVerbatim}[commandchars=\\\{\}]
\PYG{n}{python3} \PYG{n}{experiment\PYGZus{}runner}\PYG{o}{.}\PYG{n}{py} \PYG{o}{\PYGZhy{}}\PYG{o}{\PYGZhy{}}\PYG{n}{agent} \PYG{n}{ppo} \PYG{o}{\PYGZhy{}}\PYG{o}{\PYGZhy{}}\PYG{n}{env} \PYG{l+m+mi}{18}\PYG{o}{\PYGZhy{}}\PYG{n}{node}\PYG{o}{\PYGZhy{}}\PYG{n}{env}\PYG{o}{\PYGZhy{}}\PYG{n}{v0} \PYG{o}{\PYGZhy{}}\PYG{o}{\PYGZhy{}}\PYG{n}{training}\PYG{o}{\PYGZhy{}}\PYG{n}{period} \PYG{l+m+mi}{10000}
\end{sphinxVerbatim}

\sphinxAtStartPar
This will begin the training of a Proximal Policy Optimisation agent using the Stable Baselines 3 RL algorithm library within the environment outlined within the papers published by the NSA which can be
found at \sphinxurl{https://www.nsa.gov.Portals/70/documents/resources/everyone/digital-media-center/publications/the-next-wave/TNW-22-1.pdf\#page=9} for a total of 10K training timesteps. Once the training
period has completed, the agent will enter an evaluation phase, run through a specified number of evaluation episodes (defaults to 25) and then output some summary statistics.


\chapter{Enhancing Yawning Titan}
\label{\detokenize{source/enhancing_yawning_titan:enhancing-yawning-titan}}\label{\detokenize{source/enhancing_yawning_titan::doc}}
\sphinxAtStartPar
Yawning Titan has been built to allow easy addition and modification
of large sections. It is very easy to add:
\begin{itemize}
\item {}
\sphinxAtStartPar
Reward functions

\item {}
\sphinxAtStartPar
Red actions

\item {}
\sphinxAtStartPar
Blue actions

\item {}
\sphinxAtStartPar
Red agents

\end{itemize}


\section{Reward Functions}
\label{\detokenize{source/enhancing_yawning_titan:reward-functions}}
\sphinxAtStartPar
To create a new reward function navigate to:
\begin{quote}

\sphinxAtStartPar
yawning\_titan/envs/generic/core/reward\_functions.py
\end{quote}

\sphinxAtStartPar
Here you will find all of the current reward functions. To add a new
reward function yo just have to add a new function with the following
form:

\begin{sphinxVerbatim}[commandchars=\\\{\}]
\PYG{k}{def} \PYG{n+nf}{new\PYGZus{}rewards}\PYG{p}{(}\PYG{n}{args}\PYG{p}{:} \PYG{n+nb}{dict}\PYG{p}{)} \PYG{o}{\PYGZhy{}}\PYG{o}{\PYGZgt{}} \PYG{n+nb}{float}\PYG{p}{:}
\end{sphinxVerbatim}

\sphinxAtStartPar
Args will contain certain stats from the current network that you can
use to determine your rewards.
\begin{itemize}
\item {}
\sphinxAtStartPar
network\_interface: Interface with the network

\item {}
\sphinxAtStartPar
blue\_action: The action that the blue agent has taken this turn

\item {}
\sphinxAtStartPar
blue\_node: The node that the blue agent has targeted for their action

\item {}
\sphinxAtStartPar
start\_state: The state of the nodes before the blue agent has taken their action

\item {}
\sphinxAtStartPar
end\_state: The state of the nodes after the blue agent has taken their action

\item {}
\sphinxAtStartPar
start\_vulnerabilities: The vulnerabilities before blue agents turn

\item {}
\sphinxAtStartPar
end\_vulnerabilities: The vulnerabilities after the blue agents turn

\item {}
\sphinxAtStartPar
start\_isolation: The isolation status of all the nodes at the start of a turn

\item {}
\sphinxAtStartPar
end\_isolation: The isolation status of all the nodes at the end of a turn

\item {}
\sphinxAtStartPar
start\_blue: The env as the blue agent can see it before the blue agents turn

\item {}
\sphinxAtStartPar
end\_blue: The env as the blue agent can see it after the blue agents turn

\end{itemize}


\section{Red Actions}
\label{\detokenize{source/enhancing_yawning_titan:red-actions}}
\sphinxAtStartPar
To create a new red action navigate to:
\begin{quote}

\sphinxAtStartPar
yawning\_titan/envs/generic/core/red\_action\_set.py
\end{quote}

\sphinxAtStartPar
There you can see all of the current actions and create your own.
To create your own action it must follow the following form:

\begin{sphinxVerbatim}[commandchars=\\\{\}]
\PYG{k}{def} \PYG{n+nf}{new\PYGZus{}action}\PYG{p}{(}\PYG{n+nb+bp}{self}\PYG{p}{)} \PYG{o}{\PYGZhy{}}\PYG{o}{\PYGZgt{}} \PYG{n}{Tuple}\PYG{p}{[}\PYG{n+nb}{str}\PYG{p}{,} \PYG{n}{List}\PYG{p}{[}\PYG{n+nb}{bool}\PYG{p}{]}\PYG{p}{,} \PYG{n}{List}\PYG{p}{[}\PYG{n}{Union}\PYG{p}{[}\PYG{n+nb}{str}\PYG{p}{,} \PYG{k+kc}{None}\PYG{p}{]}\PYG{p}{]}\PYG{p}{,} \PYG{n}{List}\PYG{p}{[}\PYG{n}{Union}\PYG{p}{[}\PYG{n+nb}{str}\PYG{p}{,} \PYG{k+kc}{None}\PYG{p}{]}\PYG{p}{]}\PYG{p}{]}
\end{sphinxVerbatim}
\begin{description}
\item[{This means that it returns:}] \leavevmode\begin{itemize}
\item {}
\sphinxAtStartPar
The name of the action (string)

\item {}
\sphinxAtStartPar
A list of successes (boolean)

\item {}
\sphinxAtStartPar
A list of target nodes (List of strings)

\item {}
\sphinxAtStartPar
A list of attacking nodes (List of strings)

\end{itemize}

\end{description}

\sphinxAtStartPar
You then have to tell the red agent that it can use this new action.
To do this you have to add a new setting to the config file for this new action.:

\begin{sphinxVerbatim}[commandchars=\\\{\}]
\PYG{n}{RED}\PYG{p}{:}
    \PYG{n}{red\PYGZus{}uses\PYGZus{}new\PYGZus{}action}\PYG{p}{:} \PYG{k+kc}{True}
    \PYG{n}{new\PYGZus{}action\PYGZus{}likelihood}\PYG{p}{:} \PYG{l+m+mi}{1}
\end{sphinxVerbatim}

\sphinxAtStartPar
And then in the red interface:

\begin{sphinxVerbatim}[commandchars=\\\{\}]
\PYG{k}{if} \PYG{n}{settings\PYGZus{}file}\PYG{p}{[}\PYG{l+s+s2}{\PYGZdq{}}\PYG{l+s+s2}{RED}\PYG{l+s+s2}{\PYGZdq{}}\PYG{p}{]}\PYG{p}{[}\PYG{l+s+s2}{\PYGZdq{}}\PYG{l+s+s2}{red\PYGZus{}uses\PYGZus{}new\PYGZus{}action}\PYG{l+s+s2}{\PYGZdq{}}\PYG{p}{]}\PYG{p}{:}
    \PYG{n+nb+bp}{self}\PYG{o}{.}\PYG{n}{action\PYGZus{}dict}\PYG{p}{[}\PYG{n}{action\PYGZus{}number}\PYG{p}{]} \PYG{o}{=} \PYG{n+nb+bp}{self}\PYG{o}{.}\PYG{n}{new\PYGZus{}action}
    \PYG{n}{action\PYGZus{}set}\PYG{o}{.}\PYG{n}{append}\PYG{p}{(}\PYG{n}{action\PYGZus{}number}\PYG{p}{)}
    \PYG{n}{probabilities\PYGZus{}set}\PYG{o}{.}\PYG{n}{append}\PYG{p}{(}\PYG{n}{settings\PYGZus{}file}\PYG{p}{[}\PYG{l+s+s2}{\PYGZdq{}}\PYG{l+s+s2}{RED}\PYG{l+s+s2}{\PYGZdq{}}\PYG{p}{]}\PYG{p}{[}\PYG{l+s+s2}{\PYGZdq{}}\PYG{l+s+s2}{new\PYGZus{}action\PYGZus{}likelihood}\PYG{l+s+s2}{\PYGZdq{}}\PYG{p}{]}\PYG{p}{)}
    \PYG{n}{action\PYGZus{}number} \PYG{o}{+}\PYG{o}{=} \PYG{l+m+mi}{1}
\end{sphinxVerbatim}

\sphinxAtStartPar
And finally if the action is not an attacking action then it needs to be added to
the list of non attacking actions in the red\_interface:

\begin{sphinxVerbatim}[commandchars=\\\{\}]
\PYG{n+nb+bp}{self}\PYG{o}{.}\PYG{n}{non\PYGZus{}attacking\PYGZus{}actions} \PYG{o}{=} \PYG{p}{[}\PYG{l+s+s2}{\PYGZdq{}}\PYG{l+s+s2}{do\PYGZus{}nothing}\PYG{l+s+s2}{\PYGZdq{}}\PYG{p}{,} \PYG{l+s+s2}{\PYGZdq{}}\PYG{l+s+s2}{random\PYGZus{}move}\PYG{l+s+s2}{\PYGZdq{}}\PYG{p}{,} \PYG{l+s+s2}{\PYGZdq{}}\PYG{l+s+s2}{new\PYGZus{}action}\PYG{l+s+s2}{\PYGZdq{}}\PYG{p}{]}
\end{sphinxVerbatim}


\section{Blue Actions}
\label{\detokenize{source/enhancing_yawning_titan:blue-actions}}
\sphinxAtStartPar
Adding blue actions are slightly more complicated as there are 3 types
of blue actions:
\begin{itemize}
\item {} \begin{description}
\item[{General actions}] \leavevmode
\sphinxAtStartPar
Act on a single node

\end{description}

\item {} \begin{description}
\item[{Global actions}] \leavevmode
\sphinxAtStartPar
Act on every node

\end{description}

\item {} \begin{description}
\item[{Deceptive actions}] \leavevmode
\sphinxAtStartPar
Act on every link between nodes

\end{description}

\end{itemize}

\sphinxAtStartPar
First navigate to the action set location:
\begin{quote}

\sphinxAtStartPar
yawning\_titan/envs/generic/core/blue\_action\_set.py
\end{quote}


\subsection{Adding a general action}
\label{\detokenize{source/enhancing_yawning_titan:adding-a-general-action}}
\sphinxAtStartPar
Create an action with the form:

\begin{sphinxVerbatim}[commandchars=\\\{\}]
\PYG{k}{def} \PYG{n+nf}{new\PYGZus{}action}\PYG{p}{(}\PYG{n+nb+bp}{self}\PYG{p}{,} \PYG{n}{node}\PYG{p}{:} \PYG{n+nb}{str}\PYG{p}{)} \PYG{o}{\PYGZhy{}}\PYG{o}{\PYGZgt{}} \PYG{n}{Tuple}\PYG{p}{[}\PYG{n+nb}{str}\PYG{p}{,} \PYG{n+nb}{str}\PYG{p}{]}
\end{sphinxVerbatim}

\sphinxAtStartPar
Add the action to the config file:

\begin{sphinxVerbatim}[commandchars=\\\{\}]
\PYG{n}{BLUE}\PYG{p}{:}
    \PYG{n}{blue\PYGZus{}uses\PYGZus{}new\PYGZus{}action}\PYG{p}{:} \PYG{k+kc}{True}
\end{sphinxVerbatim}

\sphinxAtStartPar
Then add the action to the interface:

\begin{sphinxVerbatim}[commandchars=\\\{\}]
\PYG{k}{if} \PYG{n+nb+bp}{self}\PYG{o}{.}\PYG{n}{settings}\PYG{p}{[}\PYG{l+s+s2}{\PYGZdq{}}\PYG{l+s+s2}{BLUE}\PYG{l+s+s2}{\PYGZdq{}}\PYG{p}{]}\PYG{p}{[}\PYG{l+s+s2}{\PYGZdq{}}\PYG{l+s+s2}{blue\PYGZus{}uses\PYGZus{}new\PYGZus{}action}\PYG{l+s+s2}{\PYGZdq{}}\PYG{p}{]}\PYG{p}{:}
    \PYG{n+nb+bp}{self}\PYG{o}{.}\PYG{n}{action\PYGZus{}dict}\PYG{p}{[}\PYG{n}{action\PYGZus{}number}\PYG{p}{]} \PYG{o}{=} \PYG{n+nb+bp}{self}\PYG{o}{.}\PYG{n}{new\PYGZus{}action}
    \PYG{n}{action\PYGZus{}number} \PYG{o}{+}\PYG{o}{=} \PYG{l+m+mi}{1}
\end{sphinxVerbatim}


\subsection{Adding a Global action}
\label{\detokenize{source/enhancing_yawning_titan:adding-a-global-action}}
\sphinxAtStartPar
Create an action with the form:

\begin{sphinxVerbatim}[commandchars=\\\{\}]
\PYG{k}{def} \PYG{n+nf}{new\PYGZus{}action}\PYG{p}{(}\PYG{n+nb+bp}{self}\PYG{p}{)} \PYG{o}{\PYGZhy{}}\PYG{o}{\PYGZgt{}} \PYG{n}{Tuple}\PYG{p}{[}\PYG{n+nb}{str}\PYG{p}{,} \PYG{n+nb}{str}\PYG{p}{]}
\end{sphinxVerbatim}

\sphinxAtStartPar
Add the action to the config file:

\begin{sphinxVerbatim}[commandchars=\\\{\}]
\PYG{n}{BLUE}\PYG{p}{:}
    \PYG{n}{blue\PYGZus{}uses\PYGZus{}new\PYGZus{}action}\PYG{p}{:} \PYG{k+kc}{True}
\end{sphinxVerbatim}

\sphinxAtStartPar
Add the action to the interface:

\begin{sphinxVerbatim}[commandchars=\\\{\}]
\PYG{k}{if} \PYG{n+nb+bp}{self}\PYG{o}{.}\PYG{n}{settings}\PYG{p}{[}\PYG{l+s+s2}{\PYGZdq{}}\PYG{l+s+s2}{BLUE}\PYG{l+s+s2}{\PYGZdq{}}\PYG{p}{]}\PYG{p}{[}\PYG{l+s+s2}{\PYGZdq{}}\PYG{l+s+s2}{blue\PYGZus{}uses\PYGZus{}new\PYGZus{}action}\PYG{l+s+s2}{\PYGZdq{}}\PYG{p}{]}\PYG{p}{:}
    \PYG{n+nb+bp}{self}\PYG{o}{.}\PYG{n}{global\PYGZus{}action\PYGZus{}dict}\PYG{p}{[}\PYG{n}{global\PYGZus{}action\PYGZus{}number}\PYG{p}{]} \PYG{o}{=} \PYG{n+nb+bp}{self}\PYG{o}{.}\PYG{n}{new\PYGZus{}action}
    \PYG{n}{global\PYGZus{}action\PYGZus{}number} \PYG{o}{+}\PYG{o}{=} \PYG{l+m+mi}{1}
\end{sphinxVerbatim}


\subsection{Adding a new Red Agent}
\label{\detokenize{source/enhancing_yawning_titan:adding-a-new-red-agent}}
\sphinxAtStartPar
First step is to look at the red\_interface and advanced\_red\_interface:
\begin{quote}

\sphinxAtStartPar
yawning\_titan/envs/generic/core/red\_interface.py

\sphinxAtStartPar
yawning\_titan/envs/generic/core/advanced\_red\_interface.py
\end{quote}

\sphinxAtStartPar
Then to create a new interface you want a class with the following:
\begin{itemize}
\item {}
\sphinxAtStartPar
Inherits from RedInterface:

\begin{sphinxVerbatim}[commandchars=\\\{\}]
\PYG{k}{class} \PYG{n+nc}{NewInterface}\PYG{p}{(}\PYG{n}{RedInterface}\PYG{p}{)}\PYG{p}{:}
\end{sphinxVerbatim}

\item {}
\sphinxAtStartPar
An init method:

\begin{sphinxVerbatim}[commandchars=\\\{\}]
\PYG{k}{def} \PYG{n+nf+fm}{\PYGZus{}\PYGZus{}init\PYGZus{}\PYGZus{}}\PYG{p}{(}\PYG{n+nb+bp}{self}\PYG{p}{,} \PYG{n}{network\PYGZus{}interface}\PYG{p}{)}\PYG{p}{:}
    \PYG{n+nb}{super}\PYG{p}{(}\PYG{p}{)}\PYG{o}{.}\PYG{n+nf+fm}{\PYGZus{}\PYGZus{}init\PYGZus{}\PYGZus{}}\PYG{p}{(}\PYG{n}{network\PYGZus{}interface}\PYG{p}{)}
\end{sphinxVerbatim}

\item {}
\sphinxAtStartPar
A perform action method:

\begin{sphinxVerbatim}[commandchars=\\\{\}]
\PYG{k}{def} \PYG{n+nf}{perform\PYGZus{}action}\PYG{p}{(}\PYG{n+nb+bp}{self}\PYG{p}{)} \PYG{o}{\PYGZhy{}}\PYG{o}{\PYGZgt{}} \PYG{n}{Tuple}\PYG{p}{[}\PYG{n+nb}{str}\PYG{p}{,} \PYG{n}{Union}\PYG{p}{[}\PYG{n+nb}{bool}\PYG{p}{,} \PYG{n}{List}\PYG{p}{[}\PYG{n+nb}{bool}\PYG{p}{]}\PYG{p}{]}\PYG{p}{,} \PYG{n}{Union}\PYG{p}{[}\PYG{n}{List}\PYG{p}{[}\PYG{n+nb}{str}\PYG{p}{]}\PYG{p}{,} \PYG{n+nb}{str}\PYG{p}{]}\PYG{p}{,} \PYG{n}{Union}\PYG{p}{[}\PYG{n}{List}\PYG{p}{[}\PYG{n+nb}{str}\PYG{p}{]}\PYG{p}{,} \PYG{n+nb}{str}\PYG{p}{]}\PYG{p}{,} \PYG{n}{Tuple}\PYG{p}{[}\PYG{n}{List}\PYG{p}{[}\PYG{n+nb}{str}\PYG{p}{]}\PYG{p}{,} \PYG{n}{List}\PYG{p}{[}\PYG{n+nb}{bool}\PYG{p}{]}\PYG{p}{]}\PYG{p}{]}\PYG{p}{:}
\end{sphinxVerbatim}

\end{itemize}
\begin{description}
\item[{Where the perform action returns the:}] \leavevmode\begin{itemize}
\item {} \begin{description}
\item[{name}] \leavevmode
\sphinxAtStartPar
The name of the action performed

\end{description}

\item {} \begin{description}
\item[{success}] \leavevmode
\sphinxAtStartPar
A list of successes from the action

\end{description}

\item {} \begin{description}
\item[{target}] \leavevmode
\sphinxAtStartPar
A list of targets for the action

\end{description}

\item {} \begin{description}
\item[{attacking\_nodes}] \leavevmode
\sphinxAtStartPar
A list of attacking nodes for the action

\end{description}

\item {} \begin{description}
\item[{(n\_target, n\_success)}] \leavevmode
\sphinxAtStartPar
Target and success rates for any natural spreading that occurred

\end{description}

\end{itemize}

\end{description}

\sphinxAtStartPar
It is also important that after calling calling any abilities that attack nodes you also need to use the following to update the list of known stored attacks:

\begin{sphinxVerbatim}[commandchars=\\\{\}]
\PYG{n+nb+bp}{self}\PYG{o}{.}\PYG{n}{network\PYGZus{}interface}\PYG{o}{.}\PYG{n}{update\PYGZus{}stored\PYGZus{}attacks}\PYG{p}{(}\PYG{n}{all\PYGZus{}attacking\PYGZus{}nodes}\PYG{p}{,} \PYG{n}{all\PYGZus{}target\PYGZus{}nodes}\PYG{p}{,} \PYG{n}{all\PYGZus{}success}\PYG{p}{)}
\end{sphinxVerbatim}


\chapter{yawning\_titan}
\label{\detokenize{source/modules:yawning-titan}}\label{\detokenize{source/modules::doc}}

\section{yawning\_titan package}
\label{\detokenize{source/yawning_titan:yawning-titan-package}}\label{\detokenize{source/yawning_titan::doc}}

\subsection{Subpackages}
\label{\detokenize{source/yawning_titan:subpackages}}

\subsubsection{yawning\_titan.agents package}
\label{\detokenize{source/yawning_titan.agents:yawning-titan-agents-package}}\label{\detokenize{source/yawning_titan.agents::doc}}

\paragraph{Subpackages}
\label{\detokenize{source/yawning_titan.agents:subpackages}}

\subparagraph{yawning\_titan.agents.static\_agents package}
\label{\detokenize{source/yawning_titan.agents.static_agents:yawning-titan-agents-static-agents-package}}\label{\detokenize{source/yawning_titan.agents.static_agents::doc}}

\subparagraph{Submodules}
\label{\detokenize{source/yawning_titan.agents.static_agents:submodules}}

\subparagraph{yawning\_titan.agents.static\_agents.fixed\_red module}
\label{\detokenize{source/yawning_titan.agents.static_agents:module-yawning_titan.agents.static_agents.fixed_red}}\label{\detokenize{source/yawning_titan.agents.static_agents:yawning-titan-agents-static-agents-fixed-red-module}}\index{module@\spxentry{module}!yawning\_titan.agents.static\_agents.fixed\_red@\spxentry{yawning\_titan.agents.static\_agents.fixed\_red}}\index{yawning\_titan.agents.static\_agents.fixed\_red@\spxentry{yawning\_titan.agents.static\_agents.fixed\_red}!module@\spxentry{module}}\index{FixedRedAgent (class in yawning\_titan.agents.static\_agents.fixed\_red)@\spxentry{FixedRedAgent}\spxextra{class in yawning\_titan.agents.static\_agents.fixed\_red}}

\begin{fulllineitems}
\phantomsection\label{\detokenize{source/yawning_titan.agents.static_agents:yawning_titan.agents.static_agents.fixed_red.FixedRedAgent}}\pysiglinewithargsret{\sphinxbfcode{\sphinxupquote{class\DUrole{w}{  }}}\sphinxcode{\sphinxupquote{yawning\_titan.agents.static\_agents.fixed\_red.}}\sphinxbfcode{\sphinxupquote{FixedRedAgent}}}{\emph{\DUrole{n}{skill}\DUrole{o}{=}\DUrole{default_value}{None}}, \emph{\DUrole{n}{exploit\_capability\_dev}\DUrole{o}{=}\DUrole{default_value}{10}}, \emph{\DUrole{n}{initial\_no\_of\_zero\_days}\DUrole{o}{=}\DUrole{default_value}{1}}}{}
\sphinxAtStartPar
Bases: \sphinxcode{\sphinxupquote{object}}

\sphinxAtStartPar
Sets initial state for the red team agent

\sphinxAtStartPar
The red team agent has two linked concepts to make it more dynamic. The first is the concept of
zero days. These guarantee compromise and work 100\% of the time. It also has the concept of
zero day capability development that is built up over time and then generated more zero\sphinxhyphen{}days.
\index{do\_red\_action() (yawning\_titan.agents.static\_agents.fixed\_red.FixedRedAgent method)@\spxentry{do\_red\_action()}\spxextra{yawning\_titan.agents.static\_agents.fixed\_red.FixedRedAgent method}}

\begin{fulllineitems}
\phantomsection\label{\detokenize{source/yawning_titan.agents.static_agents:yawning_titan.agents.static_agents.fixed_red.FixedRedAgent.do_red_action}}\pysiglinewithargsret{\sphinxbfcode{\sphinxupquote{do\_red\_action}}}{\emph{\DUrole{n}{red\_action}}, \emph{\DUrole{n}{skill\_level}}, \emph{\DUrole{n}{attack\_sucess\_threshold}}, \emph{\DUrole{n}{machine\_states}}, \emph{\DUrole{n}{target}}, \emph{\DUrole{n}{able\_to\_move}\DUrole{o}{=}\DUrole{default_value}{False}}}{}
\sphinxAtStartPar
Executes a red action and changes the environment state.
\begin{quote}\begin{description}
\item[{Parameters}] \leavevmode\begin{itemize}
\item {}
\sphinxAtStartPar
\sphinxstyleliteralstrong{\sphinxupquote{red\_action}} \textendash{} The numeric value corresponding to the chosen action

\item {}
\sphinxAtStartPar
\sphinxstyleliteralstrong{\sphinxupquote{skill\_level}} \textendash{} The red team skill level

\item {}
\sphinxAtStartPar
\sphinxstyleliteralstrong{\sphinxupquote{attack\_sucess\_threshold}} \textendash{} The attack power threshold for an attack to be succesful

\item {}
\sphinxAtStartPar
\sphinxstyleliteralstrong{\sphinxupquote{machine\_states}} \textendash{} The current machine states

\item {}
\sphinxAtStartPar
\sphinxstyleliteralstrong{\sphinxupquote{target}} \textendash{} The machine being targetted by the attack

\end{itemize}

\item[{Returns}] \leavevmode
\sphinxAtStartPar
None

\end{description}\end{quote}

\end{fulllineitems}

\index{red\_current\_node (yawning\_titan.agents.static\_agents.fixed\_red.FixedRedAgent attribute)@\spxentry{red\_current\_node}\spxextra{yawning\_titan.agents.static\_agents.fixed\_red.FixedRedAgent attribute}}

\begin{fulllineitems}
\phantomsection\label{\detokenize{source/yawning_titan.agents.static_agents:yawning_titan.agents.static_agents.fixed_red.FixedRedAgent.red_current_node}}\pysigline{\sphinxbfcode{\sphinxupquote{red\_current\_node}}\sphinxbfcode{\sphinxupquote{\DUrole{w}{  }\DUrole{p}{=}\DUrole{w}{  }None}}}
\end{fulllineitems}

\index{red\_previous\_node (yawning\_titan.agents.static\_agents.fixed\_red.FixedRedAgent attribute)@\spxentry{red\_previous\_node}\spxextra{yawning\_titan.agents.static\_agents.fixed\_red.FixedRedAgent attribute}}

\begin{fulllineitems}
\phantomsection\label{\detokenize{source/yawning_titan.agents.static_agents:yawning_titan.agents.static_agents.fixed_red.FixedRedAgent.red_previous_node}}\pysigline{\sphinxbfcode{\sphinxupquote{red\_previous\_node}}\sphinxbfcode{\sphinxupquote{\DUrole{w}{  }\DUrole{p}{=}\DUrole{w}{  }None}}}
\end{fulllineitems}

\index{select\_action() (yawning\_titan.agents.static\_agents.fixed\_red.FixedRedAgent method)@\spxentry{select\_action()}\spxextra{yawning\_titan.agents.static\_agents.fixed\_red.FixedRedAgent method}}

\begin{fulllineitems}
\phantomsection\label{\detokenize{source/yawning_titan.agents.static_agents:yawning_titan.agents.static_agents.fixed_red.FixedRedAgent.select_action}}\pysiglinewithargsret{\sphinxbfcode{\sphinxupquote{select\_action}}}{\emph{\DUrole{n}{uncompromised\_nodes}}, \emph{\DUrole{n}{compromised\_nodes}}}{}
\sphinxAtStartPar
Selects the Red Teams action for a time step
\begin{quote}\begin{description}
\item[{Parameters}] \leavevmode\begin{itemize}
\item {}
\sphinxAtStartPar
\sphinxstyleliteralstrong{\sphinxupquote{uncompromised\_nodes}} \textendash{} The list of uncompromised nodes linked to the Red team current position

\item {}
\sphinxAtStartPar
\sphinxstyleliteralstrong{\sphinxupquote{compromised\_nodes}} \textendash{} The list of compromised nodes linked to the Red team current position

\end{itemize}

\item[{Returns}] \leavevmode
\sphinxAtStartPar
The action and target of the Red Team action

\end{description}\end{quote}

\end{fulllineitems}

\index{update\_location() (yawning\_titan.agents.static\_agents.fixed\_red.FixedRedAgent method)@\spxentry{update\_location()}\spxextra{yawning\_titan.agents.static\_agents.fixed\_red.FixedRedAgent method}}

\begin{fulllineitems}
\phantomsection\label{\detokenize{source/yawning_titan.agents.static_agents:yawning_titan.agents.static_agents.fixed_red.FixedRedAgent.update_location}}\pysiglinewithargsret{\sphinxbfcode{\sphinxupquote{update\_location}}}{\emph{\DUrole{n}{target}}, \emph{\DUrole{n}{red\_current\_node}}}{}
\end{fulllineitems}


\end{fulllineitems}



\subparagraph{yawning\_titan.agents.static\_agents.nsa\_red\_agent module}
\label{\detokenize{source/yawning_titan.agents.static_agents:yawning-titan-agents-static-agents-nsa-red-agent-module}}

\subparagraph{yawning\_titan.agents.static\_agents.random\_agent module}
\label{\detokenize{source/yawning_titan.agents.static_agents:module-yawning_titan.agents.static_agents.random_agent}}\label{\detokenize{source/yawning_titan.agents.static_agents:yawning-titan-agents-static-agents-random-agent-module}}\index{module@\spxentry{module}!yawning\_titan.agents.static\_agents.random\_agent@\spxentry{yawning\_titan.agents.static\_agents.random\_agent}}\index{yawning\_titan.agents.static\_agents.random\_agent@\spxentry{yawning\_titan.agents.static\_agents.random\_agent}!module@\spxentry{module}}\index{RandomAgent (class in yawning\_titan.agents.static\_agents.random\_agent)@\spxentry{RandomAgent}\spxextra{class in yawning\_titan.agents.static\_agents.random\_agent}}

\begin{fulllineitems}
\phantomsection\label{\detokenize{source/yawning_titan.agents.static_agents:yawning_titan.agents.static_agents.random_agent.RandomAgent}}\pysiglinewithargsret{\sphinxbfcode{\sphinxupquote{class\DUrole{w}{  }}}\sphinxcode{\sphinxupquote{yawning\_titan.agents.static\_agents.random\_agent.}}\sphinxbfcode{\sphinxupquote{RandomAgent}}}{\emph{\DUrole{n}{action\_space}}}{}
\sphinxAtStartPar
Bases: \sphinxcode{\sphinxupquote{object}}
\index{act() (yawning\_titan.agents.static\_agents.random\_agent.RandomAgent method)@\spxentry{act()}\spxextra{yawning\_titan.agents.static\_agents.random\_agent.RandomAgent method}}

\begin{fulllineitems}
\phantomsection\label{\detokenize{source/yawning_titan.agents.static_agents:yawning_titan.agents.static_agents.random_agent.RandomAgent.act}}\pysiglinewithargsret{\sphinxbfcode{\sphinxupquote{act}}}{\emph{\DUrole{n}{observation}}, \emph{\DUrole{n}{reward}}, \emph{\DUrole{n}{done}}}{}
\end{fulllineitems}

\index{predict() (yawning\_titan.agents.static\_agents.random\_agent.RandomAgent method)@\spxentry{predict()}\spxextra{yawning\_titan.agents.static\_agents.random\_agent.RandomAgent method}}

\begin{fulllineitems}
\phantomsection\label{\detokenize{source/yawning_titan.agents.static_agents:yawning_titan.agents.static_agents.random_agent.RandomAgent.predict}}\pysiglinewithargsret{\sphinxbfcode{\sphinxupquote{predict}}}{\emph{\DUrole{n}{observation}}, \emph{\DUrole{n}{reward}}, \emph{\DUrole{n}{done}}}{}
\end{fulllineitems}


\end{fulllineitems}



\subparagraph{yawning\_titan.agents.static\_agents.simple\_blue module}
\label{\detokenize{source/yawning_titan.agents.static_agents:module-yawning_titan.agents.static_agents.simple_blue}}\label{\detokenize{source/yawning_titan.agents.static_agents:yawning-titan-agents-static-agents-simple-blue-module}}\index{module@\spxentry{module}!yawning\_titan.agents.static\_agents.simple\_blue@\spxentry{yawning\_titan.agents.static\_agents.simple\_blue}}\index{yawning\_titan.agents.static\_agents.simple\_blue@\spxentry{yawning\_titan.agents.static\_agents.simple\_blue}!module@\spxentry{module}}\index{SimpleBlue (class in yawning\_titan.agents.static\_agents.simple\_blue)@\spxentry{SimpleBlue}\spxextra{class in yawning\_titan.agents.static\_agents.simple\_blue}}

\begin{fulllineitems}
\phantomsection\label{\detokenize{source/yawning_titan.agents.static_agents:yawning_titan.agents.static_agents.simple_blue.SimpleBlue}}\pysiglinewithargsret{\sphinxbfcode{\sphinxupquote{class\DUrole{w}{  }}}\sphinxcode{\sphinxupquote{yawning\_titan.agents.static\_agents.simple\_blue.}}\sphinxbfcode{\sphinxupquote{SimpleBlue}}}{\emph{\DUrole{n}{n\_machines}\DUrole{o}{=}\DUrole{default_value}{None}}}{}
\sphinxAtStartPar
Bases: \sphinxcode{\sphinxupquote{object}}
\index{do\_blue\_action() (yawning\_titan.agents.static\_agents.simple\_blue.SimpleBlue method)@\spxentry{do\_blue\_action()}\spxextra{yawning\_titan.agents.static\_agents.simple\_blue.SimpleBlue method}}

\begin{fulllineitems}
\phantomsection\label{\detokenize{source/yawning_titan.agents.static_agents:yawning_titan.agents.static_agents.simple_blue.SimpleBlue.do_blue_action}}\pysiglinewithargsret{\sphinxbfcode{\sphinxupquote{do\_blue\_action}}}{\emph{\DUrole{n}{action}\DUrole{p}{:}\DUrole{w}{  }\DUrole{n}{int}}, \emph{\DUrole{n}{machine\_states}\DUrole{p}{:}\DUrole{w}{  }\DUrole{n}{List\DUrole{p}{{[}}List\DUrole{p}{{[}}float\DUrole{p}{{]}}\DUrole{p}{{]}}}}, \emph{\DUrole{n}{initial\_states}\DUrole{p}{:}\DUrole{w}{  }\DUrole{n}{List\DUrole{p}{{[}}List\DUrole{p}{{[}}float\DUrole{p}{{]}}\DUrole{p}{{]}}}}}{}
\sphinxAtStartPar
Performs the chosen action
\begin{quote}\begin{description}
\item[{Parameters}] \leavevmode\begin{itemize}
\item {}
\sphinxAtStartPar
\sphinxstyleliteralstrong{\sphinxupquote{action}} \textendash{} The chosen action to perform

\item {}
\sphinxAtStartPar
\sphinxstyleliteralstrong{\sphinxupquote{machine\_states}} \textendash{} The state of the current env

\item {}
\sphinxAtStartPar
\sphinxstyleliteralstrong{\sphinxupquote{initial\_states}} \textendash{} The state of the env initially

\end{itemize}

\end{description}\end{quote}

\end{fulllineitems}

\index{nothing() (yawning\_titan.agents.static\_agents.simple\_blue.SimpleBlue method)@\spxentry{nothing()}\spxextra{yawning\_titan.agents.static\_agents.simple\_blue.SimpleBlue method}}

\begin{fulllineitems}
\phantomsection\label{\detokenize{source/yawning_titan.agents.static_agents:yawning_titan.agents.static_agents.simple_blue.SimpleBlue.nothing}}\pysiglinewithargsret{\sphinxbfcode{\sphinxupquote{nothing}}}{\emph{\DUrole{n}{action}\DUrole{p}{:}\DUrole{w}{  }\DUrole{n}{int}}}{}
\end{fulllineitems}

\index{patch\_machines() (yawning\_titan.agents.static\_agents.simple\_blue.SimpleBlue method)@\spxentry{patch\_machines()}\spxextra{yawning\_titan.agents.static\_agents.simple\_blue.SimpleBlue method}}

\begin{fulllineitems}
\phantomsection\label{\detokenize{source/yawning_titan.agents.static_agents:yawning_titan.agents.static_agents.simple_blue.SimpleBlue.patch_machines}}\pysiglinewithargsret{\sphinxbfcode{\sphinxupquote{patch\_machines}}}{\emph{\DUrole{n}{action}\DUrole{p}{:}\DUrole{w}{  }\DUrole{n}{int}}, \emph{\DUrole{n}{machine\_states}\DUrole{p}{:}\DUrole{w}{  }\DUrole{n}{List\DUrole{p}{{[}}List\DUrole{p}{{[}}float\DUrole{p}{{]}}\DUrole{p}{{]}}}}}{}
\sphinxAtStartPar
Patch a target machine and reduce its vulnerability score

\sphinxAtStartPar
This action reduces a target machines vulerability score by 0.2
per use to a lower threshold of 0.2.
\begin{quote}\begin{description}
\item[{Parameters}] \leavevmode\begin{itemize}
\item {}
\sphinxAtStartPar
\sphinxstyleliteralstrong{\sphinxupquote{action}} \textendash{} The chosen action

\item {}
\sphinxAtStartPar
\sphinxstyleliteralstrong{\sphinxupquote{machine\_states}} \textendash{} The current state of the env

\end{itemize}

\end{description}\end{quote}

\end{fulllineitems}

\index{recover\_machines() (yawning\_titan.agents.static\_agents.simple\_blue.SimpleBlue method)@\spxentry{recover\_machines()}\spxextra{yawning\_titan.agents.static\_agents.simple\_blue.SimpleBlue method}}

\begin{fulllineitems}
\phantomsection\label{\detokenize{source/yawning_titan.agents.static_agents:yawning_titan.agents.static_agents.simple_blue.SimpleBlue.recover_machines}}\pysiglinewithargsret{\sphinxbfcode{\sphinxupquote{recover\_machines}}}{\emph{\DUrole{n}{action}\DUrole{p}{:}\DUrole{w}{  }\DUrole{n}{int}}, \emph{\DUrole{n}{machine\_states}\DUrole{p}{:}\DUrole{w}{  }\DUrole{n}{List\DUrole{p}{{[}}List\DUrole{p}{{[}}float\DUrole{p}{{]}}\DUrole{p}{{]}}}}, \emph{\DUrole{n}{initial\_states}\DUrole{p}{:}\DUrole{w}{  }\DUrole{n}{List\DUrole{p}{{[}}List\DUrole{p}{{[}}float\DUrole{p}{{]}}\DUrole{p}{{]}}}}}{}
\sphinxAtStartPar
Recovers a compromised machine and resets its state to the
initial state during environment creation.
\begin{quote}\begin{description}
\item[{Parameters}] \leavevmode\begin{itemize}
\item {}
\sphinxAtStartPar
\sphinxstyleliteralstrong{\sphinxupquote{action}} \textendash{} The chosen action

\item {}
\sphinxAtStartPar
\sphinxstyleliteralstrong{\sphinxupquote{machine\_states}} \textendash{} The current state of the env

\item {}
\sphinxAtStartPar
\sphinxstyleliteralstrong{\sphinxupquote{initial\_states}} \textendash{} The state of the env initially

\end{itemize}

\end{description}\end{quote}

\end{fulllineitems}


\end{fulllineitems}



\subparagraph{Module contents}
\label{\detokenize{source/yawning_titan.agents.static_agents:module-yawning_titan.agents.static_agents}}\label{\detokenize{source/yawning_titan.agents.static_agents:module-contents}}\index{module@\spxentry{module}!yawning\_titan.agents.static\_agents@\spxentry{yawning\_titan.agents.static\_agents}}\index{yawning\_titan.agents.static\_agents@\spxentry{yawning\_titan.agents.static\_agents}!module@\spxentry{module}}

\paragraph{Submodules}
\label{\detokenize{source/yawning_titan.agents:submodules}}

\paragraph{yawning\_titan.agents.keyboard\_agent module}
\label{\detokenize{source/yawning_titan.agents:module-yawning_titan.agents.keyboard_agent}}\label{\detokenize{source/yawning_titan.agents:yawning-titan-agents-keyboard-agent-module}}\index{module@\spxentry{module}!yawning\_titan.agents.keyboard\_agent@\spxentry{yawning\_titan.agents.keyboard\_agent}}\index{yawning\_titan.agents.keyboard\_agent@\spxentry{yawning\_titan.agents.keyboard\_agent}!module@\spxentry{module}}\index{KeyboardAgent (class in yawning\_titan.agents.keyboard\_agent)@\spxentry{KeyboardAgent}\spxextra{class in yawning\_titan.agents.keyboard\_agent}}

\begin{fulllineitems}
\phantomsection\label{\detokenize{source/yawning_titan.agents:yawning_titan.agents.keyboard_agent.KeyboardAgent}}\pysiglinewithargsret{\sphinxbfcode{\sphinxupquote{class\DUrole{w}{  }}}\sphinxcode{\sphinxupquote{yawning\_titan.agents.keyboard\_agent.}}\sphinxbfcode{\sphinxupquote{KeyboardAgent}}}{\emph{\DUrole{n}{env}\DUrole{p}{:}\DUrole{w}{  }\DUrole{n}{{\hyperref[\detokenize{source/yawning_titan.envs.generic:yawning_titan.envs.generic.generic_env.GenericNetworkEnv}]{\sphinxcrossref{yawning\_titan.envs.generic.generic\_env.GenericNetworkEnv}}}}}}{}
\sphinxAtStartPar
Bases: \sphinxcode{\sphinxupquote{object}}
\index{get\_move\_set() (yawning\_titan.agents.keyboard\_agent.KeyboardAgent method)@\spxentry{get\_move\_set()}\spxextra{yawning\_titan.agents.keyboard\_agent.KeyboardAgent method}}

\begin{fulllineitems}
\phantomsection\label{\detokenize{source/yawning_titan.agents:yawning_titan.agents.keyboard_agent.KeyboardAgent.get_move_set}}\pysiglinewithargsret{\sphinxbfcode{\sphinxupquote{get\_move\_set}}}{}{}
\sphinxAtStartPar
Gets the action set for the given environment and maps it to the action numbers used by the open AI gym env
\begin{quote}\begin{description}
\item[{Returns}] \leavevmode
\sphinxAtStartPar
An action mask for top level actions to the first action number in the environment
A full dictionary of all the actions that can be taken
The action number where the standard actions start (actions that can be applied to every node)
The number of standard actions

\end{description}\end{quote}

\end{fulllineitems}

\index{play() (yawning\_titan.agents.keyboard\_agent.KeyboardAgent method)@\spxentry{play()}\spxextra{yawning\_titan.agents.keyboard\_agent.KeyboardAgent method}}

\begin{fulllineitems}
\phantomsection\label{\detokenize{source/yawning_titan.agents:yawning_titan.agents.keyboard_agent.KeyboardAgent.play}}\pysiglinewithargsret{\sphinxbfcode{\sphinxupquote{play}}}{\emph{\DUrole{n}{render\_graphically}\DUrole{p}{:}\DUrole{w}{  }\DUrole{n}{bool}\DUrole{w}{  }\DUrole{o}{=}\DUrole{w}{  }\DUrole{default_value}{True}}}{}
\sphinxAtStartPar
Plays the game as a keyboard agent. Allows the user to select an action using the console and displays the
effect of the action on the envionrment.
\begin{quote}\begin{description}
\item[{Parameters}] \leavevmode
\sphinxAtStartPar
\sphinxstyleliteralstrong{\sphinxupquote{render\_graphically}} \textendash{} If True render using the matplotlib renderer, if False display if state of the
environment in the console

\end{description}\end{quote}

\end{fulllineitems}


\end{fulllineitems}



\paragraph{Module contents}
\label{\detokenize{source/yawning_titan.agents:module-yawning_titan.agents}}\label{\detokenize{source/yawning_titan.agents:module-contents}}\index{module@\spxentry{module}!yawning\_titan.agents@\spxentry{yawning\_titan.agents}}\index{yawning\_titan.agents@\spxentry{yawning\_titan.agents}!module@\spxentry{module}}

\subsubsection{yawning\_titan.envs package}
\label{\detokenize{source/yawning_titan.envs:yawning-titan-envs-package}}\label{\detokenize{source/yawning_titan.envs::doc}}

\paragraph{Subpackages}
\label{\detokenize{source/yawning_titan.envs:subpackages}}

\subparagraph{yawning\_titan.envs.generic package}
\label{\detokenize{source/yawning_titan.envs.generic:yawning-titan-envs-generic-package}}\label{\detokenize{source/yawning_titan.envs.generic::doc}}

\subparagraph{Subpackages}
\label{\detokenize{source/yawning_titan.envs.generic:subpackages}}

\subparagraph{yawning\_titan.envs.generic.core package}
\label{\detokenize{source/yawning_titan.envs.generic.core:yawning-titan-envs-generic-core-package}}\label{\detokenize{source/yawning_titan.envs.generic.core::doc}}

\subparagraph{Submodules}
\label{\detokenize{source/yawning_titan.envs.generic.core:submodules}}

\subparagraph{yawning\_titan.envs.generic.core.action\_loops module}
\label{\detokenize{source/yawning_titan.envs.generic.core:module-yawning_titan.envs.generic.core.action_loops}}\label{\detokenize{source/yawning_titan.envs.generic.core:yawning-titan-envs-generic-core-action-loops-module}}\index{module@\spxentry{module}!yawning\_titan.envs.generic.core.action\_loops@\spxentry{yawning\_titan.envs.generic.core.action\_loops}}\index{yawning\_titan.envs.generic.core.action\_loops@\spxentry{yawning\_titan.envs.generic.core.action\_loops}!module@\spxentry{module}}\index{ActionLoop (class in yawning\_titan.envs.generic.core.action\_loops)@\spxentry{ActionLoop}\spxextra{class in yawning\_titan.envs.generic.core.action\_loops}}

\begin{fulllineitems}
\phantomsection\label{\detokenize{source/yawning_titan.envs.generic.core:yawning_titan.envs.generic.core.action_loops.ActionLoop}}\pysiglinewithargsret{\sphinxbfcode{\sphinxupquote{class\DUrole{w}{  }}}\sphinxcode{\sphinxupquote{yawning\_titan.envs.generic.core.action\_loops.}}\sphinxbfcode{\sphinxupquote{ActionLoop}}}{\emph{\DUrole{n}{env}}, \emph{\DUrole{n}{agent}}, \emph{\DUrole{n}{filename}\DUrole{o}{=}\DUrole{default_value}{None}}, \emph{\DUrole{n}{episode\_count}\DUrole{o}{=}\DUrole{default_value}{None}}}{}
\sphinxAtStartPar
Bases: \sphinxcode{\sphinxupquote{object}}
\index{gif\_action\_loop() (yawning\_titan.envs.generic.core.action\_loops.ActionLoop method)@\spxentry{gif\_action\_loop()}\spxextra{yawning\_titan.envs.generic.core.action\_loops.ActionLoop method}}

\begin{fulllineitems}
\phantomsection\label{\detokenize{source/yawning_titan.envs.generic.core:yawning_titan.envs.generic.core.action_loops.ActionLoop.gif_action_loop}}\pysiglinewithargsret{\sphinxbfcode{\sphinxupquote{gif\_action\_loop}}}{}{}
\sphinxAtStartPar
Run the agent through the env and record each step using the render function. Then create a gif output of the
loop

\end{fulllineitems}

\index{random\_action\_loop() (yawning\_titan.envs.generic.core.action\_loops.ActionLoop method)@\spxentry{random\_action\_loop()}\spxextra{yawning\_titan.envs.generic.core.action\_loops.ActionLoop method}}

\begin{fulllineitems}
\phantomsection\label{\detokenize{source/yawning_titan.envs.generic.core:yawning_titan.envs.generic.core.action_loops.ActionLoop.random_action_loop}}\pysiglinewithargsret{\sphinxbfcode{\sphinxupquote{random\_action\_loop}}}{}{}
\sphinxAtStartPar
Step the agent through the env

\end{fulllineitems}

\index{standard\_action\_loop() (yawning\_titan.envs.generic.core.action\_loops.ActionLoop method)@\spxentry{standard\_action\_loop()}\spxextra{yawning\_titan.envs.generic.core.action\_loops.ActionLoop method}}

\begin{fulllineitems}
\phantomsection\label{\detokenize{source/yawning_titan.envs.generic.core:yawning_titan.envs.generic.core.action_loops.ActionLoop.standard_action_loop}}\pysiglinewithargsret{\sphinxbfcode{\sphinxupquote{standard\_action\_loop}}}{}{}
\sphinxAtStartPar
Step the agent through the env

\end{fulllineitems}


\end{fulllineitems}



\subparagraph{yawning\_titan.envs.generic.core.blue\_action\_set module}
\label{\detokenize{source/yawning_titan.envs.generic.core:module-yawning_titan.envs.generic.core.blue_action_set}}\label{\detokenize{source/yawning_titan.envs.generic.core:yawning-titan-envs-generic-core-blue-action-set-module}}\index{module@\spxentry{module}!yawning\_titan.envs.generic.core.blue\_action\_set@\spxentry{yawning\_titan.envs.generic.core.blue\_action\_set}}\index{yawning\_titan.envs.generic.core.blue\_action\_set@\spxentry{yawning\_titan.envs.generic.core.blue\_action\_set}!module@\spxentry{module}}\index{BlueActionSet (class in yawning\_titan.envs.generic.core.blue\_action\_set)@\spxentry{BlueActionSet}\spxextra{class in yawning\_titan.envs.generic.core.blue\_action\_set}}

\begin{fulllineitems}
\phantomsection\label{\detokenize{source/yawning_titan.envs.generic.core:yawning_titan.envs.generic.core.blue_action_set.BlueActionSet}}\pysiglinewithargsret{\sphinxbfcode{\sphinxupquote{class\DUrole{w}{  }}}\sphinxcode{\sphinxupquote{yawning\_titan.envs.generic.core.blue\_action\_set.}}\sphinxbfcode{\sphinxupquote{BlueActionSet}}}{\emph{\DUrole{n}{network\_interface}\DUrole{p}{:}\DUrole{w}{  }\DUrole{n}{{\hyperref[\detokenize{source/yawning_titan.envs.generic.core:yawning_titan.envs.generic.core.network_interface.NetworkInterface}]{\sphinxcrossref{yawning\_titan.envs.generic.core.network\_interface.NetworkInterface}}}}}, \emph{\DUrole{n}{settings\_file}\DUrole{p}{:}\DUrole{w}{  }\DUrole{n}{dict}}}{}
\sphinxAtStartPar
Bases: \sphinxcode{\sphinxupquote{object}}
\index{add\_deceptive\_node() (yawning\_titan.envs.generic.core.blue\_action\_set.BlueActionSet method)@\spxentry{add\_deceptive\_node()}\spxextra{yawning\_titan.envs.generic.core.blue\_action\_set.BlueActionSet method}}

\begin{fulllineitems}
\phantomsection\label{\detokenize{source/yawning_titan.envs.generic.core:yawning_titan.envs.generic.core.blue_action_set.BlueActionSet.add_deceptive_node}}\pysiglinewithargsret{\sphinxbfcode{\sphinxupquote{add\_deceptive\_node}}}{\emph{\DUrole{n}{edge}\DUrole{p}{:}\DUrole{w}{  }\DUrole{n}{int}}}{{ $\rightarrow$ Tuple\DUrole{p}{{[}}str\DUrole{p}{,}\DUrole{w}{  }Optional\DUrole{p}{{[}}List\DUrole{p}{{]}}\DUrole{p}{{]}}}}
\sphinxAtStartPar
Adds a deceptive node into the environment. Deceptive nodes are the same as standard nodes except they have a
100\% chance (by default) to be able to detect attacks from the red agent

\sphinxAtStartPar
A deceptive node is added on an edge between two nodes
\begin{quote}\begin{description}
\item[{Parameters}] \leavevmode
\sphinxAtStartPar
\sphinxstyleliteralstrong{\sphinxupquote{edge}} \textendash{} The edge to place the deceptive node on

\item[{Returns}] \leavevmode
\sphinxAtStartPar
The name of the action performed (“add\_deceptive\_node” or “do\_nothing” depending on if action is valid)
A pair of nodes that the deceptive node was placed between (or None if no action performed)

\end{description}\end{quote}

\end{fulllineitems}

\index{do\_nothing() (yawning\_titan.envs.generic.core.blue\_action\_set.BlueActionSet method)@\spxentry{do\_nothing()}\spxextra{yawning\_titan.envs.generic.core.blue\_action\_set.BlueActionSet method}}

\begin{fulllineitems}
\phantomsection\label{\detokenize{source/yawning_titan.envs.generic.core:yawning_titan.envs.generic.core.blue_action_set.BlueActionSet.do_nothing}}\pysiglinewithargsret{\sphinxbfcode{\sphinxupquote{do\_nothing}}}{}{{ $\rightarrow$ Tuple\DUrole{p}{{[}}str\DUrole{p}{,}\DUrole{w}{  }None\DUrole{p}{{]}}}}
\sphinxAtStartPar
The agent will do nothing if this action is chosen
\begin{quote}\begin{description}
\item[{Returns}] \leavevmode
\sphinxAtStartPar
The name of the action (“do\_nothing”)
The nodes affected (None: as do nothing affects no nodes)

\end{description}\end{quote}

\end{fulllineitems}

\index{isolate\_node() (yawning\_titan.envs.generic.core.blue\_action\_set.BlueActionSet method)@\spxentry{isolate\_node()}\spxextra{yawning\_titan.envs.generic.core.blue\_action\_set.BlueActionSet method}}

\begin{fulllineitems}
\phantomsection\label{\detokenize{source/yawning_titan.envs.generic.core:yawning_titan.envs.generic.core.blue_action_set.BlueActionSet.isolate_node}}\pysiglinewithargsret{\sphinxbfcode{\sphinxupquote{isolate\_node}}}{\emph{\DUrole{n}{node}\DUrole{p}{:}\DUrole{w}{  }\DUrole{n}{str}}}{{ $\rightarrow$ Tuple\DUrole{p}{{[}}str\DUrole{p}{,}\DUrole{w}{  }str\DUrole{p}{{]}}}}
\sphinxAtStartPar
Isolates a node by disabling all of its connections to other nodes
\begin{quote}\begin{description}
\item[{Parameters}] \leavevmode
\sphinxAtStartPar
\sphinxstyleliteralstrong{\sphinxupquote{node}} \textendash{} the node to disable

\item[{Returns}] \leavevmode
\sphinxAtStartPar
The name of the action (“isolate”)
The node affected

\end{description}\end{quote}

\end{fulllineitems}

\index{make\_safe\_node() (yawning\_titan.envs.generic.core.blue\_action\_set.BlueActionSet method)@\spxentry{make\_safe\_node()}\spxextra{yawning\_titan.envs.generic.core.blue\_action\_set.BlueActionSet method}}

\begin{fulllineitems}
\phantomsection\label{\detokenize{source/yawning_titan.envs.generic.core:yawning_titan.envs.generic.core.blue_action_set.BlueActionSet.make_safe_node}}\pysiglinewithargsret{\sphinxbfcode{\sphinxupquote{make\_safe\_node}}}{\emph{\DUrole{n}{node}\DUrole{p}{:}\DUrole{w}{  }\DUrole{n}{str}}}{{ $\rightarrow$ Tuple\DUrole{p}{{[}}str\DUrole{p}{,}\DUrole{w}{  }str\DUrole{p}{{]}}}}
\sphinxAtStartPar
Makes a target node safe. Can also affect the vulnerability of the node. There are settings that can change
how this action works in the configuration file:
\begin{itemize}
\item {}
\sphinxAtStartPar
BLUE: making\_node\_safe\_modifies\_vulnerability

\item {}
\sphinxAtStartPar
BLUE: vulnerability\_change\_during\_node\_patch

\item {}
\sphinxAtStartPar
BLUE: making\_node\_safe\_gives\_random\_vulnerability

\end{itemize}
\begin{quote}\begin{description}
\item[{Parameters}] \leavevmode\begin{itemize}
\item {}
\sphinxAtStartPar
\sphinxstyleliteralstrong{\sphinxupquote{action}} (\sphinxstyleliteralemphasis{\sphinxupquote{The name of the}}) \textendash{}

\item {}
\sphinxAtStartPar
\sphinxstyleliteralstrong{\sphinxupquote{safe}} (\sphinxstyleliteralemphasis{\sphinxupquote{The name of the node to make}}) \textendash{}

\end{itemize}

\end{description}\end{quote}

\end{fulllineitems}

\index{reconnect\_node() (yawning\_titan.envs.generic.core.blue\_action\_set.BlueActionSet method)@\spxentry{reconnect\_node()}\spxextra{yawning\_titan.envs.generic.core.blue\_action\_set.BlueActionSet method}}

\begin{fulllineitems}
\phantomsection\label{\detokenize{source/yawning_titan.envs.generic.core:yawning_titan.envs.generic.core.blue_action_set.BlueActionSet.reconnect_node}}\pysiglinewithargsret{\sphinxbfcode{\sphinxupquote{reconnect\_node}}}{\emph{\DUrole{n}{node}\DUrole{p}{:}\DUrole{w}{  }\DUrole{n}{str}}}{{ $\rightarrow$ Tuple\DUrole{p}{{[}}str\DUrole{p}{,}\DUrole{w}{  }str\DUrole{p}{{]}}}}
\sphinxAtStartPar
Enables all of the connections to and from a node
\begin{quote}\begin{description}
\item[{Parameters}] \leavevmode
\sphinxAtStartPar
\sphinxstyleliteralstrong{\sphinxupquote{node}} \textendash{} the node to enable to connections to

\item[{Returns}] \leavevmode
\sphinxAtStartPar
The name of the action (“connect”)
The node affected

\end{description}\end{quote}

\end{fulllineitems}

\index{reduce\_node\_vulnerability() (yawning\_titan.envs.generic.core.blue\_action\_set.BlueActionSet method)@\spxentry{reduce\_node\_vulnerability()}\spxextra{yawning\_titan.envs.generic.core.blue\_action\_set.BlueActionSet method}}

\begin{fulllineitems}
\phantomsection\label{\detokenize{source/yawning_titan.envs.generic.core:yawning_titan.envs.generic.core.blue_action_set.BlueActionSet.reduce_node_vulnerability}}\pysiglinewithargsret{\sphinxbfcode{\sphinxupquote{reduce\_node\_vulnerability}}}{\emph{\DUrole{n}{node}\DUrole{p}{:}\DUrole{w}{  }\DUrole{n}{str}}}{{ $\rightarrow$ Tuple\DUrole{p}{{[}}str\DUrole{p}{,}\DUrole{w}{  }str\DUrole{p}{{]}}}}
\sphinxAtStartPar
Reduces the vulnerability of a given node. Will not reduce the vulnerability past the lower bound setting in the
configuration file:
\begin{itemize}
\item {}
\sphinxAtStartPar
BLUE: node\_vulnerability\_lower\_bound

\end{itemize}
\begin{quote}\begin{description}
\item[{Parameters}] \leavevmode
\sphinxAtStartPar
\sphinxstyleliteralstrong{\sphinxupquote{node}} \textendash{} the node to reduce the vulnerability of

\item[{Returns}] \leavevmode
\sphinxAtStartPar
The name of the action taken (“reduce\_vulnerability”)
The node the action was taken on

\end{description}\end{quote}

\end{fulllineitems}

\index{restore\_node() (yawning\_titan.envs.generic.core.blue\_action\_set.BlueActionSet method)@\spxentry{restore\_node()}\spxextra{yawning\_titan.envs.generic.core.blue\_action\_set.BlueActionSet method}}

\begin{fulllineitems}
\phantomsection\label{\detokenize{source/yawning_titan.envs.generic.core:yawning_titan.envs.generic.core.blue_action_set.BlueActionSet.restore_node}}\pysiglinewithargsret{\sphinxbfcode{\sphinxupquote{restore\_node}}}{\emph{\DUrole{n}{node}\DUrole{p}{:}\DUrole{w}{  }\DUrole{n}{str}}}{{ $\rightarrow$ Tuple\DUrole{p}{{[}}str\DUrole{p}{,}\DUrole{w}{  }str\DUrole{p}{{]}}}}
\sphinxAtStartPar
Restore a node to its starting state: safe and with its starting vulnerability
\begin{quote}\begin{description}
\item[{Parameters}] \leavevmode
\sphinxAtStartPar
\sphinxstyleliteralstrong{\sphinxupquote{node}} \textendash{} the node to restore

\item[{Returns}] \leavevmode
\sphinxAtStartPar
The name of the action (“restore\_node”)
The name of the node the action was taken on

\end{description}\end{quote}

\end{fulllineitems}

\index{scan\_all\_nodes() (yawning\_titan.envs.generic.core.blue\_action\_set.BlueActionSet method)@\spxentry{scan\_all\_nodes()}\spxextra{yawning\_titan.envs.generic.core.blue\_action\_set.BlueActionSet method}}

\begin{fulllineitems}
\phantomsection\label{\detokenize{source/yawning_titan.envs.generic.core:yawning_titan.envs.generic.core.blue_action_set.BlueActionSet.scan_all_nodes}}\pysiglinewithargsret{\sphinxbfcode{\sphinxupquote{scan\_all\_nodes}}}{}{{ $\rightarrow$ Tuple\DUrole{p}{{[}}str\DUrole{p}{,}\DUrole{w}{  }None\DUrole{p}{{]}}}}
\sphinxAtStartPar
Scans all of the nodes and attempts to get their states. The blue agents ability to see intrusions is based on
the values in the config file:
\begin{itemize}
\item {}
\sphinxAtStartPar
BLUE: chance\_to\_discover\_intrusion\_on\_scan

\item {}
\sphinxAtStartPar
BLUE: chance\_to\_discover\_intrusion\_on\_scan\_deceptive\_node

\end{itemize}
\begin{quote}\begin{description}
\item[{Returns}] \leavevmode
\sphinxAtStartPar
The name of the action (“scan”)
The node the action was performed on (None: as scan affects all nodes, not just 1)

\end{description}\end{quote}

\end{fulllineitems}


\end{fulllineitems}



\subparagraph{yawning\_titan.envs.generic.core.blue\_interface module}
\label{\detokenize{source/yawning_titan.envs.generic.core:module-yawning_titan.envs.generic.core.blue_interface}}\label{\detokenize{source/yawning_titan.envs.generic.core:yawning-titan-envs-generic-core-blue-interface-module}}\index{module@\spxentry{module}!yawning\_titan.envs.generic.core.blue\_interface@\spxentry{yawning\_titan.envs.generic.core.blue\_interface}}\index{yawning\_titan.envs.generic.core.blue\_interface@\spxentry{yawning\_titan.envs.generic.core.blue\_interface}!module@\spxentry{module}}\index{BlueInterface (class in yawning\_titan.envs.generic.core.blue\_interface)@\spxentry{BlueInterface}\spxextra{class in yawning\_titan.envs.generic.core.blue\_interface}}

\begin{fulllineitems}
\phantomsection\label{\detokenize{source/yawning_titan.envs.generic.core:yawning_titan.envs.generic.core.blue_interface.BlueInterface}}\pysiglinewithargsret{\sphinxbfcode{\sphinxupquote{class\DUrole{w}{  }}}\sphinxcode{\sphinxupquote{yawning\_titan.envs.generic.core.blue\_interface.}}\sphinxbfcode{\sphinxupquote{BlueInterface}}}{\emph{\DUrole{n}{network\_interface}\DUrole{p}{:}\DUrole{w}{  }\DUrole{n}{{\hyperref[\detokenize{source/yawning_titan.envs.generic.core:yawning_titan.envs.generic.core.network_interface.NetworkInterface}]{\sphinxcrossref{yawning\_titan.envs.generic.core.network\_interface.NetworkInterface}}}}}}{}
\sphinxAtStartPar
Bases: {\hyperref[\detokenize{source/yawning_titan.envs.generic.core:yawning_titan.envs.generic.core.blue_action_set.BlueActionSet}]{\sphinxcrossref{\sphinxcode{\sphinxupquote{yawning\_titan.envs.generic.core.blue\_action\_set.BlueActionSet}}}}}
\index{get\_number\_of\_actions() (yawning\_titan.envs.generic.core.blue\_interface.BlueInterface method)@\spxentry{get\_number\_of\_actions()}\spxextra{yawning\_titan.envs.generic.core.blue\_interface.BlueInterface method}}

\begin{fulllineitems}
\phantomsection\label{\detokenize{source/yawning_titan.envs.generic.core:yawning_titan.envs.generic.core.blue_interface.BlueInterface.get_number_of_actions}}\pysiglinewithargsret{\sphinxbfcode{\sphinxupquote{get\_number\_of\_actions}}}{}{{ $\rightarrow$ int}}\begin{description}
\item[{Gets the number of actions that this blue agent can perform. There are two types of actions:}] \leavevmode\begin{itemize}
\item {}
\sphinxAtStartPar
global actions (apply to all nodes) \sphinxhyphen{} need 1 action space

\item {}
\sphinxAtStartPar
standard actions (apply to a single node) \sphinxhyphen{} need 2 action space (action and node to perform on)

\end{itemize}

\end{description}
\begin{quote}\begin{description}
\item[{Returns}] \leavevmode
\sphinxAtStartPar
The number of actions that this agent can perform

\end{description}\end{quote}

\end{fulllineitems}

\index{perform\_action() (yawning\_titan.envs.generic.core.blue\_interface.BlueInterface method)@\spxentry{perform\_action()}\spxextra{yawning\_titan.envs.generic.core.blue\_interface.BlueInterface method}}

\begin{fulllineitems}
\phantomsection\label{\detokenize{source/yawning_titan.envs.generic.core:yawning_titan.envs.generic.core.blue_interface.BlueInterface.perform_action}}\pysiglinewithargsret{\sphinxbfcode{\sphinxupquote{perform\_action}}}{\emph{\DUrole{n}{action}\DUrole{p}{:}\DUrole{w}{  }\DUrole{n}{int}}}{{ $\rightarrow$ Tuple\DUrole{p}{{[}}str\DUrole{p}{,}\DUrole{w}{  }str\DUrole{p}{{]}}}}
\sphinxAtStartPar
Takes in an action number and then maps this to the correct action to perform. There are 3 different types of
actions:
\begin{itemize}
\item {}
\sphinxAtStartPar
standard actions

\item {}
\sphinxAtStartPar
deceptive actions

\item {}
\sphinxAtStartPar
global actions

\end{itemize}

\sphinxAtStartPar
\textendash{}standard actions\textendash{}
Standard actions are actions that can apply to all nodes. For each standard action there are n actions (where n
is the number of nodes in the network). An example of this action would be to isolate a node. The agent has to
pick the isolate action and then the node it is being applied to.

\sphinxAtStartPar
\textendash{}deceptive actions\textendash{}
Actions relating to deceptive nodes. Since the number of deceptive actions relate to the edges not the nodes
(see deceptive nodes for more info), the deceptive actions cannot come under the standard actions. An example
would be to place a deceptive node. The deceptive nodes can only be placed on an edge so the agent has to pick
the “place deceptive node” action and then the edge to place it on.

\sphinxAtStartPar
\textendash{}global actions\textendash{}
Global actions are actions where the agent does not need to pick any sub action other than the action. For
example an action that applies to all nodes so the agent does not need to pick a specific node to apply the
action to. “Do nothing” is an example of a global action as there is no secondary choice to be made.

\sphinxAtStartPar
The function also maps any actions outside of the action space to the “do nothing” action.

\sphinxAtStartPar
Order of operations:
1\sphinxhyphen{} check if the action is inside the action space \textendash{}\textgreater{} perform “do nothing”
2\sphinxhyphen{} check if the action is a deceptive action \textendash{}\textgreater{} perform action
3\sphinxhyphen{} check if the action is a global action \textendash{}\textgreater{} perform action
4\sphinxhyphen{} perform the standard action
\begin{quote}\begin{description}
\item[{Parameters}] \leavevmode
\sphinxAtStartPar
\sphinxstyleliteralstrong{\sphinxupquote{action}} \textendash{} the action to perform

\item[{Returns}] \leavevmode
\sphinxAtStartPar
The action that has been taken
The node the action was performed on

\end{description}\end{quote}

\end{fulllineitems}


\end{fulllineitems}



\subparagraph{yawning\_titan.envs.generic.core.network\_interface module}
\label{\detokenize{source/yawning_titan.envs.generic.core:module-yawning_titan.envs.generic.core.network_interface}}\label{\detokenize{source/yawning_titan.envs.generic.core:yawning-titan-envs-generic-core-network-interface-module}}\index{module@\spxentry{module}!yawning\_titan.envs.generic.core.network\_interface@\spxentry{yawning\_titan.envs.generic.core.network\_interface}}\index{yawning\_titan.envs.generic.core.network\_interface@\spxentry{yawning\_titan.envs.generic.core.network\_interface}!module@\spxentry{module}}
\sphinxAtStartPar
Network Interface class
This class contains the core for the generic network
\index{NetworkInterface (class in yawning\_titan.envs.generic.core.network\_interface)@\spxentry{NetworkInterface}\spxextra{class in yawning\_titan.envs.generic.core.network\_interface}}

\begin{fulllineitems}
\phantomsection\label{\detokenize{source/yawning_titan.envs.generic.core:yawning_titan.envs.generic.core.network_interface.NetworkInterface}}\pysiglinewithargsret{\sphinxbfcode{\sphinxupquote{class\DUrole{w}{  }}}\sphinxcode{\sphinxupquote{yawning\_titan.envs.generic.core.network\_interface.}}\sphinxbfcode{\sphinxupquote{NetworkInterface}}}{\emph{\DUrole{n}{matrix}\DUrole{p}{:}\DUrole{w}{  }\DUrole{n}{numpy.array}}, \emph{\DUrole{n}{positions}\DUrole{p}{:}\DUrole{w}{  }\DUrole{n}{dict}}, \emph{\DUrole{n}{settings\_path}\DUrole{p}{:}\DUrole{w}{  }\DUrole{n}{Optional\DUrole{p}{{[}}str\DUrole{p}{{]}}}\DUrole{w}{  }\DUrole{o}{=}\DUrole{w}{  }\DUrole{default_value}{None}}, \emph{\DUrole{n}{entry\_nodes}\DUrole{p}{:}\DUrole{w}{  }\DUrole{n}{Optional\DUrole{p}{{[}}List\DUrole{p}{{[}}str\DUrole{p}{{]}}\DUrole{p}{{]}}}\DUrole{w}{  }\DUrole{o}{=}\DUrole{w}{  }\DUrole{default_value}{None}}, \emph{\DUrole{n}{vulnerabilities}\DUrole{p}{:}\DUrole{w}{  }\DUrole{n}{Optional\DUrole{p}{{[}}dict\DUrole{p}{{]}}}\DUrole{w}{  }\DUrole{o}{=}\DUrole{w}{  }\DUrole{default_value}{None}}, \emph{\DUrole{n}{high\_value\_target}\DUrole{p}{:}\DUrole{w}{  }\DUrole{n}{Optional\DUrole{p}{{[}}str\DUrole{p}{{]}}}\DUrole{w}{  }\DUrole{o}{=}\DUrole{w}{  }\DUrole{default_value}{None}}}{}
\sphinxAtStartPar
Bases: \sphinxcode{\sphinxupquote{object}}
\index{add\_deceptive\_node() (yawning\_titan.envs.generic.core.network\_interface.NetworkInterface method)@\spxentry{add\_deceptive\_node()}\spxextra{yawning\_titan.envs.generic.core.network\_interface.NetworkInterface method}}

\begin{fulllineitems}
\phantomsection\label{\detokenize{source/yawning_titan.envs.generic.core:yawning_titan.envs.generic.core.network_interface.NetworkInterface.add_deceptive_node}}\pysiglinewithargsret{\sphinxbfcode{\sphinxupquote{add\_deceptive\_node}}}{\emph{\DUrole{n}{node1}\DUrole{p}{:}\DUrole{w}{  }\DUrole{n}{str}}, \emph{\DUrole{n}{node2}\DUrole{p}{:}\DUrole{w}{  }\DUrole{n}{str}}}{{ $\rightarrow$ Union\DUrole{p}{{[}}bool\DUrole{p}{,}\DUrole{w}{  }str\DUrole{p}{{]}}}}
\sphinxAtStartPar
Adds a deceptive node into the network. The deceptive node will sit between two actual nodes and act as a
normal node in all regards other than the fact that it give more information when it is attacked
\begin{quote}\begin{description}
\item[{Parameters}] \leavevmode\begin{itemize}
\item {}
\sphinxAtStartPar
\sphinxstyleliteralstrong{\sphinxupquote{node1}} \textendash{} Name of the first node to connect to the deceptive node to

\item {}
\sphinxAtStartPar
\sphinxstyleliteralstrong{\sphinxupquote{node2}} \textendash{} Name of the second to connect the deceptive node to

\end{itemize}

\item[{Returns}] \leavevmode
\sphinxAtStartPar
False if failed, the name of the new node if succeeded

\end{description}\end{quote}

\end{fulllineitems}

\index{attack\_node() (yawning\_titan.envs.generic.core.network\_interface.NetworkInterface method)@\spxentry{attack\_node()}\spxextra{yawning\_titan.envs.generic.core.network\_interface.NetworkInterface method}}

\begin{fulllineitems}
\phantomsection\label{\detokenize{source/yawning_titan.envs.generic.core:yawning_titan.envs.generic.core.network_interface.NetworkInterface.attack_node}}\pysiglinewithargsret{\sphinxbfcode{\sphinxupquote{attack\_node}}}{\emph{\DUrole{n}{node}\DUrole{p}{:}\DUrole{w}{  }\DUrole{n}{str}}, \emph{\DUrole{n}{skill}\DUrole{p}{:}\DUrole{w}{  }\DUrole{n}{float}\DUrole{w}{  }\DUrole{o}{=}\DUrole{w}{  }\DUrole{default_value}{0.5}}, \emph{\DUrole{n}{use\_skill}\DUrole{p}{:}\DUrole{w}{  }\DUrole{n}{bool}\DUrole{w}{  }\DUrole{o}{=}\DUrole{w}{  }\DUrole{default_value}{False}}, \emph{\DUrole{n}{use\_vulnerability}\DUrole{p}{:}\DUrole{w}{  }\DUrole{n}{bool}\DUrole{w}{  }\DUrole{o}{=}\DUrole{w}{  }\DUrole{default_value}{False}}, \emph{\DUrole{n}{guarantee}\DUrole{p}{:}\DUrole{w}{  }\DUrole{n}{bool}\DUrole{w}{  }\DUrole{o}{=}\DUrole{w}{  }\DUrole{default_value}{False}}}{{ $\rightarrow$ bool}}
\sphinxAtStartPar
Attacks a node and uses a random chance to suceed that is modified by the skill of the attack and the
vulnerability of the node. Both the skill and the vulnerability can be toggled to either be used or not
\begin{quote}\begin{description}
\item[{Parameters}] \leavevmode\begin{itemize}
\item {}
\sphinxAtStartPar
\sphinxstyleliteralstrong{\sphinxupquote{node}} \textendash{} The name of the node to target

\item {}
\sphinxAtStartPar
\sphinxstyleliteralstrong{\sphinxupquote{skill}} \textendash{} The skill of the attacker

\item {}
\sphinxAtStartPar
\sphinxstyleliteralstrong{\sphinxupquote{use\_skill}} \textendash{} A boolean value that is used to determine if skill is used in the calculation to check if the
attack succeeds

\item {}
\sphinxAtStartPar
\sphinxstyleliteralstrong{\sphinxupquote{use\_vulnerability}} \textendash{} A boolean value that is used to determine if vulnerability is used in the calculation to
check if the attack succeeds

\item {}
\sphinxAtStartPar
\sphinxstyleliteralstrong{\sphinxupquote{guarantee}} \textendash{} If True then attack automatically succeeds

\end{itemize}

\item[{Returns}] \leavevmode
\sphinxAtStartPar
A boolean value that represents if the attack succeeded or not

\end{description}\end{quote}

\end{fulllineitems}

\index{create\_json\_time\_step() (yawning\_titan.envs.generic.core.network\_interface.NetworkInterface method)@\spxentry{create\_json\_time\_step()}\spxextra{yawning\_titan.envs.generic.core.network\_interface.NetworkInterface method}}

\begin{fulllineitems}
\phantomsection\label{\detokenize{source/yawning_titan.envs.generic.core:yawning_titan.envs.generic.core.network_interface.NetworkInterface.create_json_time_step}}\pysiglinewithargsret{\sphinxbfcode{\sphinxupquote{create\_json\_time\_step}}}{}{{ $\rightarrow$ dict}}
\sphinxAtStartPar
Creates a dictionary that contains the current state of the environment and returns it
\begin{quote}\begin{description}
\item[{Returns}] \leavevmode
\sphinxAtStartPar
A dictionary containing the node connections, states and vulnerability scores

\end{description}\end{quote}

\end{fulllineitems}

\index{get\_all\_isolation() (yawning\_titan.envs.generic.core.network\_interface.NetworkInterface method)@\spxentry{get\_all\_isolation()}\spxextra{yawning\_titan.envs.generic.core.network\_interface.NetworkInterface method}}

\begin{fulllineitems}
\phantomsection\label{\detokenize{source/yawning_titan.envs.generic.core:yawning_titan.envs.generic.core.network_interface.NetworkInterface.get_all_isolation}}\pysiglinewithargsret{\sphinxbfcode{\sphinxupquote{get\_all\_isolation}}}{}{{ $\rightarrow$ dict}}
\sphinxAtStartPar
Returns a dictionary of the isolation status of all the nodes
\begin{quote}\begin{description}
\item[{Returns}] \leavevmode
\sphinxAtStartPar
A dictionary of isolation statuses

\end{description}\end{quote}

\end{fulllineitems}

\index{get\_all\_node\_blue\_view\_compromised\_states() (yawning\_titan.envs.generic.core.network\_interface.NetworkInterface method)@\spxentry{get\_all\_node\_blue\_view\_compromised\_states()}\spxextra{yawning\_titan.envs.generic.core.network\_interface.NetworkInterface method}}

\begin{fulllineitems}
\phantomsection\label{\detokenize{source/yawning_titan.envs.generic.core:yawning_titan.envs.generic.core.network_interface.NetworkInterface.get_all_node_blue_view_compromised_states}}\pysiglinewithargsret{\sphinxbfcode{\sphinxupquote{get\_all\_node\_blue\_view\_compromised\_states}}}{}{{ $\rightarrow$ dict}}
\sphinxAtStartPar
Returns a dictionary of compromised states
\begin{quote}\begin{description}
\item[{Returns}] \leavevmode
\sphinxAtStartPar
A dictionary of compromised states

\end{description}\end{quote}

\end{fulllineitems}

\index{get\_all\_node\_compromised\_states() (yawning\_titan.envs.generic.core.network\_interface.NetworkInterface method)@\spxentry{get\_all\_node\_compromised\_states()}\spxextra{yawning\_titan.envs.generic.core.network\_interface.NetworkInterface method}}

\begin{fulllineitems}
\phantomsection\label{\detokenize{source/yawning_titan.envs.generic.core:yawning_titan.envs.generic.core.network_interface.NetworkInterface.get_all_node_compromised_states}}\pysiglinewithargsret{\sphinxbfcode{\sphinxupquote{get\_all\_node\_compromised\_states}}}{}{{ $\rightarrow$ dict}}
\sphinxAtStartPar
Returns a dictionary of compromised states
\begin{quote}\begin{description}
\item[{Returns}] \leavevmode
\sphinxAtStartPar
A dictionary of compromised states

\end{description}\end{quote}

\end{fulllineitems}

\index{get\_all\_node\_positions() (yawning\_titan.envs.generic.core.network\_interface.NetworkInterface method)@\spxentry{get\_all\_node\_positions()}\spxextra{yawning\_titan.envs.generic.core.network\_interface.NetworkInterface method}}

\begin{fulllineitems}
\phantomsection\label{\detokenize{source/yawning_titan.envs.generic.core:yawning_titan.envs.generic.core.network_interface.NetworkInterface.get_all_node_positions}}\pysiglinewithargsret{\sphinxbfcode{\sphinxupquote{get\_all\_node\_positions}}}{}{{ $\rightarrow$ dict}}
\sphinxAtStartPar
Returns a dictionary of positions
\begin{quote}\begin{description}
\item[{Returns}] \leavevmode
\sphinxAtStartPar
A dictionary of positions

\end{description}\end{quote}

\end{fulllineitems}

\index{get\_all\_vulnerabilities() (yawning\_titan.envs.generic.core.network\_interface.NetworkInterface method)@\spxentry{get\_all\_vulnerabilities()}\spxextra{yawning\_titan.envs.generic.core.network\_interface.NetworkInterface method}}

\begin{fulllineitems}
\phantomsection\label{\detokenize{source/yawning_titan.envs.generic.core:yawning_titan.envs.generic.core.network_interface.NetworkInterface.get_all_vulnerabilities}}\pysiglinewithargsret{\sphinxbfcode{\sphinxupquote{get\_all\_vulnerabilities}}}{}{{ $\rightarrow$ dict}}
\sphinxAtStartPar
Returns a dictionary of vulnerability scores
\begin{quote}\begin{description}
\item[{Returns}] \leavevmode
\sphinxAtStartPar
A dictionary of vulnerability scores

\end{description}\end{quote}

\end{fulllineitems}

\index{get\_attributes\_from\_key() (yawning\_titan.envs.generic.core.network\_interface.NetworkInterface method)@\spxentry{get\_attributes\_from\_key()}\spxextra{yawning\_titan.envs.generic.core.network\_interface.NetworkInterface method}}

\begin{fulllineitems}
\phantomsection\label{\detokenize{source/yawning_titan.envs.generic.core:yawning_titan.envs.generic.core.network_interface.NetworkInterface.get_attributes_from_key}}\pysiglinewithargsret{\sphinxbfcode{\sphinxupquote{get\_attributes\_from\_key}}}{\emph{\DUrole{n}{key}\DUrole{p}{:}\DUrole{w}{  }\DUrole{n}{str}}}{{ $\rightarrow$ dict}}
\sphinxAtStartPar
Takes in a key and returns a dictionary where the keys are the names of the nodes and the values are the
attribute values that are stored for that node under the specified key
\begin{quote}\begin{description}
\item[{Parameters}] \leavevmode
\sphinxAtStartPar
\sphinxstyleliteralstrong{\sphinxupquote{key}} \textendash{} The name of the attribute to extract

\item[{Returns}] \leavevmode
\sphinxAtStartPar
A dictionary of attributes

\end{description}\end{quote}

\end{fulllineitems}

\index{get\_base\_connected\_nodes() (yawning\_titan.envs.generic.core.network\_interface.NetworkInterface method)@\spxentry{get\_base\_connected\_nodes()}\spxextra{yawning\_titan.envs.generic.core.network\_interface.NetworkInterface method}}

\begin{fulllineitems}
\phantomsection\label{\detokenize{source/yawning_titan.envs.generic.core:yawning_titan.envs.generic.core.network_interface.NetworkInterface.get_base_connected_nodes}}\pysiglinewithargsret{\sphinxbfcode{\sphinxupquote{get\_base\_connected\_nodes}}}{\emph{\DUrole{n}{node}\DUrole{p}{:}\DUrole{w}{  }\DUrole{n}{str}}}{{ $\rightarrow$ List\DUrole{p}{{[}}str\DUrole{p}{{]}}}}
\sphinxAtStartPar
Gets all of the nodes connected to the given node in the base graph
\begin{quote}\begin{description}
\item[{Parameters}] \leavevmode
\sphinxAtStartPar
\sphinxstyleliteralstrong{\sphinxupquote{node}} \textendash{} The name of the node to get the current connections of

\item[{Returns}] \leavevmode
\sphinxAtStartPar
A list of nodes

\end{description}\end{quote}

\end{fulllineitems}

\index{get\_current\_adj\_matrix() (yawning\_titan.envs.generic.core.network\_interface.NetworkInterface method)@\spxentry{get\_current\_adj\_matrix()}\spxextra{yawning\_titan.envs.generic.core.network\_interface.NetworkInterface method}}

\begin{fulllineitems}
\phantomsection\label{\detokenize{source/yawning_titan.envs.generic.core:yawning_titan.envs.generic.core.network_interface.NetworkInterface.get_current_adj_matrix}}\pysiglinewithargsret{\sphinxbfcode{\sphinxupquote{get\_current\_adj\_matrix}}}{}{}
\end{fulllineitems}

\index{get\_current\_connected\_nodes() (yawning\_titan.envs.generic.core.network\_interface.NetworkInterface method)@\spxentry{get\_current\_connected\_nodes()}\spxextra{yawning\_titan.envs.generic.core.network\_interface.NetworkInterface method}}

\begin{fulllineitems}
\phantomsection\label{\detokenize{source/yawning_titan.envs.generic.core:yawning_titan.envs.generic.core.network_interface.NetworkInterface.get_current_connected_nodes}}\pysiglinewithargsret{\sphinxbfcode{\sphinxupquote{get\_current\_connected\_nodes}}}{\emph{\DUrole{n}{node}\DUrole{p}{:}\DUrole{w}{  }\DUrole{n}{str}}}{{ $\rightarrow$ List\DUrole{p}{{[}}str\DUrole{p}{{]}}}}
\sphinxAtStartPar
Gets all of the nodes currently connected to a target node
\begin{quote}\begin{description}
\item[{Parameters}] \leavevmode
\sphinxAtStartPar
\sphinxstyleliteralstrong{\sphinxupquote{node}} \textendash{} The name of the node to get the current connections of

\item[{Returns}] \leavevmode
\sphinxAtStartPar
A list of nodes

\end{description}\end{quote}

\end{fulllineitems}

\index{get\_current\_observation() (yawning\_titan.envs.generic.core.network\_interface.NetworkInterface method)@\spxentry{get\_current\_observation()}\spxextra{yawning\_titan.envs.generic.core.network\_interface.NetworkInterface method}}

\begin{fulllineitems}
\phantomsection\label{\detokenize{source/yawning_titan.envs.generic.core:yawning_titan.envs.generic.core.network_interface.NetworkInterface.get_current_observation}}\pysiglinewithargsret{\sphinxbfcode{\sphinxupquote{get\_current\_observation}}}{}{{ $\rightarrow$ numpy.array}}
\sphinxAtStartPar
Gets the current observation of the environment. The environment consists of an array where each node has a
column to show if the node is compromised (0: safe, 1: compromised). Then there are number\_of\_nodes columns that
contain the current linked status to that main node. So for example in a 3 node network where node 0 was safe
and connected to node 2 but not node 1:
{[}0 0 1{]}
compromised and not connected to any other node:
{[}1, 0, 0{]}
\begin{quote}\begin{description}
\item[{Returns}] \leavevmode
\sphinxAtStartPar
numpy array containing the above details

\end{description}\end{quote}

\end{fulllineitems}

\index{get\_detected\_attacks() (yawning\_titan.envs.generic.core.network\_interface.NetworkInterface method)@\spxentry{get\_detected\_attacks()}\spxextra{yawning\_titan.envs.generic.core.network\_interface.NetworkInterface method}}

\begin{fulllineitems}
\phantomsection\label{\detokenize{source/yawning_titan.envs.generic.core:yawning_titan.envs.generic.core.network_interface.NetworkInterface.get_detected_attacks}}\pysiglinewithargsret{\sphinxbfcode{\sphinxupquote{get\_detected\_attacks}}}{}{{ $\rightarrow$ List\DUrole{p}{{[}}str\DUrole{p}{{]}}}}\begin{quote}\begin{description}
\item[{Returns}] \leavevmode
\sphinxAtStartPar
\begin{description}
\item[{A list of lists that contains the attacks that the blue agent detected ({[}{[}“3”, “4”{]}{]} \sphinxhyphen{}\textgreater{} blue detected}] \leavevmode
\sphinxAtStartPar
an attack from node “3” onto node “4”)

\end{description}


\end{description}\end{quote}

\end{fulllineitems}

\index{get\_edge\_map() (yawning\_titan.envs.generic.core.network\_interface.NetworkInterface method)@\spxentry{get\_edge\_map()}\spxextra{yawning\_titan.envs.generic.core.network\_interface.NetworkInterface method}}

\begin{fulllineitems}
\phantomsection\label{\detokenize{source/yawning_titan.envs.generic.core:yawning_titan.envs.generic.core.network_interface.NetworkInterface.get_edge_map}}\pysiglinewithargsret{\sphinxbfcode{\sphinxupquote{get\_edge\_map}}}{}{{ $\rightarrow$ dict}}\begin{quote}\begin{description}
\item[{Returns}] \leavevmode
\sphinxAtStartPar
Returns the edge map that maps the edges of the network to numbers

\end{description}\end{quote}

\end{fulllineitems}

\index{get\_entry\_nodes() (yawning\_titan.envs.generic.core.network\_interface.NetworkInterface method)@\spxentry{get\_entry\_nodes()}\spxextra{yawning\_titan.envs.generic.core.network\_interface.NetworkInterface method}}

\begin{fulllineitems}
\phantomsection\label{\detokenize{source/yawning_titan.envs.generic.core:yawning_titan.envs.generic.core.network_interface.NetworkInterface.get_entry_nodes}}\pysiglinewithargsret{\sphinxbfcode{\sphinxupquote{get\_entry\_nodes}}}{}{{ $\rightarrow$ List\DUrole{p}{{[}}str\DUrole{p}{{]}}}}\begin{quote}\begin{description}
\item[{Returns}] \leavevmode
\sphinxAtStartPar
The entry nodes for the network

\end{description}\end{quote}

\end{fulllineitems}

\index{get\_high\_value\_target() (yawning\_titan.envs.generic.core.network\_interface.NetworkInterface method)@\spxentry{get\_high\_value\_target()}\spxextra{yawning\_titan.envs.generic.core.network\_interface.NetworkInterface method}}

\begin{fulllineitems}
\phantomsection\label{\detokenize{source/yawning_titan.envs.generic.core:yawning_titan.envs.generic.core.network_interface.NetworkInterface.get_high_value_node}}\pysiglinewithargsret{\sphinxbfcode{\sphinxupquote{get\_high\_value\_target}}}{}{}\begin{quote}\begin{description}
\item[{Returns}] \leavevmode
\sphinxAtStartPar
The current high value node for the network

\end{description}\end{quote}

\end{fulllineitems}

\index{get\_midpoint() (yawning\_titan.envs.generic.core.network\_interface.NetworkInterface method)@\spxentry{get\_midpoint()}\spxextra{yawning\_titan.envs.generic.core.network\_interface.NetworkInterface method}}

\begin{fulllineitems}
\phantomsection\label{\detokenize{source/yawning_titan.envs.generic.core:yawning_titan.envs.generic.core.network_interface.NetworkInterface.get_midpoint}}\pysiglinewithargsret{\sphinxbfcode{\sphinxupquote{get\_midpoint}}}{\emph{\DUrole{n}{node1}\DUrole{p}{:}\DUrole{w}{  }\DUrole{n}{str}}, \emph{\DUrole{n}{node2}\DUrole{p}{:}\DUrole{w}{  }\DUrole{n}{str}}}{{ $\rightarrow$ Tuple\DUrole{p}{{[}}int\DUrole{p}{,}\DUrole{w}{  }int\DUrole{p}{{]}}}}
\sphinxAtStartPar
Gets the midpoint between the position of two nodes
\begin{quote}\begin{description}
\item[{Parameters}] \leavevmode\begin{itemize}
\item {}
\sphinxAtStartPar
\sphinxstyleliteralstrong{\sphinxupquote{node1}} \textendash{} the name of the first node to get the midpoint from

\item {}
\sphinxAtStartPar
\sphinxstyleliteralstrong{\sphinxupquote{node2}} \textendash{} the name of the second node to get the midpoint from

\end{itemize}

\item[{Returns}] \leavevmode
\sphinxAtStartPar
The x and y coordinates of the midpoint between two nodes

\end{description}\end{quote}

\end{fulllineitems}

\index{get\_nodes() (yawning\_titan.envs.generic.core.network\_interface.NetworkInterface method)@\spxentry{get\_nodes()}\spxextra{yawning\_titan.envs.generic.core.network\_interface.NetworkInterface method}}

\begin{fulllineitems}
\phantomsection\label{\detokenize{source/yawning_titan.envs.generic.core:yawning_titan.envs.generic.core.network_interface.NetworkInterface.get_nodes}}\pysiglinewithargsret{\sphinxbfcode{\sphinxupquote{get\_nodes}}}{\emph{\DUrole{n}{filter\_true\_compromised}\DUrole{p}{:}\DUrole{w}{  }\DUrole{n}{bool}\DUrole{w}{  }\DUrole{o}{=}\DUrole{w}{  }\DUrole{default_value}{False}}, \emph{\DUrole{n}{filter\_blue\_view\_compromised}\DUrole{p}{:}\DUrole{w}{  }\DUrole{n}{bool}\DUrole{w}{  }\DUrole{o}{=}\DUrole{w}{  }\DUrole{default_value}{False}}, \emph{\DUrole{n}{filter\_true\_safe}\DUrole{p}{:}\DUrole{w}{  }\DUrole{n}{bool}\DUrole{w}{  }\DUrole{o}{=}\DUrole{w}{  }\DUrole{default_value}{False}}, \emph{\DUrole{n}{filter\_blue\_view\_safe}\DUrole{p}{:}\DUrole{w}{  }\DUrole{n}{bool}\DUrole{w}{  }\DUrole{o}{=}\DUrole{w}{  }\DUrole{default_value}{False}}, \emph{\DUrole{n}{filter\_isolated}\DUrole{p}{:}\DUrole{w}{  }\DUrole{n}{bool}\DUrole{w}{  }\DUrole{o}{=}\DUrole{w}{  }\DUrole{default_value}{False}}, \emph{\DUrole{n}{filter\_non\_isolated}\DUrole{p}{:}\DUrole{w}{  }\DUrole{n}{bool}\DUrole{w}{  }\DUrole{o}{=}\DUrole{w}{  }\DUrole{default_value}{False}}, \emph{\DUrole{n}{filter\_deceptive}\DUrole{p}{:}\DUrole{w}{  }\DUrole{n}{bool}\DUrole{w}{  }\DUrole{o}{=}\DUrole{w}{  }\DUrole{default_value}{False}}, \emph{\DUrole{n}{filter\_non\_deceptive}\DUrole{p}{:}\DUrole{w}{  }\DUrole{n}{bool}\DUrole{w}{  }\DUrole{o}{=}\DUrole{w}{  }\DUrole{default_value}{False}}}{{ $\rightarrow$ List\DUrole{p}{{[}}str\DUrole{p}{{]}}}}
\sphinxAtStartPar
Gets all of the nodes from the network and applies a filter to them to extract a specific subset of the nodes.
\begin{quote}\begin{description}
\item[{Parameters}] \leavevmode\begin{itemize}
\item {}
\sphinxAtStartPar
\sphinxstyleliteralstrong{\sphinxupquote{filter\_true\_compromised}} \textendash{} Filter so only nodes that are compromised remain

\item {}
\sphinxAtStartPar
\sphinxstyleliteralstrong{\sphinxupquote{filter\_blue\_view\_compromised}} \textendash{} Filter so only nodes that blue can see are compromised remain

\item {}
\sphinxAtStartPar
\sphinxstyleliteralstrong{\sphinxupquote{filter\_true\_safe}} \textendash{} Filter so only nodes that are safe remain

\item {}
\sphinxAtStartPar
\sphinxstyleliteralstrong{\sphinxupquote{filter\_blue\_view\_safe}} \textendash{} Filter so only nodes that blue can see are safe remain

\item {}
\sphinxAtStartPar
\sphinxstyleliteralstrong{\sphinxupquote{filter\_isolated}} \textendash{} Filter so only isolated nodes remain

\item {}
\sphinxAtStartPar
\sphinxstyleliteralstrong{\sphinxupquote{filter\_non\_isolated}} \textendash{} Filter so only connected nodes remain

\item {}
\sphinxAtStartPar
\sphinxstyleliteralstrong{\sphinxupquote{filter\_deceptive}} \textendash{} Filter so only deceptive nodes remain

\item {}
\sphinxAtStartPar
\sphinxstyleliteralstrong{\sphinxupquote{filter\_non\_deceptive}} \textendash{} Filter so only non\sphinxhyphen{}deceptive nodes remain

\end{itemize}

\item[{Returns}] \leavevmode
\sphinxAtStartPar
A list of nodes

\end{description}\end{quote}

\end{fulllineitems}

\index{get\_number\_base\_edges() (yawning\_titan.envs.generic.core.network\_interface.NetworkInterface method)@\spxentry{get\_number\_base\_edges()}\spxextra{yawning\_titan.envs.generic.core.network\_interface.NetworkInterface method}}

\begin{fulllineitems}
\phantomsection\label{\detokenize{source/yawning_titan.envs.generic.core:yawning_titan.envs.generic.core.network_interface.NetworkInterface.get_number_base_edges}}\pysiglinewithargsret{\sphinxbfcode{\sphinxupquote{get\_number\_base\_edges}}}{}{{ $\rightarrow$ int}}
\sphinxAtStartPar
Gets the number of edges in the base original graph (not affected by nodes being isolated or modified by the
agents)
\begin{quote}\begin{description}
\item[{Returns}] \leavevmode
\sphinxAtStartPar
The number of base edges in the network

\end{description}\end{quote}

\end{fulllineitems}

\index{get\_number\_current\_edges() (yawning\_titan.envs.generic.core.network\_interface.NetworkInterface method)@\spxentry{get\_number\_current\_edges()}\spxextra{yawning\_titan.envs.generic.core.network\_interface.NetworkInterface method}}

\begin{fulllineitems}
\phantomsection\label{\detokenize{source/yawning_titan.envs.generic.core:yawning_titan.envs.generic.core.network_interface.NetworkInterface.get_number_current_edges}}\pysiglinewithargsret{\sphinxbfcode{\sphinxupquote{get\_number\_current\_edges}}}{}{{ $\rightarrow$ int}}
\sphinxAtStartPar
Gets the number of edges in the current original graph (not affected by nodes being isolated or modified by the
agents)
\begin{quote}\begin{description}
\item[{Returns}] \leavevmode
\sphinxAtStartPar
The number of base edges in the network

\end{description}\end{quote}

\end{fulllineitems}

\index{get\_number\_of\_nodes() (yawning\_titan.envs.generic.core.network\_interface.NetworkInterface method)@\spxentry{get\_number\_of\_nodes()}\spxextra{yawning\_titan.envs.generic.core.network\_interface.NetworkInterface method}}

\begin{fulllineitems}
\phantomsection\label{\detokenize{source/yawning_titan.envs.generic.core:yawning_titan.envs.generic.core.network_interface.NetworkInterface.get_number_of_nodes}}\pysiglinewithargsret{\sphinxbfcode{\sphinxupquote{get\_number\_of\_nodes}}}{}{{ $\rightarrow$ int}}\begin{quote}\begin{description}
\item[{Returns}] \leavevmode
\sphinxAtStartPar
The number of nodes in the network

\end{description}\end{quote}

\end{fulllineitems}

\index{get\_observation\_size() (yawning\_titan.envs.generic.core.network\_interface.NetworkInterface method)@\spxentry{get\_observation\_size()}\spxextra{yawning\_titan.envs.generic.core.network\_interface.NetworkInterface method}}

\begin{fulllineitems}
\phantomsection\label{\detokenize{source/yawning_titan.envs.generic.core:yawning_titan.envs.generic.core.network_interface.NetworkInterface.get_observation_size}}\pysiglinewithargsret{\sphinxbfcode{\sphinxupquote{get\_observation\_size}}}{}{{ $\rightarrow$ int}}
\sphinxAtStartPar
Gets the size of the observation space. This is based on the settings that are turned on/off.
\begin{quote}\begin{description}
\item[{Returns}] \leavevmode
\sphinxAtStartPar
The size of the observation space

\end{description}\end{quote}

\end{fulllineitems}

\index{get\_red\_location() (yawning\_titan.envs.generic.core.network\_interface.NetworkInterface method)@\spxentry{get\_red\_location()}\spxextra{yawning\_titan.envs.generic.core.network\_interface.NetworkInterface method}}

\begin{fulllineitems}
\phantomsection\label{\detokenize{source/yawning_titan.envs.generic.core:yawning_titan.envs.generic.core.network_interface.NetworkInterface.get_red_location}}\pysiglinewithargsret{\sphinxbfcode{\sphinxupquote{get\_red\_location}}}{}{{ $\rightarrow$ str}}\begin{quote}\begin{description}
\item[{Returns}] \leavevmode
\sphinxAtStartPar
The current location of the red agent

\end{description}\end{quote}

\end{fulllineitems}

\index{get\_single\_node\_blue\_view() (yawning\_titan.envs.generic.core.network\_interface.NetworkInterface method)@\spxentry{get\_single\_node\_blue\_view()}\spxextra{yawning\_titan.envs.generic.core.network\_interface.NetworkInterface method}}

\begin{fulllineitems}
\phantomsection\label{\detokenize{source/yawning_titan.envs.generic.core:yawning_titan.envs.generic.core.network_interface.NetworkInterface.get_single_node_blue_view}}\pysiglinewithargsret{\sphinxbfcode{\sphinxupquote{get\_single\_node\_blue\_view}}}{\emph{\DUrole{n}{node}\DUrole{p}{:}\DUrole{w}{  }\DUrole{n}{str}}}{{ $\rightarrow$ int}}
\sphinxAtStartPar
Gets the current state of a node (safe or compromised)
\begin{quote}\begin{description}
\item[{Parameters}] \leavevmode
\sphinxAtStartPar
\sphinxstyleliteralstrong{\sphinxupquote{node}} \textendash{} The name of the node to check the compromised status of

\item[{Returns}] \leavevmode
\sphinxAtStartPar
safe, 1: compromised

\item[{Return type}] \leavevmode
\sphinxAtStartPar
0

\end{description}\end{quote}

\end{fulllineitems}

\index{get\_single\_node\_isolation\_status() (yawning\_titan.envs.generic.core.network\_interface.NetworkInterface method)@\spxentry{get\_single\_node\_isolation\_status()}\spxextra{yawning\_titan.envs.generic.core.network\_interface.NetworkInterface method}}

\begin{fulllineitems}
\phantomsection\label{\detokenize{source/yawning_titan.envs.generic.core:yawning_titan.envs.generic.core.network_interface.NetworkInterface.get_single_node_isolation_status}}\pysiglinewithargsret{\sphinxbfcode{\sphinxupquote{get\_single\_node\_isolation\_status}}}{\emph{\DUrole{n}{node}\DUrole{p}{:}\DUrole{w}{  }\DUrole{n}{str}}}{{ $\rightarrow$ bool}}
\sphinxAtStartPar
Gets the isolation status for a single node
:param node: The name of the node to get the isolation status for
\begin{quote}\begin{description}
\item[{Returns}] \leavevmode
\sphinxAtStartPar
Boolean representing the isolation status of the node

\end{description}\end{quote}

\end{fulllineitems}

\index{get\_single\_node\_known\_intrusion\_status() (yawning\_titan.envs.generic.core.network\_interface.NetworkInterface method)@\spxentry{get\_single\_node\_known\_intrusion\_status()}\spxextra{yawning\_titan.envs.generic.core.network\_interface.NetworkInterface method}}

\begin{fulllineitems}
\phantomsection\label{\detokenize{source/yawning_titan.envs.generic.core:yawning_titan.envs.generic.core.network_interface.NetworkInterface.get_single_node_known_intrusion_status}}\pysiglinewithargsret{\sphinxbfcode{\sphinxupquote{get\_single\_node\_known\_intrusion\_status}}}{\emph{\DUrole{n}{node}\DUrole{p}{:}\DUrole{w}{  }\DUrole{n}{str}}}{{ $\rightarrow$ bool}}
\sphinxAtStartPar
Returns True if blue knows about the intrusion in this node, False if not
\begin{quote}\begin{description}
\item[{Parameters}] \leavevmode
\sphinxAtStartPar
\sphinxstyleliteralstrong{\sphinxupquote{node}} \textendash{} The name of the node to check the status of

\item[{Returns}] \leavevmode
\sphinxAtStartPar
True if blue knows about the intrusion in this node, False if not

\end{description}\end{quote}

\end{fulllineitems}

\index{get\_single\_node\_position() (yawning\_titan.envs.generic.core.network\_interface.NetworkInterface method)@\spxentry{get\_single\_node\_position()}\spxextra{yawning\_titan.envs.generic.core.network\_interface.NetworkInterface method}}

\begin{fulllineitems}
\phantomsection\label{\detokenize{source/yawning_titan.envs.generic.core:yawning_titan.envs.generic.core.network_interface.NetworkInterface.get_single_node_position}}\pysiglinewithargsret{\sphinxbfcode{\sphinxupquote{get\_single\_node\_position}}}{\emph{\DUrole{n}{node}\DUrole{p}{:}\DUrole{w}{  }\DUrole{n}{str}}}{{ $\rightarrow$ Tuple\DUrole{p}{{[}}int\DUrole{p}{,}\DUrole{w}{  }int\DUrole{p}{{]}}}}
\sphinxAtStartPar
Gets the position of a single node
:param node: The name of the node to get the position of
\begin{quote}\begin{description}
\item[{Returns}] \leavevmode
\sphinxAtStartPar
A list containing an x coord and a y coord

\end{description}\end{quote}

\end{fulllineitems}

\index{get\_single\_node\_state() (yawning\_titan.envs.generic.core.network\_interface.NetworkInterface method)@\spxentry{get\_single\_node\_state()}\spxextra{yawning\_titan.envs.generic.core.network\_interface.NetworkInterface method}}

\begin{fulllineitems}
\phantomsection\label{\detokenize{source/yawning_titan.envs.generic.core:yawning_titan.envs.generic.core.network_interface.NetworkInterface.get_single_node_state}}\pysiglinewithargsret{\sphinxbfcode{\sphinxupquote{get\_single\_node\_state}}}{\emph{\DUrole{n}{node}\DUrole{p}{:}\DUrole{w}{  }\DUrole{n}{str}}}{{ $\rightarrow$ int}}
\sphinxAtStartPar
Gets the current state of a node (safe or compromised)
\begin{quote}\begin{description}
\item[{Parameters}] \leavevmode
\sphinxAtStartPar
\sphinxstyleliteralstrong{\sphinxupquote{node}} \textendash{} The name of the node to check the compromised status of

\item[{Returns}] \leavevmode
\sphinxAtStartPar
safe, 1: compromised

\item[{Return type}] \leavevmode
\sphinxAtStartPar
0

\end{description}\end{quote}

\end{fulllineitems}

\index{get\_single\_node\_vulnerability() (yawning\_titan.envs.generic.core.network\_interface.NetworkInterface method)@\spxentry{get\_single\_node\_vulnerability()}\spxextra{yawning\_titan.envs.generic.core.network\_interface.NetworkInterface method}}

\begin{fulllineitems}
\phantomsection\label{\detokenize{source/yawning_titan.envs.generic.core:yawning_titan.envs.generic.core.network_interface.NetworkInterface.get_single_node_vulnerability}}\pysiglinewithargsret{\sphinxbfcode{\sphinxupquote{get\_single\_node\_vulnerability}}}{\emph{\DUrole{n}{node}\DUrole{p}{:}\DUrole{w}{  }\DUrole{n}{str}}}{{ $\rightarrow$ int}}
\sphinxAtStartPar
Gets the current vulnerability of a node
\begin{quote}\begin{description}
\item[{Parameters}] \leavevmode
\sphinxAtStartPar
\sphinxstyleliteralstrong{\sphinxupquote{node}} \textendash{} The name of the node to check the vulnerability of

\item[{Returns}] \leavevmode
\sphinxAtStartPar
safe, 1: compromised

\item[{Return type}] \leavevmode
\sphinxAtStartPar
0

\end{description}\end{quote}

\end{fulllineitems}

\index{get\_true\_attacks() (yawning\_titan.envs.generic.core.network\_interface.NetworkInterface method)@\spxentry{get\_true\_attacks()}\spxextra{yawning\_titan.envs.generic.core.network\_interface.NetworkInterface method}}

\begin{fulllineitems}
\phantomsection\label{\detokenize{source/yawning_titan.envs.generic.core:yawning_titan.envs.generic.core.network_interface.NetworkInterface.get_true_attacks}}\pysiglinewithargsret{\sphinxbfcode{\sphinxupquote{get\_true\_attacks}}}{}{{ $\rightarrow$ List\DUrole{p}{{[}}str\DUrole{p}{{]}}}}\begin{quote}\begin{description}
\item[{Returns}] \leavevmode
\sphinxAtStartPar
\begin{description}
\item[{A list of lists that contains the attacks that took place that round ({[}{[}“3”, “4”{]}{]} \sphinxhyphen{}\textgreater{} the red agent}] \leavevmode
\sphinxAtStartPar
attacked node “4” from node “3”)

\end{description}


\end{description}\end{quote}

\end{fulllineitems}

\index{immediate\_attempt\_view\_update() (yawning\_titan.envs.generic.core.network\_interface.NetworkInterface method)@\spxentry{immediate\_attempt\_view\_update()}\spxextra{yawning\_titan.envs.generic.core.network\_interface.NetworkInterface method}}

\begin{fulllineitems}
\phantomsection\label{\detokenize{source/yawning_titan.envs.generic.core:yawning_titan.envs.generic.core.network_interface.NetworkInterface.immediate_attempt_view_update}}\pysiglinewithargsret{\sphinxbfcode{\sphinxupquote{immediate\_attempt\_view\_update}}}{\emph{\DUrole{n}{node}\DUrole{p}{:}\DUrole{w}{  }\DUrole{n}{str}}}{}
\sphinxAtStartPar
Attempts to update the view of a specific node for the blue agent. There is a chance that intrusions will not
be detected
\begin{quote}\begin{description}
\item[{Parameters}] \leavevmode
\sphinxAtStartPar
\sphinxstyleliteralstrong{\sphinxupquote{node}} \textendash{} the node to try and update the view for

\end{description}\end{quote}

\end{fulllineitems}

\index{immediate\_attempt\_view\_update\_with\_specified\_chance() (yawning\_titan.envs.generic.core.network\_interface.NetworkInterface method)@\spxentry{immediate\_attempt\_view\_update\_with\_specified\_chance()}\spxextra{yawning\_titan.envs.generic.core.network\_interface.NetworkInterface method}}

\begin{fulllineitems}
\phantomsection\label{\detokenize{source/yawning_titan.envs.generic.core:yawning_titan.envs.generic.core.network_interface.NetworkInterface.immediate_attempt_view_update_with_specified_chance}}\pysiglinewithargsret{\sphinxbfcode{\sphinxupquote{immediate\_attempt\_view\_update\_with\_specified\_chance}}}{\emph{\DUrole{n}{node}\DUrole{p}{:}\DUrole{w}{  }\DUrole{n}{str}}, \emph{\DUrole{n}{chance}\DUrole{p}{:}\DUrole{w}{  }\DUrole{n}{float}}}{}
\sphinxAtStartPar
Update the blue view of a node but with a specified chance and not the chance used in the settings file
\begin{quote}\begin{description}
\item[{Parameters}] \leavevmode\begin{itemize}
\item {}
\sphinxAtStartPar
\sphinxstyleliteralstrong{\sphinxupquote{node}} \textendash{} The node to attempt to update

\item {}
\sphinxAtStartPar
\sphinxstyleliteralstrong{\sphinxupquote{chance}} \textendash{} The chance blue has of updating the node and detecting a red intrusion

\end{itemize}

\end{description}\end{quote}

\end{fulllineitems}

\index{isolate\_node() (yawning\_titan.envs.generic.core.network\_interface.NetworkInterface method)@\spxentry{isolate\_node()}\spxextra{yawning\_titan.envs.generic.core.network\_interface.NetworkInterface method}}

\begin{fulllineitems}
\phantomsection\label{\detokenize{source/yawning_titan.envs.generic.core:yawning_titan.envs.generic.core.network_interface.NetworkInterface.isolate_node}}\pysiglinewithargsret{\sphinxbfcode{\sphinxupquote{isolate\_node}}}{\emph{\DUrole{n}{node}\DUrole{p}{:}\DUrole{w}{  }\DUrole{n}{str}}}{}
\sphinxAtStartPar
Isolate a node (disable all of the nodes connections)
\begin{quote}\begin{description}
\item[{Parameters}] \leavevmode
\sphinxAtStartPar
\sphinxstyleliteralstrong{\sphinxupquote{node}} \textendash{} the node to disable the connections of

\end{description}\end{quote}

\end{fulllineitems}

\index{make\_node\_safe() (yawning\_titan.envs.generic.core.network\_interface.NetworkInterface method)@\spxentry{make\_node\_safe()}\spxextra{yawning\_titan.envs.generic.core.network\_interface.NetworkInterface method}}

\begin{fulllineitems}
\phantomsection\label{\detokenize{source/yawning_titan.envs.generic.core:yawning_titan.envs.generic.core.network_interface.NetworkInterface.make_node_safe}}\pysiglinewithargsret{\sphinxbfcode{\sphinxupquote{make\_node\_safe}}}{\emph{\DUrole{n}{node}\DUrole{p}{:}\DUrole{w}{  }\DUrole{n}{str}}}{}
\sphinxAtStartPar
Makes the state for a given node safe
\begin{quote}\begin{description}
\item[{Parameters}] \leavevmode
\sphinxAtStartPar
\sphinxstyleliteralstrong{\sphinxupquote{node}} \textendash{} the node to make safe

\end{description}\end{quote}

\end{fulllineitems}

\index{push\_red() (yawning\_titan.envs.generic.core.network\_interface.NetworkInterface method)@\spxentry{push\_red()}\spxextra{yawning\_titan.envs.generic.core.network\_interface.NetworkInterface method}}

\begin{fulllineitems}
\phantomsection\label{\detokenize{source/yawning_titan.envs.generic.core:yawning_titan.envs.generic.core.network_interface.NetworkInterface.push_red}}\pysiglinewithargsret{\sphinxbfcode{\sphinxupquote{push\_red}}}{}{}
\sphinxAtStartPar
If the blue agent patches the node that the red agent is in the red agent will be pushed to a connected
compromised node. If there are none then the red agent will be pushed out of the network

\end{fulllineitems}

\index{reconnect\_node() (yawning\_titan.envs.generic.core.network\_interface.NetworkInterface method)@\spxentry{reconnect\_node()}\spxextra{yawning\_titan.envs.generic.core.network\_interface.NetworkInterface method}}

\begin{fulllineitems}
\phantomsection\label{\detokenize{source/yawning_titan.envs.generic.core:yawning_titan.envs.generic.core.network_interface.NetworkInterface.reconnect_node}}\pysiglinewithargsret{\sphinxbfcode{\sphinxupquote{reconnect\_node}}}{\emph{\DUrole{n}{node}\DUrole{p}{:}\DUrole{w}{  }\DUrole{n}{str}}}{}
\sphinxAtStartPar
Ren\sphinxhyphen{}enable any connections that may have previously been disabled
\begin{quote}\begin{description}
\item[{Parameters}] \leavevmode
\sphinxAtStartPar
\sphinxstyleliteralstrong{\sphinxupquote{node}} \textendash{} the node to re\sphinxhyphen{}enable

\end{description}\end{quote}

\end{fulllineitems}

\index{reset() (yawning\_titan.envs.generic.core.network\_interface.NetworkInterface method)@\spxentry{reset()}\spxextra{yawning\_titan.envs.generic.core.network\_interface.NetworkInterface method}}

\begin{fulllineitems}
\phantomsection\label{\detokenize{source/yawning_titan.envs.generic.core:yawning_titan.envs.generic.core.network_interface.NetworkInterface.reset}}\pysiglinewithargsret{\sphinxbfcode{\sphinxupquote{reset}}}{}{}
\sphinxAtStartPar
Resets the network back to its default state

\end{fulllineitems}

\index{reset\_single\_node\_vulnerability() (yawning\_titan.envs.generic.core.network\_interface.NetworkInterface method)@\spxentry{reset\_single\_node\_vulnerability()}\spxextra{yawning\_titan.envs.generic.core.network\_interface.NetworkInterface method}}

\begin{fulllineitems}
\phantomsection\label{\detokenize{source/yawning_titan.envs.generic.core:yawning_titan.envs.generic.core.network_interface.NetworkInterface.reset_single_node_vulnerability}}\pysiglinewithargsret{\sphinxbfcode{\sphinxupquote{reset\_single\_node\_vulnerability}}}{\emph{\DUrole{n}{node}\DUrole{p}{:}\DUrole{w}{  }\DUrole{n}{str}}}{}
\sphinxAtStartPar
Resets a nodes vulnerability score back to the same value it started with
\begin{quote}\begin{description}
\item[{Parameters}] \leavevmode
\sphinxAtStartPar
\sphinxstyleliteralstrong{\sphinxupquote{node}} \textendash{} The name of the node to change the vulnerability of

\end{description}\end{quote}

\end{fulllineitems}

\index{reset\_stored\_attacks() (yawning\_titan.envs.generic.core.network\_interface.NetworkInterface method)@\spxentry{reset\_stored\_attacks()}\spxextra{yawning\_titan.envs.generic.core.network\_interface.NetworkInterface method}}

\begin{fulllineitems}
\phantomsection\label{\detokenize{source/yawning_titan.envs.generic.core:yawning_titan.envs.generic.core.network_interface.NetworkInterface.reset_stored_attacks}}\pysiglinewithargsret{\sphinxbfcode{\sphinxupquote{reset\_stored\_attacks}}}{}{}
\sphinxAtStartPar
Resets the attacks list. This needs to be called every timestep to ensure that only the current attacks are
contained

\end{fulllineitems}

\index{save\_json() (yawning\_titan.envs.generic.core.network\_interface.NetworkInterface method)@\spxentry{save\_json()}\spxextra{yawning\_titan.envs.generic.core.network\_interface.NetworkInterface method}}

\begin{fulllineitems}
\phantomsection\label{\detokenize{source/yawning_titan.envs.generic.core:yawning_titan.envs.generic.core.network_interface.NetworkInterface.save_json}}\pysiglinewithargsret{\sphinxbfcode{\sphinxupquote{save\_json}}}{\emph{\DUrole{n}{data\_dict}\DUrole{p}{:}\DUrole{w}{  }\DUrole{n}{dict}}, \emph{\DUrole{n}{ts}\DUrole{p}{:}\DUrole{w}{  }\DUrole{n}{int}}}{}
\sphinxAtStartPar
Saves a given dictionary to a json file
\begin{quote}\begin{description}
\item[{Parameters}] \leavevmode\begin{itemize}
\item {}
\sphinxAtStartPar
\sphinxstyleliteralstrong{\sphinxupquote{data\_dict}} \textendash{} Data to save to the json file

\item {}
\sphinxAtStartPar
\sphinxstyleliteralstrong{\sphinxupquote{ts}} \textendash{} The current timestamp of the data

\end{itemize}

\end{description}\end{quote}

\end{fulllineitems}

\index{scan\_node() (yawning\_titan.envs.generic.core.network\_interface.NetworkInterface method)@\spxentry{scan\_node()}\spxextra{yawning\_titan.envs.generic.core.network\_interface.NetworkInterface method}}

\begin{fulllineitems}
\phantomsection\label{\detokenize{source/yawning_titan.envs.generic.core:yawning_titan.envs.generic.core.network_interface.NetworkInterface.scan_node}}\pysiglinewithargsret{\sphinxbfcode{\sphinxupquote{scan\_node}}}{\emph{\DUrole{n}{node}\DUrole{p}{:}\DUrole{w}{  }\DUrole{n}{str}}}{}
\end{fulllineitems}

\index{update\_red\_location() (yawning\_titan.envs.generic.core.network\_interface.NetworkInterface method)@\spxentry{update\_red\_location()}\spxextra{yawning\_titan.envs.generic.core.network\_interface.NetworkInterface method}}

\begin{fulllineitems}
\phantomsection\label{\detokenize{source/yawning_titan.envs.generic.core:yawning_titan.envs.generic.core.network_interface.NetworkInterface.update_red_location}}\pysiglinewithargsret{\sphinxbfcode{\sphinxupquote{update\_red\_location}}}{\emph{\DUrole{n}{location}\DUrole{p}{:}\DUrole{w}{  }\DUrole{n}{str}}}{}
\sphinxAtStartPar
Modifies the value of the red location
\begin{quote}\begin{description}
\item[{Parameters}] \leavevmode
\sphinxAtStartPar
\sphinxstyleliteralstrong{\sphinxupquote{location}} \textendash{} The name of the node the red agent is now occupying

\end{description}\end{quote}

\end{fulllineitems}

\index{update\_single\_node\_blue\_view() (yawning\_titan.envs.generic.core.network\_interface.NetworkInterface method)@\spxentry{update\_single\_node\_blue\_view()}\spxextra{yawning\_titan.envs.generic.core.network\_interface.NetworkInterface method}}

\begin{fulllineitems}
\phantomsection\label{\detokenize{source/yawning_titan.envs.generic.core:yawning_titan.envs.generic.core.network_interface.NetworkInterface.update_single_node_blue_view}}\pysiglinewithargsret{\sphinxbfcode{\sphinxupquote{update\_single\_node\_blue\_view}}}{\emph{\DUrole{n}{node}\DUrole{p}{:}\DUrole{w}{  }\DUrole{n}{str}}, \emph{\DUrole{n}{status}\DUrole{p}{:}\DUrole{w}{  }\DUrole{n}{int}}}{}
\sphinxAtStartPar
Updates the blue’s view of the compromised status of a node
\begin{quote}\begin{description}
\item[{Parameters}] \leavevmode\begin{itemize}
\item {}
\sphinxAtStartPar
\sphinxstyleliteralstrong{\sphinxupquote{node}} \textendash{} The name of the node to update the status for

\item {}
\sphinxAtStartPar
\sphinxstyleliteralstrong{\sphinxupquote{status}} \textendash{} The new status of the node

\end{itemize}

\end{description}\end{quote}

\end{fulllineitems}

\index{update\_single\_node\_compromised\_status() (yawning\_titan.envs.generic.core.network\_interface.NetworkInterface method)@\spxentry{update\_single\_node\_compromised\_status()}\spxextra{yawning\_titan.envs.generic.core.network\_interface.NetworkInterface method}}

\begin{fulllineitems}
\phantomsection\label{\detokenize{source/yawning_titan.envs.generic.core:yawning_titan.envs.generic.core.network_interface.NetworkInterface.update_single_node_compromised_status}}\pysiglinewithargsret{\sphinxbfcode{\sphinxupquote{update\_single\_node\_compromised\_status}}}{\emph{\DUrole{n}{node}\DUrole{p}{:}\DUrole{w}{  }\DUrole{n}{str}}, \emph{\DUrole{n}{value}\DUrole{p}{:}\DUrole{w}{  }\DUrole{n}{int}}}{}
\sphinxAtStartPar
Modifies the value of the compromised status of a single node
\begin{quote}\begin{description}
\item[{Parameters}] \leavevmode\begin{itemize}
\item {}
\sphinxAtStartPar
\sphinxstyleliteralstrong{\sphinxupquote{node}} \textendash{} The name of the node to affect

\item {}
\sphinxAtStartPar
\sphinxstyleliteralstrong{\sphinxupquote{value}} \textendash{} The new value of the compromised status for the node

\end{itemize}

\end{description}\end{quote}

\end{fulllineitems}

\index{update\_single\_node\_known\_intrusions() (yawning\_titan.envs.generic.core.network\_interface.NetworkInterface method)@\spxentry{update\_single\_node\_known\_intrusions()}\spxextra{yawning\_titan.envs.generic.core.network\_interface.NetworkInterface method}}

\begin{fulllineitems}
\phantomsection\label{\detokenize{source/yawning_titan.envs.generic.core:yawning_titan.envs.generic.core.network_interface.NetworkInterface.update_single_node_known_intrusions}}\pysiglinewithargsret{\sphinxbfcode{\sphinxupquote{update\_single\_node\_known\_intrusions}}}{\emph{\DUrole{n}{node}\DUrole{p}{:}\DUrole{w}{  }\DUrole{n}{str}}, \emph{\DUrole{n}{value}\DUrole{p}{:}\DUrole{w}{  }\DUrole{n}{bool}}}{}
\sphinxAtStartPar
Modifies the value of the known intrusion status of a single node
\begin{quote}\begin{description}
\item[{Parameters}] \leavevmode\begin{itemize}
\item {}
\sphinxAtStartPar
\sphinxstyleliteralstrong{\sphinxupquote{node}} \textendash{} The name of the node to affect

\item {}
\sphinxAtStartPar
\sphinxstyleliteralstrong{\sphinxupquote{value}} \textendash{} The new value of the known intrusion status for the node

\end{itemize}

\end{description}\end{quote}

\end{fulllineitems}

\index{update\_single\_node\_vulnerability() (yawning\_titan.envs.generic.core.network\_interface.NetworkInterface method)@\spxentry{update\_single\_node\_vulnerability()}\spxextra{yawning\_titan.envs.generic.core.network\_interface.NetworkInterface method}}

\begin{fulllineitems}
\phantomsection\label{\detokenize{source/yawning_titan.envs.generic.core:yawning_titan.envs.generic.core.network_interface.NetworkInterface.update_single_node_vulnerability}}\pysiglinewithargsret{\sphinxbfcode{\sphinxupquote{update\_single\_node\_vulnerability}}}{\emph{\DUrole{n}{node}\DUrole{p}{:}\DUrole{w}{  }\DUrole{n}{str}}, \emph{\DUrole{n}{value}\DUrole{p}{:}\DUrole{w}{  }\DUrole{n}{float}}}{}
\sphinxAtStartPar
Modifies the value of the vulnerability status of a single node
\begin{quote}\begin{description}
\item[{Parameters}] \leavevmode\begin{itemize}
\item {}
\sphinxAtStartPar
\sphinxstyleliteralstrong{\sphinxupquote{node}} \textendash{} The name of the node to affect

\item {}
\sphinxAtStartPar
\sphinxstyleliteralstrong{\sphinxupquote{value}} \textendash{} The new value of the vulnerability for the node

\end{itemize}

\end{description}\end{quote}

\end{fulllineitems}

\index{update\_stored\_attacks() (yawning\_titan.envs.generic.core.network\_interface.NetworkInterface method)@\spxentry{update\_stored\_attacks()}\spxextra{yawning\_titan.envs.generic.core.network\_interface.NetworkInterface method}}

\begin{fulllineitems}
\phantomsection\label{\detokenize{source/yawning_titan.envs.generic.core:yawning_titan.envs.generic.core.network_interface.NetworkInterface.update_stored_attacks}}\pysiglinewithargsret{\sphinxbfcode{\sphinxupquote{update\_stored\_attacks}}}{\emph{\DUrole{n}{attacking\_nodes}\DUrole{p}{:}\DUrole{w}{  }\DUrole{n}{List\DUrole{p}{{[}}str\DUrole{p}{{]}}}}, \emph{\DUrole{n}{target\_nodes}\DUrole{p}{:}\DUrole{w}{  }\DUrole{n}{List\DUrole{p}{{[}}str\DUrole{p}{{]}}}}, \emph{\DUrole{n}{success}\DUrole{p}{:}\DUrole{w}{  }\DUrole{n}{List\DUrole{p}{{[}}bool\DUrole{p}{{]}}}}}{}
\sphinxAtStartPar
Updates this turns current attacks. This function collects all of the failed attacks and stores them for the
blue agent to use in their action decision
\begin{quote}\begin{description}
\item[{Parameters}] \leavevmode\begin{itemize}
\item {}
\sphinxAtStartPar
\sphinxstyleliteralstrong{\sphinxupquote{attacking\_nodes}} \textendash{} Nodes red has attacked from

\item {}
\sphinxAtStartPar
\sphinxstyleliteralstrong{\sphinxupquote{target\_nodes}} \textendash{} Nodes red is attacking

\item {}
\sphinxAtStartPar
\sphinxstyleliteralstrong{\sphinxupquote{success}} \textendash{} If the attacks were a success or not

\end{itemize}

\end{description}\end{quote}

\end{fulllineitems}


\end{fulllineitems}

\index{choice() (in module yawning\_titan.envs.generic.core.network\_interface)@\spxentry{choice()}\spxextra{in module yawning\_titan.envs.generic.core.network\_interface}}

\begin{fulllineitems}
\phantomsection\label{\detokenize{source/yawning_titan.envs.generic.core:yawning_titan.envs.generic.core.network_interface.choice}}\pysiglinewithargsret{\sphinxcode{\sphinxupquote{yawning\_titan.envs.generic.core.network\_interface.}}\sphinxbfcode{\sphinxupquote{choice}}}{\emph{\DUrole{n}{a}}, \emph{\DUrole{n}{size}\DUrole{o}{=}\DUrole{default_value}{None}}, \emph{\DUrole{n}{replace}\DUrole{o}{=}\DUrole{default_value}{True}}, \emph{\DUrole{n}{p}\DUrole{o}{=}\DUrole{default_value}{None}}}{}
\sphinxAtStartPar
Generates a random sample from a given 1\sphinxhyphen{}D array

\sphinxAtStartPar
\DUrole{versionmodified,added}{New in version 1.7.0.}

\begin{sphinxadmonition}{note}{Note:}
\sphinxAtStartPar
New code should use the \sphinxcode{\sphinxupquote{choice}} method of a \sphinxcode{\sphinxupquote{default\_rng()}}
instance instead; please see the \DUrole{xref,std,std-ref}{random\sphinxhyphen{}quick\sphinxhyphen{}start}.
\end{sphinxadmonition}
\begin{quote}\begin{description}
\item[{Parameters}] \leavevmode\begin{itemize}
\item {}
\sphinxAtStartPar
\sphinxstyleliteralstrong{\sphinxupquote{a}} (\sphinxstyleliteralemphasis{\sphinxupquote{1\sphinxhyphen{}D array\sphinxhyphen{}like}}\sphinxstyleliteralemphasis{\sphinxupquote{ or }}\sphinxstyleliteralemphasis{\sphinxupquote{int}}) \textendash{} If an ndarray, a random sample is generated from its elements.
If an int, the random sample is generated as if a were np.arange(a)

\item {}
\sphinxAtStartPar
\sphinxstyleliteralstrong{\sphinxupquote{size}} (\sphinxstyleliteralemphasis{\sphinxupquote{int}}\sphinxstyleliteralemphasis{\sphinxupquote{ or }}\sphinxstyleliteralemphasis{\sphinxupquote{tuple of ints}}\sphinxstyleliteralemphasis{\sphinxupquote{, }}\sphinxstyleliteralemphasis{\sphinxupquote{optional}}) \textendash{} Output shape.  If the given shape is, e.g., \sphinxcode{\sphinxupquote{(m, n, k)}}, then
\sphinxcode{\sphinxupquote{m * n * k}} samples are drawn.  Default is None, in which case a
single value is returned.

\item {}
\sphinxAtStartPar
\sphinxstyleliteralstrong{\sphinxupquote{replace}} (\sphinxstyleliteralemphasis{\sphinxupquote{boolean}}\sphinxstyleliteralemphasis{\sphinxupquote{, }}\sphinxstyleliteralemphasis{\sphinxupquote{optional}}) \textendash{} Whether the sample is with or without replacement

\item {}
\sphinxAtStartPar
\sphinxstyleliteralstrong{\sphinxupquote{p}} (\sphinxstyleliteralemphasis{\sphinxupquote{1\sphinxhyphen{}D array\sphinxhyphen{}like}}\sphinxstyleliteralemphasis{\sphinxupquote{, }}\sphinxstyleliteralemphasis{\sphinxupquote{optional}}) \textendash{} The probabilities associated with each entry in a.
If not given the sample assumes a uniform distribution over all
entries in a.

\end{itemize}

\item[{Returns}] \leavevmode
\sphinxAtStartPar
\sphinxstylestrong{samples} \textendash{} The generated random samples

\item[{Return type}] \leavevmode
\sphinxAtStartPar
single item or ndarray

\item[{Raises}] \leavevmode
\sphinxAtStartPar
\sphinxstyleliteralstrong{\sphinxupquote{ValueError}} \textendash{} If a is an int and less than zero, if a or p are not 1\sphinxhyphen{}dimensional,
    if a is an array\sphinxhyphen{}like of size 0, if p is not a vector of
    probabilities, if a and p have different lengths, or if
    replace=False and the sample size is greater than the population
    size

\end{description}\end{quote}


\sphinxstrong{See also:}
\nopagebreak


\sphinxAtStartPar
\sphinxcode{\sphinxupquote{randint}}, \sphinxcode{\sphinxupquote{shuffle}}, \sphinxcode{\sphinxupquote{permutation}}
\begin{description}
\item[{\sphinxcode{\sphinxupquote{Generator.choice}}}] \leavevmode
\sphinxAtStartPar
which should be used in new code

\end{description}


\subsubsection*{Notes}

\sphinxAtStartPar
Sampling random rows from a 2\sphinxhyphen{}D array is not possible with this function,
but is possible with \sphinxtitleref{Generator.choice} through its \sphinxcode{\sphinxupquote{axis}} keyword.
\subsubsection*{Examples}

\sphinxAtStartPar
Generate a uniform random sample from np.arange(5) of size 3:

\begin{sphinxVerbatim}[commandchars=\\\{\}]
\PYG{g+gp}{\PYGZgt{}\PYGZgt{}\PYGZgt{} }\PYG{n}{np}\PYG{o}{.}\PYG{n}{random}\PYG{o}{.}\PYG{n}{choice}\PYG{p}{(}\PYG{l+m+mi}{5}\PYG{p}{,} \PYG{l+m+mi}{3}\PYG{p}{)}
\PYG{g+go}{array([0, 3, 4]) \PYGZsh{} random}
\PYG{g+gp}{\PYGZgt{}\PYGZgt{}\PYGZgt{} }\PYG{c+c1}{\PYGZsh{}This is equivalent to np.random.randint(0,5,3)}
\end{sphinxVerbatim}

\sphinxAtStartPar
Generate a non\sphinxhyphen{}uniform random sample from np.arange(5) of size 3:

\begin{sphinxVerbatim}[commandchars=\\\{\}]
\PYG{g+gp}{\PYGZgt{}\PYGZgt{}\PYGZgt{} }\PYG{n}{np}\PYG{o}{.}\PYG{n}{random}\PYG{o}{.}\PYG{n}{choice}\PYG{p}{(}\PYG{l+m+mi}{5}\PYG{p}{,} \PYG{l+m+mi}{3}\PYG{p}{,} \PYG{n}{p}\PYG{o}{=}\PYG{p}{[}\PYG{l+m+mf}{0.1}\PYG{p}{,} \PYG{l+m+mi}{0}\PYG{p}{,} \PYG{l+m+mf}{0.3}\PYG{p}{,} \PYG{l+m+mf}{0.6}\PYG{p}{,} \PYG{l+m+mi}{0}\PYG{p}{]}\PYG{p}{)}
\PYG{g+go}{array([3, 3, 0]) \PYGZsh{} random}
\end{sphinxVerbatim}

\sphinxAtStartPar
Generate a uniform random sample from np.arange(5) of size 3 without
replacement:

\begin{sphinxVerbatim}[commandchars=\\\{\}]
\PYG{g+gp}{\PYGZgt{}\PYGZgt{}\PYGZgt{} }\PYG{n}{np}\PYG{o}{.}\PYG{n}{random}\PYG{o}{.}\PYG{n}{choice}\PYG{p}{(}\PYG{l+m+mi}{5}\PYG{p}{,} \PYG{l+m+mi}{3}\PYG{p}{,} \PYG{n}{replace}\PYG{o}{=}\PYG{k+kc}{False}\PYG{p}{)}
\PYG{g+go}{array([3,1,0]) \PYGZsh{} random}
\PYG{g+gp}{\PYGZgt{}\PYGZgt{}\PYGZgt{} }\PYG{c+c1}{\PYGZsh{}This is equivalent to np.random.permutation(np.arange(5))[:3]}
\end{sphinxVerbatim}

\sphinxAtStartPar
Generate a non\sphinxhyphen{}uniform random sample from np.arange(5) of size
3 without replacement:

\begin{sphinxVerbatim}[commandchars=\\\{\}]
\PYG{g+gp}{\PYGZgt{}\PYGZgt{}\PYGZgt{} }\PYG{n}{np}\PYG{o}{.}\PYG{n}{random}\PYG{o}{.}\PYG{n}{choice}\PYG{p}{(}\PYG{l+m+mi}{5}\PYG{p}{,} \PYG{l+m+mi}{3}\PYG{p}{,} \PYG{n}{replace}\PYG{o}{=}\PYG{k+kc}{False}\PYG{p}{,} \PYG{n}{p}\PYG{o}{=}\PYG{p}{[}\PYG{l+m+mf}{0.1}\PYG{p}{,} \PYG{l+m+mi}{0}\PYG{p}{,} \PYG{l+m+mf}{0.3}\PYG{p}{,} \PYG{l+m+mf}{0.6}\PYG{p}{,} \PYG{l+m+mi}{0}\PYG{p}{]}\PYG{p}{)}
\PYG{g+go}{array([2, 3, 0]) \PYGZsh{} random}
\end{sphinxVerbatim}

\sphinxAtStartPar
Any of the above can be repeated with an arbitrary array\sphinxhyphen{}like
instead of just integers. For instance:

\begin{sphinxVerbatim}[commandchars=\\\{\}]
\PYG{g+gp}{\PYGZgt{}\PYGZgt{}\PYGZgt{} }\PYG{n}{aa\PYGZus{}milne\PYGZus{}arr} \PYG{o}{=} \PYG{p}{[}\PYG{l+s+s1}{\PYGZsq{}}\PYG{l+s+s1}{pooh}\PYG{l+s+s1}{\PYGZsq{}}\PYG{p}{,} \PYG{l+s+s1}{\PYGZsq{}}\PYG{l+s+s1}{rabbit}\PYG{l+s+s1}{\PYGZsq{}}\PYG{p}{,} \PYG{l+s+s1}{\PYGZsq{}}\PYG{l+s+s1}{piglet}\PYG{l+s+s1}{\PYGZsq{}}\PYG{p}{,} \PYG{l+s+s1}{\PYGZsq{}}\PYG{l+s+s1}{Christopher}\PYG{l+s+s1}{\PYGZsq{}}\PYG{p}{]}
\PYG{g+gp}{\PYGZgt{}\PYGZgt{}\PYGZgt{} }\PYG{n}{np}\PYG{o}{.}\PYG{n}{random}\PYG{o}{.}\PYG{n}{choice}\PYG{p}{(}\PYG{n}{aa\PYGZus{}milne\PYGZus{}arr}\PYG{p}{,} \PYG{l+m+mi}{5}\PYG{p}{,} \PYG{n}{p}\PYG{o}{=}\PYG{p}{[}\PYG{l+m+mf}{0.5}\PYG{p}{,} \PYG{l+m+mf}{0.1}\PYG{p}{,} \PYG{l+m+mf}{0.1}\PYG{p}{,} \PYG{l+m+mf}{0.3}\PYG{p}{]}\PYG{p}{)}
\PYG{g+go}{array([\PYGZsq{}pooh\PYGZsq{}, \PYGZsq{}pooh\PYGZsq{}, \PYGZsq{}pooh\PYGZsq{}, \PYGZsq{}Christopher\PYGZsq{}, \PYGZsq{}piglet\PYGZsq{}], \PYGZsh{} random}
\PYG{g+go}{      dtype=\PYGZsq{}\PYGZlt{}U11\PYGZsq{})}
\end{sphinxVerbatim}

\end{fulllineitems}

\index{insert\_node\_between() (in module yawning\_titan.envs.generic.core.network\_interface)@\spxentry{insert\_node\_between()}\spxextra{in module yawning\_titan.envs.generic.core.network\_interface}}

\begin{fulllineitems}
\phantomsection\label{\detokenize{source/yawning_titan.envs.generic.core:yawning_titan.envs.generic.core.network_interface.insert_node_between}}\pysiglinewithargsret{\sphinxcode{\sphinxupquote{yawning\_titan.envs.generic.core.network\_interface.}}\sphinxbfcode{\sphinxupquote{insert\_node\_between}}}{\emph{\DUrole{n}{new\_node}\DUrole{p}{:}\DUrole{w}{  }\DUrole{n}{str}}, \emph{\DUrole{n}{node1}\DUrole{p}{:}\DUrole{w}{  }\DUrole{n}{str}}, \emph{\DUrole{n}{node2}\DUrole{p}{:}\DUrole{w}{  }\DUrole{n}{str}}, \emph{\DUrole{n}{graph}\DUrole{p}{:}\DUrole{w}{  }\DUrole{n}{networkx.classes.graph.Graph}}}{{ $\rightarrow$ None}}
\sphinxAtStartPar
Inserts a node in between two nodes
\begin{quote}\begin{description}
\item[{Parameters}] \leavevmode\begin{itemize}
\item {}
\sphinxAtStartPar
\sphinxstyleliteralstrong{\sphinxupquote{new\_node}} \textendash{} the name of the new node

\item {}
\sphinxAtStartPar
\sphinxstyleliteralstrong{\sphinxupquote{node1}} \textendash{} the name of the first node the new node will be connected to

\item {}
\sphinxAtStartPar
\sphinxstyleliteralstrong{\sphinxupquote{node2}} \textendash{} the name of the second node the new node will be connected to

\item {}
\sphinxAtStartPar
\sphinxstyleliteralstrong{\sphinxupquote{graph}} \textendash{} the networkx graph to add the new node to

\end{itemize}

\end{description}\end{quote}

\end{fulllineitems}

\index{remove\_node() (in module yawning\_titan.envs.generic.core.network\_interface)@\spxentry{remove\_node()}\spxextra{in module yawning\_titan.envs.generic.core.network\_interface}}

\begin{fulllineitems}
\phantomsection\label{\detokenize{source/yawning_titan.envs.generic.core:yawning_titan.envs.generic.core.network_interface.remove_node}}\pysiglinewithargsret{\sphinxcode{\sphinxupquote{yawning\_titan.envs.generic.core.network\_interface.}}\sphinxbfcode{\sphinxupquote{remove\_node}}}{\emph{\DUrole{n}{node\_name}\DUrole{p}{:}\DUrole{w}{  }\DUrole{n}{str}}, \emph{\DUrole{n}{graph}\DUrole{p}{:}\DUrole{w}{  }\DUrole{n}{networkx.classes.graph.Graph}}}{{ $\rightarrow$ None}}
\sphinxAtStartPar
Removes a node from a graph. Removing a node removes all connections to and from that node
\begin{quote}\begin{description}
\item[{Parameters}] \leavevmode\begin{itemize}
\item {}
\sphinxAtStartPar
\sphinxstyleliteralstrong{\sphinxupquote{node\_name}} \textendash{} the name of the node to remove

\item {}
\sphinxAtStartPar
\sphinxstyleliteralstrong{\sphinxupquote{graph}} \textendash{} the networkx graph to remove the node from

\end{itemize}

\end{description}\end{quote}

\end{fulllineitems}



\subparagraph{yawning\_titan.envs.generic.core.red\_action\_set module}
\label{\detokenize{source/yawning_titan.envs.generic.core:module-yawning_titan.envs.generic.core.red_action_set}}\label{\detokenize{source/yawning_titan.envs.generic.core:yawning-titan-envs-generic-core-red-action-set-module}}\index{module@\spxentry{module}!yawning\_titan.envs.generic.core.red\_action\_set@\spxentry{yawning\_titan.envs.generic.core.red\_action\_set}}\index{yawning\_titan.envs.generic.core.red\_action\_set@\spxentry{yawning\_titan.envs.generic.core.red\_action\_set}!module@\spxentry{module}}
\sphinxAtStartPar
A Parent red agent. This red agent acts as a container for any move that a red agent could want to make. An actual red
agent that would interface with the generic game loop would use a subset of the methods available here.

\sphinxAtStartPar
Interfaces with the networkx interface
\index{RedActionSet (class in yawning\_titan.envs.generic.core.red\_action\_set)@\spxentry{RedActionSet}\spxextra{class in yawning\_titan.envs.generic.core.red\_action\_set}}

\begin{fulllineitems}
\phantomsection\label{\detokenize{source/yawning_titan.envs.generic.core:yawning_titan.envs.generic.core.red_action_set.RedActionSet}}\pysiglinewithargsret{\sphinxbfcode{\sphinxupquote{class\DUrole{w}{  }}}\sphinxcode{\sphinxupquote{yawning\_titan.envs.generic.core.red\_action\_set.}}\sphinxbfcode{\sphinxupquote{RedActionSet}}}{\emph{\DUrole{n}{network\_interface}\DUrole{p}{:}\DUrole{w}{  }\DUrole{n}{{\hyperref[\detokenize{source/yawning_titan.envs.generic.core:yawning_titan.envs.generic.core.network_interface.NetworkInterface}]{\sphinxcrossref{yawning\_titan.envs.generic.core.network\_interface.NetworkInterface}}}}}, \emph{\DUrole{n}{settings\_file}\DUrole{p}{:}\DUrole{w}{  }\DUrole{n}{dict}}, \emph{\DUrole{n}{action\_set}\DUrole{p}{:}\DUrole{w}{  }\DUrole{n}{List\DUrole{p}{{[}}int\DUrole{p}{{]}}}}, \emph{\DUrole{n}{action\_probabilities}\DUrole{p}{:}\DUrole{w}{  }\DUrole{n}{List\DUrole{p}{{[}}float\DUrole{p}{{]}}}}}{}
\sphinxAtStartPar
Bases: \sphinxcode{\sphinxupquote{object}}
\index{action\_probabilities (yawning\_titan.envs.generic.core.red\_action\_set.RedActionSet attribute)@\spxentry{action\_probabilities}\spxextra{yawning\_titan.envs.generic.core.red\_action\_set.RedActionSet attribute}}

\begin{fulllineitems}
\phantomsection\label{\detokenize{source/yawning_titan.envs.generic.core:yawning_titan.envs.generic.core.red_action_set.RedActionSet.action_probabilities}}\pysigline{\sphinxbfcode{\sphinxupquote{action\_probabilities}}\sphinxbfcode{\sphinxupquote{\DUrole{w}{  }\DUrole{p}{=}\DUrole{w}{  }{[}{]}}}}
\end{fulllineitems}

\index{action\_set (yawning\_titan.envs.generic.core.red\_action\_set.RedActionSet attribute)@\spxentry{action\_set}\spxextra{yawning\_titan.envs.generic.core.red\_action\_set.RedActionSet attribute}}

\begin{fulllineitems}
\phantomsection\label{\detokenize{source/yawning_titan.envs.generic.core:yawning_titan.envs.generic.core.red_action_set.RedActionSet.action_set}}\pysigline{\sphinxbfcode{\sphinxupquote{action\_set}}\sphinxbfcode{\sphinxupquote{\DUrole{w}{  }\DUrole{p}{=}\DUrole{w}{  }{[}{]}}}}
\end{fulllineitems}

\index{basic\_attack() (yawning\_titan.envs.generic.core.red\_action\_set.RedActionSet method)@\spxentry{basic\_attack()}\spxextra{yawning\_titan.envs.generic.core.red\_action\_set.RedActionSet method}}

\begin{fulllineitems}
\phantomsection\label{\detokenize{source/yawning_titan.envs.generic.core:yawning_titan.envs.generic.core.red_action_set.RedActionSet.basic_attack}}\pysiglinewithargsret{\sphinxbfcode{\sphinxupquote{basic\_attack}}}{}{{ $\rightarrow$ Tuple\DUrole{p}{{[}}str\DUrole{p}{,}\DUrole{w}{  }List\DUrole{p}{{[}}bool\DUrole{p}{{]}}\DUrole{p}{,}\DUrole{w}{  }List\DUrole{p}{{[}}Optional\DUrole{p}{{[}}str\DUrole{p}{{]}}\DUrole{p}{{]}}\DUrole{p}{,}\DUrole{w}{  }List\DUrole{p}{{[}}Optional\DUrole{p}{{[}}str\DUrole{p}{{]}}\DUrole{p}{{]}}\DUrole{p}{{]}}}}
\sphinxAtStartPar
The red agent will attempt to compromise a target node using the predefined attack method
\begin{quote}\begin{description}
\item[{Returns}] \leavevmode
\sphinxAtStartPar
The name of the action taken
If the action succeeded
The target node
The attacking node

\end{description}\end{quote}

\end{fulllineitems}

\index{choose\_action() (yawning\_titan.envs.generic.core.red\_action\_set.RedActionSet method)@\spxentry{choose\_action()}\spxextra{yawning\_titan.envs.generic.core.red\_action\_set.RedActionSet method}}

\begin{fulllineitems}
\phantomsection\label{\detokenize{source/yawning_titan.envs.generic.core:yawning_titan.envs.generic.core.red_action_set.RedActionSet.choose_action}}\pysiglinewithargsret{\sphinxbfcode{\sphinxupquote{choose\_action}}}{}{{ $\rightarrow$ int}}
\sphinxAtStartPar
Chooses an action to perform
\begin{quote}\begin{description}
\item[{Returns}] \leavevmode
\sphinxAtStartPar
The chosen action to perform

\end{description}\end{quote}

\end{fulllineitems}

\index{choose\_target\_node() (yawning\_titan.envs.generic.core.red\_action\_set.RedActionSet method)@\spxentry{choose\_target\_node()}\spxextra{yawning\_titan.envs.generic.core.red\_action\_set.RedActionSet method}}

\begin{fulllineitems}
\phantomsection\label{\detokenize{source/yawning_titan.envs.generic.core:yawning_titan.envs.generic.core.red_action_set.RedActionSet.choose_target_node}}\pysiglinewithargsret{\sphinxbfcode{\sphinxupquote{choose\_target\_node}}}{}{{ $\rightarrow$ Union\DUrole{p}{{[}}Tuple\DUrole{p}{{[}}str\DUrole{p}{,}\DUrole{w}{  }str\DUrole{p}{{]}}\DUrole{p}{,}\DUrole{w}{  }Tuple\DUrole{p}{{[}}bool\DUrole{p}{,}\DUrole{w}{  }bool\DUrole{p}{{]}}\DUrole{p}{{]}}}}
\sphinxAtStartPar
Chooses a node to target an attack at
\begin{quote}\begin{description}
\item[{Returns}] \leavevmode
\sphinxAtStartPar
The target node (False if no possible nodes to attack)
The node attacking the target node (False if no possible nodes to attack)

\end{description}\end{quote}

\end{fulllineitems}

\index{do\_nothing() (yawning\_titan.envs.generic.core.red\_action\_set.RedActionSet method)@\spxentry{do\_nothing()}\spxextra{yawning\_titan.envs.generic.core.red\_action\_set.RedActionSet method}}

\begin{fulllineitems}
\phantomsection\label{\detokenize{source/yawning_titan.envs.generic.core:yawning_titan.envs.generic.core.red_action_set.RedActionSet.do_nothing}}\pysiglinewithargsret{\sphinxbfcode{\sphinxupquote{do\_nothing}}}{}{{ $\rightarrow$ Tuple\DUrole{p}{{[}}str\DUrole{p}{,}\DUrole{w}{  }List\DUrole{p}{{[}}bool\DUrole{p}{{]}}\DUrole{p}{,}\DUrole{w}{  }List\DUrole{p}{{[}}Optional\DUrole{p}{{[}}str\DUrole{p}{{]}}\DUrole{p}{{]}}\DUrole{p}{,}\DUrole{w}{  }List\DUrole{p}{{[}}Optional\DUrole{p}{{[}}str\DUrole{p}{{]}}\DUrole{p}{{]}}\DUrole{p}{{]}}}}
\sphinxAtStartPar
The agent does nothing
\begin{quote}\begin{description}
\item[{Returns}] \leavevmode
\sphinxAtStartPar
The name of the action (“do\_nothing”)
If the move succeeded
The target node {[}{]}
The current node {[}{]}

\end{description}\end{quote}

\end{fulllineitems}

\index{get\_amount\_zero\_day() (yawning\_titan.envs.generic.core.red\_action\_set.RedActionSet method)@\spxentry{get\_amount\_zero\_day()}\spxextra{yawning\_titan.envs.generic.core.red\_action\_set.RedActionSet method}}

\begin{fulllineitems}
\phantomsection\label{\detokenize{source/yawning_titan.envs.generic.core:yawning_titan.envs.generic.core.red_action_set.RedActionSet.get_amount_zero_day}}\pysiglinewithargsret{\sphinxbfcode{\sphinxupquote{get\_amount\_zero\_day}}}{}{{ $\rightarrow$ int}}
\sphinxAtStartPar
Gets the amount of zero day attacks that the red agent has stored up
\begin{quote}\begin{description}
\item[{Returns}] \leavevmode
\sphinxAtStartPar
Integer number \sphinxhyphen{} amount of zero day attacks

\end{description}\end{quote}

\end{fulllineitems}

\index{increment\_day() (yawning\_titan.envs.generic.core.red\_action\_set.RedActionSet method)@\spxentry{increment\_day()}\spxextra{yawning\_titan.envs.generic.core.red\_action\_set.RedActionSet method}}

\begin{fulllineitems}
\phantomsection\label{\detokenize{source/yawning_titan.envs.generic.core:yawning_titan.envs.generic.core.red_action_set.RedActionSet.increment_day}}\pysiglinewithargsret{\sphinxbfcode{\sphinxupquote{increment\_day}}}{}{}
\sphinxAtStartPar
Increment the day for zero day attacks

\end{fulllineitems}

\index{intrude() (yawning\_titan.envs.generic.core.red\_action\_set.RedActionSet method)@\spxentry{intrude()}\spxextra{yawning\_titan.envs.generic.core.red\_action\_set.RedActionSet method}}

\begin{fulllineitems}
\phantomsection\label{\detokenize{source/yawning_titan.envs.generic.core:yawning_titan.envs.generic.core.red_action_set.RedActionSet.intrude}}\pysiglinewithargsret{\sphinxbfcode{\sphinxupquote{intrude}}}{}{{ $\rightarrow$ Tuple\DUrole{p}{{[}}str\DUrole{p}{,}\DUrole{w}{  }List\DUrole{p}{{[}}bool\DUrole{p}{{]}}\DUrole{p}{,}\DUrole{w}{  }List\DUrole{p}{{[}}Optional\DUrole{p}{{[}}str\DUrole{p}{{]}}\DUrole{p}{{]}}\DUrole{p}{,}\DUrole{w}{  }List\DUrole{p}{{[}}None\DUrole{p}{{]}}\DUrole{p}{{]}}}}
\sphinxAtStartPar
Intrude attack function. The red agent will try to infect every safe node at once. The chance for the red agent
to compromise a node is independent to each of the other nodes
\begin{quote}\begin{description}
\item[{Returns}] \leavevmode
\sphinxAtStartPar
The name of the action
A list of success status for each node attacked
A list of the target nodes
A list of the attacking nodes

\end{description}\end{quote}

\end{fulllineitems}

\index{natural\_spread() (yawning\_titan.envs.generic.core.red\_action\_set.RedActionSet method)@\spxentry{natural\_spread()}\spxextra{yawning\_titan.envs.generic.core.red\_action\_set.RedActionSet method}}

\begin{fulllineitems}
\phantomsection\label{\detokenize{source/yawning_titan.envs.generic.core:yawning_titan.envs.generic.core.red_action_set.RedActionSet.natural_spread}}\pysiglinewithargsret{\sphinxbfcode{\sphinxupquote{natural\_spread}}}{}{{ $\rightarrow$ Tuple\DUrole{p}{{[}}List\DUrole{p}{{[}}bool\DUrole{p}{{]}}\DUrole{p}{,}\DUrole{w}{  }List\DUrole{p}{{[}}Optional\DUrole{p}{{[}}str\DUrole{p}{{]}}\DUrole{p}{{]}}\DUrole{p}{,}\DUrole{w}{  }List\DUrole{p}{{[}}Optional\DUrole{p}{{[}}str\DUrole{p}{{]}}\DUrole{p}{{]}}\DUrole{p}{{]}}}}
\sphinxAtStartPar
The red agent naturally spreads throughout the network. Nodes that are connected to compromised nodes can have
a different chance to become compromised. The settings for how likely nodes are to become compromised are in
the config file.
\begin{quote}\begin{description}
\item[{Returns}] \leavevmode
\sphinxAtStartPar
The success status of all the attacks
The target nodes
The attacking nodes

\end{description}\end{quote}

\end{fulllineitems}

\index{node\_set (yawning\_titan.envs.generic.core.red\_action\_set.RedActionSet attribute)@\spxentry{node\_set}\spxextra{yawning\_titan.envs.generic.core.red\_action\_set.RedActionSet attribute}}

\begin{fulllineitems}
\phantomsection\label{\detokenize{source/yawning_titan.envs.generic.core:yawning_titan.envs.generic.core.red_action_set.RedActionSet.node_set}}\pysigline{\sphinxbfcode{\sphinxupquote{node\_set}}\sphinxbfcode{\sphinxupquote{\DUrole{w}{  }\DUrole{p}{=}\DUrole{w}{  }{[}{]}}}}
\end{fulllineitems}

\index{random\_move() (yawning\_titan.envs.generic.core.red\_action\_set.RedActionSet method)@\spxentry{random\_move()}\spxextra{yawning\_titan.envs.generic.core.red\_action\_set.RedActionSet method}}

\begin{fulllineitems}
\phantomsection\label{\detokenize{source/yawning_titan.envs.generic.core:yawning_titan.envs.generic.core.red_action_set.RedActionSet.random_move}}\pysiglinewithargsret{\sphinxbfcode{\sphinxupquote{random\_move}}}{}{{ $\rightarrow$ Tuple\DUrole{p}{{[}}str\DUrole{p}{,}\DUrole{w}{  }List\DUrole{p}{{[}}bool\DUrole{p}{{]}}\DUrole{p}{,}\DUrole{w}{  }List\DUrole{p}{{[}}Optional\DUrole{p}{{[}}str\DUrole{p}{{]}}\DUrole{p}{{]}}\DUrole{p}{,}\DUrole{w}{  }List\DUrole{p}{{[}}Optional\DUrole{p}{{[}}str\DUrole{p}{{]}}\DUrole{p}{{]}}\DUrole{p}{{]}}}}
\sphinxAtStartPar
The red agent will attempt to move to a connected compromised node
\begin{quote}\begin{description}
\item[{Returns}] \leavevmode
\sphinxAtStartPar
The name of the action (“random\_move”)
If the move succeeded
The new red location (node)
The old red location (node)

\end{description}\end{quote}

\end{fulllineitems}

\index{spread() (yawning\_titan.envs.generic.core.red\_action\_set.RedActionSet method)@\spxentry{spread()}\spxextra{yawning\_titan.envs.generic.core.red\_action\_set.RedActionSet method}}

\begin{fulllineitems}
\phantomsection\label{\detokenize{source/yawning_titan.envs.generic.core:yawning_titan.envs.generic.core.red_action_set.RedActionSet.spread}}\pysiglinewithargsret{\sphinxbfcode{\sphinxupquote{spread}}}{}{{ $\rightarrow$ Tuple\DUrole{p}{{[}}str\DUrole{p}{,}\DUrole{w}{  }List\DUrole{p}{{[}}bool\DUrole{p}{{]}}\DUrole{p}{,}\DUrole{w}{  }List\DUrole{p}{{[}}Optional\DUrole{p}{{[}}str\DUrole{p}{{]}}\DUrole{p}{{]}}\DUrole{p}{,}\DUrole{w}{  }List\DUrole{p}{{[}}Optional\DUrole{p}{{[}}str\DUrole{p}{{]}}\DUrole{p}{{]}}\DUrole{p}{{]}}}}
\sphinxAtStartPar
Spread attack function. The red agent will try and spread from every infected node to every connected safe node.
The chance to spread between two nodes is independent of any other spreading.
\begin{quote}\begin{description}
\item[{Returns}] \leavevmode
\sphinxAtStartPar
The name of the action
A list of success status for each node attacked
A list of the target nodes
A list of the attacking nodes

\end{description}\end{quote}

\end{fulllineitems}

\index{zero\_day\_attack() (yawning\_titan.envs.generic.core.red\_action\_set.RedActionSet method)@\spxentry{zero\_day\_attack()}\spxextra{yawning\_titan.envs.generic.core.red\_action\_set.RedActionSet method}}

\begin{fulllineitems}
\phantomsection\label{\detokenize{source/yawning_titan.envs.generic.core:yawning_titan.envs.generic.core.red_action_set.RedActionSet.zero_day_attack}}\pysiglinewithargsret{\sphinxbfcode{\sphinxupquote{zero\_day\_attack}}}{}{{ $\rightarrow$ Tuple\DUrole{p}{{[}}str\DUrole{p}{,}\DUrole{w}{  }List\DUrole{p}{{[}}bool\DUrole{p}{{]}}\DUrole{p}{,}\DUrole{w}{  }List\DUrole{p}{{[}}Optional\DUrole{p}{{[}}str\DUrole{p}{{]}}\DUrole{p}{{]}}\DUrole{p}{,}\DUrole{w}{  }List\DUrole{p}{{[}}Optional\DUrole{p}{{[}}str\DUrole{p}{{]}}\DUrole{p}{{]}}\DUrole{p}{{]}}}}
\sphinxAtStartPar
If there is an available zero day attack available then performs one
\begin{quote}\begin{description}
\item[{Returns}] \leavevmode
\sphinxAtStartPar
The name of the action taken
If the action succeeded
The target node
The attacking node

\end{description}\end{quote}

\end{fulllineitems}


\end{fulllineitems}



\subparagraph{yawning\_titan.envs.generic.core.red\_interface module}
\label{\detokenize{source/yawning_titan.envs.generic.core:module-yawning_titan.envs.generic.core.red_interface}}\label{\detokenize{source/yawning_titan.envs.generic.core:yawning-titan-envs-generic-core-red-interface-module}}\index{module@\spxentry{module}!yawning\_titan.envs.generic.core.red\_interface@\spxentry{yawning\_titan.envs.generic.core.red\_interface}}\index{yawning\_titan.envs.generic.core.red\_interface@\spxentry{yawning\_titan.envs.generic.core.red\_interface}!module@\spxentry{module}}\index{RedInterface (class in yawning\_titan.envs.generic.core.red\_interface)@\spxentry{RedInterface}\spxextra{class in yawning\_titan.envs.generic.core.red\_interface}}

\begin{fulllineitems}
\phantomsection\label{\detokenize{source/yawning_titan.envs.generic.core:yawning_titan.envs.generic.core.red_interface.RedInterface}}\pysiglinewithargsret{\sphinxbfcode{\sphinxupquote{class\DUrole{w}{  }}}\sphinxcode{\sphinxupquote{yawning\_titan.envs.generic.core.red\_interface.}}\sphinxbfcode{\sphinxupquote{RedInterface}}}{\emph{\DUrole{n}{network\_interface}}}{}
\sphinxAtStartPar
Bases: {\hyperref[\detokenize{source/yawning_titan.envs.generic.core:yawning_titan.envs.generic.core.red_action_set.RedActionSet}]{\sphinxcrossref{\sphinxcode{\sphinxupquote{yawning\_titan.envs.generic.core.red\_action\_set.RedActionSet}}}}}
\index{perform\_action() (yawning\_titan.envs.generic.core.red\_interface.RedInterface method)@\spxentry{perform\_action()}\spxextra{yawning\_titan.envs.generic.core.red\_interface.RedInterface method}}

\begin{fulllineitems}
\phantomsection\label{\detokenize{source/yawning_titan.envs.generic.core:yawning_titan.envs.generic.core.red_interface.RedInterface.perform_action}}\pysiglinewithargsret{\sphinxbfcode{\sphinxupquote{perform\_action}}}{}{{ $\rightarrow$ Tuple\DUrole{p}{{[}}str\DUrole{p}{,}\DUrole{w}{  }Union\DUrole{p}{{[}}bool\DUrole{p}{,}\DUrole{w}{  }List\DUrole{p}{{[}}bool\DUrole{p}{{]}}\DUrole{p}{{]}}\DUrole{p}{,}\DUrole{w}{  }Union\DUrole{p}{{[}}List\DUrole{p}{{[}}str\DUrole{p}{{]}}\DUrole{p}{,}\DUrole{w}{  }str\DUrole{p}{{]}}\DUrole{p}{,}\DUrole{w}{  }Union\DUrole{p}{{[}}List\DUrole{p}{{[}}str\DUrole{p}{{]}}\DUrole{p}{,}\DUrole{w}{  }str\DUrole{p}{{]}}\DUrole{p}{,}\DUrole{w}{  }Tuple\DUrole{p}{{[}}List\DUrole{p}{{[}}str\DUrole{p}{{]}}\DUrole{p}{,}\DUrole{w}{  }List\DUrole{p}{{[}}bool\DUrole{p}{{]}}\DUrole{p}{{]}}\DUrole{p}{{]}}}}
\sphinxAtStartPar
Chooses and then performs an action. This is called for every one of the red agents turns
\begin{quote}\begin{description}
\item[{Returns}] \leavevmode
\sphinxAtStartPar
A tuple containing the name of the action, the success status, the target, the attacking node and any natural spreading attacks

\end{description}\end{quote}

\end{fulllineitems}


\end{fulllineitems}



\subparagraph{yawning\_titan.envs.generic.core.reward\_functions module}
\label{\detokenize{source/yawning_titan.envs.generic.core:module-yawning_titan.envs.generic.core.reward_functions}}\label{\detokenize{source/yawning_titan.envs.generic.core:yawning-titan-envs-generic-core-reward-functions-module}}\index{module@\spxentry{module}!yawning\_titan.envs.generic.core.reward\_functions@\spxentry{yawning\_titan.envs.generic.core.reward\_functions}}\index{yawning\_titan.envs.generic.core.reward\_functions@\spxentry{yawning\_titan.envs.generic.core.reward\_functions}!module@\spxentry{module}}
\sphinxAtStartPar
A collection of reward functions used be the generic network environment. You can select the reward function that you
wish to use in the config file under settings. The reward functions take in a parameter called args. args is a dictionary
that contains the following information:
\begin{quote}

\sphinxAtStartPar
\sphinxhyphen{}network\_interface: Interface with the network
\sphinxhyphen{}blue\_action: The action that the blue agent has taken this turn
\sphinxhyphen{}blue\_node: The node that the blue agent has targeted for their action
\sphinxhyphen{}start\_state: The state of the nodes before the blue agent has taken their action
\sphinxhyphen{}end\_state: The state of the nodes after the blue agent has taken their action
\sphinxhyphen{}start\_vulnerabilities: The vulnerabilities before blue agents turn
\sphinxhyphen{}end\_vulnerabilities: The vulnerabilities after the blue agents turn
\sphinxhyphen{}start\_isolation: The isolation status of all the nodes at the start of a turn
\sphinxhyphen{}end\_isolation: The isolation status of all the nodes at the end of a turn
\sphinxhyphen{}start\_blue: The env as the blue agent can see it before the blue agents turn
\sphinxhyphen{}end\_blue: The env as the blue agent can see it after the blue agents turn
\end{quote}

\sphinxAtStartPar
The reward function returns a single number (integer or float) that is the blue agents reward for that turn.
\index{experimental\_rewards() (in module yawning\_titan.envs.generic.core.reward\_functions)@\spxentry{experimental\_rewards()}\spxextra{in module yawning\_titan.envs.generic.core.reward\_functions}}

\begin{fulllineitems}
\phantomsection\label{\detokenize{source/yawning_titan.envs.generic.core:yawning_titan.envs.generic.core.reward_functions.experimental_rewards}}\pysiglinewithargsret{\sphinxcode{\sphinxupquote{yawning\_titan.envs.generic.core.reward\_functions.}}\sphinxbfcode{\sphinxupquote{experimental\_rewards}}}{\emph{\DUrole{n}{args}\DUrole{p}{:}\DUrole{w}{  }\DUrole{n}{dict}}}{{ $\rightarrow$ float}}
\sphinxAtStartPar
Calculates the reward for the current state of the environment. Actions cost a certain amount and blue gets rewarded
for removing red nodes and reducing the vulnerability of nodes
\begin{quote}\begin{description}
\item[{Parameters}] \leavevmode
\sphinxAtStartPar
\sphinxstyleliteralstrong{\sphinxupquote{args}} \textendash{} A dictionary containing the following items:
network\_interface: Interface with the network
blue\_action: The action that the blue agent has taken this turn
blue\_node: The node that the blue agent has targeted for their action
start\_state: The state of the nodes before the blue agent has taken their action
end\_state: The state of the nodes after the blue agent has taken their action
start\_vulnerabilities: The vulnerabilities before blue agents turn
end\_vulnerabilities: The vulnerabilities after the blue agents turn
start\_isolation: The isolation status of all the nodes at the start of a turn
end\_isolation: The isolation status of all the nodes at the end of a turn
start\_blue: The env as the blue agent can see it before the blue agents turn
end\_blue: The env as the blue agent can see it after the blue agents turn

\item[{Returns}] \leavevmode
\sphinxAtStartPar
The reward earned for this specific turn for the blue agent

\end{description}\end{quote}

\end{fulllineitems}

\index{num\_nodes\_safe() (in module yawning\_titan.envs.generic.core.reward\_functions)@\spxentry{num\_nodes\_safe()}\spxextra{in module yawning\_titan.envs.generic.core.reward\_functions}}

\begin{fulllineitems}
\phantomsection\label{\detokenize{source/yawning_titan.envs.generic.core:yawning_titan.envs.generic.core.reward_functions.num_nodes_safe}}\pysiglinewithargsret{\sphinxcode{\sphinxupquote{yawning\_titan.envs.generic.core.reward\_functions.}}\sphinxbfcode{\sphinxupquote{num\_nodes\_safe}}}{\emph{\DUrole{n}{args}\DUrole{p}{:}\DUrole{w}{  }\DUrole{n}{dict}}}{{ $\rightarrow$ float}}
\sphinxAtStartPar
Simple reward function based on the number of nodes safe
within the environment
\begin{quote}\begin{description}
\item[{Parameters}] \leavevmode\begin{itemize}
\item {}
\sphinxAtStartPar
\sphinxstyleliteralstrong{\sphinxupquote{args}} \textendash{} A dictionary containing information from the

\item {}
\sphinxAtStartPar
\sphinxstyleliteralstrong{\sphinxupquote{timestep}} (\sphinxstyleliteralemphasis{\sphinxupquote{environment for the given}}) \textendash{}

\end{itemize}

\item[{Returns}] \leavevmode
\sphinxAtStartPar
The calculated reward

\end{description}\end{quote}

\end{fulllineitems}

\index{one\_per\_timestep() (in module yawning\_titan.envs.generic.core.reward\_functions)@\spxentry{one\_per\_timestep()}\spxextra{in module yawning\_titan.envs.generic.core.reward\_functions}}

\begin{fulllineitems}
\phantomsection\label{\detokenize{source/yawning_titan.envs.generic.core:yawning_titan.envs.generic.core.reward_functions.one_per_timestep}}\pysiglinewithargsret{\sphinxcode{\sphinxupquote{yawning\_titan.envs.generic.core.reward\_functions.}}\sphinxbfcode{\sphinxupquote{one\_per\_timestep}}}{\emph{\DUrole{n}{args}\DUrole{p}{:}\DUrole{w}{  }\DUrole{n}{dict}}}{{ $\rightarrow$ float}}
\sphinxAtStartPar
Gives a reward for 0.1 for every timestep that the blue agent is alive
\begin{quote}\begin{description}
\item[{Parameters}] \leavevmode
\sphinxAtStartPar
\sphinxstyleliteralstrong{\sphinxupquote{args}} \textendash{} A dictionary containing the following items:
network\_interface: Interface with the network
blue\_action: The action that the blue agent has taken this turn
blue\_node: The node that the blue agent has targeted for their action
start\_state: The state of the nodes before the blue agent has taken their action
end\_state: The state of the nodes after the blue agent has taken their action
start\_vulnerabilities: The vulnerabilities before blue agents turn
end\_vulnerabilities: The vulnerabilities after the blue agents turn
start\_isolation: The isolation status of all the nodes at the start of a turn
end\_isolation: The isolation status of all the nodes at the end of a turn
start\_blue: The env as the blue agent can see it before the blue agents turn
end\_blue: The env as the blue agent can see it after the blue agents turn

\item[{Returns}] \leavevmode
\sphinxAtStartPar
0.1

\end{description}\end{quote}

\end{fulllineitems}

\index{punish\_bad\_actions() (in module yawning\_titan.envs.generic.core.reward\_functions)@\spxentry{punish\_bad\_actions()}\spxextra{in module yawning\_titan.envs.generic.core.reward\_functions}}

\begin{fulllineitems}
\phantomsection\label{\detokenize{source/yawning_titan.envs.generic.core:yawning_titan.envs.generic.core.reward_functions.punish_bad_actions}}\pysiglinewithargsret{\sphinxcode{\sphinxupquote{yawning\_titan.envs.generic.core.reward\_functions.}}\sphinxbfcode{\sphinxupquote{punish\_bad\_actions}}}{\emph{\DUrole{n}{args}\DUrole{p}{:}\DUrole{w}{  }\DUrole{n}{dict}}}{{ $\rightarrow$ float}}
\sphinxAtStartPar
Just punishes bad actions bad moves
\begin{quote}\begin{description}
\item[{Parameters}] \leavevmode
\sphinxAtStartPar
\sphinxstyleliteralstrong{\sphinxupquote{args}} \textendash{} A dictionary containing the following items:
network\_interface: Interface with the network
blue\_action: The action that the blue agent has taken this turn
blue\_node: The node that the blue agent has targeted for their action
start\_state: The state of the nodes before the blue agent has taken their action
end\_state: The state of the nodes after the blue agent has taken their action
start\_vulnerabilities: The vulnerabilities before blue agents turn
end\_vulnerabilities: The vulnerabilities after the blue agents turn
start\_isolation: The isolation status of all the nodes at the start of a turn
end\_isolation: The isolation status of all the nodes at the end of a turn
start\_blue: The env as the blue agent can see it before the blue agents turn
end\_blue: The env as the blue agent can see it after the blue agents turn

\item[{Returns}] \leavevmode
\sphinxAtStartPar
The reward earned for this specific turn for the blue agent

\end{description}\end{quote}

\end{fulllineitems}

\index{safe\_nodes\_give\_rewards() (in module yawning\_titan.envs.generic.core.reward\_functions)@\spxentry{safe\_nodes\_give\_rewards()}\spxextra{in module yawning\_titan.envs.generic.core.reward\_functions}}

\begin{fulllineitems}
\phantomsection\label{\detokenize{source/yawning_titan.envs.generic.core:yawning_titan.envs.generic.core.reward_functions.safe_nodes_give_rewards}}\pysiglinewithargsret{\sphinxcode{\sphinxupquote{yawning\_titan.envs.generic.core.reward\_functions.}}\sphinxbfcode{\sphinxupquote{safe\_nodes\_give\_rewards}}}{\emph{\DUrole{n}{args}\DUrole{p}{:}\DUrole{w}{  }\DUrole{n}{dict}}}{{ $\rightarrow$ float}}
\sphinxAtStartPar
Gives 1 reward for every safe node at that timestep
\begin{quote}\begin{description}
\item[{Parameters}] \leavevmode
\sphinxAtStartPar
\sphinxstyleliteralstrong{\sphinxupquote{args}} \textendash{} A dictionary containing the following items:
network\_interface: Interface with the network
blue\_action: The action that the blue agent has taken this turn
blue\_node: The node that the blue agent has targeted for their action
start\_state: The state of the nodes before the blue agent has taken their action
end\_state: The state of the nodes after the blue agent has taken their action
start\_vulnerabilities: The vulnerabilities before blue agents turn
end\_vulnerabilities: The vulnerabilities after the blue agents turn
start\_isolation: The isolation status of all the nodes at the start of a turn
end\_isolation: The isolation status of all the nodes at the end of a turn
start\_blue: The env as the blue agent can see it before the blue agents turn
end\_blue: The env as the blue agent can see it after the blue agents turn

\item[{Returns}] \leavevmode
\sphinxAtStartPar
The reward earned for this specific turn for the blue agent

\end{description}\end{quote}

\end{fulllineitems}

\index{standard\_rewards() (in module yawning\_titan.envs.generic.core.reward\_functions)@\spxentry{standard\_rewards()}\spxextra{in module yawning\_titan.envs.generic.core.reward\_functions}}

\begin{fulllineitems}
\phantomsection\label{\detokenize{source/yawning_titan.envs.generic.core:yawning_titan.envs.generic.core.reward_functions.standard_rewards}}\pysiglinewithargsret{\sphinxcode{\sphinxupquote{yawning\_titan.envs.generic.core.reward\_functions.}}\sphinxbfcode{\sphinxupquote{standard\_rewards}}}{\emph{\DUrole{n}{args}\DUrole{p}{:}\DUrole{w}{  }\DUrole{n}{dict}}}{{ $\rightarrow$ float}}
\sphinxAtStartPar
Calculates the reward for the current state of the environment. Actions cost a certain amount and blue gets rewarded
for removing red nodes and reducing the vulnerability of nodes
\begin{quote}\begin{description}
\item[{Parameters}] \leavevmode
\sphinxAtStartPar
\sphinxstyleliteralstrong{\sphinxupquote{args}} \textendash{} A dictionary containing the following items:
network\_interface: Interface with the network
blue\_action: The action that the blue agent has taken this turn
blue\_node: The node that the blue agent has targeted for their action
start\_state: The state of the nodes before the blue agent has taken their action
end\_state: The state of the nodes after the blue agent has taken their action
start\_vulnerabilities: The vulnerabilities before blue agents turn
end\_vulnerabilities: The vulnerabilities after the blue agents turn
start\_isolation: The isolation status of all the nodes at the start of a turn
end\_isolation: The isolation status of all the nodes at the end of a turn
start\_blue: The env as the blue agent can see it before the blue agents turn
end\_blue: The env as the blue agent can see it after the blue agents turn

\item[{Returns}] \leavevmode
\sphinxAtStartPar
The reward earned for this specific turn for the blue agent

\end{description}\end{quote}

\end{fulllineitems}

\index{zero\_reward() (in module yawning\_titan.envs.generic.core.reward\_functions)@\spxentry{zero\_reward()}\spxextra{in module yawning\_titan.envs.generic.core.reward\_functions}}

\begin{fulllineitems}
\phantomsection\label{\detokenize{source/yawning_titan.envs.generic.core:yawning_titan.envs.generic.core.reward_functions.zero_reward}}\pysiglinewithargsret{\sphinxcode{\sphinxupquote{yawning\_titan.envs.generic.core.reward\_functions.}}\sphinxbfcode{\sphinxupquote{zero\_reward}}}{\emph{\DUrole{n}{args}\DUrole{p}{:}\DUrole{w}{  }\DUrole{n}{dict}}}{{ $\rightarrow$ float}}
\sphinxAtStartPar
Does not give any reward per timestep
\begin{quote}\begin{description}
\item[{Parameters}] \leavevmode
\sphinxAtStartPar
\sphinxstyleliteralstrong{\sphinxupquote{args}} \textendash{} A dictionary containing the following items:
network\_interface: Interface with the network
blue\_action: The action that the blue agent has taken this turn
blue\_node: The node that the blue agent has targeted for their action
start\_state: The state of the nodes before the blue agent has taken their action
end\_state: The state of the nodes after the blue agent has taken their action
start\_vulnerabilities: The vulnerabilities before blue agents turn
end\_vulnerabilities: The vulnerabilities after the blue agents turn
start\_isolation: The isolation status of all the nodes at the start of a turn
end\_isolation: The isolation status of all the nodes at the end of a turn
start\_blue: The env as the blue agent can see it before the blue agents turn
end\_blue: The env as the blue agent can see it after the blue agents turn

\item[{Returns}] \leavevmode
\sphinxAtStartPar
0

\end{description}\end{quote}

\end{fulllineitems}



\subparagraph{Module contents}
\label{\detokenize{source/yawning_titan.envs.generic.core:module-yawning_titan.envs.generic.core}}\label{\detokenize{source/yawning_titan.envs.generic.core:module-contents}}\index{module@\spxentry{module}!yawning\_titan.envs.generic.core@\spxentry{yawning\_titan.envs.generic.core}}\index{yawning\_titan.envs.generic.core@\spxentry{yawning\_titan.envs.generic.core}!module@\spxentry{module}}

\subparagraph{yawning\_titan.envs.generic.helpers package}
\label{\detokenize{source/yawning_titan.envs.generic.helpers:yawning-titan-envs-generic-helpers-package}}\label{\detokenize{source/yawning_titan.envs.generic.helpers::doc}}

\subparagraph{Submodules}
\label{\detokenize{source/yawning_titan.envs.generic.helpers:submodules}}

\subparagraph{yawning\_titan.envs.generic.helpers.environment\_input\_validation module}
\label{\detokenize{source/yawning_titan.envs.generic.helpers:module-yawning_titan.envs.generic.helpers.environment_input_validation}}\label{\detokenize{source/yawning_titan.envs.generic.helpers:yawning-titan-envs-generic-helpers-environment-input-validation-module}}\index{module@\spxentry{module}!yawning\_titan.envs.generic.helpers.environment\_input\_validation@\spxentry{yawning\_titan.envs.generic.helpers.environment\_input\_validation}}\index{yawning\_titan.envs.generic.helpers.environment\_input\_validation@\spxentry{yawning\_titan.envs.generic.helpers.environment\_input\_validation}!module@\spxentry{module}}\index{check\_blue() (in module yawning\_titan.envs.generic.helpers.environment\_input\_validation)@\spxentry{check\_blue()}\spxextra{in module yawning\_titan.envs.generic.helpers.environment\_input\_validation}}

\begin{fulllineitems}
\phantomsection\label{\detokenize{source/yawning_titan.envs.generic.helpers:yawning_titan.envs.generic.helpers.environment_input_validation.check_blue}}\pysiglinewithargsret{\sphinxcode{\sphinxupquote{yawning\_titan.envs.generic.helpers.environment\_input\_validation.}}\sphinxbfcode{\sphinxupquote{check\_blue}}}{\emph{\DUrole{n}{data}\DUrole{p}{:}\DUrole{w}{  }\DUrole{n}{dict}}}{}
\sphinxAtStartPar
Checks all of the settings relating to the blue agent
\begin{quote}\begin{description}
\item[{Parameters}] \leavevmode
\sphinxAtStartPar
\sphinxstyleliteralstrong{\sphinxupquote{data}} \textendash{} The dictionary of settings for the blue agent

\end{description}\end{quote}

\end{fulllineitems}

\index{check\_game\_rules() (in module yawning\_titan.envs.generic.helpers.environment\_input\_validation)@\spxentry{check\_game\_rules()}\spxextra{in module yawning\_titan.envs.generic.helpers.environment\_input\_validation}}

\begin{fulllineitems}
\phantomsection\label{\detokenize{source/yawning_titan.envs.generic.helpers:yawning_titan.envs.generic.helpers.environment_input_validation.check_game_rules}}\pysiglinewithargsret{\sphinxcode{\sphinxupquote{yawning\_titan.envs.generic.helpers.environment\_input\_validation.}}\sphinxbfcode{\sphinxupquote{check\_game\_rules}}}{\emph{\DUrole{n}{data}\DUrole{p}{:}\DUrole{w}{  }\DUrole{n}{dict}}, \emph{\DUrole{n}{number\_of\_nodes}\DUrole{p}{:}\DUrole{w}{  }\DUrole{n}{int}}}{}
\sphinxAtStartPar
Checks the settings relating to the game rules
\begin{quote}\begin{description}
\item[{Parameters}] \leavevmode\begin{itemize}
\item {}
\sphinxAtStartPar
\sphinxstyleliteralstrong{\sphinxupquote{data}} \textendash{} The dictionary relating to the game rules settings

\item {}
\sphinxAtStartPar
\sphinxstyleliteralstrong{\sphinxupquote{number\_of\_nodes}} \textendash{} The number of nodes in the network

\end{itemize}

\end{description}\end{quote}

\end{fulllineitems}

\index{check\_input() (in module yawning\_titan.envs.generic.helpers.environment\_input\_validation)@\spxentry{check\_input()}\spxextra{in module yawning\_titan.envs.generic.helpers.environment\_input\_validation}}

\begin{fulllineitems}
\phantomsection\label{\detokenize{source/yawning_titan.envs.generic.helpers:yawning_titan.envs.generic.helpers.environment_input_validation.check_input}}\pysiglinewithargsret{\sphinxcode{\sphinxupquote{yawning\_titan.envs.generic.helpers.environment\_input\_validation.}}\sphinxbfcode{\sphinxupquote{check\_input}}}{\emph{\DUrole{n}{data}\DUrole{p}{:}\DUrole{w}{  }\DUrole{n}{dict}}, \emph{\DUrole{n}{number\_of\_nodes}\DUrole{p}{:}\DUrole{w}{  }\DUrole{n}{int}}}{}
\sphinxAtStartPar
Checks the settings file making sure that all the required settings are there and that they contain suitable values
\begin{quote}\begin{description}
\item[{Parameters}] \leavevmode\begin{itemize}
\item {}
\sphinxAtStartPar
\sphinxstyleliteralstrong{\sphinxupquote{data}} \textendash{} The settings file (A dictionary)

\item {}
\sphinxAtStartPar
\sphinxstyleliteralstrong{\sphinxupquote{number\_of\_nodes}} \textendash{} The number of nodes in the network

\end{itemize}

\end{description}\end{quote}

\end{fulllineitems}

\index{check\_misc() (in module yawning\_titan.envs.generic.helpers.environment\_input\_validation)@\spxentry{check\_misc()}\spxextra{in module yawning\_titan.envs.generic.helpers.environment\_input\_validation}}

\begin{fulllineitems}
\phantomsection\label{\detokenize{source/yawning_titan.envs.generic.helpers:yawning_titan.envs.generic.helpers.environment_input_validation.check_misc}}\pysiglinewithargsret{\sphinxcode{\sphinxupquote{yawning\_titan.envs.generic.helpers.environment\_input\_validation.}}\sphinxbfcode{\sphinxupquote{check\_misc}}}{\emph{\DUrole{n}{data}\DUrole{p}{:}\DUrole{w}{  }\DUrole{n}{dict}}}{}
\sphinxAtStartPar
Checks the misc settings
\begin{quote}\begin{description}
\item[{Parameters}] \leavevmode
\sphinxAtStartPar
\sphinxstyleliteralstrong{\sphinxupquote{data}} \textendash{} The dictionary for the misc settings

\end{description}\end{quote}

\end{fulllineitems}

\index{check\_observation\_space() (in module yawning\_titan.envs.generic.helpers.environment\_input\_validation)@\spxentry{check\_observation\_space()}\spxextra{in module yawning\_titan.envs.generic.helpers.environment\_input\_validation}}

\begin{fulllineitems}
\phantomsection\label{\detokenize{source/yawning_titan.envs.generic.helpers:yawning_titan.envs.generic.helpers.environment_input_validation.check_observation_space}}\pysiglinewithargsret{\sphinxcode{\sphinxupquote{yawning\_titan.envs.generic.helpers.environment\_input\_validation.}}\sphinxbfcode{\sphinxupquote{check\_observation\_space}}}{\emph{\DUrole{n}{data}\DUrole{p}{:}\DUrole{w}{  }\DUrole{n}{dict}}}{}
\end{fulllineitems}

\index{check\_red() (in module yawning\_titan.envs.generic.helpers.environment\_input\_validation)@\spxentry{check\_red()}\spxextra{in module yawning\_titan.envs.generic.helpers.environment\_input\_validation}}

\begin{fulllineitems}
\phantomsection\label{\detokenize{source/yawning_titan.envs.generic.helpers:yawning_titan.envs.generic.helpers.environment_input_validation.check_red}}\pysiglinewithargsret{\sphinxcode{\sphinxupquote{yawning\_titan.envs.generic.helpers.environment\_input\_validation.}}\sphinxbfcode{\sphinxupquote{check\_red}}}{\emph{\DUrole{n}{data}\DUrole{p}{:}\DUrole{w}{  }\DUrole{n}{dict}}}{}
\sphinxAtStartPar
Checks all of the settings relating to the red agent
\begin{quote}\begin{description}
\item[{Parameters}] \leavevmode
\sphinxAtStartPar
\sphinxstyleliteralstrong{\sphinxupquote{data}} \textendash{} The dictionary for settings relating to the red agent

\end{description}\end{quote}

\end{fulllineitems}

\index{check\_reset() (in module yawning\_titan.envs.generic.helpers.environment\_input\_validation)@\spxentry{check\_reset()}\spxextra{in module yawning\_titan.envs.generic.helpers.environment\_input\_validation}}

\begin{fulllineitems}
\phantomsection\label{\detokenize{source/yawning_titan.envs.generic.helpers:yawning_titan.envs.generic.helpers.environment_input_validation.check_reset}}\pysiglinewithargsret{\sphinxcode{\sphinxupquote{yawning\_titan.envs.generic.helpers.environment\_input\_validation.}}\sphinxbfcode{\sphinxupquote{check\_reset}}}{\emph{\DUrole{n}{data}\DUrole{p}{:}\DUrole{w}{  }\DUrole{n}{dict}}}{}
\sphinxAtStartPar
Checks the settings relating to resets
\begin{quote}\begin{description}
\item[{Parameters}] \leavevmode
\sphinxAtStartPar
\sphinxstyleliteralstrong{\sphinxupquote{data}} \textendash{} The settings related to resets

\end{description}\end{quote}

\end{fulllineitems}

\index{check\_reward\_function\_exists() (in module yawning\_titan.envs.generic.helpers.environment\_input\_validation)@\spxentry{check\_reward\_function\_exists()}\spxextra{in module yawning\_titan.envs.generic.helpers.environment\_input\_validation}}

\begin{fulllineitems}
\phantomsection\label{\detokenize{source/yawning_titan.envs.generic.helpers:yawning_titan.envs.generic.helpers.environment_input_validation.check_reward_function_exists}}\pysiglinewithargsret{\sphinxcode{\sphinxupquote{yawning\_titan.envs.generic.helpers.environment\_input\_validation.}}\sphinxbfcode{\sphinxupquote{check\_reward\_function\_exists}}}{\emph{\DUrole{n}{data}\DUrole{p}{:}\DUrole{w}{  }\DUrole{n}{dict}}}{}
\sphinxAtStartPar
Checks that the reward function that the settings file contains exists
\begin{quote}\begin{description}
\item[{Parameters}] \leavevmode
\sphinxAtStartPar
\sphinxstyleliteralstrong{\sphinxupquote{data}} \textendash{} The dictionary containing reward settings

\end{description}\end{quote}

\end{fulllineitems}

\index{check\_rewards() (in module yawning\_titan.envs.generic.helpers.environment\_input\_validation)@\spxentry{check\_rewards()}\spxextra{in module yawning\_titan.envs.generic.helpers.environment\_input\_validation}}

\begin{fulllineitems}
\phantomsection\label{\detokenize{source/yawning_titan.envs.generic.helpers:yawning_titan.envs.generic.helpers.environment_input_validation.check_rewards}}\pysiglinewithargsret{\sphinxcode{\sphinxupquote{yawning\_titan.envs.generic.helpers.environment\_input\_validation.}}\sphinxbfcode{\sphinxupquote{check\_rewards}}}{\emph{\DUrole{n}{data}\DUrole{p}{:}\DUrole{w}{  }\DUrole{n}{dict}}}{}
\sphinxAtStartPar
Checks the settings relating to the rewards
\begin{quote}\begin{description}
\item[{Parameters}] \leavevmode
\sphinxAtStartPar
\sphinxstyleliteralstrong{\sphinxupquote{data}} \textendash{} The dictionary of settings for the rewards

\end{description}\end{quote}

\end{fulllineitems}

\index{check\_type() (in module yawning\_titan.envs.generic.helpers.environment\_input\_validation)@\spxentry{check\_type()}\spxextra{in module yawning\_titan.envs.generic.helpers.environment\_input\_validation}}

\begin{fulllineitems}
\phantomsection\label{\detokenize{source/yawning_titan.envs.generic.helpers:yawning_titan.envs.generic.helpers.environment_input_validation.check_type}}\pysiglinewithargsret{\sphinxcode{\sphinxupquote{yawning\_titan.envs.generic.helpers.environment\_input\_validation.}}\sphinxbfcode{\sphinxupquote{check\_type}}}{\emph{\DUrole{n}{data}\DUrole{p}{:}\DUrole{w}{  }\DUrole{n}{dict}}, \emph{\DUrole{n}{name}\DUrole{p}{:}\DUrole{w}{  }\DUrole{n}{str}}, \emph{\DUrole{n}{types}\DUrole{p}{:}\DUrole{w}{  }\DUrole{n}{list}}}{}
\sphinxAtStartPar
Checks that a bit of data contained within a dictionary is one of a list of types
\begin{quote}\begin{description}
\item[{Parameters}] \leavevmode\begin{itemize}
\item {}
\sphinxAtStartPar
\sphinxstyleliteralstrong{\sphinxupquote{data}} \textendash{} The dictionary

\item {}
\sphinxAtStartPar
\sphinxstyleliteralstrong{\sphinxupquote{name}} \textendash{} The name of the key of the item to check

\item {}
\sphinxAtStartPar
\sphinxstyleliteralstrong{\sphinxupquote{types}} \textendash{} A list of types that the item must belong to

\end{itemize}

\end{description}\end{quote}

\end{fulllineitems}

\index{check\_within\_range() (in module yawning\_titan.envs.generic.helpers.environment\_input\_validation)@\spxentry{check\_within\_range()}\spxextra{in module yawning\_titan.envs.generic.helpers.environment\_input\_validation}}

\begin{fulllineitems}
\phantomsection\label{\detokenize{source/yawning_titan.envs.generic.helpers:yawning_titan.envs.generic.helpers.environment_input_validation.check_within_range}}\pysiglinewithargsret{\sphinxcode{\sphinxupquote{yawning\_titan.envs.generic.helpers.environment\_input\_validation.}}\sphinxbfcode{\sphinxupquote{check\_within\_range}}}{\emph{\DUrole{n}{data}\DUrole{p}{:}\DUrole{w}{  }\DUrole{n}{dict}}, \emph{\DUrole{n}{name}\DUrole{p}{:}\DUrole{w}{  }\DUrole{n}{str}}, \emph{\DUrole{n}{lower}\DUrole{p}{:}\DUrole{w}{  }\DUrole{n}{Union\DUrole{p}{{[}}None\DUrole{p}{,}\DUrole{w}{  }float\DUrole{p}{{]}}}}, \emph{\DUrole{n}{upper}\DUrole{p}{:}\DUrole{w}{  }\DUrole{n}{Union\DUrole{p}{{[}}None\DUrole{p}{,}\DUrole{w}{  }float\DUrole{p}{{]}}}}, \emph{\DUrole{n}{l\_inclusive}\DUrole{p}{:}\DUrole{w}{  }\DUrole{n}{bool}}, \emph{\DUrole{n}{u\_inclusive}\DUrole{p}{:}\DUrole{w}{  }\DUrole{n}{bool}}}{}
\sphinxAtStartPar
Check that an item belonging to a dictionary fits within a certain numerical range (either inclusive or not). If upper or lower are None then ignores that direction.
\begin{quote}\begin{description}
\item[{Parameters}] \leavevmode\begin{itemize}
\item {}
\sphinxAtStartPar
\sphinxstyleliteralstrong{\sphinxupquote{data}} \textendash{} The dictionary where the item is held

\item {}
\sphinxAtStartPar
\sphinxstyleliteralstrong{\sphinxupquote{name}} \textendash{} The name of the key that corresponds to the item

\item {}
\sphinxAtStartPar
\sphinxstyleliteralstrong{\sphinxupquote{lower}} \textendash{} The lower bound for the range (None means no lower bound)

\item {}
\sphinxAtStartPar
\sphinxstyleliteralstrong{\sphinxupquote{upper}} \textendash{} The upper bound for the range (None means no upper bound)

\item {}
\sphinxAtStartPar
\sphinxstyleliteralstrong{\sphinxupquote{l\_inclusive}} \textendash{} Boolean \sphinxhyphen{} True for inclusive, False for not

\item {}
\sphinxAtStartPar
\sphinxstyleliteralstrong{\sphinxupquote{u\_inclusive}} \textendash{} Boolean \sphinxhyphen{} True for inclusive, False for not

\end{itemize}

\end{description}\end{quote}

\end{fulllineitems}



\subparagraph{yawning\_titan.envs.generic.helpers.eval\_printout module}
\label{\detokenize{source/yawning_titan.envs.generic.helpers:module-yawning_titan.envs.generic.helpers.eval_printout}}\label{\detokenize{source/yawning_titan.envs.generic.helpers:yawning-titan-envs-generic-helpers-eval-printout-module}}\index{module@\spxentry{module}!yawning\_titan.envs.generic.helpers.eval\_printout@\spxentry{yawning\_titan.envs.generic.helpers.eval\_printout}}\index{yawning\_titan.envs.generic.helpers.eval\_printout@\spxentry{yawning\_titan.envs.generic.helpers.eval\_printout}!module@\spxentry{module}}\begin{description}
\item[{Util to print out agent evaluation metrics:}] \leavevmode\begin{itemize}
\item {}
\sphinxAtStartPar
Total episodes elapsed

\item {}
\sphinxAtStartPar
Absolute wins for red and blue

\item {}
\sphinxAtStartPar
Percentage win rate for red and blue

\item {}
\sphinxAtStartPar
Average episode length

\item {}
\sphinxAtStartPar
Actions taken by blue each game/Average actions taken by blue over n games

\end{itemize}

\end{description}
\index{EvalPrintout (class in yawning\_titan.envs.generic.helpers.eval\_printout)@\spxentry{EvalPrintout}\spxextra{class in yawning\_titan.envs.generic.helpers.eval\_printout}}

\begin{fulllineitems}
\phantomsection\label{\detokenize{source/yawning_titan.envs.generic.helpers:yawning_titan.envs.generic.helpers.eval_printout.EvalPrintout}}\pysiglinewithargsret{\sphinxbfcode{\sphinxupquote{class\DUrole{w}{  }}}\sphinxcode{\sphinxupquote{yawning\_titan.envs.generic.helpers.eval\_printout.}}\sphinxbfcode{\sphinxupquote{EvalPrintout}}}{\emph{\DUrole{n}{avg\_every}\DUrole{p}{:}\DUrole{w}{  }\DUrole{n}{int}}}{}
\sphinxAtStartPar
Bases: \sphinxcode{\sphinxupquote{object}}
\index{calculate\_metrics() (yawning\_titan.envs.generic.helpers.eval\_printout.EvalPrintout method)@\spxentry{calculate\_metrics()}\spxextra{yawning\_titan.envs.generic.helpers.eval\_printout.EvalPrintout method}}

\begin{fulllineitems}
\phantomsection\label{\detokenize{source/yawning_titan.envs.generic.helpers:yawning_titan.envs.generic.helpers.eval_printout.EvalPrintout.calculate_metrics}}\pysiglinewithargsret{\sphinxbfcode{\sphinxupquote{calculate\_metrics}}}{\emph{\DUrole{n}{game\_stats\_list}\DUrole{p}{:}\DUrole{w}{  }\DUrole{n}{List\DUrole{p}{{[}}dict\DUrole{p}{{]}}}}}{{ $\rightarrow$ Tuple\DUrole{p}{{[}}int\DUrole{p}{,}\DUrole{w}{  }int\DUrole{p}{,}\DUrole{w}{  }float\DUrole{p}{,}\DUrole{w}{  }float\DUrole{p}{,}\DUrole{w}{  }int\DUrole{p}{,}\DUrole{w}{  }list\DUrole{p}{{]}}}}\begin{quote}\begin{description}
\item[{Parameters}] \leavevmode
\sphinxAtStartPar
\sphinxstyleliteralstrong{\sphinxupquote{game\_stats\_list}} \textendash{} List of dictionaries containing the last avg\_every number of game stats

\item[{Returns}] \leavevmode
\sphinxAtStartPar

\sphinxAtStartPar
Number of games blue won in the last avg\_every number of games
red\_wins: Number of games red won in the last avg\_every number of games
percentage\_blue: Percentage of games blue won in the last avg\_every number of games
percentage\_red: Percentage of games red won in the last avg\_every number of games
avg\_duration: Average number of timesteps per episode over the last avg\_every number of games
sorted\_actions: Dictionary of actions taken by blue, averaged over the last avg\_every number of games
\begin{quote}

\sphinxAtStartPar
and ordered by frequency of each action from highest to lowest. Dictionary values are
tuples: (average frequency of action, action usage percentage)
\end{quote}


\item[{Return type}] \leavevmode
\sphinxAtStartPar
blue\_wins

\end{description}\end{quote}

\end{fulllineitems}

\index{print\_stats() (yawning\_titan.envs.generic.helpers.eval\_printout.EvalPrintout method)@\spxentry{print\_stats()}\spxextra{yawning\_titan.envs.generic.helpers.eval\_printout.EvalPrintout method}}

\begin{fulllineitems}
\phantomsection\label{\detokenize{source/yawning_titan.envs.generic.helpers:yawning_titan.envs.generic.helpers.eval_printout.EvalPrintout.print_stats}}\pysiglinewithargsret{\sphinxbfcode{\sphinxupquote{print\_stats}}}{\emph{\DUrole{n}{game\_stats\_list}\DUrole{p}{:}\DUrole{w}{  }\DUrole{n}{List\DUrole{p}{{[}}dict\DUrole{p}{{]}}}}, \emph{\DUrole{n}{total\_games}\DUrole{p}{:}\DUrole{w}{  }\DUrole{n}{int}}}{}
\sphinxAtStartPar
Prints out the (averaged) stats from the last avg\_every number of games to the console
\begin{quote}\begin{description}
\item[{Parameters}] \leavevmode\begin{itemize}
\item {}
\sphinxAtStartPar
\sphinxstyleliteralstrong{\sphinxupquote{game\_stats\_list}} \textendash{} List of dictionaries containing the last avg\_every number of game stats

\item {}
\sphinxAtStartPar
\sphinxstyleliteralstrong{\sphinxupquote{total\_games}} \textendash{} Total games played since starting

\end{itemize}

\end{description}\end{quote}

\end{fulllineitems}


\end{fulllineitems}



\subparagraph{yawning\_titan.envs.generic.helpers.graph2plot module}
\label{\detokenize{source/yawning_titan.envs.generic.helpers:module-yawning_titan.envs.generic.helpers.graph2plot}}\label{\detokenize{source/yawning_titan.envs.generic.helpers:yawning-titan-envs-generic-helpers-graph2plot-module}}\index{module@\spxentry{module}!yawning\_titan.envs.generic.helpers.graph2plot@\spxentry{yawning\_titan.envs.generic.helpers.graph2plot}}\index{yawning\_titan.envs.generic.helpers.graph2plot@\spxentry{yawning\_titan.envs.generic.helpers.graph2plot}!module@\spxentry{module}}\index{CustomEnvGraph (class in yawning\_titan.envs.generic.helpers.graph2plot)@\spxentry{CustomEnvGraph}\spxextra{class in yawning\_titan.envs.generic.helpers.graph2plot}}

\begin{fulllineitems}
\phantomsection\label{\detokenize{source/yawning_titan.envs.generic.helpers:yawning_titan.envs.generic.helpers.graph2plot.CustomEnvGraph}}\pysiglinewithargsret{\sphinxbfcode{\sphinxupquote{class\DUrole{w}{  }}}\sphinxcode{\sphinxupquote{yawning\_titan.envs.generic.helpers.graph2plot.}}\sphinxbfcode{\sphinxupquote{CustomEnvGraph}}}{\emph{\DUrole{n}{title}\DUrole{p}{:}\DUrole{w}{  }\DUrole{n}{Optional\DUrole{p}{{[}}str\DUrole{p}{{]}}}\DUrole{w}{  }\DUrole{o}{=}\DUrole{w}{  }\DUrole{default_value}{None}}}{}
\sphinxAtStartPar
Bases: \sphinxcode{\sphinxupquote{object}}

\sphinxAtStartPar
A network graph rendering environment for Open AI Gym use
\index{close() (yawning\_titan.envs.generic.helpers.graph2plot.CustomEnvGraph method)@\spxentry{close()}\spxextra{yawning\_titan.envs.generic.helpers.graph2plot.CustomEnvGraph method}}

\begin{fulllineitems}
\phantomsection\label{\detokenize{source/yawning_titan.envs.generic.helpers:yawning_titan.envs.generic.helpers.graph2plot.CustomEnvGraph.close}}\pysiglinewithargsret{\sphinxbfcode{\sphinxupquote{close}}}{}{}
\sphinxAtStartPar
Close all handles to external renderers

\end{fulllineitems}

\index{render() (yawning\_titan.envs.generic.helpers.graph2plot.CustomEnvGraph method)@\spxentry{render()}\spxextra{yawning\_titan.envs.generic.helpers.graph2plot.CustomEnvGraph method}}

\begin{fulllineitems}
\phantomsection\label{\detokenize{source/yawning_titan.envs.generic.helpers:yawning_titan.envs.generic.helpers.graph2plot.CustomEnvGraph.render}}\pysiglinewithargsret{\sphinxbfcode{\sphinxupquote{render}}}{\emph{\DUrole{n}{current\_step}\DUrole{p}{:}\DUrole{w}{  }\DUrole{n}{int}}, \emph{\DUrole{n}{g}\DUrole{p}{:}\DUrole{w}{  }\DUrole{n}{networkx.classes.graph.Graph}}, \emph{\DUrole{n}{pos}\DUrole{p}{:}\DUrole{w}{  }\DUrole{n}{dict}}, \emph{\DUrole{n}{compromised\_nodes}\DUrole{p}{:}\DUrole{w}{  }\DUrole{n}{dict}}, \emph{\DUrole{n}{uncompromised\_nodes}\DUrole{p}{:}\DUrole{w}{  }\DUrole{n}{list}}, \emph{\DUrole{n}{attacks}\DUrole{p}{:}\DUrole{w}{  }\DUrole{n}{list}}, \emph{\DUrole{n}{current\_time\_step\_reward}\DUrole{p}{:}\DUrole{w}{  }\DUrole{n}{float}}, \emph{\DUrole{n}{red\_previous\_node}}, \emph{\DUrole{n}{vulnerability\_dict}\DUrole{p}{:}\DUrole{w}{  }\DUrole{n}{dict}}, \emph{\DUrole{n}{made\_safe\_nodes}\DUrole{p}{:}\DUrole{w}{  }\DUrole{n}{list}}, \emph{\DUrole{n}{title}\DUrole{p}{:}\DUrole{w}{  }\DUrole{n}{str}}, \emph{\DUrole{n}{special\_nodes}\DUrole{p}{:}\DUrole{w}{  }\DUrole{n}{Optional\DUrole{p}{{[}}dict\DUrole{p}{{]}}}\DUrole{w}{  }\DUrole{o}{=}\DUrole{w}{  }\DUrole{default_value}{None}}, \emph{\DUrole{n}{entrance\_nodes}\DUrole{p}{:}\DUrole{w}{  }\DUrole{n}{Optional\DUrole{p}{{[}}list\DUrole{p}{{]}}}\DUrole{w}{  }\DUrole{o}{=}\DUrole{w}{  }\DUrole{default_value}{None}}, \emph{\DUrole{n}{show\_only\_blue\_view}\DUrole{p}{:}\DUrole{w}{  }\DUrole{n}{bool}\DUrole{w}{  }\DUrole{o}{=}\DUrole{w}{  }\DUrole{default_value}{False}}, \emph{\DUrole{n}{show\_node\_names}\DUrole{p}{:}\DUrole{w}{  }\DUrole{n}{bool}\DUrole{w}{  }\DUrole{o}{=}\DUrole{w}{  }\DUrole{default_value}{False}}}{}
\sphinxAtStartPar
Render the current network into an axis
\begin{quote}\begin{description}
\item[{Parameters}] \leavevmode\begin{itemize}
\item {}
\sphinxAtStartPar
\sphinxstyleliteralstrong{\sphinxupquote{current\_step}} \textendash{} the current step in the environment (int)

\item {}
\sphinxAtStartPar
\sphinxstyleliteralstrong{\sphinxupquote{g}} \textendash{} a networkx object that stores the current connectivity (networkx graph)

\item {}
\sphinxAtStartPar
\sphinxstyleliteralstrong{\sphinxupquote{pos}} \textendash{} a dictionary that contains the points and their positions

\item {}
\sphinxAtStartPar
\sphinxstyleliteralstrong{\sphinxupquote{compromised\_nodes}} \textendash{} a dictionary of all the compromised nodes and a boolean value if blue can see the intrusion or not

\item {}
\sphinxAtStartPar
\sphinxstyleliteralstrong{\sphinxupquote{uncompromised\_nodes}} \textendash{} a list of all the uncompromised nodes

\item {}
\sphinxAtStartPar
\sphinxstyleliteralstrong{\sphinxupquote{attacks}} \textendash{} a list of the nodes where an attack is happening
(infected node, target node)

\item {}
\sphinxAtStartPar
\sphinxstyleliteralstrong{\sphinxupquote{current\_time\_step\_reward}} \textendash{} the current total reward

\item {}
\sphinxAtStartPar
\sphinxstyleliteralstrong{\sphinxupquote{red\_previous\_node}} \textendash{} CURRENTLY NOT USED

\item {}
\sphinxAtStartPar
\sphinxstyleliteralstrong{\sphinxupquote{vulnerability\_dict}} \textendash{} A dictionary that stores the vulnerability of the nodes

\item {}
\sphinxAtStartPar
\sphinxstyleliteralstrong{\sphinxupquote{made\_safe\_nodes}} \textendash{} a list of nodes that the blue agent has made safe this turn

\item {}
\sphinxAtStartPar
\sphinxstyleliteralstrong{\sphinxupquote{title}} \textendash{} The title for the render

\item {}
\sphinxAtStartPar
\sphinxstyleliteralstrong{\sphinxupquote{special\_nodes}} \textendash{} A dictionary containing dictionaries of: nodes, node descriptions and colours for the nodes

\item {}
\sphinxAtStartPar
\sphinxstyleliteralstrong{\sphinxupquote{entrance\_nodes}} \textendash{} Nodes that serve as a gateway for the red agent to be able to access the network

\item {}
\sphinxAtStartPar
\sphinxstyleliteralstrong{\sphinxupquote{show\_only\_blue\_view}} \textendash{} If true only shows what the blue agent can see

\item {}
\sphinxAtStartPar
\sphinxstyleliteralstrong{\sphinxupquote{show\_node\_names}} \textendash{} Show the names of nodes

\end{itemize}

\end{description}\end{quote}

\end{fulllineitems}


\end{fulllineitems}



\subparagraph{yawning\_titan.envs.generic.helpers.network\_creator module}
\label{\detokenize{source/yawning_titan.envs.generic.helpers:module-yawning_titan.envs.generic.helpers.network_creator}}\label{\detokenize{source/yawning_titan.envs.generic.helpers:yawning-titan-envs-generic-helpers-network-creator-module}}\index{module@\spxentry{module}!yawning\_titan.envs.generic.helpers.network\_creator@\spxentry{yawning\_titan.envs.generic.helpers.network\_creator}}\index{yawning\_titan.envs.generic.helpers.network\_creator@\spxentry{yawning\_titan.envs.generic.helpers.network\_creator}!module@\spxentry{module}}\index{check\_nearby\_in() (in module yawning\_titan.envs.generic.helpers.network\_creator)@\spxentry{check\_nearby\_in()}\spxextra{in module yawning\_titan.envs.generic.helpers.network\_creator}}

\begin{fulllineitems}
\phantomsection\label{\detokenize{source/yawning_titan.envs.generic.helpers:yawning_titan.envs.generic.helpers.network_creator.check_if_nearby}}\pysiglinewithargsret{\sphinxcode{\sphinxupquote{yawning\_titan.envs.generic.helpers.network\_creator.}}\sphinxbfcode{\sphinxupquote{check\_nearby\_in}}}{\emph{\DUrole{n}{pos}\DUrole{p}{:}\DUrole{w}{  }\DUrole{n}{List\DUrole{p}{{[}}float\DUrole{p}{{]}}}}, \emph{\DUrole{n}{full\_list}\DUrole{p}{:}\DUrole{w}{  }\DUrole{n}{dict}}, \emph{\DUrole{n}{value}\DUrole{p}{:}\DUrole{w}{  }\DUrole{n}{int}}}{{ $\rightarrow$ bool}}
\end{fulllineitems}

\index{create\_18\_node\_network() (in module yawning\_titan.envs.generic.helpers.network\_creator)@\spxentry{create\_18\_node\_network()}\spxextra{in module yawning\_titan.envs.generic.helpers.network\_creator}}

\begin{fulllineitems}
\phantomsection\label{\detokenize{source/yawning_titan.envs.generic.helpers:yawning_titan.envs.generic.helpers.network_creator.create_18_node_network}}\pysiglinewithargsret{\sphinxcode{\sphinxupquote{yawning\_titan.envs.generic.helpers.network\_creator.}}\sphinxbfcode{\sphinxupquote{create\_18\_node\_network}}}{}{{ $\rightarrow$ Tuple\DUrole{p}{{[}}Union\DUrole{p}{{[}}int\DUrole{p}{,}\DUrole{w}{  }float\DUrole{p}{,}\DUrole{w}{  }complex\DUrole{p}{,}\DUrole{w}{  }str\DUrole{p}{,}\DUrole{w}{  }bytes\DUrole{p}{,}\DUrole{w}{  }numpy.generic\DUrole{p}{,}\DUrole{w}{  }Sequence\DUrole{p}{{[}}Union\DUrole{p}{{[}}int\DUrole{p}{,}\DUrole{w}{  }float\DUrole{p}{,}\DUrole{w}{  }complex\DUrole{p}{,}\DUrole{w}{  }str\DUrole{p}{,}\DUrole{w}{  }bytes\DUrole{p}{,}\DUrole{w}{  }numpy.generic\DUrole{p}{{]}}\DUrole{p}{{]}}\DUrole{p}{,}\DUrole{w}{  }Sequence\DUrole{p}{{[}}Sequence\DUrole{p}{{[}}Any\DUrole{p}{{]}}\DUrole{p}{{]}}\DUrole{p}{,}\DUrole{w}{  }numpy.typing.\_array\_like.\_SupportsArray\DUrole{p}{{]}}\DUrole{p}{,}\DUrole{w}{  }dict\DUrole{p}{{]}}}}
\sphinxAtStartPar
Creates the standard 18 node network found in the research paper
\begin{quote}\begin{description}
\item[{Returns}] \leavevmode
\sphinxAtStartPar
The adjacency matrix that represents the network
A dictionary of positions of the nodes

\end{description}\end{quote}

\end{fulllineitems}

\index{create\_mesh() (in module yawning\_titan.envs.generic.helpers.network\_creator)@\spxentry{create\_mesh()}\spxextra{in module yawning\_titan.envs.generic.helpers.network\_creator}}

\begin{fulllineitems}
\phantomsection\label{\detokenize{source/yawning_titan.envs.generic.helpers:yawning_titan.envs.generic.helpers.network_creator.create_mesh}}\pysiglinewithargsret{\sphinxcode{\sphinxupquote{yawning\_titan.envs.generic.helpers.network\_creator.}}\sphinxbfcode{\sphinxupquote{create\_mesh}}}{\emph{\DUrole{n}{size}\DUrole{p}{:}\DUrole{w}{  }\DUrole{n}{int}\DUrole{w}{  }\DUrole{o}{=}\DUrole{w}{  }\DUrole{default_value}{100}}, \emph{\DUrole{n}{connectivity}\DUrole{p}{:}\DUrole{w}{  }\DUrole{n}{float}\DUrole{w}{  }\DUrole{o}{=}\DUrole{w}{  }\DUrole{default_value}{0.7}}}{{ $\rightarrow$ Tuple\DUrole{p}{{[}}Union\DUrole{p}{{[}}int\DUrole{p}{,}\DUrole{w}{  }float\DUrole{p}{,}\DUrole{w}{  }complex\DUrole{p}{,}\DUrole{w}{  }str\DUrole{p}{,}\DUrole{w}{  }bytes\DUrole{p}{,}\DUrole{w}{  }numpy.generic\DUrole{p}{,}\DUrole{w}{  }Sequence\DUrole{p}{{[}}Union\DUrole{p}{{[}}int\DUrole{p}{,}\DUrole{w}{  }float\DUrole{p}{,}\DUrole{w}{  }complex\DUrole{p}{,}\DUrole{w}{  }str\DUrole{p}{,}\DUrole{w}{  }bytes\DUrole{p}{,}\DUrole{w}{  }numpy.generic\DUrole{p}{{]}}\DUrole{p}{{]}}\DUrole{p}{,}\DUrole{w}{  }Sequence\DUrole{p}{{[}}Sequence\DUrole{p}{{[}}Any\DUrole{p}{{]}}\DUrole{p}{{]}}\DUrole{p}{,}\DUrole{w}{  }numpy.typing.\_array\_like.\_SupportsArray\DUrole{p}{{]}}\DUrole{p}{,}\DUrole{w}{  }dict\DUrole{p}{{]}}}}
\sphinxAtStartPar
Creates a mesh node environment
\begin{quote}\begin{description}
\item[{Parameters}] \leavevmode\begin{itemize}
\item {}
\sphinxAtStartPar
\sphinxstyleliteralstrong{\sphinxupquote{size}} \textendash{} the number of nodes in the environment

\item {}
\sphinxAtStartPar
\sphinxstyleliteralstrong{\sphinxupquote{connectivity}} \textendash{} how connected each of the nodes should be (percentage chance for any node to be connected to

\item {}
\sphinxAtStartPar
\sphinxstyleliteralstrong{\sphinxupquote{other}}\sphinxstyleliteralstrong{\sphinxupquote{)}} (\sphinxstyleliteralemphasis{\sphinxupquote{any}}) \textendash{}

\end{itemize}

\item[{Returns}] \leavevmode
\sphinxAtStartPar
The adjacency matrix that represents the network
A dictionary of positions of the nodes

\end{description}\end{quote}

\end{fulllineitems}

\index{create\_p2p() (in module yawning\_titan.envs.generic.helpers.network\_creator)@\spxentry{create\_p2p()}\spxextra{in module yawning\_titan.envs.generic.helpers.network\_creator}}

\begin{fulllineitems}
\phantomsection\label{\detokenize{source/yawning_titan.envs.generic.helpers:yawning_titan.envs.generic.helpers.network_creator.create_p2p}}\pysiglinewithargsret{\sphinxcode{\sphinxupquote{yawning\_titan.envs.generic.helpers.network\_creator.}}\sphinxbfcode{\sphinxupquote{create\_p2p}}}{\emph{\DUrole{n}{group\_size}\DUrole{p}{:}\DUrole{w}{  }\DUrole{n}{int}\DUrole{w}{  }\DUrole{o}{=}\DUrole{w}{  }\DUrole{default_value}{5}}, \emph{\DUrole{n}{inter\_group\_connectivity}\DUrole{p}{:}\DUrole{w}{  }\DUrole{n}{float}\DUrole{w}{  }\DUrole{o}{=}\DUrole{w}{  }\DUrole{default_value}{0.1}}, \emph{\DUrole{n}{group\_connectivity}\DUrole{p}{:}\DUrole{w}{  }\DUrole{n}{int}\DUrole{w}{  }\DUrole{o}{=}\DUrole{w}{  }\DUrole{default_value}{1}}}{{ $\rightarrow$ Tuple\DUrole{p}{{[}}Union\DUrole{p}{{[}}int\DUrole{p}{,}\DUrole{w}{  }float\DUrole{p}{,}\DUrole{w}{  }complex\DUrole{p}{,}\DUrole{w}{  }str\DUrole{p}{,}\DUrole{w}{  }bytes\DUrole{p}{,}\DUrole{w}{  }numpy.generic\DUrole{p}{,}\DUrole{w}{  }Sequence\DUrole{p}{{[}}Union\DUrole{p}{{[}}int\DUrole{p}{,}\DUrole{w}{  }float\DUrole{p}{,}\DUrole{w}{  }complex\DUrole{p}{,}\DUrole{w}{  }str\DUrole{p}{,}\DUrole{w}{  }bytes\DUrole{p}{,}\DUrole{w}{  }numpy.generic\DUrole{p}{{]}}\DUrole{p}{{]}}\DUrole{p}{,}\DUrole{w}{  }Sequence\DUrole{p}{{[}}Sequence\DUrole{p}{{[}}Any\DUrole{p}{{]}}\DUrole{p}{{]}}\DUrole{p}{,}\DUrole{w}{  }numpy.typing.\_array\_like.\_SupportsArray\DUrole{p}{{]}}\DUrole{p}{,}\DUrole{w}{  }dict\DUrole{p}{{]}}}}
\sphinxAtStartPar
Creates a two group network. You can modify the connectivity between the two groups and the connectivity within the
groups
\begin{quote}\begin{description}
\item[{Parameters}] \leavevmode\begin{itemize}
\item {}
\sphinxAtStartPar
\sphinxstyleliteralstrong{\sphinxupquote{group\_size}} \textendash{} the amount of nodes in each group (before random variance)

\item {}
\sphinxAtStartPar
\sphinxstyleliteralstrong{\sphinxupquote{inter\_group\_connectivity}} \textendash{} the connectivity between the two groups

\item {}
\sphinxAtStartPar
\sphinxstyleliteralstrong{\sphinxupquote{group\_connectivity}} \textendash{} the connectivity within the group

\end{itemize}

\item[{Returns}] \leavevmode
\sphinxAtStartPar
The adjacency matrix that represents the network
A dictionary of positions of the nodes

\end{description}\end{quote}

\end{fulllineitems}

\index{create\_ring() (in module yawning\_titan.envs.generic.helpers.network\_creator)@\spxentry{create\_ring()}\spxextra{in module yawning\_titan.envs.generic.helpers.network\_creator}}

\begin{fulllineitems}
\phantomsection\label{\detokenize{source/yawning_titan.envs.generic.helpers:yawning_titan.envs.generic.helpers.network_creator.create_ring}}\pysiglinewithargsret{\sphinxcode{\sphinxupquote{yawning\_titan.envs.generic.helpers.network\_creator.}}\sphinxbfcode{\sphinxupquote{create\_ring}}}{\emph{\DUrole{n}{break\_probability}\DUrole{p}{:}\DUrole{w}{  }\DUrole{n}{float}\DUrole{w}{  }\DUrole{o}{=}\DUrole{w}{  }\DUrole{default_value}{0.3}}, \emph{\DUrole{n}{ring\_size}\DUrole{p}{:}\DUrole{w}{  }\DUrole{n}{int}\DUrole{w}{  }\DUrole{o}{=}\DUrole{w}{  }\DUrole{default_value}{60}}}{{ $\rightarrow$ Tuple\DUrole{p}{{[}}Union\DUrole{p}{{[}}int\DUrole{p}{,}\DUrole{w}{  }float\DUrole{p}{,}\DUrole{w}{  }complex\DUrole{p}{,}\DUrole{w}{  }str\DUrole{p}{,}\DUrole{w}{  }bytes\DUrole{p}{,}\DUrole{w}{  }numpy.generic\DUrole{p}{,}\DUrole{w}{  }Sequence\DUrole{p}{{[}}Union\DUrole{p}{{[}}int\DUrole{p}{,}\DUrole{w}{  }float\DUrole{p}{,}\DUrole{w}{  }complex\DUrole{p}{,}\DUrole{w}{  }str\DUrole{p}{,}\DUrole{w}{  }bytes\DUrole{p}{,}\DUrole{w}{  }numpy.generic\DUrole{p}{{]}}\DUrole{p}{{]}}\DUrole{p}{,}\DUrole{w}{  }Sequence\DUrole{p}{{[}}Sequence\DUrole{p}{{[}}Any\DUrole{p}{{]}}\DUrole{p}{{]}}\DUrole{p}{,}\DUrole{w}{  }numpy.typing.\_array\_like.\_SupportsArray\DUrole{p}{{]}}\DUrole{p}{,}\DUrole{w}{  }dict\DUrole{p}{{]}}}}
\sphinxAtStartPar
Creates a ring network
\begin{quote}\begin{description}
\item[{Parameters}] \leavevmode\begin{itemize}
\item {}
\sphinxAtStartPar
\sphinxstyleliteralstrong{\sphinxupquote{break\_probability}} \textendash{} the probability that two nodes will not be connected

\item {}
\sphinxAtStartPar
\sphinxstyleliteralstrong{\sphinxupquote{ring\_size}} \textendash{} the number of nodes in the network

\end{itemize}

\item[{Returns}] \leavevmode
\sphinxAtStartPar
The adjacency matrix that represents the network
A dictionary of positions of the nodes

\end{description}\end{quote}

\end{fulllineitems}

\index{create\_star() (in module yawning\_titan.envs.generic.helpers.network\_creator)@\spxentry{create\_star()}\spxextra{in module yawning\_titan.envs.generic.helpers.network\_creator}}

\begin{fulllineitems}
\phantomsection\label{\detokenize{source/yawning_titan.envs.generic.helpers:yawning_titan.envs.generic.helpers.network_creator.create_star}}\pysiglinewithargsret{\sphinxcode{\sphinxupquote{yawning\_titan.envs.generic.helpers.network\_creator.}}\sphinxbfcode{\sphinxupquote{create\_star}}}{\emph{\DUrole{n}{first\_layer\_size}\DUrole{p}{:}\DUrole{w}{  }\DUrole{n}{int}\DUrole{w}{  }\DUrole{o}{=}\DUrole{w}{  }\DUrole{default_value}{8}}, \emph{\DUrole{n}{group\_size}\DUrole{p}{:}\DUrole{w}{  }\DUrole{n}{int}\DUrole{w}{  }\DUrole{o}{=}\DUrole{w}{  }\DUrole{default_value}{5}}, \emph{\DUrole{n}{group\_connectivity}\DUrole{p}{:}\DUrole{w}{  }\DUrole{n}{float}\DUrole{w}{  }\DUrole{o}{=}\DUrole{w}{  }\DUrole{default_value}{0.5}}}{{ $\rightarrow$ Tuple\DUrole{p}{{[}}Union\DUrole{p}{{[}}int\DUrole{p}{,}\DUrole{w}{  }float\DUrole{p}{,}\DUrole{w}{  }complex\DUrole{p}{,}\DUrole{w}{  }str\DUrole{p}{,}\DUrole{w}{  }bytes\DUrole{p}{,}\DUrole{w}{  }numpy.generic\DUrole{p}{,}\DUrole{w}{  }Sequence\DUrole{p}{{[}}Union\DUrole{p}{{[}}int\DUrole{p}{,}\DUrole{w}{  }float\DUrole{p}{,}\DUrole{w}{  }complex\DUrole{p}{,}\DUrole{w}{  }str\DUrole{p}{,}\DUrole{w}{  }bytes\DUrole{p}{,}\DUrole{w}{  }numpy.generic\DUrole{p}{{]}}\DUrole{p}{{]}}\DUrole{p}{,}\DUrole{w}{  }Sequence\DUrole{p}{{[}}Sequence\DUrole{p}{{[}}Any\DUrole{p}{{]}}\DUrole{p}{{]}}\DUrole{p}{,}\DUrole{w}{  }numpy.typing.\_array\_like.\_SupportsArray\DUrole{p}{{]}}\DUrole{p}{,}\DUrole{w}{  }dict\DUrole{p}{{]}}}}
\sphinxAtStartPar
Creates a star node environment. This is one node in the middle with groups of nodes around it. There is only one
connection between a group and the center node. Groups cannot connect to each other.
\begin{quote}\begin{description}
\item[{Parameters}] \leavevmode\begin{itemize}
\item {}
\sphinxAtStartPar
\sphinxstyleliteralstrong{\sphinxupquote{first\_layer\_size}} \textendash{} the number of collections of nodes in first “outer ring”

\item {}
\sphinxAtStartPar
\sphinxstyleliteralstrong{\sphinxupquote{group\_size}} \textendash{} how many nodes are in each collection

\item {}
\sphinxAtStartPar
\sphinxstyleliteralstrong{\sphinxupquote{group\_connectivity}} \textendash{} how connected the nodes in the connections are

\end{itemize}

\item[{Returns}] \leavevmode
\sphinxAtStartPar
The adjacency matrix that represents the network
A dictionary of positions of the nodes

\end{description}\end{quote}

\end{fulllineitems}

\index{custom\_network() (in module yawning\_titan.envs.generic.helpers.network\_creator)@\spxentry{custom\_network()}\spxextra{in module yawning\_titan.envs.generic.helpers.network\_creator}}

\begin{fulllineitems}
\phantomsection\label{\detokenize{source/yawning_titan.envs.generic.helpers:yawning_titan.envs.generic.helpers.network_creator.custom_network}}\pysiglinewithargsret{\sphinxcode{\sphinxupquote{yawning\_titan.envs.generic.helpers.network\_creator.}}\sphinxbfcode{\sphinxupquote{custom\_network}}}{}{{ $\rightarrow$ Optional\DUrole{p}{{[}}Tuple\DUrole{p}{{[}}Union\DUrole{p}{{[}}int\DUrole{p}{,}\DUrole{w}{  }float\DUrole{p}{,}\DUrole{w}{  }complex\DUrole{p}{,}\DUrole{w}{  }str\DUrole{p}{,}\DUrole{w}{  }bytes\DUrole{p}{,}\DUrole{w}{  }numpy.generic\DUrole{p}{,}\DUrole{w}{  }Sequence\DUrole{p}{{[}}Union\DUrole{p}{{[}}int\DUrole{p}{,}\DUrole{w}{  }float\DUrole{p}{,}\DUrole{w}{  }complex\DUrole{p}{,}\DUrole{w}{  }str\DUrole{p}{,}\DUrole{w}{  }bytes\DUrole{p}{,}\DUrole{w}{  }numpy.generic\DUrole{p}{{]}}\DUrole{p}{{]}}\DUrole{p}{,}\DUrole{w}{  }Sequence\DUrole{p}{{[}}Sequence\DUrole{p}{{[}}Any\DUrole{p}{{]}}\DUrole{p}{{]}}\DUrole{p}{,}\DUrole{w}{  }numpy.typing.\_array\_like.\_SupportsArray\DUrole{p}{{]}}\DUrole{p}{,}\DUrole{w}{  }dict\DUrole{p}{{]}}\DUrole{p}{{]}}}}
\sphinxAtStartPar
Creates custom network through user interaction
\begin{quote}\begin{description}
\item[{Returns}] \leavevmode
\sphinxAtStartPar
The adjacency matrix that represents the network
A dictionary of positions of the nodes

\end{description}\end{quote}

\end{fulllineitems}

\index{generate\_node\_positions() (in module yawning\_titan.envs.generic.helpers.network\_creator)@\spxentry{generate\_node\_positions()}\spxextra{in module yawning\_titan.envs.generic.helpers.network\_creator}}

\begin{fulllineitems}
\phantomsection\label{\detokenize{source/yawning_titan.envs.generic.helpers:yawning_titan.envs.generic.helpers.network_creator.generate_node_positions}}\pysiglinewithargsret{\sphinxcode{\sphinxupquote{yawning\_titan.envs.generic.helpers.network\_creator.}}\sphinxbfcode{\sphinxupquote{generate\_node\_positions}}}{\emph{\DUrole{n}{adj\_matrix}\DUrole{p}{:}\DUrole{w}{  }\DUrole{n}{Union\DUrole{p}{{[}}int\DUrole{p}{,}\DUrole{w}{  }float\DUrole{p}{,}\DUrole{w}{  }complex\DUrole{p}{,}\DUrole{w}{  }str\DUrole{p}{,}\DUrole{w}{  }bytes\DUrole{p}{,}\DUrole{w}{  }numpy.generic\DUrole{p}{,}\DUrole{w}{  }Sequence\DUrole{p}{{[}}Union\DUrole{p}{{[}}int\DUrole{p}{,}\DUrole{w}{  }float\DUrole{p}{,}\DUrole{w}{  }complex\DUrole{p}{,}\DUrole{w}{  }str\DUrole{p}{,}\DUrole{w}{  }bytes\DUrole{p}{,}\DUrole{w}{  }numpy.generic\DUrole{p}{{]}}\DUrole{p}{{]}}\DUrole{p}{,}\DUrole{w}{  }Sequence\DUrole{p}{{[}}Sequence\DUrole{p}{{[}}Any\DUrole{p}{{]}}\DUrole{p}{{]}}\DUrole{p}{,}\DUrole{w}{  }numpy.typing.\_array\_like.\_SupportsArray\DUrole{p}{{]}}}}}{{ $\rightarrow$ dict}}
\sphinxAtStartPar
Generates a random position for each node and saves it as a dictionary
\begin{quote}\begin{description}
\item[{Parameters}] \leavevmode
\sphinxAtStartPar
\sphinxstyleliteralstrong{\sphinxupquote{adj\_matrix}} \textendash{} The adjacency matrix for the network

\item[{Returns}] \leavevmode
\sphinxAtStartPar
A dictionary of node positions

\end{description}\end{quote}

\end{fulllineitems}

\index{gnp\_random\_connected\_graph() (in module yawning\_titan.envs.generic.helpers.network\_creator)@\spxentry{gnp\_random\_connected\_graph()}\spxextra{in module yawning\_titan.envs.generic.helpers.network\_creator}}

\begin{fulllineitems}
\phantomsection\label{\detokenize{source/yawning_titan.envs.generic.helpers:yawning_titan.envs.generic.helpers.network_creator.gnp_random_connected_graph}}\pysiglinewithargsret{\sphinxcode{\sphinxupquote{yawning\_titan.envs.generic.helpers.network\_creator.}}\sphinxbfcode{\sphinxupquote{gnp\_random\_connected\_graph}}}{\emph{\DUrole{n}{n\_nodes}\DUrole{p}{:}\DUrole{w}{  }\DUrole{n}{int}}, \emph{\DUrole{n}{probability\_of\_edge}\DUrole{p}{:}\DUrole{w}{  }\DUrole{n}{float}}}{{ $\rightarrow$ Optional\DUrole{p}{{[}}Tuple\DUrole{p}{{[}}Union\DUrole{p}{{[}}int\DUrole{p}{,}\DUrole{w}{  }float\DUrole{p}{,}\DUrole{w}{  }complex\DUrole{p}{,}\DUrole{w}{  }str\DUrole{p}{,}\DUrole{w}{  }bytes\DUrole{p}{,}\DUrole{w}{  }numpy.generic\DUrole{p}{,}\DUrole{w}{  }Sequence\DUrole{p}{{[}}Union\DUrole{p}{{[}}int\DUrole{p}{,}\DUrole{w}{  }float\DUrole{p}{,}\DUrole{w}{  }complex\DUrole{p}{,}\DUrole{w}{  }str\DUrole{p}{,}\DUrole{w}{  }bytes\DUrole{p}{,}\DUrole{w}{  }numpy.generic\DUrole{p}{{]}}\DUrole{p}{{]}}\DUrole{p}{,}\DUrole{w}{  }Sequence\DUrole{p}{{[}}Sequence\DUrole{p}{{[}}Any\DUrole{p}{{]}}\DUrole{p}{{]}}\DUrole{p}{,}\DUrole{w}{  }numpy.typing.\_array\_like.\_SupportsArray\DUrole{p}{{]}}\DUrole{p}{,}\DUrole{w}{  }dict\DUrole{p}{{]}}\DUrole{p}{{]}}}}
\sphinxAtStartPar
Creates a randomly connected graph but with the guarntee that
each node will have at least on connection

\sphinxAtStartPar
This is taken from the following stack overflow Q\&A with a bit of a refactor
for clarity
\begin{quote}\begin{description}
\item[{Parameters}] \leavevmode\begin{itemize}
\item {}
\sphinxAtStartPar
\sphinxstyleliteralstrong{\sphinxupquote{n\_nodes}} \textendash{} the number of nodes in the graph

\item {}
\sphinxAtStartPar
\sphinxstyleliteralstrong{\sphinxupquote{probability\_of\_edge}} \textendash{} the probability for a node to have an edge

\end{itemize}

\item[{Returns}] \leavevmode
\sphinxAtStartPar
The adjacency matrix that represents the network
A dictionary of positions of the nodes

\end{description}\end{quote}

\end{fulllineitems}

\index{load\_network() (in module yawning\_titan.envs.generic.helpers.network\_creator)@\spxentry{load\_network()}\spxextra{in module yawning\_titan.envs.generic.helpers.network\_creator}}

\begin{fulllineitems}
\phantomsection\label{\detokenize{source/yawning_titan.envs.generic.helpers:yawning_titan.envs.generic.helpers.network_creator.load_network}}\pysiglinewithargsret{\sphinxcode{\sphinxupquote{yawning\_titan.envs.generic.helpers.network\_creator.}}\sphinxbfcode{\sphinxupquote{load\_network}}}{\emph{\DUrole{n}{network\_name}\DUrole{p}{:}\DUrole{w}{  }\DUrole{n}{str}}}{{ $\rightarrow$ Tuple\DUrole{p}{{[}}Union\DUrole{p}{{[}}int\DUrole{p}{,}\DUrole{w}{  }float\DUrole{p}{,}\DUrole{w}{  }complex\DUrole{p}{,}\DUrole{w}{  }str\DUrole{p}{,}\DUrole{w}{  }bytes\DUrole{p}{,}\DUrole{w}{  }numpy.generic\DUrole{p}{,}\DUrole{w}{  }Sequence\DUrole{p}{{[}}Union\DUrole{p}{{[}}int\DUrole{p}{,}\DUrole{w}{  }float\DUrole{p}{,}\DUrole{w}{  }complex\DUrole{p}{,}\DUrole{w}{  }str\DUrole{p}{,}\DUrole{w}{  }bytes\DUrole{p}{,}\DUrole{w}{  }numpy.generic\DUrole{p}{{]}}\DUrole{p}{{]}}\DUrole{p}{,}\DUrole{w}{  }Sequence\DUrole{p}{{[}}Sequence\DUrole{p}{{[}}Any\DUrole{p}{{]}}\DUrole{p}{{]}}\DUrole{p}{,}\DUrole{w}{  }numpy.typing.\_array\_like.\_SupportsArray\DUrole{p}{{]}}\DUrole{p}{,}\DUrole{w}{  }dict\DUrole{p}{{]}}}}
\sphinxAtStartPar
Loads a saved network so that it can be used by the generic network environment
\begin{quote}\begin{description}
\item[{Parameters}] \leavevmode
\sphinxAtStartPar
\sphinxstyleliteralstrong{\sphinxupquote{network\_name}} \textendash{} The name of the network

\item[{Returns}] \leavevmode
\sphinxAtStartPar
The adjacency matrix
A position dictionary for all the nodes

\end{description}\end{quote}

\end{fulllineitems}

\index{procedural\_network() (in module yawning\_titan.envs.generic.helpers.network\_creator)@\spxentry{procedural\_network()}\spxextra{in module yawning\_titan.envs.generic.helpers.network\_creator}}

\begin{fulllineitems}
\phantomsection\label{\detokenize{source/yawning_titan.envs.generic.helpers:yawning_titan.envs.generic.helpers.network_creator.procedural_network}}\pysiglinewithargsret{\sphinxcode{\sphinxupquote{yawning\_titan.envs.generic.helpers.network\_creator.}}\sphinxbfcode{\sphinxupquote{procedural\_network}}}{\emph{\DUrole{n}{size}\DUrole{p}{:}\DUrole{w}{  }\DUrole{n}{int}\DUrole{w}{  }\DUrole{o}{=}\DUrole{w}{  }\DUrole{default_value}{30}}, \emph{\DUrole{n}{highly\_connected\_nodes}\DUrole{p}{:}\DUrole{w}{  }\DUrole{n}{float}\DUrole{w}{  }\DUrole{o}{=}\DUrole{w}{  }\DUrole{default_value}{0.01}}, \emph{\DUrole{n}{overall\_connectivity}\DUrole{p}{:}\DUrole{w}{  }\DUrole{n}{float}\DUrole{w}{  }\DUrole{o}{=}\DUrole{w}{  }\DUrole{default_value}{0.4}}}{{ $\rightarrow$ Tuple\DUrole{p}{{[}}Union\DUrole{p}{{[}}int\DUrole{p}{,}\DUrole{w}{  }float\DUrole{p}{,}\DUrole{w}{  }complex\DUrole{p}{,}\DUrole{w}{  }str\DUrole{p}{,}\DUrole{w}{  }bytes\DUrole{p}{,}\DUrole{w}{  }numpy.generic\DUrole{p}{,}\DUrole{w}{  }Sequence\DUrole{p}{{[}}Union\DUrole{p}{{[}}int\DUrole{p}{,}\DUrole{w}{  }float\DUrole{p}{,}\DUrole{w}{  }complex\DUrole{p}{,}\DUrole{w}{  }str\DUrole{p}{,}\DUrole{w}{  }bytes\DUrole{p}{,}\DUrole{w}{  }numpy.generic\DUrole{p}{{]}}\DUrole{p}{{]}}\DUrole{p}{,}\DUrole{w}{  }Sequence\DUrole{p}{{[}}Sequence\DUrole{p}{{[}}Any\DUrole{p}{{]}}\DUrole{p}{{]}}\DUrole{p}{,}\DUrole{w}{  }numpy.typing.\_array\_like.\_SupportsArray\DUrole{p}{{]}}\DUrole{p}{,}\DUrole{w}{  }dict\DUrole{p}{{]}}}}\begin{description}
\item[{Types of nodes:}] \leavevmode\begin{itemize}
\item {}
\sphinxAtStartPar
Highly connected node: can connect to up to size/2 nodes

\item {}
\sphinxAtStartPar
Gateway nodes: can connect  to 2 or 3 nodes

\item {}
\sphinxAtStartPar
Single nodes: can only connect to one other node

\item {}
\sphinxAtStartPar
Isolated nodes: cannot connect to any nodes

\end{itemize}

\end{description}
\begin{quote}\begin{description}
\item[{Parameters}] \leavevmode\begin{itemize}
\item {}
\sphinxAtStartPar
\sphinxstyleliteralstrong{\sphinxupquote{size}} \textendash{}

\item {}
\sphinxAtStartPar
\sphinxstyleliteralstrong{\sphinxupquote{highly\_connected\_nodes}} \textendash{}

\item {}
\sphinxAtStartPar
\sphinxstyleliteralstrong{\sphinxupquote{overall\_connectivity}} \textendash{}

\end{itemize}

\item[{Returns}] \leavevmode
\sphinxAtStartPar
The adjacency matrix that represents the network
A dictionary of positions of the nodes

\end{description}\end{quote}

\end{fulllineitems}

\index{save\_network() (in module yawning\_titan.envs.generic.helpers.network\_creator)@\spxentry{save\_network()}\spxextra{in module yawning\_titan.envs.generic.helpers.network\_creator}}

\begin{fulllineitems}
\phantomsection\label{\detokenize{source/yawning_titan.envs.generic.helpers:yawning_titan.envs.generic.helpers.network_creator.save_network}}\pysiglinewithargsret{\sphinxcode{\sphinxupquote{yawning\_titan.envs.generic.helpers.network\_creator.}}\sphinxbfcode{\sphinxupquote{save\_network}}}{\emph{\DUrole{n}{network\_name}\DUrole{p}{:}\DUrole{w}{  }\DUrole{n}{str}}, \emph{\DUrole{n}{adj\_matrix}\DUrole{p}{:}\DUrole{w}{  }\DUrole{n}{Union\DUrole{p}{{[}}int\DUrole{p}{,}\DUrole{w}{  }float\DUrole{p}{,}\DUrole{w}{  }complex\DUrole{p}{,}\DUrole{w}{  }str\DUrole{p}{,}\DUrole{w}{  }bytes\DUrole{p}{,}\DUrole{w}{  }numpy.generic\DUrole{p}{,}\DUrole{w}{  }Sequence\DUrole{p}{{[}}Union\DUrole{p}{{[}}int\DUrole{p}{,}\DUrole{w}{  }float\DUrole{p}{,}\DUrole{w}{  }complex\DUrole{p}{,}\DUrole{w}{  }str\DUrole{p}{,}\DUrole{w}{  }bytes\DUrole{p}{,}\DUrole{w}{  }numpy.generic\DUrole{p}{{]}}\DUrole{p}{{]}}\DUrole{p}{,}\DUrole{w}{  }Sequence\DUrole{p}{{[}}Sequence\DUrole{p}{{[}}Any\DUrole{p}{{]}}\DUrole{p}{{]}}\DUrole{p}{,}\DUrole{w}{  }numpy.typing.\_array\_like.\_SupportsArray\DUrole{p}{{]}}}}, \emph{\DUrole{n}{positions}\DUrole{p}{:}\DUrole{w}{  }\DUrole{n}{dict}}}{}
\sphinxAtStartPar
Saves a network in a text file so that it can be used multiple times
\begin{quote}\begin{description}
\item[{Parameters}] \leavevmode\begin{itemize}
\item {}
\sphinxAtStartPar
\sphinxstyleliteralstrong{\sphinxupquote{network\_name}} \textendash{} The name of the network

\item {}
\sphinxAtStartPar
\sphinxstyleliteralstrong{\sphinxupquote{adj\_matrix}} \textendash{} The adjacency matrix for the network

\item {}
\sphinxAtStartPar
\sphinxstyleliteralstrong{\sphinxupquote{positions}} \textendash{} A dictionary of node positions

\end{itemize}

\end{description}\end{quote}

\end{fulllineitems}



\subparagraph{yawning\_titan.envs.generic.helpers.node\_attribute\_gen module}
\label{\detokenize{source/yawning_titan.envs.generic.helpers:module-yawning_titan.envs.generic.helpers.node_attribute_gen}}\label{\detokenize{source/yawning_titan.envs.generic.helpers:yawning-titan-envs-generic-helpers-node-attribute-gen-module}}\index{module@\spxentry{module}!yawning\_titan.envs.generic.helpers.node\_attribute\_gen@\spxentry{yawning\_titan.envs.generic.helpers.node\_attribute\_gen}}\index{yawning\_titan.envs.generic.helpers.node\_attribute\_gen@\spxentry{yawning\_titan.envs.generic.helpers.node\_attribute\_gen}!module@\spxentry{module}}\index{generate\_vulnerabilities() (in module yawning\_titan.envs.generic.helpers.node\_attribute\_gen)@\spxentry{generate\_vulnerabilities()}\spxextra{in module yawning\_titan.envs.generic.helpers.node\_attribute\_gen}}

\begin{fulllineitems}
\phantomsection\label{\detokenize{source/yawning_titan.envs.generic.helpers:yawning_titan.envs.generic.helpers.node_attribute_gen.generate_vulnerabilities}}\pysiglinewithargsret{\sphinxcode{\sphinxupquote{yawning\_titan.envs.generic.helpers.node\_attribute\_gen.}}\sphinxbfcode{\sphinxupquote{generate\_vulnerabilities}}}{\emph{\DUrole{n}{n\_nodes}\DUrole{p}{:}\DUrole{w}{  }\DUrole{n}{int}}, \emph{\DUrole{n}{settings\_data}\DUrole{p}{:}\DUrole{w}{  }\DUrole{n}{dict}}}{{ $\rightarrow$ dict}}
\sphinxAtStartPar
Generates vulnerability values for n nodes. These values are
randomly generated between the upper and lower bounds within the
settings.
\begin{quote}\begin{description}
\item[{Parameters}] \leavevmode\begin{itemize}
\item {}
\sphinxAtStartPar
\sphinxstyleliteralstrong{\sphinxupquote{n\_nodes}} \textendash{} Number of nodes within the environment

\item {}
\sphinxAtStartPar
\sphinxstyleliteralstrong{\sphinxupquote{settings\_data}} \textendash{} The environment settings object

\end{itemize}

\item[{Returns}] \leavevmode
\sphinxAtStartPar
A dictionary containing the vulnerabilities

\item[{Return type}] \leavevmode
\sphinxAtStartPar
vulnerabilities

\end{description}\end{quote}

\end{fulllineitems}



\subparagraph{Module contents}
\label{\detokenize{source/yawning_titan.envs.generic.helpers:module-yawning_titan.envs.generic.helpers}}\label{\detokenize{source/yawning_titan.envs.generic.helpers:module-contents}}\index{module@\spxentry{module}!yawning\_titan.envs.generic.helpers@\spxentry{yawning\_titan.envs.generic.helpers}}\index{yawning\_titan.envs.generic.helpers@\spxentry{yawning\_titan.envs.generic.helpers}!module@\spxentry{module}}

\subparagraph{yawning\_titan.envs.generic.wrappers package}
\label{\detokenize{source/yawning_titan.envs.generic.wrappers:yawning-titan-envs-generic-wrappers-package}}\label{\detokenize{source/yawning_titan.envs.generic.wrappers::doc}}

\subparagraph{Submodules}
\label{\detokenize{source/yawning_titan.envs.generic.wrappers:submodules}}

\subparagraph{yawning\_titan.envs.generic.wrappers.graph\_embedding\_observations module}
\label{\detokenize{source/yawning_titan.envs.generic.wrappers:module-yawning_titan.envs.generic.wrappers.graph_embedding_observations}}\label{\detokenize{source/yawning_titan.envs.generic.wrappers:yawning-titan-envs-generic-wrappers-graph-embedding-observations-module}}\index{module@\spxentry{module}!yawning\_titan.envs.generic.wrappers.graph\_embedding\_observations@\spxentry{yawning\_titan.envs.generic.wrappers.graph\_embedding\_observations}}\index{yawning\_titan.envs.generic.wrappers.graph\_embedding\_observations@\spxentry{yawning\_titan.envs.generic.wrappers.graph\_embedding\_observations}!module@\spxentry{module}}\index{FeatherGraphEmbedObservation (class in yawning\_titan.envs.generic.wrappers.graph\_embedding\_observations)@\spxentry{FeatherGraphEmbedObservation}\spxextra{class in yawning\_titan.envs.generic.wrappers.graph\_embedding\_observations}}

\begin{fulllineitems}
\phantomsection\label{\detokenize{source/yawning_titan.envs.generic.wrappers:yawning_titan.envs.generic.wrappers.graph_embedding_observations.FeatherGraphEmbedObservation}}\pysiglinewithargsret{\sphinxbfcode{\sphinxupquote{class\DUrole{w}{  }}}\sphinxcode{\sphinxupquote{yawning\_titan.envs.generic.wrappers.graph\_embedding\_observations.}}\sphinxbfcode{\sphinxupquote{FeatherGraphEmbedObservation}}}{\emph{\DUrole{n}{env}\DUrole{p}{:}\DUrole{w}{  }\DUrole{n}{gym.core.Env}}, \emph{\DUrole{n}{max\_num\_nodes}\DUrole{p}{:}\DUrole{w}{  }\DUrole{n}{int}\DUrole{w}{  }\DUrole{o}{=}\DUrole{w}{  }\DUrole{default_value}{100}}}{}
\sphinxAtStartPar
Bases: \sphinxcode{\sphinxupquote{gym.core.ObservationWrapper}}

\sphinxAtStartPar
Gym Observation Space Wrapper that uses the Feather\sphinxhyphen{}G Whole
Graph embedding algorithm to embed the underlying environment
graph alongisde padding vulnerability scores and vulnerability
status to support up to 100 nodes (arbitaryily set)
\index{make\_embedding() (yawning\_titan.envs.generic.wrappers.graph\_embedding\_observations.FeatherGraphEmbedObservation method)@\spxentry{make\_embedding()}\spxextra{yawning\_titan.envs.generic.wrappers.graph\_embedding\_observations.FeatherGraphEmbedObservation method}}

\begin{fulllineitems}
\phantomsection\label{\detokenize{source/yawning_titan.envs.generic.wrappers:yawning_titan.envs.generic.wrappers.graph_embedding_observations.FeatherGraphEmbedObservation.make_embedding}}\pysiglinewithargsret{\sphinxbfcode{\sphinxupquote{make\_embedding}}}{}{{ $\rightarrow$ numpy.ndarray}}
\sphinxAtStartPar
Creates a FeaterGraph embedding from the inputted
NetworkX graph
\begin{quote}\begin{description}
\item[{Returns}] \leavevmode
\sphinxAtStartPar
A numpy array containing the Feather embedding

\end{description}\end{quote}

\end{fulllineitems}

\index{observation() (yawning\_titan.envs.generic.wrappers.graph\_embedding\_observations.FeatherGraphEmbedObservation method)@\spxentry{observation()}\spxextra{yawning\_titan.envs.generic.wrappers.graph\_embedding\_observations.FeatherGraphEmbedObservation method}}

\begin{fulllineitems}
\phantomsection\label{\detokenize{source/yawning_titan.envs.generic.wrappers:yawning_titan.envs.generic.wrappers.graph_embedding_observations.FeatherGraphEmbedObservation.observation}}\pysiglinewithargsret{\sphinxbfcode{\sphinxupquote{observation}}}{\emph{\DUrole{n}{observation}\DUrole{p}{:}\DUrole{w}{  }\DUrole{n}{numpy.ndarray}}}{{ $\rightarrow$ numpy.ndarray}}
\sphinxAtStartPar
Observation Tranformation Function
\begin{enumerate}
\sphinxsetlistlabels{\arabic}{enumi}{enumii}{}{.}%
\item {}
\sphinxAtStartPar
Generates a networkx graph object from the current adjency matrix

\item {}
\sphinxAtStartPar
Collects the current vulnerability scores and node status’s

\item {}
\sphinxAtStartPar
Pads the returned arrays to ensure length is 100 (currently arbitaryily set)

\item {}
\sphinxAtStartPar
Embeds the networkx graph using the Feather Graph alogirhtm from Karateclub

\item {}
\sphinxAtStartPar
Concats the graph embedding, padded vulnerability scores and padded node status’s together

\item {}
\sphinxAtStartPar
Returns new observation

\end{enumerate}
\begin{quote}\begin{description}
\item[{Parameters}] \leavevmode
\sphinxAtStartPar
\sphinxstyleliteralstrong{\sphinxupquote{observation}} \textendash{} The base, unwrapped observation generated by the environment

\item[{Returns}] \leavevmode
\sphinxAtStartPar
A newly formatted environment observation

\end{description}\end{quote}

\end{fulllineitems}


\end{fulllineitems}



\subparagraph{Module contents}
\label{\detokenize{source/yawning_titan.envs.generic.wrappers:module-yawning_titan.envs.generic.wrappers}}\label{\detokenize{source/yawning_titan.envs.generic.wrappers:module-contents}}\index{module@\spxentry{module}!yawning\_titan.envs.generic.wrappers@\spxentry{yawning\_titan.envs.generic.wrappers}}\index{yawning\_titan.envs.generic.wrappers@\spxentry{yawning\_titan.envs.generic.wrappers}!module@\spxentry{module}}

\subparagraph{Submodules}
\label{\detokenize{source/yawning_titan.envs.generic:submodules}}

\subparagraph{yawning\_titan.envs.generic.generic\_env module}
\label{\detokenize{source/yawning_titan.envs.generic:module-yawning_titan.envs.generic.generic_env}}\label{\detokenize{source/yawning_titan.envs.generic:yawning-titan-envs-generic-generic-env-module}}\index{module@\spxentry{module}!yawning\_titan.envs.generic.generic\_env@\spxentry{yawning\_titan.envs.generic.generic\_env}}\index{yawning\_titan.envs.generic.generic\_env@\spxentry{yawning\_titan.envs.generic.generic\_env}!module@\spxentry{module}}
\sphinxAtStartPar
This is a generic Network Open AI environment. This is a very generic Open AI environment that allows you to heavily
customise what the goals, agents and network looks like. It follows this format:
\begin{itemize}
\item {}
\sphinxAtStartPar
red agents action

\item {}
\sphinxAtStartPar
check if the game is over

\item {}
\sphinxAtStartPar
blue agents action

\item {}
\sphinxAtStartPar
calculate rewards for the current state of the network

\end{itemize}

\sphinxAtStartPar
There is also a render method that uses methods from the graph2plot to create a matplotlib graph to show the current
state of the environment.
\index{GenericNetworkEnv (class in yawning\_titan.envs.generic.generic\_env)@\spxentry{GenericNetworkEnv}\spxextra{class in yawning\_titan.envs.generic.generic\_env}}

\begin{fulllineitems}
\phantomsection\label{\detokenize{source/yawning_titan.envs.generic:yawning_titan.envs.generic.generic_env.GenericNetworkEnv}}\pysiglinewithargsret{\sphinxbfcode{\sphinxupquote{class\DUrole{w}{  }}}\sphinxcode{\sphinxupquote{yawning\_titan.envs.generic.generic\_env.}}\sphinxbfcode{\sphinxupquote{GenericNetworkEnv}}}{\emph{\DUrole{n}{red\_agent}\DUrole{p}{:}\DUrole{w}{  }\DUrole{n}{{\hyperref[\detokenize{source/yawning_titan.envs.generic.core:yawning_titan.envs.generic.core.red_interface.RedInterface}]{\sphinxcrossref{yawning\_titan.envs.generic.core.red\_interface.RedInterface}}}}}, \emph{\DUrole{n}{blue\_agent}\DUrole{p}{:}\DUrole{w}{  }\DUrole{n}{{\hyperref[\detokenize{source/yawning_titan.envs.generic.core:yawning_titan.envs.generic.core.blue_interface.BlueInterface}]{\sphinxcrossref{yawning\_titan.envs.generic.core.blue\_interface.BlueInterface}}}}}, \emph{\DUrole{n}{network\_interface}\DUrole{p}{:}\DUrole{w}{  }\DUrole{n}{{\hyperref[\detokenize{source/yawning_titan.envs.generic.core:yawning_titan.envs.generic.core.network_interface.NetworkInterface}]{\sphinxcrossref{yawning\_titan.envs.generic.core.network\_interface.NetworkInterface}}}}}, \emph{\DUrole{n}{number\_of\_actions}\DUrole{p}{:}\DUrole{w}{  }\DUrole{n}{int}}, \emph{\DUrole{n}{print\_metrics}\DUrole{p}{:}\DUrole{w}{  }\DUrole{n}{bool}\DUrole{w}{  }\DUrole{o}{=}\DUrole{w}{  }\DUrole{default_value}{False}}, \emph{\DUrole{n}{show\_metrics\_every}\DUrole{p}{:}\DUrole{w}{  }\DUrole{n}{int}\DUrole{w}{  }\DUrole{o}{=}\DUrole{w}{  }\DUrole{default_value}{1}}, \emph{\DUrole{n}{collect\_data}\DUrole{p}{:}\DUrole{w}{  }\DUrole{n}{bool}\DUrole{w}{  }\DUrole{o}{=}\DUrole{w}{  }\DUrole{default_value}{True}}}{}
\sphinxAtStartPar
Bases: \sphinxcode{\sphinxupquote{gym.core.Env}}
\index{render() (yawning\_titan.envs.generic.generic\_env.GenericNetworkEnv method)@\spxentry{render()}\spxextra{yawning\_titan.envs.generic.generic\_env.GenericNetworkEnv method}}

\begin{fulllineitems}
\phantomsection\label{\detokenize{source/yawning_titan.envs.generic:yawning_titan.envs.generic.generic_env.GenericNetworkEnv.render}}\pysiglinewithargsret{\sphinxbfcode{\sphinxupquote{render}}}{\emph{\DUrole{n}{mode}\DUrole{p}{:}\DUrole{w}{  }\DUrole{n}{str}\DUrole{w}{  }\DUrole{o}{=}\DUrole{w}{  }\DUrole{default_value}{\textquotesingle{}human\textquotesingle{}}}, \emph{\DUrole{n}{show\_only\_blue\_view}\DUrole{p}{:}\DUrole{w}{  }\DUrole{n}{bool}\DUrole{w}{  }\DUrole{o}{=}\DUrole{w}{  }\DUrole{default_value}{False}}, \emph{\DUrole{n}{show\_node\_names}\DUrole{p}{:}\DUrole{w}{  }\DUrole{n}{bool}\DUrole{w}{  }\DUrole{o}{=}\DUrole{w}{  }\DUrole{default_value}{False}}}{}
\sphinxAtStartPar
Renders the network using the graph2plot class. This uses a networkx representation of the network
\begin{quote}\begin{description}
\item[{Parameters}] \leavevmode\begin{itemize}
\item {}
\sphinxAtStartPar
\sphinxstyleliteralstrong{\sphinxupquote{mode}} \textendash{} the mode of the rendering

\item {}
\sphinxAtStartPar
\sphinxstyleliteralstrong{\sphinxupquote{show\_only\_blue\_view}} \textendash{} If true shows only what the blue agent can see

\item {}
\sphinxAtStartPar
\sphinxstyleliteralstrong{\sphinxupquote{show\_node\_names}} \textendash{} Show the names of the nodes

\end{itemize}

\end{description}\end{quote}

\end{fulllineitems}

\index{reset() (yawning\_titan.envs.generic.generic\_env.GenericNetworkEnv method)@\spxentry{reset()}\spxextra{yawning\_titan.envs.generic.generic\_env.GenericNetworkEnv method}}

\begin{fulllineitems}
\phantomsection\label{\detokenize{source/yawning_titan.envs.generic:yawning_titan.envs.generic.generic_env.GenericNetworkEnv.reset}}\pysiglinewithargsret{\sphinxbfcode{\sphinxupquote{reset}}}{}{{ $\rightarrow$ numpy.array}}
\sphinxAtStartPar
Resets the environment to the default state
\begin{quote}\begin{description}
\item[{Returns}] \leavevmode
\sphinxAtStartPar
A new starting observation (numpy array)

\end{description}\end{quote}

\end{fulllineitems}

\index{step() (yawning\_titan.envs.generic.generic\_env.GenericNetworkEnv method)@\spxentry{step()}\spxextra{yawning\_titan.envs.generic.generic\_env.GenericNetworkEnv method}}

\begin{fulllineitems}
\phantomsection\label{\detokenize{source/yawning_titan.envs.generic:yawning_titan.envs.generic.generic_env.GenericNetworkEnv.step}}\pysiglinewithargsret{\sphinxbfcode{\sphinxupquote{step}}}{\emph{\DUrole{n}{action}\DUrole{p}{:}\DUrole{w}{  }\DUrole{n}{int}}}{{ $\rightarrow$ Tuple\DUrole{p}{{[}}numpy.array\DUrole{p}{,}\DUrole{w}{  }float\DUrole{p}{,}\DUrole{w}{  }bool\DUrole{p}{,}\DUrole{w}{  }dict\DUrole{p}{{]}}}}
\sphinxAtStartPar
Takes a time step and executes the actions for both Blue RL agent and
hard\sphinxhyphen{}hard coded Red agent.
\begin{quote}\begin{description}
\item[{Parameters}] \leavevmode
\sphinxAtStartPar
\sphinxstyleliteralstrong{\sphinxupquote{action}} \textendash{} The action value generated from the Blue RL agent (int)

\item[{Returns}] \leavevmode
\sphinxAtStartPar
The next environment observation (numpy array)
reward: The reward value for that timestep (int)
done: Whether the episode is done (bool)
info: a dictionary containing info about the current state

\item[{Return type}] \leavevmode
\sphinxAtStartPar
observation

\end{description}\end{quote}

\end{fulllineitems}


\end{fulllineitems}



\subparagraph{Module contents}
\label{\detokenize{source/yawning_titan.envs.generic:module-yawning_titan.envs.generic}}\label{\detokenize{source/yawning_titan.envs.generic:module-contents}}\index{module@\spxentry{module}!yawning\_titan.envs.generic@\spxentry{yawning\_titan.envs.generic}}\index{yawning\_titan.envs.generic@\spxentry{yawning\_titan.envs.generic}!module@\spxentry{module}}

\subparagraph{yawning\_titan.envs.specific package}
\label{\detokenize{source/yawning_titan.envs.specific:yawning-titan-envs-specific-package}}\label{\detokenize{source/yawning_titan.envs.specific::doc}}

\subparagraph{Subpackages}
\label{\detokenize{source/yawning_titan.envs.specific:subpackages}}

\subparagraph{yawning\_titan.envs.specific.core package}
\label{\detokenize{source/yawning_titan.envs.specific.core:yawning-titan-envs-specific-core-package}}\label{\detokenize{source/yawning_titan.envs.specific.core::doc}}

\subparagraph{Submodules}
\label{\detokenize{source/yawning_titan.envs.specific.core:submodules}}

\subparagraph{yawning\_titan.envs.specific.core.machines module}
\label{\detokenize{source/yawning_titan.envs.specific.core:module-yawning_titan.envs.specific.core.machines}}\label{\detokenize{source/yawning_titan.envs.specific.core:yawning-titan-envs-specific-core-machines-module}}\index{module@\spxentry{module}!yawning\_titan.envs.specific.core.machines@\spxentry{yawning\_titan.envs.specific.core.machines}}\index{yawning\_titan.envs.specific.core.machines@\spxentry{yawning\_titan.envs.specific.core.machines}!module@\spxentry{module}}\index{Machines (class in yawning\_titan.envs.specific.core.machines)@\spxentry{Machines}\spxextra{class in yawning\_titan.envs.specific.core.machines}}

\begin{fulllineitems}
\phantomsection\label{\detokenize{source/yawning_titan.envs.specific.core:yawning_titan.envs.specific.core.machines.Machines}}\pysiglinewithargsret{\sphinxbfcode{\sphinxupquote{class\DUrole{w}{  }}}\sphinxcode{\sphinxupquote{yawning\_titan.envs.specific.core.machines.}}\sphinxbfcode{\sphinxupquote{Machines}}}{\emph{\DUrole{n}{n\_machines}\DUrole{p}{:}\DUrole{w}{  }\DUrole{n}{int}\DUrole{w}{  }\DUrole{o}{=}\DUrole{w}{  }\DUrole{default_value}{5}}, \emph{\DUrole{n}{vuln\_score\_ub}\DUrole{p}{:}\DUrole{w}{  }\DUrole{n}{float}\DUrole{w}{  }\DUrole{o}{=}\DUrole{w}{  }\DUrole{default_value}{0.8}}, \emph{\DUrole{n}{vuln\_score\_lb}\DUrole{p}{:}\DUrole{w}{  }\DUrole{n}{float}\DUrole{w}{  }\DUrole{o}{=}\DUrole{w}{  }\DUrole{default_value}{0.4}}}{}
\sphinxAtStartPar
Bases: \sphinxcode{\sphinxupquote{object}}

\sphinxAtStartPar
Sets the initial state for the machines within the environment and
randomly generates vulnerability scores for each machine
\index{get\_initial\_state() (yawning\_titan.envs.specific.core.machines.Machines method)@\spxentry{get\_initial\_state()}\spxextra{yawning\_titan.envs.specific.core.machines.Machines method}}

\begin{fulllineitems}
\phantomsection\label{\detokenize{source/yawning_titan.envs.specific.core:yawning_titan.envs.specific.core.machines.Machines.get_initial_state}}\pysiglinewithargsret{\sphinxbfcode{\sphinxupquote{get\_initial\_state}}}{}{{ $\rightarrow$ List\DUrole{p}{{[}}List\DUrole{p}{{[}}float\DUrole{p}{{]}}\DUrole{p}{{]}}}}
\sphinxAtStartPar
Gets the initial states of the machines
\begin{quote}\begin{description}
\item[{Returns}] \leavevmode
\sphinxAtStartPar
The initial machine states

\end{description}\end{quote}

\sphinxAtStartPar
Notes: This is required in order to ensure that the initial
states are saved properly.

\end{fulllineitems}

\index{init\_machines() (yawning\_titan.envs.specific.core.machines.Machines method)@\spxentry{init\_machines()}\spxextra{yawning\_titan.envs.specific.core.machines.Machines method}}

\begin{fulllineitems}
\phantomsection\label{\detokenize{source/yawning_titan.envs.specific.core:yawning_titan.envs.specific.core.machines.Machines.init_machines}}\pysiglinewithargsret{\sphinxbfcode{\sphinxupquote{init\_machines}}}{}{{ $\rightarrow$ List\DUrole{p}{{[}}List\DUrole{p}{{[}}float\DUrole{p}{{]}}\DUrole{p}{{]}}}}
\sphinxAtStartPar
Generates a set of machines state pairs.
Each pair has a vulnerability score between the
upper and lower bound values provided and a 0
to denote uncompromised state
\begin{quote}\begin{description}
\item[{Returns}] \leavevmode
\sphinxAtStartPar
A list of fresh machine states pairs.

\end{description}\end{quote}
\subsubsection*{Example}

\sphinxAtStartPar
{[}{[}0.74,0{]},{[}0.47,0{]},{[}0.62, 0{]},{[}0.52, 0{]},{[}0.83,0{]}{]}

\end{fulllineitems}


\end{fulllineitems}



\subparagraph{yawning\_titan.envs.specific.core.node\_states module}
\label{\detokenize{source/yawning_titan.envs.specific.core:module-yawning_titan.envs.specific.core.node_states}}\label{\detokenize{source/yawning_titan.envs.specific.core:yawning-titan-envs-specific-core-node-states-module}}\index{module@\spxentry{module}!yawning\_titan.envs.specific.core.node\_states@\spxentry{yawning\_titan.envs.specific.core.node\_states}}\index{yawning\_titan.envs.specific.core.node\_states@\spxentry{yawning\_titan.envs.specific.core.node\_states}!module@\spxentry{module}}\index{get\_compromised\_nodes() (in module yawning\_titan.envs.specific.core.node\_states)@\spxentry{get\_compromised\_nodes()}\spxextra{in module yawning\_titan.envs.specific.core.node\_states}}

\begin{fulllineitems}
\phantomsection\label{\detokenize{source/yawning_titan.envs.specific.core:yawning_titan.envs.specific.core.node_states.get_compromised_nodes}}\pysiglinewithargsret{\sphinxcode{\sphinxupquote{yawning\_titan.envs.specific.core.node\_states.}}\sphinxbfcode{\sphinxupquote{get\_compromised\_nodes}}}{\emph{\DUrole{n}{machine\_states}\DUrole{p}{:}\DUrole{w}{  }\DUrole{n}{List\DUrole{p}{{[}}List\DUrole{p}{{[}}float\DUrole{p}{{]}}\DUrole{p}{{]}}}}}{{ $\rightarrow$ List\DUrole{p}{{[}}int\DUrole{p}{{]}}}}
\sphinxAtStartPar
Returns a list of compromised nodes
\begin{quote}\begin{description}
\item[{Parameters}] \leavevmode
\sphinxAtStartPar
\sphinxstyleliteralstrong{\sphinxupquote{machine\_states}} \textendash{} The current machine states

\item[{Returns}] \leavevmode
\sphinxAtStartPar
A list of compromised nodes

\end{description}\end{quote}
\subsubsection*{Notes}

\sphinxAtStartPar
This differs from the similar function above
because this function does not support returning
uncompromised nodes based on an agents current
position

\end{fulllineitems}

\index{get\_linked\_compromised\_nodes() (in module yawning\_titan.envs.specific.core.node\_states)@\spxentry{get\_linked\_compromised\_nodes()}\spxextra{in module yawning\_titan.envs.specific.core.node\_states}}

\begin{fulllineitems}
\phantomsection\label{\detokenize{source/yawning_titan.envs.specific.core:yawning_titan.envs.specific.core.node_states.get_linked_compromised_nodes}}\pysiglinewithargsret{\sphinxcode{\sphinxupquote{yawning\_titan.envs.specific.core.node\_states.}}\sphinxbfcode{\sphinxupquote{get\_linked\_compromised\_nodes}}}{\emph{\DUrole{n}{current\_position}\DUrole{p}{:}\DUrole{w}{  }\DUrole{n}{int}}, \emph{\DUrole{n}{network}\DUrole{p}{:}\DUrole{w}{  }\DUrole{n}{networkx.classes.graph.Graph}}, \emph{\DUrole{n}{machine\_states}\DUrole{p}{:}\DUrole{w}{  }\DUrole{n}{List\DUrole{p}{{[}}List\DUrole{p}{{[}}float\DUrole{p}{{]}}\DUrole{p}{{]}}}}}{{ $\rightarrow$ List\DUrole{p}{{[}}int\DUrole{p}{{]}}}}
\sphinxAtStartPar
Returns a list containing all of the linked compromised nodes relative
to the red agents current position
\begin{quote}\begin{description}
\item[{Parameters}] \leavevmode\begin{itemize}
\item {}
\sphinxAtStartPar
\sphinxstyleliteralstrong{\sphinxupquote{current\_position}} \textendash{} The red teams current position

\item {}
\sphinxAtStartPar
\sphinxstyleliteralstrong{\sphinxupquote{network}} \textendash{} A networkx graph representation of the network

\item {}
\sphinxAtStartPar
\sphinxstyleliteralstrong{\sphinxupquote{machine\_states}} \textendash{} The current machine states

\end{itemize}

\item[{Returns}] \leavevmode
\sphinxAtStartPar
A list of comrpomised linked nodes

\item[{Return type}] \leavevmode
\sphinxAtStartPar
compromised nodes

\end{description}\end{quote}

\end{fulllineitems}

\index{get\_linked\_node\_state() (in module yawning\_titan.envs.specific.core.node\_states)@\spxentry{get\_linked\_node\_state()}\spxextra{in module yawning\_titan.envs.specific.core.node\_states}}

\begin{fulllineitems}
\phantomsection\label{\detokenize{source/yawning_titan.envs.specific.core:yawning_titan.envs.specific.core.node_states.get_linked_node_state}}\pysiglinewithargsret{\sphinxcode{\sphinxupquote{yawning\_titan.envs.specific.core.node\_states.}}\sphinxbfcode{\sphinxupquote{get\_linked\_node\_state}}}{\emph{\DUrole{n}{current\_position}\DUrole{p}{:}\DUrole{w}{  }\DUrole{n}{int}}, \emph{\DUrole{n}{network}\DUrole{p}{:}\DUrole{w}{  }\DUrole{n}{networkx.classes.graph.Graph}}, \emph{\DUrole{n}{machine\_states}\DUrole{p}{:}\DUrole{w}{  }\DUrole{n}{List\DUrole{p}{{[}}List\DUrole{p}{{[}}float\DUrole{p}{{]}}\DUrole{p}{{]}}}}}{{ $\rightarrow$ Tuple\DUrole{p}{{[}}List\DUrole{p}{{[}}int\DUrole{p}{{]}}\DUrole{p}{,}\DUrole{w}{  }List\DUrole{p}{{[}}int\DUrole{p}{{]}}\DUrole{p}{{]}}}}
\sphinxAtStartPar
Calculates the states of the nodes linked the red team’s current position
\begin{quote}\begin{description}
\item[{Parameters}] \leavevmode\begin{itemize}
\item {}
\sphinxAtStartPar
\sphinxstyleliteralstrong{\sphinxupquote{current\_position}} \textendash{} The current Red Team’s location

\item {}
\sphinxAtStartPar
\sphinxstyleliteralstrong{\sphinxupquote{network}} \textendash{} A networkx graph representation of the network

\item {}
\sphinxAtStartPar
\sphinxstyleliteralstrong{\sphinxupquote{machine\_states}} \textendash{} The current machine states

\end{itemize}

\item[{Returns}] \leavevmode
\sphinxAtStartPar
A list of uncompromised linked nodes
compromised\_nodes: A list of compromised linked nodes

\item[{Return type}] \leavevmode
\sphinxAtStartPar
uncompromised\_nodes

\end{description}\end{quote}

\end{fulllineitems}

\index{get\_linked\_uncompromised\_nodes() (in module yawning\_titan.envs.specific.core.node\_states)@\spxentry{get\_linked\_uncompromised\_nodes()}\spxextra{in module yawning\_titan.envs.specific.core.node\_states}}

\begin{fulllineitems}
\phantomsection\label{\detokenize{source/yawning_titan.envs.specific.core:yawning_titan.envs.specific.core.node_states.get_linked_uncompromised_nodes}}\pysiglinewithargsret{\sphinxcode{\sphinxupquote{yawning\_titan.envs.specific.core.node\_states.}}\sphinxbfcode{\sphinxupquote{get\_linked\_uncompromised\_nodes}}}{\emph{\DUrole{n}{current\_position}\DUrole{p}{:}\DUrole{w}{  }\DUrole{n}{int}}, \emph{\DUrole{n}{network}\DUrole{p}{:}\DUrole{w}{  }\DUrole{n}{networkx.classes.graph.Graph}}, \emph{\DUrole{n}{machine\_states}\DUrole{p}{:}\DUrole{w}{  }\DUrole{n}{List\DUrole{p}{{[}}List\DUrole{p}{{[}}float\DUrole{p}{{]}}\DUrole{p}{{]}}}}}{{ $\rightarrow$ List\DUrole{p}{{[}}int\DUrole{p}{{]}}}}
\sphinxAtStartPar
Returns a list containing all of the linked uncompromised nodes relative
to the red agents current position
\begin{quote}\begin{description}
\item[{Parameters}] \leavevmode\begin{itemize}
\item {}
\sphinxAtStartPar
\sphinxstyleliteralstrong{\sphinxupquote{current\_position}} \textendash{} The red teams current position

\item {}
\sphinxAtStartPar
\sphinxstyleliteralstrong{\sphinxupquote{network}} \textendash{} A networkx graph representation of the network

\item {}
\sphinxAtStartPar
\sphinxstyleliteralstrong{\sphinxupquote{machine\_states}} \textendash{} The current machine states

\end{itemize}

\item[{Returns}] \leavevmode
\sphinxAtStartPar
A list of uncomrpomised linked nodes

\item[{Return type}] \leavevmode
\sphinxAtStartPar
compromised nodes

\end{description}\end{quote}

\end{fulllineitems}

\index{get\_uncompromised\_nodes() (in module yawning\_titan.envs.specific.core.node\_states)@\spxentry{get\_uncompromised\_nodes()}\spxextra{in module yawning\_titan.envs.specific.core.node\_states}}

\begin{fulllineitems}
\phantomsection\label{\detokenize{source/yawning_titan.envs.specific.core:yawning_titan.envs.specific.core.node_states.get_uncompromised_nodes}}\pysiglinewithargsret{\sphinxcode{\sphinxupquote{yawning\_titan.envs.specific.core.node\_states.}}\sphinxbfcode{\sphinxupquote{get\_uncompromised\_nodes}}}{\emph{\DUrole{n}{machine\_states}\DUrole{p}{:}\DUrole{w}{  }\DUrole{n}{List\DUrole{p}{{[}}List\DUrole{p}{{[}}float\DUrole{p}{{]}}\DUrole{p}{{]}}}}}{{ $\rightarrow$ List\DUrole{p}{{[}}int\DUrole{p}{{]}}}}
\sphinxAtStartPar
Returns a list of uncompromised nodes
\begin{quote}\begin{description}
\item[{Parameters}] \leavevmode
\sphinxAtStartPar
\sphinxstyleliteralstrong{\sphinxupquote{machine\_states}} \textendash{} The current machine states

\item[{Returns}] \leavevmode
\sphinxAtStartPar
A list of uncompromised nodes

\end{description}\end{quote}
\subsubsection*{Notes}

\sphinxAtStartPar
This differs from the similar function above
because this function does not support returning
uncompromised nodes based on an agents current
position

\end{fulllineitems}



\subparagraph{yawning\_titan.envs.specific.core.nsa\_node module}
\label{\detokenize{source/yawning_titan.envs.specific.core:module-yawning_titan.envs.specific.core.nsa_node}}\label{\detokenize{source/yawning_titan.envs.specific.core:yawning-titan-envs-specific-core-nsa-node-module}}\index{module@\spxentry{module}!yawning\_titan.envs.specific.core.nsa\_node@\spxentry{yawning\_titan.envs.specific.core.nsa\_node}}\index{yawning\_titan.envs.specific.core.nsa\_node@\spxentry{yawning\_titan.envs.specific.core.nsa\_node}!module@\spxentry{module}}\index{Node (class in yawning\_titan.envs.specific.core.nsa\_node)@\spxentry{Node}\spxextra{class in yawning\_titan.envs.specific.core.nsa\_node}}

\begin{fulllineitems}
\phantomsection\label{\detokenize{source/yawning_titan.envs.specific.core:yawning_titan.envs.specific.core.nsa_node.Node}}\pysigline{\sphinxbfcode{\sphinxupquote{class\DUrole{w}{  }}}\sphinxcode{\sphinxupquote{yawning\_titan.envs.specific.core.nsa\_node.}}\sphinxbfcode{\sphinxupquote{Node}}}
\sphinxAtStartPar
Bases: \sphinxcode{\sphinxupquote{object}}
\index{change\_compromised() (yawning\_titan.envs.specific.core.nsa\_node.Node method)@\spxentry{change\_compromised()}\spxextra{yawning\_titan.envs.specific.core.nsa\_node.Node method}}

\begin{fulllineitems}
\phantomsection\label{\detokenize{source/yawning_titan.envs.specific.core:yawning_titan.envs.specific.core.nsa_node.Node.change_compromised}}\pysiglinewithargsret{\sphinxbfcode{\sphinxupquote{change\_compromised}}}{\emph{\DUrole{n}{mode}\DUrole{p}{:}\DUrole{w}{  }\DUrole{n}{int}}}{}
\sphinxAtStartPar
Changes the compromised status of a node
\begin{quote}\begin{description}
\item[{Parameters}] \leavevmode
\sphinxAtStartPar
\sphinxstyleliteralstrong{\sphinxupquote{mode}} \textendash{} either 0, 1 or 2
0: does nothing
1: changes the node to safe
2: changes the node to compromised

\end{description}\end{quote}

\end{fulllineitems}

\index{change\_isolated() (yawning\_titan.envs.specific.core.nsa\_node.Node method)@\spxentry{change\_isolated()}\spxextra{yawning\_titan.envs.specific.core.nsa\_node.Node method}}

\begin{fulllineitems}
\phantomsection\label{\detokenize{source/yawning_titan.envs.specific.core:yawning_titan.envs.specific.core.nsa_node.Node.change_isolated}}\pysiglinewithargsret{\sphinxbfcode{\sphinxupquote{change\_isolated}}}{}{}
\sphinxAtStartPar
Changes the isolation status of a node.

\sphinxAtStartPar
Flips it so if it was true it becomes false and vice versa

\end{fulllineitems}

\index{get\_condition() (yawning\_titan.envs.specific.core.nsa\_node.Node method)@\spxentry{get\_condition()}\spxextra{yawning\_titan.envs.specific.core.nsa\_node.Node method}}

\begin{fulllineitems}
\phantomsection\label{\detokenize{source/yawning_titan.envs.specific.core:yawning_titan.envs.specific.core.nsa_node.Node.get_condition}}\pysiglinewithargsret{\sphinxbfcode{\sphinxupquote{get\_condition}}}{}{{ $\rightarrow$ Tuple\DUrole{p}{{[}}int\DUrole{p}{,}\DUrole{w}{  }int\DUrole{p}{{]}}}}
\sphinxAtStartPar
Returns the condition of the node
\begin{quote}\begin{description}
\item[{Returns}] \leavevmode
\sphinxAtStartPar
a list containing the isolation and compromised status of the node ({[}bool, bool{]})

\item[{Return type}] \leavevmode
\sphinxAtStartPar
reward

\end{description}\end{quote}

\end{fulllineitems}


\end{fulllineitems}



\subparagraph{yawning\_titan.envs.specific.core.nsa\_node\_collection module}
\label{\detokenize{source/yawning_titan.envs.specific.core:module-yawning_titan.envs.specific.core.nsa_node_collection}}\label{\detokenize{source/yawning_titan.envs.specific.core:yawning-titan-envs-specific-core-nsa-node-collection-module}}\index{module@\spxentry{module}!yawning\_titan.envs.specific.core.nsa\_node\_collection@\spxentry{yawning\_titan.envs.specific.core.nsa\_node\_collection}}\index{yawning\_titan.envs.specific.core.nsa\_node\_collection@\spxentry{yawning\_titan.envs.specific.core.nsa\_node\_collection}!module@\spxentry{module}}\index{NodeCollection (class in yawning\_titan.envs.specific.core.nsa\_node\_collection)@\spxentry{NodeCollection}\spxextra{class in yawning\_titan.envs.specific.core.nsa\_node\_collection}}

\begin{fulllineitems}
\phantomsection\label{\detokenize{source/yawning_titan.envs.specific.core:yawning_titan.envs.specific.core.nsa_node_collection.NodeCollection}}\pysiglinewithargsret{\sphinxbfcode{\sphinxupquote{class\DUrole{w}{  }}}\sphinxcode{\sphinxupquote{yawning\_titan.envs.specific.core.nsa\_node\_collection.}}\sphinxbfcode{\sphinxupquote{NodeCollection}}}{\emph{\DUrole{n}{network}\DUrole{p}{:}\DUrole{w}{  }\DUrole{n}{Tuple\DUrole{p}{{[}}numpy.array\DUrole{p}{,}\DUrole{w}{  }dict\DUrole{p}{{]}}}}, \emph{\DUrole{n}{chance\_to\_spread\_during\_patch}}}{}
\sphinxAtStartPar
Bases: \sphinxcode{\sphinxupquote{object}}
\index{calculate\_reward() (yawning\_titan.envs.specific.core.nsa\_node\_collection.NodeCollection method)@\spxentry{calculate\_reward()}\spxextra{yawning\_titan.envs.specific.core.nsa\_node\_collection.NodeCollection method}}

\begin{fulllineitems}
\phantomsection\label{\detokenize{source/yawning_titan.envs.specific.core:yawning_titan.envs.specific.core.nsa_node_collection.NodeCollection.calculate_reward}}\pysiglinewithargsret{\sphinxbfcode{\sphinxupquote{calculate\_reward}}}{}{{ $\rightarrow$ float}}
\sphinxAtStartPar
Calculates a reward for the current networks state
\begin{quote}\begin{description}
\item[{Returns}] \leavevmode
\sphinxAtStartPar
the reward for being in the current state

\item[{Return type}] \leavevmode
\sphinxAtStartPar
reward

\end{description}\end{quote}

\end{fulllineitems}

\index{get\_compromised\_nodes() (yawning\_titan.envs.specific.core.nsa\_node\_collection.NodeCollection method)@\spxentry{get\_compromised\_nodes()}\spxextra{yawning\_titan.envs.specific.core.nsa\_node\_collection.NodeCollection method}}

\begin{fulllineitems}
\phantomsection\label{\detokenize{source/yawning_titan.envs.specific.core:yawning_titan.envs.specific.core.nsa_node_collection.NodeCollection.get_compromised_nodes}}\pysiglinewithargsret{\sphinxbfcode{\sphinxupquote{get\_compromised\_nodes}}}{}{{ $\rightarrow$ List\DUrole{p}{{[}}int\DUrole{p}{{]}}}}
\sphinxAtStartPar
Creates a list of all the nodes in the network that are compromised.
\begin{quote}\begin{description}
\item[{Returns}] \leavevmode
\sphinxAtStartPar
A list of nodes that are compromised (list of ints)

\item[{Return type}] \leavevmode
\sphinxAtStartPar
compromised\_nodes

\end{description}\end{quote}

\end{fulllineitems}

\index{get\_connected\_nodes() (yawning\_titan.envs.specific.core.nsa\_node\_collection.NodeCollection method)@\spxentry{get\_connected\_nodes()}\spxextra{yawning\_titan.envs.specific.core.nsa\_node\_collection.NodeCollection method}}

\begin{fulllineitems}
\phantomsection\label{\detokenize{source/yawning_titan.envs.specific.core:yawning_titan.envs.specific.core.nsa_node_collection.NodeCollection.get_connected_nodes}}\pysiglinewithargsret{\sphinxbfcode{\sphinxupquote{get\_connected\_nodes}}}{\emph{\DUrole{n}{number}\DUrole{p}{:}\DUrole{w}{  }\DUrole{n}{int}}}{{ $\rightarrow$ List\DUrole{p}{{[}}int\DUrole{p}{{]}}}}
\sphinxAtStartPar
When given a node returns a list of all of the nodes connected to that node
\begin{quote}\begin{description}
\item[{Parameters}] \leavevmode
\sphinxAtStartPar
\sphinxstyleliteralstrong{\sphinxupquote{number}} \textendash{} the number of the node to run on

\item[{Returns}] \leavevmode
\sphinxAtStartPar
a list of all the nodes connected to a specified node (list of ints)

\end{description}\end{quote}

\end{fulllineitems}

\index{get\_isolated\_nodes() (yawning\_titan.envs.specific.core.nsa\_node\_collection.NodeCollection method)@\spxentry{get\_isolated\_nodes()}\spxextra{yawning\_titan.envs.specific.core.nsa\_node\_collection.NodeCollection method}}

\begin{fulllineitems}
\phantomsection\label{\detokenize{source/yawning_titan.envs.specific.core:yawning_titan.envs.specific.core.nsa_node_collection.NodeCollection.get_isolated_nodes}}\pysiglinewithargsret{\sphinxbfcode{\sphinxupquote{get\_isolated\_nodes}}}{}{{ $\rightarrow$ List\DUrole{p}{{[}}int\DUrole{p}{{]}}}}
\sphinxAtStartPar
Creates a list of all the isolated nodes in the network
\begin{quote}\begin{description}
\item[{Returns}] \leavevmode
\sphinxAtStartPar
A list of nodes that are isolated (list of ints)

\item[{Return type}] \leavevmode
\sphinxAtStartPar
isolated\_nodes

\end{description}\end{quote}

\end{fulllineitems}

\index{get\_netx\_graph() (yawning\_titan.envs.specific.core.nsa\_node\_collection.NodeCollection method)@\spxentry{get\_netx\_graph()}\spxextra{yawning\_titan.envs.specific.core.nsa\_node\_collection.NodeCollection method}}

\begin{fulllineitems}
\phantomsection\label{\detokenize{source/yawning_titan.envs.specific.core:yawning_titan.envs.specific.core.nsa_node_collection.NodeCollection.get_netx_graph}}\pysiglinewithargsret{\sphinxbfcode{\sphinxupquote{get\_netx\_graph}}}{}{{ $\rightarrow$ networkx.classes.graph.Graph}}
\end{fulllineitems}

\index{get\_netx\_pos() (yawning\_titan.envs.specific.core.nsa\_node\_collection.NodeCollection method)@\spxentry{get\_netx\_pos()}\spxextra{yawning\_titan.envs.specific.core.nsa\_node\_collection.NodeCollection method}}

\begin{fulllineitems}
\phantomsection\label{\detokenize{source/yawning_titan.envs.specific.core:yawning_titan.envs.specific.core.nsa_node_collection.NodeCollection.get_netx_pos}}\pysiglinewithargsret{\sphinxbfcode{\sphinxupquote{get\_netx\_pos}}}{}{{ $\rightarrow$ dict}}
\end{fulllineitems}

\index{get\_number\_of\_isolated() (yawning\_titan.envs.specific.core.nsa\_node\_collection.NodeCollection method)@\spxentry{get\_number\_of\_isolated()}\spxextra{yawning\_titan.envs.specific.core.nsa\_node\_collection.NodeCollection method}}

\begin{fulllineitems}
\phantomsection\label{\detokenize{source/yawning_titan.envs.specific.core:yawning_titan.envs.specific.core.nsa_node_collection.NodeCollection.get_number_of_isolated}}\pysiglinewithargsret{\sphinxbfcode{\sphinxupquote{get\_number\_of\_isolated}}}{}{{ $\rightarrow$ int}}
\sphinxAtStartPar
Gets the number of isolated nodes in the network
\begin{quote}\begin{description}
\item[{Returns}] \leavevmode
\sphinxAtStartPar
the number of isolated nodes in the network (int)

\end{description}\end{quote}

\end{fulllineitems}

\index{get\_number\_of\_nodes() (yawning\_titan.envs.specific.core.nsa\_node\_collection.NodeCollection method)@\spxentry{get\_number\_of\_nodes()}\spxextra{yawning\_titan.envs.specific.core.nsa\_node\_collection.NodeCollection method}}

\begin{fulllineitems}
\phantomsection\label{\detokenize{source/yawning_titan.envs.specific.core:yawning_titan.envs.specific.core.nsa_node_collection.NodeCollection.get_number_of_nodes}}\pysiglinewithargsret{\sphinxbfcode{\sphinxupquote{get\_number\_of\_nodes}}}{}{{ $\rightarrow$ int}}
\sphinxAtStartPar
Returns the number of nodes in the network
\begin{quote}\begin{description}
\item[{Returns}] \leavevmode
\sphinxAtStartPar
The number of nodes in the network (int)

\end{description}\end{quote}

\end{fulllineitems}

\index{get\_number\_of\_un\_compromised() (yawning\_titan.envs.specific.core.nsa\_node\_collection.NodeCollection method)@\spxentry{get\_number\_of\_un\_compromised()}\spxextra{yawning\_titan.envs.specific.core.nsa\_node\_collection.NodeCollection method}}

\begin{fulllineitems}
\phantomsection\label{\detokenize{source/yawning_titan.envs.specific.core:yawning_titan.envs.specific.core.nsa_node_collection.NodeCollection.get_number_of_un_compromised}}\pysiglinewithargsret{\sphinxbfcode{\sphinxupquote{get\_number\_of\_un\_compromised}}}{}{{ $\rightarrow$ int}}
\sphinxAtStartPar
Gets the number of safe nodes in the network
\begin{quote}\begin{description}
\item[{Returns}] \leavevmode
\sphinxAtStartPar
the number of safe nodes in the network (int)

\end{description}\end{quote}

\end{fulllineitems}

\index{get\_observation() (yawning\_titan.envs.specific.core.nsa\_node\_collection.NodeCollection method)@\spxentry{get\_observation()}\spxextra{yawning\_titan.envs.specific.core.nsa\_node\_collection.NodeCollection method}}

\begin{fulllineitems}
\phantomsection\label{\detokenize{source/yawning_titan.envs.specific.core:yawning_titan.envs.specific.core.nsa_node_collection.NodeCollection.get_observation}}\pysiglinewithargsret{\sphinxbfcode{\sphinxupquote{get\_observation}}}{}{{ $\rightarrow$ numpy.array}}
\sphinxAtStartPar
Gets the states of all the nodes in the network
\begin{quote}\begin{description}
\item[{Returns}] \leavevmode
\sphinxAtStartPar
The current state of the environment (numpy array)

\item[{Return type}] \leavevmode
\sphinxAtStartPar
observation

\end{description}\end{quote}

\end{fulllineitems}

\index{get\_un\_compromised\_nodes() (yawning\_titan.envs.specific.core.nsa\_node\_collection.NodeCollection method)@\spxentry{get\_un\_compromised\_nodes()}\spxextra{yawning\_titan.envs.specific.core.nsa\_node\_collection.NodeCollection method}}

\begin{fulllineitems}
\phantomsection\label{\detokenize{source/yawning_titan.envs.specific.core:yawning_titan.envs.specific.core.nsa_node_collection.NodeCollection.get_un_compromised_nodes}}\pysiglinewithargsret{\sphinxbfcode{\sphinxupquote{get\_un\_compromised\_nodes}}}{}{{ $\rightarrow$ List\DUrole{p}{{[}}int\DUrole{p}{{]}}}}
\sphinxAtStartPar
Creates a list of all the safe nodes in the network
\begin{quote}\begin{description}
\item[{Returns}] \leavevmode
\sphinxAtStartPar
A list of nodes that are safe (list of ints)

\item[{Return type}] \leavevmode
\sphinxAtStartPar
un\_compromised\_nodes

\end{description}\end{quote}

\end{fulllineitems}

\index{modify\_node() (yawning\_titan.envs.specific.core.nsa\_node\_collection.NodeCollection method)@\spxentry{modify\_node()}\spxextra{yawning\_titan.envs.specific.core.nsa\_node\_collection.NodeCollection method}}

\begin{fulllineitems}
\phantomsection\label{\detokenize{source/yawning_titan.envs.specific.core:yawning_titan.envs.specific.core.nsa_node_collection.NodeCollection.modify_node}}\pysiglinewithargsret{\sphinxbfcode{\sphinxupquote{modify\_node}}}{\emph{\DUrole{n}{number}\DUrole{p}{:}\DUrole{w}{  }\DUrole{n}{int}}, \emph{\DUrole{n}{changes}\DUrole{p}{:}\DUrole{w}{  }\DUrole{n}{Tuple\DUrole{p}{{[}}bool\DUrole{p}{,}\DUrole{w}{  }int\DUrole{p}{{]}}}}}{}
\sphinxAtStartPar
Changes the state of a single node
\begin{quote}\begin{description}
\item[{Parameters}] \leavevmode\begin{itemize}
\item {}
\sphinxAtStartPar
\sphinxstyleliteralstrong{\sphinxupquote{number}} \textendash{} the number of the node to change

\item {}
\sphinxAtStartPar
\sphinxstyleliteralstrong{\sphinxupquote{changes}} \textendash{}
\sphinxAtStartPar
a list with two variables in {[}isolate, compromise{]}
isolate: A boolean that will if true change the isolation status of the node (true \sphinxhyphen{}\textgreater{} false,
\begin{quote}

\sphinxAtStartPar
false \sphinxhyphen{}\textgreater{} true) (boolean)
\end{quote}
\begin{description}
\item[{compromise: a mode signal that will change the state of a node. 0 does nothing, 1 makes it safe and}] \leavevmode
\sphinxAtStartPar
2 compromises the node (int)

\end{description}


\end{itemize}

\end{description}\end{quote}

\end{fulllineitems}

\index{spread() (yawning\_titan.envs.specific.core.nsa\_node\_collection.NodeCollection method)@\spxentry{spread()}\spxextra{yawning\_titan.envs.specific.core.nsa\_node\_collection.NodeCollection method}}

\begin{fulllineitems}
\phantomsection\label{\detokenize{source/yawning_titan.envs.specific.core:yawning_titan.envs.specific.core.nsa_node_collection.NodeCollection.spread}}\pysiglinewithargsret{\sphinxbfcode{\sphinxupquote{spread}}}{\emph{\DUrole{n}{number}\DUrole{p}{:}\DUrole{w}{  }\DUrole{n}{int}}}{}
\sphinxAtStartPar
Spreads the red agent through all connected nodes
\begin{quote}\begin{description}
\item[{Parameters}] \leavevmode
\sphinxAtStartPar
\sphinxstyleliteralstrong{\sphinxupquote{number}} \textendash{} the number of the node to spread from

\end{description}\end{quote}

\end{fulllineitems}


\end{fulllineitems}



\subparagraph{Module contents}
\label{\detokenize{source/yawning_titan.envs.specific.core:module-yawning_titan.envs.specific.core}}\label{\detokenize{source/yawning_titan.envs.specific.core:module-contents}}\index{module@\spxentry{module}!yawning\_titan.envs.specific.core@\spxentry{yawning\_titan.envs.specific.core}}\index{yawning\_titan.envs.specific.core@\spxentry{yawning\_titan.envs.specific.core}!module@\spxentry{module}}

\subparagraph{Submodules}
\label{\detokenize{source/yawning_titan.envs.specific:submodules}}

\subparagraph{yawning\_titan.envs.specific.five\_node\_def module}
\label{\detokenize{source/yawning_titan.envs.specific:module-yawning_titan.envs.specific.five_node_def}}\label{\detokenize{source/yawning_titan.envs.specific:yawning-titan-envs-specific-five-node-def-module}}\index{module@\spxentry{module}!yawning\_titan.envs.specific.five\_node\_def@\spxentry{yawning\_titan.envs.specific.five\_node\_def}}\index{yawning\_titan.envs.specific.five\_node\_def@\spxentry{yawning\_titan.envs.specific.five\_node\_def}!module@\spxentry{module}}
\sphinxAtStartPar
NOTE: This environment is deprecated but has been included as an example of how to create a specific environment


\subparagraph{Five Node Environment AKA Cyber Whack A Mole}
\label{\detokenize{source/yawning_titan.envs.specific:five-node-environment-aka-cyber-whack-a-mole}}
\sphinxAtStartPar
This environment is made up of five nodes in the following topology:

\sphinxAtStartPar
+————+  +————+  +————+  +————+  +————+
|            |  |            |  |            |  |            |  |            |
|  Node 1    |  |   Node 2   |  |    Node 3  |  |   Node 4   |  |   Node 5   |
|            |  |            |  |            |  |            |  |            |
+————+  +————+  +————+  +————+  +————+

\sphinxAtStartPar
Configurable Parameters:
\begin{quote}

\sphinxAtStartPar
Number of Machines \sphinxhyphen{} This value determines the number of machines within the environment and defaults to 5.
Number of Compromised Machines for Loss \sphinxhyphen{} This value determines how many compromised machines equal a loss.
Attack Success Threshold \sphinxhyphen{} This value determines what the red agents attack value must be to be successful.
\end{quote}
\index{FiveNodeDef (class in yawning\_titan.envs.specific.five\_node\_def)@\spxentry{FiveNodeDef}\spxextra{class in yawning\_titan.envs.specific.five\_node\_def}}

\begin{fulllineitems}
\phantomsection\label{\detokenize{source/yawning_titan.envs.specific:yawning_titan.envs.specific.five_node_def.FiveNodeDef}}\pysiglinewithargsret{\sphinxbfcode{\sphinxupquote{class\DUrole{w}{  }}}\sphinxcode{\sphinxupquote{yawning\_titan.envs.specific.five\_node\_def.}}\sphinxbfcode{\sphinxupquote{FiveNodeDef}}}{\emph{\DUrole{n}{attacker\_skill}\DUrole{p}{:}\DUrole{w}{  }\DUrole{n}{float}\DUrole{w}{  }\DUrole{o}{=}\DUrole{w}{  }\DUrole{default_value}{50}}, \emph{\DUrole{n}{n\_machines}\DUrole{p}{:}\DUrole{w}{  }\DUrole{n}{int}\DUrole{w}{  }\DUrole{o}{=}\DUrole{w}{  }\DUrole{default_value}{5}}, \emph{\DUrole{n}{attack\_success\_threshold}\DUrole{p}{:}\DUrole{w}{  }\DUrole{n}{float}\DUrole{w}{  }\DUrole{o}{=}\DUrole{w}{  }\DUrole{default_value}{0.3}}, \emph{\DUrole{n}{no\_compromised\_machine\_loss}\DUrole{p}{:}\DUrole{w}{  }\DUrole{n}{int}\DUrole{w}{  }\DUrole{o}{=}\DUrole{w}{  }\DUrole{default_value}{4}}}{}
\sphinxAtStartPar
Bases: \sphinxcode{\sphinxupquote{gym.core.Env}}
\index{reset() (yawning\_titan.envs.specific.five\_node\_def.FiveNodeDef method)@\spxentry{reset()}\spxextra{yawning\_titan.envs.specific.five\_node\_def.FiveNodeDef method}}

\begin{fulllineitems}
\phantomsection\label{\detokenize{source/yawning_titan.envs.specific:yawning_titan.envs.specific.five_node_def.FiveNodeDef.reset}}\pysiglinewithargsret{\sphinxbfcode{\sphinxupquote{reset}}}{}{{ $\rightarrow$ numpy.array}}
\sphinxAtStartPar
Resets the environment to the default state
\begin{quote}\begin{description}
\item[{Returns}] \leavevmode
\sphinxAtStartPar
A new starting observation (numpy array)

\end{description}\end{quote}

\end{fulllineitems}

\index{step() (yawning\_titan.envs.specific.five\_node\_def.FiveNodeDef method)@\spxentry{step()}\spxextra{yawning\_titan.envs.specific.five\_node\_def.FiveNodeDef method}}

\begin{fulllineitems}
\phantomsection\label{\detokenize{source/yawning_titan.envs.specific:yawning_titan.envs.specific.five_node_def.FiveNodeDef.step}}\pysiglinewithargsret{\sphinxbfcode{\sphinxupquote{step}}}{\emph{\DUrole{n}{action}\DUrole{p}{:}\DUrole{w}{  }\DUrole{n}{int}}}{{ $\rightarrow$ Tuple\DUrole{p}{{[}}numpy.array\DUrole{p}{,}\DUrole{w}{  }float\DUrole{p}{,}\DUrole{w}{  }bool\DUrole{p}{,}\DUrole{w}{  }dict\DUrole{p}{{]}}}}
\sphinxAtStartPar
Takes a time step and executes the actions for both Blue RL agent and
hard\sphinxhyphen{}hard coded Red agent.
\begin{quote}\begin{description}
\item[{Parameters}] \leavevmode
\sphinxAtStartPar
\sphinxstyleliteralstrong{\sphinxupquote{action}} \textendash{} The action value generated from the Blue RL agent (int)

\item[{Returns}] \leavevmode
\sphinxAtStartPar
The next environment observation (numpy array)
reward: The reward value for that timestep (int)
done: Whether the epsiode is done (bool)

\item[{Return type}] \leavevmode
\sphinxAtStartPar
observation

\end{description}\end{quote}

\end{fulllineitems}


\end{fulllineitems}



\subparagraph{yawning\_titan.envs.specific.four\_node\_def module}
\label{\detokenize{source/yawning_titan.envs.specific:module-yawning_titan.envs.specific.four_node_def}}\label{\detokenize{source/yawning_titan.envs.specific:yawning-titan-envs-specific-four-node-def-module}}\index{module@\spxentry{module}!yawning\_titan.envs.specific.four\_node\_def@\spxentry{yawning\_titan.envs.specific.four\_node\_def}}\index{yawning\_titan.envs.specific.four\_node\_def@\spxentry{yawning\_titan.envs.specific.four\_node\_def}!module@\spxentry{module}}
\sphinxAtStartPar
NOTE: This environment is deprecated but has been included as an example of how to create a specific environment


\subparagraph{Four Node Connected Environment}
\label{\detokenize{source/yawning_titan.envs.specific:four-node-connected-environment}}\begin{description}
\item[{This environment is made up of four nodes in the following topology:}] \leavevmode\begin{quote}
\end{quote}

\sphinxAtStartPar
+————\textendash{}+   Node 1 \sphinxhyphen{} UAD    +—————+
|              |     Red Start     |               |
|              |                   |     +———+———+

\end{description}

\sphinxAtStartPar
+——\textendash{}+———\sphinxhyphen{}+   |                   |     |                   |
|                   |   +——————\sphinxhyphen{}+     |                   |
|                   |                             |    Node 3 \sphinxhyphen{} UAD   |
|  Node 2 \sphinxhyphen{} UAD     |                             |                   |
|                   |   +——————\sphinxhyphen{}+     |                   |
|                   |   |                   |     +———+———+
+——\sphinxhyphen{}+———\textendash{}+   |   Node 4 \sphinxhyphen{} UAD    |               |
\begin{quote}

\begin{DUlineblock}{0em}
\item[] |    Objective      |               |
\end{DUlineblock}
\begin{description}
\item[{+—————+                   +—————+}] \leavevmode
\begin{DUlineblock}{0em}
\item[] {\color{red}\bfseries{}|}
\item[] {\color{red}\bfseries{}|}
\end{DUlineblock}


\begin{savenotes}\sphinxattablestart
\centering
\begin{tabulary}{\linewidth}[t]{|}
\hline

\end{tabulary}
\par
\sphinxattableend\end{savenotes}

\end{description}
\end{quote}

\sphinxAtStartPar
The aim of this environment is for a blue team agent, which has full observability of the
environment, to successfully stop the hard code red agent from getting to the objective.

\sphinxAtStartPar
Inspired by:
\sphinxhyphen{} \sphinxurl{https://github.com/panlybero/MARL-POMDP-CYBERSECURITY}
\sphinxhyphen{} \sphinxurl{https://www.scitepress.org/Link.aspx?doi=10.5220\%2f0006197105590566}
\index{FourNodeDef (class in yawning\_titan.envs.specific.four\_node\_def)@\spxentry{FourNodeDef}\spxextra{class in yawning\_titan.envs.specific.four\_node\_def}}

\begin{fulllineitems}
\phantomsection\label{\detokenize{source/yawning_titan.envs.specific:yawning_titan.envs.specific.four_node_def.FourNodeDef}}\pysiglinewithargsret{\sphinxbfcode{\sphinxupquote{class\DUrole{w}{  }}}\sphinxcode{\sphinxupquote{yawning\_titan.envs.specific.four\_node\_def.}}\sphinxbfcode{\sphinxupquote{FourNodeDef}}}{\emph{\DUrole{n}{attacker\_skill}\DUrole{p}{:}\DUrole{w}{  }\DUrole{n}{float}\DUrole{w}{  }\DUrole{o}{=}\DUrole{w}{  }\DUrole{default_value}{90}}, \emph{\DUrole{n}{red\_start\_node}\DUrole{p}{:}\DUrole{w}{  }\DUrole{n}{int}\DUrole{w}{  }\DUrole{o}{=}\DUrole{w}{  }\DUrole{default_value}{0}}, \emph{\DUrole{n}{objective\_node}\DUrole{p}{:}\DUrole{w}{  }\DUrole{n}{int}\DUrole{w}{  }\DUrole{o}{=}\DUrole{w}{  }\DUrole{default_value}{3}}, \emph{\DUrole{n}{n\_machines}\DUrole{p}{:}\DUrole{w}{  }\DUrole{n}{int}\DUrole{w}{  }\DUrole{o}{=}\DUrole{w}{  }\DUrole{default_value}{4}}, \emph{\DUrole{n}{attack\_success\_threshold}\DUrole{p}{:}\DUrole{w}{  }\DUrole{n}{float}\DUrole{w}{  }\DUrole{o}{=}\DUrole{w}{  }\DUrole{default_value}{0.6}}}{}
\sphinxAtStartPar
Bases: \sphinxcode{\sphinxupquote{gym.core.Env}}
\index{render() (yawning\_titan.envs.specific.four\_node\_def.FourNodeDef method)@\spxentry{render()}\spxextra{yawning\_titan.envs.specific.four\_node\_def.FourNodeDef method}}

\begin{fulllineitems}
\phantomsection\label{\detokenize{source/yawning_titan.envs.specific:yawning_titan.envs.specific.four_node_def.FourNodeDef.render}}\pysiglinewithargsret{\sphinxbfcode{\sphinxupquote{render}}}{\emph{\DUrole{n}{mode}\DUrole{p}{:}\DUrole{w}{  }\DUrole{n}{str}\DUrole{w}{  }\DUrole{o}{=}\DUrole{w}{  }\DUrole{default_value}{\textquotesingle{}human\textquotesingle{}}}}{}
\sphinxAtStartPar
Render the environment to the screen so that it can be played in
realtime

\end{fulllineitems}

\index{reset() (yawning\_titan.envs.specific.four\_node\_def.FourNodeDef method)@\spxentry{reset()}\spxextra{yawning\_titan.envs.specific.four\_node\_def.FourNodeDef method}}

\begin{fulllineitems}
\phantomsection\label{\detokenize{source/yawning_titan.envs.specific:yawning_titan.envs.specific.four_node_def.FourNodeDef.reset}}\pysiglinewithargsret{\sphinxbfcode{\sphinxupquote{reset}}}{}{{ $\rightarrow$ numpy.array}}
\sphinxAtStartPar
Resets the environment to the default state
\begin{quote}\begin{description}
\item[{Returns}] \leavevmode
\sphinxAtStartPar
A new starting observation (numpy array)

\end{description}\end{quote}

\end{fulllineitems}

\index{step() (yawning\_titan.envs.specific.four\_node\_def.FourNodeDef method)@\spxentry{step()}\spxextra{yawning\_titan.envs.specific.four\_node\_def.FourNodeDef method}}

\begin{fulllineitems}
\phantomsection\label{\detokenize{source/yawning_titan.envs.specific:yawning_titan.envs.specific.four_node_def.FourNodeDef.step}}\pysiglinewithargsret{\sphinxbfcode{\sphinxupquote{step}}}{\emph{\DUrole{n}{action}\DUrole{p}{:}\DUrole{w}{  }\DUrole{n}{int}}}{{ $\rightarrow$ Tuple\DUrole{p}{{[}}numpy.array\DUrole{p}{,}\DUrole{w}{  }float\DUrole{p}{,}\DUrole{w}{  }bool\DUrole{p}{,}\DUrole{w}{  }dict\DUrole{p}{{]}}}}
\sphinxAtStartPar
Takes a time step and executes the actions for both Blue RL agent and
hard\sphinxhyphen{}hard coded Red agent.
\begin{quote}\begin{description}
\item[{Parameters}] \leavevmode
\sphinxAtStartPar
\sphinxstyleliteralstrong{\sphinxupquote{action}} \textendash{} The action value generated from the Blue RL agent (int)

\item[{Returns}] \leavevmode
\sphinxAtStartPar
The next environment observation (numpy array)
reward: The reward value for that timestep (int)
done: Whether the epsiode is done (bool)
notes: An empty notes dictionary

\item[{Return type}] \leavevmode
\sphinxAtStartPar
observation

\end{description}\end{quote}

\end{fulllineitems}


\end{fulllineitems}



\subparagraph{yawning\_titan.envs.specific.graph\_explore module}
\label{\detokenize{source/yawning_titan.envs.specific:module-yawning_titan.envs.specific.graph_explore}}\label{\detokenize{source/yawning_titan.envs.specific:yawning-titan-envs-specific-graph-explore-module}}\index{module@\spxentry{module}!yawning\_titan.envs.specific.graph\_explore@\spxentry{yawning\_titan.envs.specific.graph\_explore}}\index{yawning\_titan.envs.specific.graph\_explore@\spxentry{yawning\_titan.envs.specific.graph\_explore}!module@\spxentry{module}}\index{GraphExplore (class in yawning\_titan.envs.specific.graph\_explore)@\spxentry{GraphExplore}\spxextra{class in yawning\_titan.envs.specific.graph\_explore}}

\begin{fulllineitems}
\phantomsection\label{\detokenize{source/yawning_titan.envs.specific:yawning_titan.envs.specific.graph_explore.GraphExplore}}\pysigline{\sphinxbfcode{\sphinxupquote{class\DUrole{w}{  }}}\sphinxcode{\sphinxupquote{yawning\_titan.envs.specific.graph\_explore.}}\sphinxbfcode{\sphinxupquote{GraphExplore}}}
\sphinxAtStartPar
Bases: \sphinxcode{\sphinxupquote{gym.core.Env}}

\sphinxAtStartPar
A custom environment that follows gym interface, this environment
emulates a network and enables an agent to select which node
to visit, if it is not possible to move to the node the agent is denied
the move.
\index{GAME\_MAX (yawning\_titan.envs.specific.graph\_explore.GraphExplore attribute)@\spxentry{GAME\_MAX}\spxextra{yawning\_titan.envs.specific.graph\_explore.GraphExplore attribute}}

\begin{fulllineitems}
\phantomsection\label{\detokenize{source/yawning_titan.envs.specific:yawning_titan.envs.specific.graph_explore.GraphExplore.GAME_MAX}}\pysigline{\sphinxbfcode{\sphinxupquote{GAME\_MAX}}\sphinxbfcode{\sphinxupquote{\DUrole{w}{  }\DUrole{p}{=}\DUrole{w}{  }1000}}}
\end{fulllineitems}

\index{NODES (yawning\_titan.envs.specific.graph\_explore.GraphExplore attribute)@\spxentry{NODES}\spxextra{yawning\_titan.envs.specific.graph\_explore.GraphExplore attribute}}

\begin{fulllineitems}
\phantomsection\label{\detokenize{source/yawning_titan.envs.specific:yawning_titan.envs.specific.graph_explore.GraphExplore.NODES}}\pysigline{\sphinxbfcode{\sphinxupquote{NODES}}\sphinxbfcode{\sphinxupquote{\DUrole{w}{  }\DUrole{p}{=}\DUrole{w}{  }10}}}
\end{fulllineitems}

\index{SEED (yawning\_titan.envs.specific.graph\_explore.GraphExplore attribute)@\spxentry{SEED}\spxextra{yawning\_titan.envs.specific.graph\_explore.GraphExplore attribute}}

\begin{fulllineitems}
\phantomsection\label{\detokenize{source/yawning_titan.envs.specific:yawning_titan.envs.specific.graph_explore.GraphExplore.SEED}}\pysigline{\sphinxbfcode{\sphinxupquote{SEED}}\sphinxbfcode{\sphinxupquote{\DUrole{w}{  }\DUrole{p}{=}\DUrole{w}{  }1010}}}
\end{fulllineitems}

\index{close() (yawning\_titan.envs.specific.graph\_explore.GraphExplore method)@\spxentry{close()}\spxextra{yawning\_titan.envs.specific.graph\_explore.GraphExplore method}}

\begin{fulllineitems}
\phantomsection\label{\detokenize{source/yawning_titan.envs.specific:yawning_titan.envs.specific.graph_explore.GraphExplore.close}}\pysiglinewithargsret{\sphinxbfcode{\sphinxupquote{close}}}{}{}
\sphinxAtStartPar
Remove all open visualisations

\end{fulllineitems}

\index{metadata (yawning\_titan.envs.specific.graph\_explore.GraphExplore attribute)@\spxentry{metadata}\spxextra{yawning\_titan.envs.specific.graph\_explore.GraphExplore attribute}}

\begin{fulllineitems}
\phantomsection\label{\detokenize{source/yawning_titan.envs.specific:yawning_titan.envs.specific.graph_explore.GraphExplore.metadata}}\pysigline{\sphinxbfcode{\sphinxupquote{metadata}}\sphinxbfcode{\sphinxupquote{\DUrole{w}{  }\DUrole{p}{=}\DUrole{w}{  }\{\textquotesingle{}render.modes\textquotesingle{}: {[}\textquotesingle{}human\textquotesingle{}{]}\}}}}
\end{fulllineitems}

\index{render() (yawning\_titan.envs.specific.graph\_explore.GraphExplore method)@\spxentry{render()}\spxextra{yawning\_titan.envs.specific.graph\_explore.GraphExplore method}}

\begin{fulllineitems}
\phantomsection\label{\detokenize{source/yawning_titan.envs.specific:yawning_titan.envs.specific.graph_explore.GraphExplore.render}}\pysiglinewithargsret{\sphinxbfcode{\sphinxupquote{render}}}{\emph{\DUrole{n}{mode}\DUrole{p}{:}\DUrole{w}{  }\DUrole{n}{str}\DUrole{w}{  }\DUrole{o}{=}\DUrole{w}{  }\DUrole{default_value}{\textquotesingle{}live\textquotesingle{}}}, \emph{\DUrole{n}{close}\DUrole{p}{:}\DUrole{w}{  }\DUrole{n}{bool}\DUrole{w}{  }\DUrole{o}{=}\DUrole{w}{  }\DUrole{default_value}{False}}}{}
\sphinxAtStartPar
Render the environment to the screen so that it can be played in
realtime

\end{fulllineitems}

\index{reset() (yawning\_titan.envs.specific.graph\_explore.GraphExplore method)@\spxentry{reset()}\spxextra{yawning\_titan.envs.specific.graph\_explore.GraphExplore method}}

\begin{fulllineitems}
\phantomsection\label{\detokenize{source/yawning_titan.envs.specific:yawning_titan.envs.specific.graph_explore.GraphExplore.reset}}\pysiglinewithargsret{\sphinxbfcode{\sphinxupquote{reset}}}{}{{ $\rightarrow$ numpy.array}}
\sphinxAtStartPar
Reset the initial game configurations

\end{fulllineitems}

\index{step() (yawning\_titan.envs.specific.graph\_explore.GraphExplore method)@\spxentry{step()}\spxextra{yawning\_titan.envs.specific.graph\_explore.GraphExplore method}}

\begin{fulllineitems}
\phantomsection\label{\detokenize{source/yawning_titan.envs.specific:yawning_titan.envs.specific.graph_explore.GraphExplore.step}}\pysiglinewithargsret{\sphinxbfcode{\sphinxupquote{step}}}{\emph{\DUrole{n}{action}\DUrole{p}{:}\DUrole{w}{  }\DUrole{n}{int}}}{{ $\rightarrow$ Tuple\DUrole{p}{{[}}numpy.array\DUrole{p}{,}\DUrole{w}{  }float\DUrole{p}{,}\DUrole{w}{  }bool\DUrole{p}{,}\DUrole{w}{  }dict\DUrole{p}{{]}}}}
\sphinxAtStartPar
Execute one time step within the environment

\end{fulllineitems}

\index{visualisation (yawning\_titan.envs.specific.graph\_explore.GraphExplore attribute)@\spxentry{visualisation}\spxextra{yawning\_titan.envs.specific.graph\_explore.GraphExplore attribute}}

\begin{fulllineitems}
\phantomsection\label{\detokenize{source/yawning_titan.envs.specific:yawning_titan.envs.specific.graph_explore.GraphExplore.visualisation}}\pysigline{\sphinxbfcode{\sphinxupquote{visualisation}}\sphinxbfcode{\sphinxupquote{\DUrole{w}{  }\DUrole{p}{=}\DUrole{w}{  }None}}}
\end{fulllineitems}


\end{fulllineitems}



\subparagraph{yawning\_titan.envs.specific.nsa\_node\_def module}
\label{\detokenize{source/yawning_titan.envs.specific:module-yawning_titan.envs.specific.nsa_node_def}}\label{\detokenize{source/yawning_titan.envs.specific:yawning-titan-envs-specific-nsa-node-def-module}}\index{module@\spxentry{module}!yawning\_titan.envs.specific.nsa\_node\_def@\spxentry{yawning\_titan.envs.specific.nsa\_node\_def}}\index{yawning\_titan.envs.specific.nsa\_node\_def@\spxentry{yawning\_titan.envs.specific.nsa\_node\_def}!module@\spxentry{module}}
\sphinxAtStartPar
A new node network that can be configured for multiple different configurations.

\sphinxAtStartPar
Paper: \sphinxurl{https://www.nsa.gov.Portals/70/documents/resources/everyone/digital-media-center/publications/the-next-wave/TNW-22-1.pdf\#page=9}

\sphinxAtStartPar
Currently suppports:
\sphinxhyphen{} 18 node network from the research paper
\sphinxhyphen{} a network creator that allows you to use multiple topologies and change the connectivity of the network

\sphinxAtStartPar
Red agent actions:
\begin{quote}

\sphinxAtStartPar
Spread \sphinxhyphen{} Tries to spread to each node connected to a compromised node
Randomly infect \sphinxhyphen{} Tries to randomly infect every currently un\sphinxhyphen{}compromised node
\end{quote}

\sphinxAtStartPar
Configurable parameters:
\begin{quote}

\sphinxAtStartPar
chance\_to\_spread \sphinxhyphen{} This is the chance for the red agent to spread between nodes
chance\_to\_spread\_during\_patch \sphinxhyphen{} There is a chance that when a compromised node is patched the red agent “escapes” to
\begin{quote}

\sphinxAtStartPar
neaby nodes and compromises them
\end{quote}

\sphinxAtStartPar
chance\_to\_randomly\_compromise \sphinxhyphen{} This is the chance that the red agent randomly infects a un\sphinxhyphen{}compromised node
cost\_of\_isolate \sphinxhyphen{} The cost (negative reward) associated with performing the isolate action (initially set to 10 based on
\begin{quote}

\sphinxAtStartPar
data from the paper)
\end{quote}
\begin{description}
\item[{cost\_of\_patch \sphinxhyphen{} The cost (negative reward) associated with performing the patch action (initially set to 5 based on}] \leavevmode
\sphinxAtStartPar
data from the paper)

\item[{cost\_of\_nothing \sphinxhyphen{} The cost (negative reward) associated with performing the do nothing action (initially set to 0 based on}] \leavevmode
\sphinxAtStartPar
data from the paper)

\end{description}

\sphinxAtStartPar
end \sphinxhyphen{} The number of steps that the blue agent must survive for to win
spread\_vs\_random\_intrusion \sphinxhyphen{} The chance that the red agent will choose the spread action on its turn as aposed to
\begin{quote}

\sphinxAtStartPar
the random intrusion action
\end{quote}
\begin{description}
\item[{punish\_for\_isolate \sphinxhyphen{} Either True or False. If True then each step the agent is punished based on the number of}] \leavevmode
\sphinxAtStartPar
isolated nodes there are

\item[{reward\_method \sphinxhyphen{} Either 0, 1 or 2. Each constitutes a different method of rewarding the agent:}] \leavevmode\begin{itemize}
\item {}
\sphinxAtStartPar
0 is the papers reward system

\item {}
\sphinxAtStartPar
1 is my reward system rewarding based on number of un\sphinxhyphen{}compromised nodes

\item {}
\sphinxAtStartPar
2 is the minimal reward system. The agent gets 1 for a win or \sphinxhyphen{}1 for a loss

\end{itemize}

\end{description}
\end{quote}
\index{NodeEnv (class in yawning\_titan.envs.specific.nsa\_node\_def)@\spxentry{NodeEnv}\spxextra{class in yawning\_titan.envs.specific.nsa\_node\_def}}

\begin{fulllineitems}
\phantomsection\label{\detokenize{source/yawning_titan.envs.specific:yawning_titan.envs.specific.nsa_node_def.NodeEnv}}\pysiglinewithargsret{\sphinxbfcode{\sphinxupquote{class\DUrole{w}{  }}}\sphinxcode{\sphinxupquote{yawning\_titan.envs.specific.nsa\_node\_def.}}\sphinxbfcode{\sphinxupquote{NodeEnv}}}{\emph{\DUrole{n}{chance\_to\_spread}\DUrole{p}{:}\DUrole{w}{  }\DUrole{n}{float}\DUrole{w}{  }\DUrole{o}{=}\DUrole{w}{  }\DUrole{default_value}{0.01}}, \emph{\DUrole{n}{chance\_to\_spread\_during\_patch}\DUrole{p}{:}\DUrole{w}{  }\DUrole{n}{float}\DUrole{w}{  }\DUrole{o}{=}\DUrole{w}{  }\DUrole{default_value}{0.01}}, \emph{\DUrole{n}{chance\_to\_randomly\_compromise}\DUrole{p}{:}\DUrole{w}{  }\DUrole{n}{float}\DUrole{w}{  }\DUrole{o}{=}\DUrole{w}{  }\DUrole{default_value}{0.15}}, \emph{\DUrole{n}{cost\_of\_isolate}\DUrole{p}{:}\DUrole{w}{  }\DUrole{n}{float}\DUrole{w}{  }\DUrole{o}{=}\DUrole{w}{  }\DUrole{default_value}{10}}, \emph{\DUrole{n}{cost\_of\_patch}\DUrole{p}{:}\DUrole{w}{  }\DUrole{n}{float}\DUrole{w}{  }\DUrole{o}{=}\DUrole{w}{  }\DUrole{default_value}{5}}, \emph{\DUrole{n}{cost\_of\_nothing}\DUrole{p}{:}\DUrole{w}{  }\DUrole{n}{float}\DUrole{w}{  }\DUrole{o}{=}\DUrole{w}{  }\DUrole{default_value}{0}}, \emph{\DUrole{n}{end}\DUrole{p}{:}\DUrole{w}{  }\DUrole{n}{int}\DUrole{w}{  }\DUrole{o}{=}\DUrole{w}{  }\DUrole{default_value}{1000}}, \emph{\DUrole{n}{spread\_vs\_random\_intrusion}\DUrole{p}{:}\DUrole{w}{  }\DUrole{n}{float}\DUrole{w}{  }\DUrole{o}{=}\DUrole{w}{  }\DUrole{default_value}{0.5}}, \emph{\DUrole{n}{punish\_for\_isolate}\DUrole{p}{:}\DUrole{w}{  }\DUrole{n}{bool}\DUrole{w}{  }\DUrole{o}{=}\DUrole{w}{  }\DUrole{default_value}{False}}, \emph{\DUrole{n}{reward\_method}\DUrole{p}{:}\DUrole{w}{  }\DUrole{n}{int}\DUrole{w}{  }\DUrole{o}{=}\DUrole{w}{  }\DUrole{default_value}{1}}, \emph{\DUrole{n}{network}\DUrole{p}{:}\DUrole{w}{  }\DUrole{n}{Tuple\DUrole{p}{{[}}numpy.array\DUrole{p}{,}\DUrole{w}{  }dict\DUrole{p}{{]}}}\DUrole{w}{  }\DUrole{o}{=}\DUrole{w}{  }\DUrole{default_value}{(array({[}{[}0, 0, 0, 0, 0, 1, 0, 0, 0, 0, 0, 0, 0, 0, 0, 0, 0, 0{]}, {[}0, 0, 0, 0, 0, 1, 0, 0, 0, 0, 0, 0, 0, 0, 0, 0, 0, 0{]}, {[}0, 0, 0, 0, 0, 1, 0, 0, 0, 0, 0, 0, 0, 0, 0, 0, 0, 0{]}, {[}0, 0, 0, 0, 0, 1, 0, 0, 0, 0, 0, 0, 0, 0, 0, 0, 0, 0{]}, {[}0, 0, 0, 0, 0, 1, 0, 0, 0, 0, 0, 0, 0, 0, 0, 0, 0, 0{]}, {[}1, 1, 1, 1, 1, 0, 0, 1, 0, 0, 0, 0, 0, 0, 0, 0, 0, 0{]}, {[}0, 0, 0, 0, 0, 0, 0, 1, 0, 0, 0, 0, 0, 0, 0, 0, 0, 0{]}, {[}0, 0, 0, 0, 0, 1, 1, 0, 1, 0, 0, 0, 1, 0, 0, 0, 0, 0{]}, {[}0, 0, 0, 0, 0, 0, 0, 1, 0, 1, 1, 1, 0, 0, 0, 0, 0, 0{]}, {[}0, 0, 0, 0, 0, 0, 0, 0, 1, 0, 0, 0, 0, 0, 0, 0, 0, 0{]}, {[}0, 0, 0, 0, 0, 0, 0, 0, 1, 0, 0, 0, 0, 0, 0, 0, 0, 0{]}, {[}0, 0, 0, 0, 0, 0, 0, 0, 1, 0, 0, 0, 0, 0, 0, 0, 0, 0{]}, {[}0, 0, 0, 0, 0, 0, 0, 1, 0, 0, 0, 0, 0, 1, 1, 1, 1, 1{]}, {[}0, 0, 0, 0, 0, 0, 0, 0, 0, 0, 0, 0, 1, 0, 0, 0, 0, 0{]}, {[}0, 0, 0, 0, 0, 0, 0, 0, 0, 0, 0, 0, 1, 0, 0, 0, 0, 0{]}, {[}0, 0, 0, 0, 0, 0, 0, 0, 0, 0, 0, 0, 1, 0, 0, 0, 0, 0{]}, {[}0, 0, 0, 0, 0, 0, 0, 0, 0, 0, 0, 0, 1, 0, 0, 0, 0, 0{]}, {[}0, 0, 0, 0, 0, 0, 0, 0, 0, 0, 0, 0, 1, 0, 0, 0, 0, 0{]}{]}), \{\textquotesingle{}0\textquotesingle{}: {[}1, 7{]}, \textquotesingle{}1\textquotesingle{}: {[}2, 7{]}, \textquotesingle{}2\textquotesingle{}: {[}3, 7{]}, \textquotesingle{}3\textquotesingle{}: {[}4, 7{]}, \textquotesingle{}4\textquotesingle{}: {[}5, 7{]}, \textquotesingle{}5\textquotesingle{}: {[}3, 6{]}, \textquotesingle{}6\textquotesingle{}: {[}1, 4{]}, \textquotesingle{}7\textquotesingle{}: {[}3, 4{]}, \textquotesingle{}8\textquotesingle{}: {[}4, 4{]}, \textquotesingle{}9\textquotesingle{}: {[}6, 5{]}, \textquotesingle{}10\textquotesingle{}: {[}6, 4{]}, \textquotesingle{}11\textquotesingle{}: {[}6, 3{]}, \textquotesingle{}12\textquotesingle{}: {[}3, 2{]}, \textquotesingle{}13\textquotesingle{}: {[}1, 1{]}, \textquotesingle{}14\textquotesingle{}: {[}2, 1{]}, \textquotesingle{}15\textquotesingle{}: {[}3, 1{]}, \textquotesingle{}16\textquotesingle{}: {[}4, 1{]}, \textquotesingle{}17\textquotesingle{}: {[}5, 1{]}\})}}}{}
\sphinxAtStartPar
Bases: \sphinxcode{\sphinxupquote{gym.core.Env}}
\index{render() (yawning\_titan.envs.specific.nsa\_node\_def.NodeEnv method)@\spxentry{render()}\spxextra{yawning\_titan.envs.specific.nsa\_node\_def.NodeEnv method}}

\begin{fulllineitems}
\phantomsection\label{\detokenize{source/yawning_titan.envs.specific:yawning_titan.envs.specific.nsa_node_def.NodeEnv.render}}\pysiglinewithargsret{\sphinxbfcode{\sphinxupquote{render}}}{\emph{\DUrole{n}{mode}\DUrole{p}{:}\DUrole{w}{  }\DUrole{n}{str}\DUrole{w}{  }\DUrole{o}{=}\DUrole{w}{  }\DUrole{default_value}{\textquotesingle{}human\textquotesingle{}}}}{}
\sphinxAtStartPar
Renders the network using the graph2plot class. This uses a networkx representation of the network
\begin{quote}\begin{description}
\item[{Parameters}] \leavevmode
\sphinxAtStartPar
\sphinxstyleliteralstrong{\sphinxupquote{mode}} \textendash{} the mode of the rendering

\end{description}\end{quote}

\end{fulllineitems}

\index{reset() (yawning\_titan.envs.specific.nsa\_node\_def.NodeEnv method)@\spxentry{reset()}\spxextra{yawning\_titan.envs.specific.nsa\_node\_def.NodeEnv method}}

\begin{fulllineitems}
\phantomsection\label{\detokenize{source/yawning_titan.envs.specific:yawning_titan.envs.specific.nsa_node_def.NodeEnv.reset}}\pysiglinewithargsret{\sphinxbfcode{\sphinxupquote{reset}}}{}{{ $\rightarrow$ numpy.array}}
\sphinxAtStartPar
Resets the environment to the default state
\begin{quote}\begin{description}
\item[{Returns}] \leavevmode
\sphinxAtStartPar
A new starting observation (numpy array)

\end{description}\end{quote}

\end{fulllineitems}

\index{step() (yawning\_titan.envs.specific.nsa\_node\_def.NodeEnv method)@\spxentry{step()}\spxextra{yawning\_titan.envs.specific.nsa\_node\_def.NodeEnv method}}

\begin{fulllineitems}
\phantomsection\label{\detokenize{source/yawning_titan.envs.specific:yawning_titan.envs.specific.nsa_node_def.NodeEnv.step}}\pysiglinewithargsret{\sphinxbfcode{\sphinxupquote{step}}}{\emph{\DUrole{n}{action}\DUrole{p}{:}\DUrole{w}{  }\DUrole{n}{int}}}{{ $\rightarrow$ Tuple\DUrole{p}{{[}}numpy.array\DUrole{p}{,}\DUrole{w}{  }float\DUrole{p}{,}\DUrole{w}{  }bool\DUrole{p}{,}\DUrole{w}{  }dict\DUrole{p}{{]}}}}
\sphinxAtStartPar
Takes a time step and executes the actions for both Blue RL agent and
hard\sphinxhyphen{}hard coded Red agent.
\begin{quote}\begin{description}
\item[{Parameters}] \leavevmode
\sphinxAtStartPar
\sphinxstyleliteralstrong{\sphinxupquote{action}} \textendash{} The action value generated from the Blue RL agent (int)

\item[{Returns}] \leavevmode
\sphinxAtStartPar
The next environment observation (numpy array)
reward: The reward value for that timestep (int)
done: Whether the epsiode is done (bool)
info: a dictionary containing info about the current state

\item[{Return type}] \leavevmode
\sphinxAtStartPar
observation

\end{description}\end{quote}

\end{fulllineitems}


\end{fulllineitems}



\subparagraph{Module contents}
\label{\detokenize{source/yawning_titan.envs.specific:module-yawning_titan.envs.specific}}\label{\detokenize{source/yawning_titan.envs.specific:module-contents}}\index{module@\spxentry{module}!yawning\_titan.envs.specific@\spxentry{yawning\_titan.envs.specific}}\index{yawning\_titan.envs.specific@\spxentry{yawning\_titan.envs.specific}!module@\spxentry{module}}

\paragraph{Module contents}
\label{\detokenize{source/yawning_titan.envs:module-yawning_titan.envs}}\label{\detokenize{source/yawning_titan.envs:module-contents}}\index{module@\spxentry{module}!yawning\_titan.envs@\spxentry{yawning\_titan.envs}}\index{yawning\_titan.envs@\spxentry{yawning\_titan.envs}!module@\spxentry{module}}

\subsubsection{yawning\_titan.experiment\_helpers package}
\label{\detokenize{source/yawning_titan.experiment_helpers:yawning-titan-experiment-helpers-package}}\label{\detokenize{source/yawning_titan.experiment_helpers::doc}}

\paragraph{Submodules}
\label{\detokenize{source/yawning_titan.experiment_helpers:submodules}}

\paragraph{yawning\_titan.experiment\_helpers.constants module}
\label{\detokenize{source/yawning_titan.experiment_helpers:module-yawning_titan.experiment_helpers.constants}}\label{\detokenize{source/yawning_titan.experiment_helpers:yawning-titan-experiment-helpers-constants-module}}\index{module@\spxentry{module}!yawning\_titan.experiment\_helpers.constants@\spxentry{yawning\_titan.experiment\_helpers.constants}}\index{yawning\_titan.experiment\_helpers.constants@\spxentry{yawning\_titan.experiment\_helpers.constants}!module@\spxentry{module}}

\paragraph{yawning\_titan.experiment\_helpers.graph\_metrics module}
\label{\detokenize{source/yawning_titan.experiment_helpers:module-yawning_titan.experiment_helpers.graph_metrics}}\label{\detokenize{source/yawning_titan.experiment_helpers:yawning-titan-experiment-helpers-graph-metrics-module}}\index{module@\spxentry{module}!yawning\_titan.experiment\_helpers.graph\_metrics@\spxentry{yawning\_titan.experiment\_helpers.graph\_metrics}}\index{yawning\_titan.experiment\_helpers.graph\_metrics@\spxentry{yawning\_titan.experiment\_helpers.graph\_metrics}!module@\spxentry{module}}
\sphinxAtStartPar
graph\_metrics.py  \sphinxhyphen{} Collection of functions to help generating metrics and summary statistics for networkx graphs
\index{flatten\_list() (in module yawning\_titan.experiment\_helpers.graph\_metrics)@\spxentry{flatten\_list()}\spxextra{in module yawning\_titan.experiment\_helpers.graph\_metrics}}

\begin{fulllineitems}
\phantomsection\label{\detokenize{source/yawning_titan.experiment_helpers:yawning_titan.experiment_helpers.graph_metrics.flatten_list}}\pysiglinewithargsret{\sphinxcode{\sphinxupquote{yawning\_titan.experiment\_helpers.graph\_metrics.}}\sphinxbfcode{\sphinxupquote{flatten\_list}}}{\emph{\DUrole{n}{list\_input}\DUrole{p}{:}\DUrole{w}{  }\DUrole{n}{list}}}{{ $\rightarrow$ list}}
\sphinxAtStartPar
Takes a list of lists and flattens them into a single list
\begin{quote}\begin{description}
\item[{Parameters}] \leavevmode
\sphinxAtStartPar
\sphinxstyleliteralstrong{\sphinxupquote{list\_input}} \textendash{} The input list of lists to be processed

\item[{Returns}] \leavevmode
\sphinxAtStartPar
A single list containing all elements

\end{description}\end{quote}

\end{fulllineitems}

\index{geometric\_mean\_overflow() (in module yawning\_titan.experiment\_helpers.graph\_metrics)@\spxentry{geometric\_mean\_overflow()}\spxextra{in module yawning\_titan.experiment\_helpers.graph\_metrics}}

\begin{fulllineitems}
\phantomsection\label{\detokenize{source/yawning_titan.experiment_helpers:yawning_titan.experiment_helpers.graph_metrics.geometric_mean_overflow}}\pysiglinewithargsret{\sphinxcode{\sphinxupquote{yawning\_titan.experiment\_helpers.graph\_metrics.}}\sphinxbfcode{\sphinxupquote{geometric\_mean\_overflow}}}{\emph{\DUrole{n}{input\_list}\DUrole{p}{:}\DUrole{w}{  }\DUrole{n}{List}}}{{ $\rightarrow$ float}}
\sphinxAtStartPar
Calculates the geometric mean accounting for the
potential of overflow through using logs
\begin{quote}\begin{description}
\item[{Parameters}] \leavevmode
\sphinxAtStartPar
\sphinxstyleliteralstrong{\sphinxupquote{input\_list}} \textendash{} A list of values

\item[{Returns}] \leavevmode
\sphinxAtStartPar
Geometric mean as a float

\end{description}\end{quote}

\sphinxAtStartPar
Note: There is actually a function included in the ‘statistics’
python module that does this but is only available in python 3.8 onwards

\end{fulllineitems}

\index{get\_assortativity\_metrics() (in module yawning\_titan.experiment\_helpers.graph\_metrics)@\spxentry{get\_assortativity\_metrics()}\spxextra{in module yawning\_titan.experiment\_helpers.graph\_metrics}}

\begin{fulllineitems}
\phantomsection\label{\detokenize{source/yawning_titan.experiment_helpers:yawning_titan.experiment_helpers.graph_metrics.get_assortativity_metrics}}\pysiglinewithargsret{\sphinxcode{\sphinxupquote{yawning\_titan.experiment\_helpers.graph\_metrics.}}\sphinxbfcode{\sphinxupquote{get\_assortativity\_metrics}}}{\emph{\DUrole{n}{graph}\DUrole{p}{:}\DUrole{w}{  }\DUrole{n}{networkx.classes.graph.Graph}}}{}
\sphinxAtStartPar
Get’s assortativity metrics for an input graph using
networkx’s in\sphinxhyphen{}built algorithms
\begin{quote}\begin{description}
\item[{Parameters}] \leavevmode
\sphinxAtStartPar
\sphinxstyleliteralstrong{\sphinxupquote{graph}} \textendash{} A networkx graph

\item[{Returns}] \leavevmode
\sphinxAtStartPar
A two\sphinxhyphen{}tuple with the metrics

\end{description}\end{quote}

\end{fulllineitems}

\index{get\_func\_summary\_statistics() (in module yawning\_titan.experiment\_helpers.graph\_metrics)@\spxentry{get\_func\_summary\_statistics()}\spxextra{in module yawning\_titan.experiment\_helpers.graph\_metrics}}

\begin{fulllineitems}
\phantomsection\label{\detokenize{source/yawning_titan.experiment_helpers:yawning_titan.experiment_helpers.graph_metrics.get_func_summary_statistics}}\pysiglinewithargsret{\sphinxcode{\sphinxupquote{yawning\_titan.experiment\_helpers.graph\_metrics.}}\sphinxbfcode{\sphinxupquote{get\_func\_summary\_statistics}}}{\emph{\DUrole{n}{func}\DUrole{p}{:}\DUrole{w}{  }\DUrole{n}{Callable}}}{{ $\rightarrow$ list}}
\sphinxAtStartPar
Generates a list of summary statistics based on the
output of a networkx in\sphinxhyphen{}build algorithm
\begin{quote}\begin{description}
\item[{Parameters}] \leavevmode
\sphinxAtStartPar
\sphinxstyleliteralstrong{\sphinxupquote{func}} \textendash{} A networkx algorithm function

\item[{Returns}] \leavevmode
\sphinxAtStartPar
\begin{itemize}
\item {}
\sphinxAtStartPar
Arithmetic Mean

\item {}
\sphinxAtStartPar
Geometric Mean

\item {}
\sphinxAtStartPar
Harmonic Mean

\item {}
\sphinxAtStartPar
Standard Deviation

\item {}
\sphinxAtStartPar
Variance

\item {}
\sphinxAtStartPar
Median

\end{itemize}


\item[{Return type}] \leavevmode
\sphinxAtStartPar
A list containing

\end{description}\end{quote}
\subsubsection*{Example}

\sphinxAtStartPar
\textgreater{} generate\_summary\_statistics(nx.degree\_centrality(graph))
\textgreater{} (3.3095238095238098, 2.5333333333333337, 2.9015675801088014, 1.9824913893491538, 2.620181405895692, 3.0)

\end{fulllineitems}

\index{get\_graph\_metric\_bundle() (in module yawning\_titan.experiment\_helpers.graph\_metrics)@\spxentry{get\_graph\_metric\_bundle()}\spxextra{in module yawning\_titan.experiment\_helpers.graph\_metrics}}

\begin{fulllineitems}
\phantomsection\label{\detokenize{source/yawning_titan.experiment_helpers:yawning_titan.experiment_helpers.graph_metrics.get_graph_metric_bundle}}\pysiglinewithargsret{\sphinxcode{\sphinxupquote{yawning\_titan.experiment\_helpers.graph\_metrics.}}\sphinxbfcode{\sphinxupquote{get\_graph\_metric\_bundle}}}{\emph{\DUrole{n}{graph}\DUrole{p}{:}\DUrole{w}{  }\DUrole{n}{networkx.classes.graph.Graph}}}{{ $\rightarrow$ List\DUrole{p}{{[}}List\DUrole{p}{{]}}}}
\sphinxAtStartPar
Generates a graph metric bundle that includes the summary statistics
for a collection of networkx in\sphinxhyphen{}built algorithms.
\begin{description}
\item[{Algorithms used:}] \leavevmode\begin{itemize}
\item {}
\sphinxAtStartPar
Average Degree Connectivity

\item {}
\sphinxAtStartPar
Closeness Centrality

\item {}
\sphinxAtStartPar
Degree Centrality

\item {}
\sphinxAtStartPar
Eigenvector Centrality

\item {}
\sphinxAtStartPar
Communicability Between\sphinxhyphen{}ness Centrality

\end{itemize}

\end{description}
\begin{quote}\begin{description}
\item[{Parameters}] \leavevmode
\sphinxAtStartPar
\sphinxstyleliteralstrong{\sphinxupquote{graph}} \textendash{} A networkx graph

\item[{Returns}] \leavevmode
\sphinxAtStartPar
A list of lists containing

\end{description}\end{quote}

\end{fulllineitems}

\index{pprint\_metric\_table() (in module yawning\_titan.experiment\_helpers.graph\_metrics)@\spxentry{pprint\_metric\_table()}\spxextra{in module yawning\_titan.experiment\_helpers.graph\_metrics}}

\begin{fulllineitems}
\phantomsection\label{\detokenize{source/yawning_titan.experiment_helpers:yawning_titan.experiment_helpers.graph_metrics.pprint_metric_table}}\pysiglinewithargsret{\sphinxcode{\sphinxupquote{yawning\_titan.experiment\_helpers.graph\_metrics.}}\sphinxbfcode{\sphinxupquote{pprint\_metric\_table}}}{\emph{\DUrole{n}{metric\_output}\DUrole{p}{:}\DUrole{w}{  }\DUrole{n}{List\DUrole{p}{{[}}List\DUrole{p}{{]}}}}, \emph{\DUrole{n}{headers}\DUrole{o}{=}\DUrole{default_value}{None}}}{}
\sphinxAtStartPar
Pretty prints graph metrics to the terminal using the tabulate
module.
\begin{quote}\begin{description}
\item[{Parameters}] \leavevmode\begin{itemize}
\item {}
\sphinxAtStartPar
\sphinxstyleliteralstrong{\sphinxupquote{metric\_output}} \textendash{} A list of lists containing the values to be printed.

\item {}
\sphinxAtStartPar
\sphinxstyleliteralstrong{\sphinxupquote{headers}} \textendash{} A list of heading names (optional)

\end{itemize}

\item[{Returns}] \leavevmode
\sphinxAtStartPar
A formatted table to terminal

\end{description}\end{quote}

\end{fulllineitems}



\paragraph{yawning\_titan.experiment\_helpers.model\_evaluation module}
\label{\detokenize{source/yawning_titan.experiment_helpers:module-yawning_titan.experiment_helpers.model_evaluation}}\label{\detokenize{source/yawning_titan.experiment_helpers:yawning-titan-experiment-helpers-model-evaluation-module}}\index{module@\spxentry{module}!yawning\_titan.experiment\_helpers.model\_evaluation@\spxentry{yawning\_titan.experiment\_helpers.model\_evaluation}}\index{yawning\_titan.experiment\_helpers.model\_evaluation@\spxentry{yawning\_titan.experiment\_helpers.model\_evaluation}!module@\spxentry{module}}
\sphinxAtStartPar
model\_evaluation.py

\sphinxAtStartPar
This file is currently depreciated and not maintained.
\index{show\_eval\_data() (in module yawning\_titan.experiment\_helpers.model\_evaluation)@\spxentry{show\_eval\_data()}\spxextra{in module yawning\_titan.experiment\_helpers.model\_evaluation}}

\begin{fulllineitems}
\phantomsection\label{\detokenize{source/yawning_titan.experiment_helpers:yawning_titan.experiment_helpers.model_evaluation.show_eval_data}}\pysiglinewithargsret{\sphinxcode{\sphinxupquote{yawning\_titan.experiment\_helpers.model\_evaluation.}}\sphinxbfcode{\sphinxupquote{show\_eval\_data}}}{}{}
\sphinxAtStartPar
Function that creates a box plot that shows the spread of the rewards from the different models

\end{fulllineitems}

\index{show\_safe\_nodes() (in module yawning\_titan.experiment\_helpers.model\_evaluation)@\spxentry{show\_safe\_nodes()}\spxextra{in module yawning\_titan.experiment\_helpers.model\_evaluation}}

\begin{fulllineitems}
\phantomsection\label{\detokenize{source/yawning_titan.experiment_helpers:yawning_titan.experiment_helpers.model_evaluation.show_safe_nodes}}\pysiglinewithargsret{\sphinxcode{\sphinxupquote{yawning\_titan.experiment\_helpers.model\_evaluation.}}\sphinxbfcode{\sphinxupquote{show\_safe\_nodes}}}{}{}
\sphinxAtStartPar
Function that gets the number of safe nodes at each time step of the model. Uses this data to then make a graph to
be able to better visualise this data

\end{fulllineitems}

\index{show\_training\_data() (in module yawning\_titan.experiment\_helpers.model\_evaluation)@\spxentry{show\_training\_data()}\spxextra{in module yawning\_titan.experiment\_helpers.model\_evaluation}}

\begin{fulllineitems}
\phantomsection\label{\detokenize{source/yawning_titan.experiment_helpers:yawning_titan.experiment_helpers.model_evaluation.show_training_data}}\pysiglinewithargsret{\sphinxcode{\sphinxupquote{yawning\_titan.experiment\_helpers.model\_evaluation.}}\sphinxbfcode{\sphinxupquote{show\_training\_data}}}{}{}\begin{description}
\item[{Function that creates a plot with the following graphs:}] \leavevmode\begin{itemize}
\item {}
\sphinxAtStartPar
Reward vs time step

\item {}
\sphinxAtStartPar
Episode length vs time step

\item {}
\sphinxAtStartPar
Episode length vs Reward

\end{itemize}

\end{description}

\sphinxAtStartPar
This data is taken from the tensorboards created while the models are training

\end{fulllineitems}



\paragraph{yawning\_titan.experiment\_helpers.rllib module}
\label{\detokenize{source/yawning_titan.experiment_helpers:module-yawning_titan.experiment_helpers.rllib}}\label{\detokenize{source/yawning_titan.experiment_helpers:yawning-titan-experiment-helpers-rllib-module}}\index{module@\spxentry{module}!yawning\_titan.experiment\_helpers.rllib@\spxentry{yawning\_titan.experiment\_helpers.rllib}}\index{yawning\_titan.experiment\_helpers.rllib@\spxentry{yawning\_titan.experiment\_helpers.rllib}!module@\spxentry{module}}\index{env\_creator() (in module yawning\_titan.experiment\_helpers.rllib)@\spxentry{env\_creator()}\spxextra{in module yawning\_titan.experiment\_helpers.rllib}}

\begin{fulllineitems}
\phantomsection\label{\detokenize{source/yawning_titan.experiment_helpers:yawning_titan.experiment_helpers.rllib.env_creator}}\pysiglinewithargsret{\sphinxcode{\sphinxupquote{yawning\_titan.experiment\_helpers.rllib.}}\sphinxbfcode{\sphinxupquote{env\_creator}}}{\emph{\DUrole{n}{env\_name}}}{}
\end{fulllineitems}

\index{train\_impala() (in module yawning\_titan.experiment\_helpers.rllib)@\spxentry{train\_impala()}\spxextra{in module yawning\_titan.experiment\_helpers.rllib}}

\begin{fulllineitems}
\phantomsection\label{\detokenize{source/yawning_titan.experiment_helpers:yawning_titan.experiment_helpers.rllib.train_impala}}\pysiglinewithargsret{\sphinxcode{\sphinxupquote{yawning\_titan.experiment\_helpers.rllib.}}\sphinxbfcode{\sphinxupquote{train\_impala}}}{\emph{\DUrole{n}{env\_name}\DUrole{p}{:}\DUrole{w}{  }\DUrole{n}{str}}, \emph{\DUrole{n}{dl\_backend}\DUrole{p}{:}\DUrole{w}{  }\DUrole{n}{str}}, \emph{\DUrole{n}{training\_timesteps}\DUrole{p}{:}\DUrole{w}{  }\DUrole{n}{int}}}{}
\end{fulllineitems}

\index{train\_ppo() (in module yawning\_titan.experiment\_helpers.rllib)@\spxentry{train\_ppo()}\spxextra{in module yawning\_titan.experiment\_helpers.rllib}}

\begin{fulllineitems}
\phantomsection\label{\detokenize{source/yawning_titan.experiment_helpers:yawning_titan.experiment_helpers.rllib.train_ppo}}\pysiglinewithargsret{\sphinxcode{\sphinxupquote{yawning\_titan.experiment\_helpers.rllib.}}\sphinxbfcode{\sphinxupquote{train\_ppo}}}{\emph{\DUrole{n}{env\_name}\DUrole{p}{:}\DUrole{w}{  }\DUrole{n}{str}}, \emph{\DUrole{n}{dl\_backend}\DUrole{p}{:}\DUrole{w}{  }\DUrole{n}{str}}, \emph{\DUrole{n}{training\_timesteps}\DUrole{p}{:}\DUrole{w}{  }\DUrole{n}{int}}}{}
\sphinxAtStartPar
Uses RLLib to train a PPO agent
\begin{quote}\begin{description}
\item[{Parameters}] \leavevmode\begin{itemize}
\item {}
\sphinxAtStartPar
\sphinxstyleliteralstrong{\sphinxupquote{dl\_backend}} \textendash{} The deep learning backend to be used (str)

\item {}
\sphinxAtStartPar
\sphinxstyleliteralstrong{\sphinxupquote{training\_timesteps}} \textendash{} number of training timesteps for the agent

\end{itemize}

\item[{Returns}] \leavevmode
\sphinxAtStartPar
A trained Rllib agent

\item[{Return type}] \leavevmode
\sphinxAtStartPar
trainer

\end{description}\end{quote}

\sphinxAtStartPar
Notes: There seems to be some terminology conflicts across Sb3 and Rllib.
Rllib uses training\sphinxhyphen{}timesteps in the same way sb3 uses episode counts.

\end{fulllineitems}



\paragraph{yawning\_titan.experiment\_helpers.sb3 module}
\label{\detokenize{source/yawning_titan.experiment_helpers:module-yawning_titan.experiment_helpers.sb3}}\label{\detokenize{source/yawning_titan.experiment_helpers:yawning-titan-experiment-helpers-sb3-module}}\index{module@\spxentry{module}!yawning\_titan.experiment\_helpers.sb3@\spxentry{yawning\_titan.experiment\_helpers.sb3}}\index{yawning\_titan.experiment\_helpers.sb3@\spxentry{yawning\_titan.experiment\_helpers.sb3}!module@\spxentry{module}}\index{init\_env() (in module yawning\_titan.experiment\_helpers.sb3)@\spxentry{init\_env()}\spxextra{in module yawning\_titan.experiment\_helpers.sb3}}

\begin{fulllineitems}
\phantomsection\label{\detokenize{source/yawning_titan.experiment_helpers:yawning_titan.experiment_helpers.sb3.init_env}}\pysiglinewithargsret{\sphinxcode{\sphinxupquote{yawning\_titan.experiment\_helpers.sb3.}}\sphinxbfcode{\sphinxupquote{init\_env}}}{\emph{\DUrole{n}{env}\DUrole{p}{:}\DUrole{w}{  }\DUrole{n}{str}}, \emph{\DUrole{n}{experiment\_id}\DUrole{p}{:}\DUrole{w}{  }\DUrole{n}{str}}}{}
\sphinxAtStartPar
Uses the Stable Baselines 3 Monitor wrappper to wrap an
environment in order to enable monitoring
\begin{quote}\begin{description}
\item[{Parameters}] \leavevmode\begin{itemize}
\item {}
\sphinxAtStartPar
\sphinxstyleliteralstrong{\sphinxupquote{env}} \textendash{} the registered name of an OpenAI gym environment (str)

\item {}
\sphinxAtStartPar
\sphinxstyleliteralstrong{\sphinxupquote{experiment\_id}} \textendash{} a UID for the experiment (str)

\end{itemize}

\item[{Returns}] \leavevmode
\sphinxAtStartPar
A Stable Baselines 3 Monitor Wrapped Gym Environment

\end{description}\end{quote}

\end{fulllineitems}

\index{print\_metric\_stats() (in module yawning\_titan.experiment\_helpers.sb3)@\spxentry{print\_metric\_stats()}\spxextra{in module yawning\_titan.experiment\_helpers.sb3}}

\begin{fulllineitems}
\phantomsection\label{\detokenize{source/yawning_titan.experiment_helpers:yawning_titan.experiment_helpers.sb3.print_metric_stats}}\pysiglinewithargsret{\sphinxcode{\sphinxupquote{yawning\_titan.experiment\_helpers.sb3.}}\sphinxbfcode{\sphinxupquote{print\_metric\_stats}}}{\emph{\DUrole{n}{metric\_name}}, \emph{\DUrole{n}{metrics}\DUrole{p}{:}\DUrole{w}{  }\DUrole{n}{list}}, \emph{\DUrole{n}{raw\_metrics}\DUrole{o}{=}\DUrole{default_value}{False}}}{}
\sphinxAtStartPar
Takes a metric name and a list of values and
prints stats
\begin{quote}\begin{description}
\item[{Parameters}] \leavevmode\begin{itemize}
\item {}
\sphinxAtStartPar
\sphinxstyleliteralstrong{\sphinxupquote{metric\_name}} \textendash{} The metric name (str)

\item {}
\sphinxAtStartPar
\sphinxstyleliteralstrong{\sphinxupquote{metrics}} \textendash{} A list of ints/float metric readings(list)

\end{itemize}

\end{description}\end{quote}
\begin{description}
\item[{Example Output:}] \leavevmode\begin{quote}

\sphinxAtStartPar
Agent: ppo
\end{quote}

\sphinxAtStartPar
—\sphinxhyphen{} Episode Reward —\sphinxhyphen{}
———\sphinxhyphen{}  ———\textendash{}
No. of Obs    50
Mean        \sphinxhyphen{}443.372
IQR           15
Min         \sphinxhyphen{}473.2
Max         \sphinxhyphen{}399.6
Variance     236.255
Skewness       0.610495
Kurtosis       0.527695
———\sphinxhyphen{}  ———\textendash{}
—\sphinxhyphen{} Episode Length —\sphinxhyphen{}
———\sphinxhyphen{}  ———\sphinxhyphen{}
No. of Obs   50
Mean         60
IQR          28.75
Min          27
Max         132
Variance    496.204
Skewness      0.956199
Kurtosis      1.01591
———\sphinxhyphen{}  ———\sphinxhyphen{}

\end{description}

\end{fulllineitems}

\index{print\_policy\_eval\_metrics() (in module yawning\_titan.experiment\_helpers.sb3)@\spxentry{print\_policy\_eval\_metrics()}\spxextra{in module yawning\_titan.experiment\_helpers.sb3}}

\begin{fulllineitems}
\phantomsection\label{\detokenize{source/yawning_titan.experiment_helpers:yawning_titan.experiment_helpers.sb3.print_policy_eval_metrics}}\pysiglinewithargsret{\sphinxcode{\sphinxupquote{yawning\_titan.experiment\_helpers.sb3.}}\sphinxbfcode{\sphinxupquote{print\_policy\_eval\_metrics}}}{\emph{\DUrole{n}{agents}\DUrole{p}{:}\DUrole{w}{  }\DUrole{n}{list}}, \emph{\DUrole{n}{evals}\DUrole{p}{:}\DUrole{w}{  }\DUrole{n}{list}}, \emph{\DUrole{n}{raw\_metrics}\DUrole{o}{=}\DUrole{default_value}{False}}}{}
\sphinxAtStartPar
Outputs policy evaluation metrics and summary statistics
\begin{quote}\begin{description}
\item[{Parameters}] \leavevmode\begin{itemize}
\item {}
\sphinxAtStartPar
\sphinxstyleliteralstrong{\sphinxupquote{agents}} \textendash{} An index of the agents that were evaluated(list)

\item {}
\sphinxAtStartPar
\sphinxstyleliteralstrong{\sphinxupquote{evals}} \textendash{} The output from Stable Baselines 3 ‘evaluate\_policy’
for each of the agents trained

\end{itemize}

\end{description}\end{quote}

\end{fulllineitems}

\index{train\_and\_eval() (in module yawning\_titan.experiment\_helpers.sb3)@\spxentry{train\_and\_eval()}\spxextra{in module yawning\_titan.experiment\_helpers.sb3}}

\begin{fulllineitems}
\phantomsection\label{\detokenize{source/yawning_titan.experiment_helpers:yawning_titan.experiment_helpers.sb3.train_and_eval}}\pysiglinewithargsret{\sphinxcode{\sphinxupquote{yawning\_titan.experiment\_helpers.sb3.}}\sphinxbfcode{\sphinxupquote{train\_and\_eval}}}{\emph{\DUrole{n}{agent\_name}\DUrole{p}{:}\DUrole{w}{  }\DUrole{n}{str}}, \emph{\DUrole{n}{environment}}, \emph{\DUrole{n}{training\_timesteps}\DUrole{p}{:}\DUrole{w}{  }\DUrole{n}{int}}, \emph{\DUrole{n}{n\_eval\_episodes}\DUrole{p}{:}\DUrole{w}{  }\DUrole{n}{int}}}{}
\sphinxAtStartPar
Trains and Evaluates an agent
\begin{quote}\begin{description}
\item[{Parameters}] \leavevmode\begin{itemize}
\item {}
\sphinxAtStartPar
\sphinxstyleliteralstrong{\sphinxupquote{agent\_name}} \textendash{} the algorithm name (str)

\item {}
\sphinxAtStartPar
\sphinxstyleliteralstrong{\sphinxupquote{environment}} \textendash{} An initlaised Open AI Gym environment

\item {}
\sphinxAtStartPar
\sphinxstyleliteralstrong{\sphinxupquote{training\_timesteps}} \textendash{} total no. of training timesteps (int)

\end{itemize}

\item[{Returns}] \leavevmode
\sphinxAtStartPar
a trained Stable Baselines 3 agent
eval\_pol: the output from the Stable Baselines 3 ‘evaluate\_policy’ function

\item[{Return type}] \leavevmode
\sphinxAtStartPar
chosen\_agent

\end{description}\end{quote}

\end{fulllineitems}



\paragraph{yawning\_titan.experiment\_helpers.tb\_eval\_plotter module}
\label{\detokenize{source/yawning_titan.experiment_helpers:module-yawning_titan.experiment_helpers.tb_eval_plotter}}\label{\detokenize{source/yawning_titan.experiment_helpers:yawning-titan-experiment-helpers-tb-eval-plotter-module}}\index{module@\spxentry{module}!yawning\_titan.experiment\_helpers.tb\_eval\_plotter@\spxentry{yawning\_titan.experiment\_helpers.tb\_eval\_plotter}}\index{yawning\_titan.experiment\_helpers.tb\_eval\_plotter@\spxentry{yawning\_titan.experiment\_helpers.tb\_eval\_plotter}!module@\spxentry{module}}
\sphinxAtStartPar
tb\_eval\_plotter.py \sphinxhyphen{} Evaluation Plotting Functions for data collected in Tensorboard
\index{show\_training\_data() (in module yawning\_titan.experiment\_helpers.tb\_eval\_plotter)@\spxentry{show\_training\_data()}\spxextra{in module yawning\_titan.experiment\_helpers.tb\_eval\_plotter}}

\begin{fulllineitems}
\phantomsection\label{\detokenize{source/yawning_titan.experiment_helpers:yawning_titan.experiment_helpers.tb_eval_plotter.show_training_data}}\pysiglinewithargsret{\sphinxcode{\sphinxupquote{yawning\_titan.experiment\_helpers.tb\_eval\_plotter.}}\sphinxbfcode{\sphinxupquote{show\_training\_data}}}{\emph{\DUrole{n}{filepath}}}{}\begin{description}
\item[{Function that creates a plot with the following graphs:}] \leavevmode\begin{itemize}
\item {}
\sphinxAtStartPar
Reward vs time step

\item {}
\sphinxAtStartPar
Episode length vs time step

\item {}
\sphinxAtStartPar
Episode length vs Reward

\end{itemize}

\end{description}

\sphinxAtStartPar
This data is taken from the tensorboards created while the models are training

\end{fulllineitems}



\paragraph{Module contents}
\label{\detokenize{source/yawning_titan.experiment_helpers:module-yawning_titan.experiment_helpers}}\label{\detokenize{source/yawning_titan.experiment_helpers:module-contents}}\index{module@\spxentry{module}!yawning\_titan.experiment\_helpers@\spxentry{yawning\_titan.experiment\_helpers}}\index{yawning\_titan.experiment\_helpers@\spxentry{yawning\_titan.experiment\_helpers}!module@\spxentry{module}}

\subsection{Module contents}
\label{\detokenize{source/yawning_titan:module-yawning_titan}}\label{\detokenize{source/yawning_titan:module-contents}}\index{module@\spxentry{module}!yawning\_titan@\spxentry{yawning\_titan}}\index{yawning\_titan@\spxentry{yawning\_titan}!module@\spxentry{module}}

\chapter{Indices and tables}
\label{\detokenize{index:indices-and-tables}}\begin{itemize}
\item {}
\sphinxAtStartPar
\DUrole{xref,std,std-ref}{genindex}

\item {}
\sphinxAtStartPar
\DUrole{xref,std,std-ref}{modindex}

\item {}
\sphinxAtStartPar
\DUrole{xref,std,std-ref}{search}

\end{itemize}


\renewcommand{\indexname}{Python Module Index}
\begin{sphinxtheindex}
\let\bigletter\sphinxstyleindexlettergroup
\bigletter{y}
\item\relax\sphinxstyleindexentry{yawning\_titan}\sphinxstyleindexpageref{source/yawning_titan:\detokenize{module-yawning_titan}}
\item\relax\sphinxstyleindexentry{yawning\_titan.agents}\sphinxstyleindexpageref{source/yawning_titan.agents:\detokenize{module-yawning_titan.agents}}
\item\relax\sphinxstyleindexentry{yawning\_titan.agents.keyboard\_agent}\sphinxstyleindexpageref{source/yawning_titan.agents:\detokenize{module-yawning_titan.agents.keyboard_agent}}
\item\relax\sphinxstyleindexentry{yawning\_titan.agents.static\_agents}\sphinxstyleindexpageref{source/yawning_titan.agents.static_agents:\detokenize{module-yawning_titan.agents.static_agents}}
\item\relax\sphinxstyleindexentry{yawning\_titan.agents.static\_agents.fixed\_red}\sphinxstyleindexpageref{source/yawning_titan.agents.static_agents:\detokenize{module-yawning_titan.agents.static_agents.fixed_red}}
\item\relax\sphinxstyleindexentry{yawning\_titan.agents.static\_agents.random\_agent}\sphinxstyleindexpageref{source/yawning_titan.agents.static_agents:\detokenize{module-yawning_titan.agents.static_agents.random_agent}}
\item\relax\sphinxstyleindexentry{yawning\_titan.agents.static\_agents.simple\_blue}\sphinxstyleindexpageref{source/yawning_titan.agents.static_agents:\detokenize{module-yawning_titan.agents.static_agents.simple_blue}}
\item\relax\sphinxstyleindexentry{yawning\_titan.envs}\sphinxstyleindexpageref{source/yawning_titan.envs:\detokenize{module-yawning_titan.envs}}
\item\relax\sphinxstyleindexentry{yawning\_titan.envs.generic}\sphinxstyleindexpageref{source/yawning_titan.envs.generic:\detokenize{module-yawning_titan.envs.generic}}
\item\relax\sphinxstyleindexentry{yawning\_titan.envs.generic.core}\sphinxstyleindexpageref{source/yawning_titan.envs.generic.core:\detokenize{module-yawning_titan.envs.generic.core}}
\item\relax\sphinxstyleindexentry{yawning\_titan.envs.generic.core.action\_loops}\sphinxstyleindexpageref{source/yawning_titan.envs.generic.core:\detokenize{module-yawning_titan.envs.generic.core.action_loops}}
\item\relax\sphinxstyleindexentry{yawning\_titan.envs.generic.core.blue\_action\_set}\sphinxstyleindexpageref{source/yawning_titan.envs.generic.core:\detokenize{module-yawning_titan.envs.generic.core.blue_action_set}}
\item\relax\sphinxstyleindexentry{yawning\_titan.envs.generic.core.blue\_interface}\sphinxstyleindexpageref{source/yawning_titan.envs.generic.core:\detokenize{module-yawning_titan.envs.generic.core.blue_interface}}
\item\relax\sphinxstyleindexentry{yawning\_titan.envs.generic.core.network\_interface}\sphinxstyleindexpageref{source/yawning_titan.envs.generic.core:\detokenize{module-yawning_titan.envs.generic.core.network_interface}}
\item\relax\sphinxstyleindexentry{yawning\_titan.envs.generic.core.red\_action\_set}\sphinxstyleindexpageref{source/yawning_titan.envs.generic.core:\detokenize{module-yawning_titan.envs.generic.core.red_action_set}}
\item\relax\sphinxstyleindexentry{yawning\_titan.envs.generic.core.red\_interface}\sphinxstyleindexpageref{source/yawning_titan.envs.generic.core:\detokenize{module-yawning_titan.envs.generic.core.red_interface}}
\item\relax\sphinxstyleindexentry{yawning\_titan.envs.generic.core.reward\_functions}\sphinxstyleindexpageref{source/yawning_titan.envs.generic.core:\detokenize{module-yawning_titan.envs.generic.core.reward_functions}}
\item\relax\sphinxstyleindexentry{yawning\_titan.envs.generic.generic\_env}\sphinxstyleindexpageref{source/yawning_titan.envs.generic:\detokenize{module-yawning_titan.envs.generic.generic_env}}
\item\relax\sphinxstyleindexentry{yawning\_titan.envs.generic.helpers}\sphinxstyleindexpageref{source/yawning_titan.envs.generic.helpers:\detokenize{module-yawning_titan.envs.generic.helpers}}
\item\relax\sphinxstyleindexentry{yawning\_titan.envs.generic.helpers.environment\_input\_validation}\sphinxstyleindexpageref{source/yawning_titan.envs.generic.helpers:\detokenize{module-yawning_titan.envs.generic.helpers.environment_input_validation}}
\item\relax\sphinxstyleindexentry{yawning\_titan.envs.generic.helpers.eval\_printout}\sphinxstyleindexpageref{source/yawning_titan.envs.generic.helpers:\detokenize{module-yawning_titan.envs.generic.helpers.eval_printout}}
\item\relax\sphinxstyleindexentry{yawning\_titan.envs.generic.helpers.graph2plot}\sphinxstyleindexpageref{source/yawning_titan.envs.generic.helpers:\detokenize{module-yawning_titan.envs.generic.helpers.graph2plot}}
\item\relax\sphinxstyleindexentry{yawning\_titan.envs.generic.helpers.network\_creator}\sphinxstyleindexpageref{source/yawning_titan.envs.generic.helpers:\detokenize{module-yawning_titan.envs.generic.helpers.network_creator}}
\item\relax\sphinxstyleindexentry{yawning\_titan.envs.generic.helpers.node\_attribute\_gen}\sphinxstyleindexpageref{source/yawning_titan.envs.generic.helpers:\detokenize{module-yawning_titan.envs.generic.helpers.node_attribute_gen}}
\item\relax\sphinxstyleindexentry{yawning\_titan.envs.generic.wrappers}\sphinxstyleindexpageref{source/yawning_titan.envs.generic.wrappers:\detokenize{module-yawning_titan.envs.generic.wrappers}}
\item\relax\sphinxstyleindexentry{yawning\_titan.envs.generic.wrappers.graph\_embedding\_observations}\sphinxstyleindexpageref{source/yawning_titan.envs.generic.wrappers:\detokenize{module-yawning_titan.envs.generic.wrappers.graph_embedding_observations}}
\item\relax\sphinxstyleindexentry{yawning\_titan.envs.specific}\sphinxstyleindexpageref{source/yawning_titan.envs.specific:\detokenize{module-yawning_titan.envs.specific}}
\item\relax\sphinxstyleindexentry{yawning\_titan.envs.specific.core}\sphinxstyleindexpageref{source/yawning_titan.envs.specific.core:\detokenize{module-yawning_titan.envs.specific.core}}
\item\relax\sphinxstyleindexentry{yawning\_titan.envs.specific.core.machines}\sphinxstyleindexpageref{source/yawning_titan.envs.specific.core:\detokenize{module-yawning_titan.envs.specific.core.machines}}
\item\relax\sphinxstyleindexentry{yawning\_titan.envs.specific.core.node\_states}\sphinxstyleindexpageref{source/yawning_titan.envs.specific.core:\detokenize{module-yawning_titan.envs.specific.core.node_states}}
\item\relax\sphinxstyleindexentry{yawning\_titan.envs.specific.core.nsa\_node}\sphinxstyleindexpageref{source/yawning_titan.envs.specific.core:\detokenize{module-yawning_titan.envs.specific.core.nsa_node}}
\item\relax\sphinxstyleindexentry{yawning\_titan.envs.specific.core.nsa\_node\_collection}\sphinxstyleindexpageref{source/yawning_titan.envs.specific.core:\detokenize{module-yawning_titan.envs.specific.core.nsa_node_collection}}
\item\relax\sphinxstyleindexentry{yawning\_titan.envs.specific.five\_node\_def}\sphinxstyleindexpageref{source/yawning_titan.envs.specific:\detokenize{module-yawning_titan.envs.specific.five_node_def}}
\item\relax\sphinxstyleindexentry{yawning\_titan.envs.specific.four\_node\_def}\sphinxstyleindexpageref{source/yawning_titan.envs.specific:\detokenize{module-yawning_titan.envs.specific.four_node_def}}
\item\relax\sphinxstyleindexentry{yawning\_titan.envs.specific.graph\_explore}\sphinxstyleindexpageref{source/yawning_titan.envs.specific:\detokenize{module-yawning_titan.envs.specific.graph_explore}}
\item\relax\sphinxstyleindexentry{yawning\_titan.envs.specific.nsa\_node\_def}\sphinxstyleindexpageref{source/yawning_titan.envs.specific:\detokenize{module-yawning_titan.envs.specific.nsa_node_def}}
\item\relax\sphinxstyleindexentry{yawning\_titan.experiment\_helpers}\sphinxstyleindexpageref{source/yawning_titan.experiment_helpers:\detokenize{module-yawning_titan.experiment_helpers}}
\item\relax\sphinxstyleindexentry{yawning\_titan.experiment\_helpers.constants}\sphinxstyleindexpageref{source/yawning_titan.experiment_helpers:\detokenize{module-yawning_titan.experiment_helpers.constants}}
\item\relax\sphinxstyleindexentry{yawning\_titan.experiment\_helpers.graph\_metrics}\sphinxstyleindexpageref{source/yawning_titan.experiment_helpers:\detokenize{module-yawning_titan.experiment_helpers.graph_metrics}}
\item\relax\sphinxstyleindexentry{yawning\_titan.experiment\_helpers.model\_evaluation}\sphinxstyleindexpageref{source/yawning_titan.experiment_helpers:\detokenize{module-yawning_titan.experiment_helpers.model_evaluation}}
\item\relax\sphinxstyleindexentry{yawning\_titan.experiment\_helpers.rllib}\sphinxstyleindexpageref{source/yawning_titan.experiment_helpers:\detokenize{module-yawning_titan.experiment_helpers.rllib}}
\item\relax\sphinxstyleindexentry{yawning\_titan.experiment\_helpers.sb3}\sphinxstyleindexpageref{source/yawning_titan.experiment_helpers:\detokenize{module-yawning_titan.experiment_helpers.sb3}}
\item\relax\sphinxstyleindexentry{yawning\_titan.experiment\_helpers.tb\_eval\_plotter}\sphinxstyleindexpageref{source/yawning_titan.experiment_helpers:\detokenize{module-yawning_titan.experiment_helpers.tb_eval_plotter}}
\end{sphinxtheindex}

\renewcommand{\indexname}{Index}
\printindex
\end{document}
